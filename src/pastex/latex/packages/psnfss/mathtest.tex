\documentclass{article}
\usepackage{mathptm}
\pagestyle{empty}
\begin{document}
\begin{center}
\bf\Large Sebastian's math test
\end{center}
The default math mode font is $math\ italic$. This should not be
confused with ordinary \emph{text italic}.   
\verb|\mathbf| produces $\mathbf{bold\ face\ roman}$ letters. If you wish to have
{\boldmath $bold\ face\ math\ italic$} letters, and bold face Greek
letters and mathematical symbols, use the \verb|\boldmath| command
\emph{before} going into math mode.  This changes the default math
fonts to bold. Greek is available in upper and lower case:
\(\alpha,\beta, \Gamma, \Delta \dots \omega, \Omega\)
 
\begin{tabular}{ll}
normal: \( x = 2\pi \Rightarrow x \simeq 6.28 \)\\
mathbf \(\mathbf{x} = 2\pi \Rightarrow x \simeq 6.28 \)\\
boldmath {\boldmath \(x = \mathbf{2}\pi \Rightarrow x 
           \simeq{\mathbf{6.28}}              \)}\\
\end{tabular}
 
\noindent There is also a calligraphic font for upper case letters;
these are produced by the \verb|\mathcal| command:\( \mathcal{ABCDE} \)
 

\begin{equation}
  \phi(t)=\frac{1}{\sqrt{2\pi}}
  \int^t_0 e^{-x^2/2} dx 
\end{equation}

\begin{equation}
  \prod_{j\geq 0}
  \left(\sum_{k\geq 0}a_{jk} z^k\right) 
= \sum_{k\geq 0} z^n
  \left( \sum_{{k_0,k_1,\ldots\geq 0}
          \atop{k_0+k_1+\ldots=n}    }
        a{_0k_0}a_{1k_1}\ldots  \right) 
\end{equation}

\begin{equation}
\pi(n) = \sum_{m=2}^{n}
  \left\lfloor \left(\sum_{k=1}^{m-1}
       \lfloor(m/k)/\lceil m/k\rceil 
       \rfloor \right)^{-1}
  \right\rfloor
\end{equation}

\begin{equation}
\{\underbrace{%
    \overbrace{\mathstrut a,\ldots,a}^{k\ a's},
    \overbrace{\mathstrut b,\ldots,b}^{l\ b's}}
  _{k+1\ \mathrm{elements}}                   \}
\end{equation}

\begin{displaymath}
\mbox{W}^+\
\begin{array}{l}
\nearrow\raise5pt\hbox{$\mu^+ + \nu_{\mu}$}\\
\rightarrow         \pi^+ +\pi^0         \\[5pt]
\rightarrow \kappa^+ +\pi^0              \\
\searrow\lower5pt\hbox{$\mathrm{e}^+ 
          +\nu_{\scriptstyle\mathrm{e}}$}
\end{array}
\end{displaymath}

\begin{displaymath}
\frac{\pm
\left|\begin{array}{ccc}
x_1-x_2  & y_1-y_2 & z_1-z_2 \\
l_1      & m_1     & n_1     \\
l_2      & m_2     & n_2
\end{array}\right|}{
\sqrt{\left|\begin{array}{cc}l_1&m_1\\
l_2&m_2\end{array}\right|^2
+     \left|\begin{array}{cc}m_1&n_1\\
n_1&l_1\end{array}\right|^2
+     \left|\begin{array}{cc}m_2&n_2\\
n_2&l_2\end{array}\right|^2}}
\end{displaymath}

\begin{displaymath}
\mbox{ acute=}\acute{a}
\mbox{ grave=}\grave{a}
\mbox{ ddot=}\ddot {a}
\mbox{ tilde=}\tilde{a}
\mbox{ bar=}\bar  {a}
\mbox{ breve=}\breve{a}
\mbox{ check=}\check{a}
\mbox{ hat=}\hat  {a}
\mbox{ vec=}\vec  {a}
\mbox{ dot=}\dot  {a}
\end{displaymath}

$$z=\sqrt{x}+{y\over a+b} + \alpha_2+{\Gamma\over\Omega}$$

\boldmath

$$z=\sqrt{x}+{y\over a+b} + \alpha_2+{\Gamma\over\Omega}$$

\unboldmath

$$z=\sqrt{x}+{y\over a+b} + \alpha_2+{\Gamma\over\Omega}$$

\end{document}
