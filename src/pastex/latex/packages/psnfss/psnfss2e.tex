\def\fileversion{3.2}
\def\filedate{1995/11/21}
%
% This document describes a LaTeX2e setup with
% package files for using PostScript fonts.
%
\def\PS{\textsc{PostScript}}
\documentclass{article}
\usepackage{shortvrb}
\begin{document}
\MakeShortVerb{+}
\title{Notes on setup of PostScript fonts for \LaTeXe}
\author{Sebastian Rahtz\\\texttt{s.rahtz@elsevier.co.uk}}
\date{\filedate}
\maketitle
%
\tableofcontents
\section{Introduction}

This set of files\footnote{Thanks are due to many, many, people for
  correcting errors and helping, including Michel Goossens, Alois
  Steindl, Peter Dyballa, Aloysius
  Helminck, Constantin Kahn, Berthold Horn, Alan Jeffrey, 
  Howard Marvel, Frank Poppe, 
  Andrew Trevorrow, Piet Tutelaars\ldots} offers a sample working setup for \LaTeXe's
NFSS and \PS\ fonts; it is based on the system I created between 1989
and 1991 for Southampton Computer Science department, checked and
updated for NFSS2, and subsequently \LaTeXe.  You should be familiar with
the standard \LaTeXe\ files and their use to follow this document.
It is assumed that Karl Berry's naming scheme is followed.  

Packages are offered to set documents in the common PostScript fonts,
plus common free fonts (Charter, Utopia etc).  All Lucida fonts are
supported. The MathTime fonts are not supported fully because
I do not understand them and do not have them; a package by Aloysius
Helminck is provided as-is in the `unsupported/mathtim'
subdirectory, about which enquiries should be addressed
to him.  

\section{Finding font metrics}
\textbf{IMPORTANT NOTE:} .fd files and .tfm files for the common setup are
\emph{not} included in this package. They can be found in the CTAN
archives in the companion collection (\texttt{fonts/psfonts}) for each
font family separately.  That collection also includes simpler \LaTeXe\ 
package files for each font family. To install a font family, take the
following steps:
\begin{enumerate}
\item Locate the font family in the \texttt{fonts/psfonts} directory,
  which is divided up by foundry (eg adobe, monotype etc). If, for
  instance, you want the `normal' Times Roman, this is in
  \texttt{adobe/times}. The family names for the directories are
  listed in Karl Berry's font-naming documentation.
\item Each family directory has subdirectories containing +.tfm+,
  +.vf+ and +.fd+ files. You need to install these where \LaTeXe\ and
  your driver will   find them. \LaTeXe\ will need the +.tfm+ and
  +.fd+ files, and the driver needs the +.vf+ files (and possibly the
  +.tfm+ ones too).
\item You now need to tell your driver that you are going to use these
  new PostScript fonts (to stop it looking for Metafont sources and
  +.pk+ files). Using +dvips+ this is accomplished by taking the +.map+
  file in the family directory and appending it to the standard
  +psfonts.map+ file of dvips. Alternatively you can install the
  +.map+ and the config file (named +config.+\emph{short family name} (eg
  Times' short name is ptm, so the config file is +config.ptm+) where
  dvips' support files live, and use eg +dvips -Pptm+ to load the
  extra +.map+ file on the fly. Refer to the dvips manual for full
  details of map and config files.

  If you don't use dvips, refer to your driver's manual for how to
  tell it about new PostScript fonts.
\item If the font is not resident in the printer, you'll have to
  download it. You can use software supplied with the font to do this,
  or have it done on the fly by some drivers. \emph{However}, note
  that the names in the +.map+ files supplied on CTAN assume strict
  conformancy with the Berry scheme --- you may have to rename your
  +.pfb+ or +.pfa+ font file.
\item If you just want to load one font family, and have it replace
  the default roman, sans or typewriter family in your document, each
  of the font family directories has a simple \LaTeX2e\ .sty package
  file.
\end{enumerate}

It is also \textbf{Very Important} to understand the naming system and
generation of the fonts! This setup follows the latest version of the
scheme by Karl Berry (on CTAN in +info/fontname+) religiously.  The
metric and +.fd+ files are named \emph{differently} from the AFM files
distributed by all font suppliers as this time (August 1995).  Thus, for
Times Roman, the OT1-encoded font is called +ptmr7t+, the T1-encoded
font is +ptmr8t+ and the raw font is +ptmr8r+. This system will be
followed exactly in all PostScript font support in \LaTeXe.

The font metric files whose use we assume are those generated using
Alan Jeffrey's \emph{fontinst} package. These used to generate quite
tight setting compared to other systems, which produced a lot of
hyphenation or overfull boxes if you were not careful. However, since
February 1995, the metrics have changed, so the advice to reset
various \TeX\ tolerance parameters etc no longer applies.

\section{Standard installation}
This distribution is provided as a set of \verb+.dtx+ files which need
to be unpacked using \emph{docstrip} to create user files.  The
resulting \verb+.sty+ files change the font defaults to use some new
group of fonts (sometimes just one default is changed).

Scripts are provided for \emph{docstrip} in the form of \texttt{.ins}
files, which simply need to be run through \TeX; when that has been
done, install all the \texttt{.sty} files that result in a
directory where \LaTeX\ will find them.

You have an important decision to make at some point --- are you going
to use fonts encoded in the `Cork' layout, or the old ones which look
like the CM fonts described in the \TeX\ book? This manual will not
attempt to explain why you should or should not use Cork fonts\ldots
Font description (\texttt{.fd}) files are available for both T1 and OT1
encoding in the CTAN \texttt{fonts/psfonts} directories.

If you follow the Cork-encoding route, you need different \TeX\ font
metric files and virtual font files. To activate this, use the package
\texttt{t1enc}.

\emph{Important.} If you use the Cork (T1 in \LaTeXe\ scheme) encoding, you
will probably also need the `dc' CM fonts to go with them, for maths
and so on.

The standard `35' \PS\ fonts built into most \PS\ printers
are known by their `Berry' names:
\begin{quote}
  \begin{tabular}{|ll|}
\hline
Family name & Full name\\
\hline
pag&Adobe AvantGarde\\
pbk&Adobe Bookman\\
pcr&Adobe Courier\\
phv&Adobe Helvetica\\
pnc&Adobe NewCenturySchoolbook\\
ppl&Adobe Palatino\\
ptm&Adobe TimesRoman\\
pzc&Adobe ZapfChancery\\
psy&Adobe Symbol font\\
pzd&Adobe ZapfDingbats\\
\hline
\end{tabular}
\end{quote}

To create font and package files for Lucida and Lucida Bright (including
Lucida Bright maths), run \TeX\ on \texttt{lucida.ins}. 

\section{Testing}
All installers should run \texttt{test0.tex} through \LaTeX\ and print
the result, after installing their chosen setup, to ensure that things
are more or less working. The OT1 encoding demonstration will
\emph{not} have a proper set of pounds signs! They will all be italic.
\texttt{test1.tex} will exercise your supply of
PostScript fonts.

\textbf{Do not worry if nothing but \texttt{test0.tex} works!}.
\texttt{test0.ps} is a prebuilt version of \texttt{test0.tex} for you
to compare.

\section{User interface}

The daily user will simply use one of the packages \texttt{times},
\texttt{newcent}, \texttt{helvet}, \texttt{palatino} etc to change the
default text fonts for one or more of the roman, sans-serif and
typewriter faces.  Table \ref{styles} lists the effects of the package
files created in the installation procedure.

The special package \texttt{pifont} gives access to the Dingbat
and Symbol fonts. This is described in \emph{The \LaTeX\ Companion}.
\begin{table}
\begin{small}
  \begin{tabular}{|l|lll|}
\hline
Package & Sans font & Roman font & Typewriter font\\
\hline
times.sty & Helvetica & Times & Courier\\
palatino.sty & Helvetica & Palatino & Courier\\
helvet.sty   & Helvetica & &\\
avant.sty   & AvantGarde & &\\
newcent.sty & AvantGarde & NewCenturySchoolbook & Courier\\
bookman.sty &  AvantGarde & Bookman & Courier\\
\hline
\end{tabular}
\end{small}
\caption{Effect of package files\label{styles}}
\end{table}
Note that maths fonts will stay the same unless you have suitable
fonts to load. If the Adobe Lucida Maths fonts have been purchased, and
appropriate metrics obtained, loading \texttt{lucmath} will remove
all reference to CMR fonts in the document. Alternatively, purchase
the Lucida Bright font set and use the \texttt{lucbr} package.

\subsection{Variant OT1 font encoding}
The package files assume that you have already made the choice of
which text font encoding scheme you prefer (T1 or OT1), and that it is
the default when the \LaTeX\ job starts. If you end up using
older OT1 \verb+tfm+ files distributed with dvips before mid 1995,
you'll find some characters are not in the expected places.
Similarly, older Textures users will find things not quite right. Y\&Y
users may be loading reencoding packages which moves things around.
Prior to 1995, PSNFSS provided
a package  \verb+ot1var+ to cope with this sort of situation.
It is now \emph{not supported or maintained}. The remnants are in
the obsolete subdirectory for the curious.

\section{Font family names}
\begin{tabular}{|ll|}
\hline
Family name & Full name\\
\hline
bch & Bitstream Charter\\
hlc&B\&H Lucida Bright\\
hlcs&B\&H Lucida Sans\\
hlct&B\&H Lucida Bright Typewriter\\
pgm&Adobe Garamond\\
mim&Monotype Imprint\\
mnt&Monotype Times New Roman\\
pgm&Adobe ITC Garamond\\
pgs&Adobe MGillSans\\
pgs&Adobe MGillSans\\
plc&Adobe Lucida\\
plcs&Adobe Lucida Sans\\
pnb&NewBaskerville\\
pop&Adobe Optima\\
pun&Adobe Univers\\
put&Adobe Utopia-Regular\\
unmr&URW NimbusRoman-Regular\\
unmrs&URW NimbusSans-Regular\\
\hline
\end{tabular}

\begin{table}
\begin{small}
  \begin{tabular}{|l|lll|}
\hline
Package & Sans font & Roman font & Typewriter font\\
\hline
basker.sty & & Monotype Baskerville &\\
bembo.sty &  & Bembo & \\
charter & & Bitstream Charter & \\
garamond.sty & & Adobe Garamond & \\
mtimes & & Monotype Times & \\
nimbus  & URW NimbusSans-Regular & URW NimbusRoman-Regular & \\
utopia  & & Utopia & \\
lucid.sty & LucidaSans & Lucida & Courier\\
lucbr.sty & LucidaSans & LucidaBright & LucidaTypewriter\\
\hline
\end{tabular}
\end{small}
\caption{Effect of extra package files\label{exstyles}}

Notes: a) \texttt{lucbr.sty} uses the font names for Lucida Bright
which conform to Karl Berry's scheme. Use package option `yy' to use
the font names supplied by Y\&Y. b) If you want to use just standard
PostScript fonts for math, Alan Jeffrey's \emph{mathptm} package does
as good a job as possible (though it still needs access to some CMR
math fonts). The extra metric and virtual font files that this needs
are supplied with Adobe Times Roman in the CTAN
\verb|fonts/psfonts/adobe/times| directory.
\end{table}

\end{document}







