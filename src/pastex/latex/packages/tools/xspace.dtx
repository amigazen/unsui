% \iffalse
%% File: xspace.dtx Copyright (C) 1991-1996 David Carlisle
%
%<*dtx>
          \ProvidesFile{xspace.dtx}
%</dtx>
%<package>\NeedsTeXFormat{LaTeX2e}
%<package>\ProvidesPackage{xspace}
% \fi
%         \ProvidesFile{xspace.dtx}
          [1996/05/17 v1.04 Space after command names (DPC)]
%
% \iffalse
%<*driver>
\documentclass{ltxdoc}
\usepackage{xspace}
\begin{document}
\DocInput{xspace.dtx}
\end{document}
%</driver>
% \fi
%
% \GetFileInfo{xspace.dtx}
% \title{The \textsf{xspace} package\thanks{This file
%        has version number \fileversion, last
%        revised \filedate.}}
% \author{David Carlisle}
% \date{\filedate}
% \maketitle
%
% %%%%%%%%%%%%%%%%%%%%%%%%%%%%%%%%%%%%%%%%%%%%%%%%%%%%%%%%%%%%%%%%%%%%
%
% \CheckSum{72}
%
% \changes{v1.00}{1991/08/30}{Initial version}
% \changes{v1.01}{1992/06/26}{Re-issue for the new doc and docstrip}
% \changes{v1.02}{1994/01/31}{Re-issue for LaTeX2e (no change to code)}
%
% \begin{abstract}
% |\xspace| should be used at the end of a macro designed to be used
% mainly in text. It adds a space unless the macro is followed by
% certain punctuation characters.
% \end{abstract}
%
% \section{Introduction}
% After |\newcommand{\gb}{Great Britain\xspace}|\\
% \newcommand{\gb}{Great Britain\xspace}
% |\gb is a very nice place to live.|\\
% \gb is a very nice place to live.\\
% |\gb, a small island off the coast of France.|\\
% \gb, a small island off the coast of France.
%
% |\xspace| saves the user from having to type \verb*+\ + or |{}| after
% most occurrences of a macro name in text. However if either of these
% constructions follows |\xspace|, a space is not added by |\xspace|.
% This means that it is safe to add |\xspace| to the end of an existing
% macro without making too many changes in your document.
%
% Sometimes |\xspace| may make the wrong decision, and add a space when
% it is not required. In these cases follow the macro with |{}|, as this
% has the effect of suppressing the space.
%
% Note that this package must be loaded \emph{after} any language (or
% other) packages that make punctuation characters `active'.
%
% \StopEventually{}
%
% \section{The Macros}
%
%    \begin{macrocode}
%<*package>
%    \end{macrocode}
%
% \begin{macro}{\xspace}
% |\xspace| just looks ahead, and then calls |\@xspace|.
% \changes{v1.03}{1994/11/15}{Make robust}
%    \begin{macrocode}
\DeclareRobustCommand\xspace{\futurelet\@let@token\@xspace}
%    \end{macrocode}
% \end{macro}
% \begin{macro}{\@xspace}
% If the next token is one of a specified list of characters, do
% nothing, otherwise add a space. If you often use a different
% punctuation character, add the appropriate line (do not forget the
% |\fi| at the  end!)
% \changes{v1.03}{1994/11/15}{Add exclamation mark}
% \changes{v1.04}{1996/05/17}{Add slash}
%    \begin{macrocode}
\def\@xspace{%
  \ifx\@let@token\bgroup\else
  \ifx\@let@token\egroup\else
  \ifx\@let@token\/\else
  \ifx\@let@token\ \else
  \ifx\@let@token~\else
  \ifx\@let@token.\else
  \ifx\@let@token!\else
  \ifx\@let@token,\else
  \ifx\@let@token:\else
  \ifx\@let@token;\else
  \ifx\@let@token?\else
  \ifx\@let@token/\else
  \ifx\@let@token'\else
  \ifx\@let@token)\else
  \ifx\@let@token-\else
   \space
   \fi\fi\fi\fi\fi\fi\fi\fi\fi\fi\fi\fi\fi\fi\fi}
%</package>
%    \end{macrocode}
% \end{macro}
%
% \Finale
%

