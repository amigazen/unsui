% \iffalse meta-comment
%
% Copyright 1989-1995 Johannes L. Braams and any individual authors
% listed elsewhere in this file.  All rights reserved.
% 
% For further copyright information any other copyright notices in this
% file.
% 
% This file is part of the Babel system release 3.5.
% --------------------------------------------------
%   This system is distributed in the hope that it will be useful,
%   but WITHOUT ANY WARRANTY; without even the implied warranty of
%   MERCHANTABILITY or FITNESS FOR A PARTICULAR PURPOSE.
% 
%   For error reports concerning UNCHANGED versions of this file no more
%   than one year old, see bugs.txt.
% 
%   Please do not request updates from me directly.  Primary
%   distribution is through the CTAN archives.
% 
% 
% IMPORTANT COPYRIGHT NOTICE:
% 
% You are NOT ALLOWED to distribute this file alone.
% 
% You are allowed to distribute this file under the condition that it is
% distributed together with all the files listed in manifest.txt.
% 
% If you receive only some of these files from someone, complain!
% 
% Permission is granted to copy this file to another file with a clearly
% different name and to customize the declarations in that copy to serve
% the needs of your installation, provided that you comply with
% the conditions in the file legal.txt from the LaTeX2e distribution.
% 
% However, NO PERMISSION is granted to produce or to distribute a
% modified version of this file under its original name.
%  
% You are NOT ALLOWED to change this file.
% 
% 
% \fi
%%% ====================================================================
%%%  @TeX-file{
%%%     author          = "Johannes L. Braams",
%%%     version         = "1.1",
%%%     date            = "03 May 1995",
%%%     time            = "14:34:53 MET",
%%%     filename        = "tb1202.tex",
%%%     address         = "Kooienswater 62
%%%                        2715 AJ Zoetermeer
%%%                        The Netherlands",
%%%     telephone       = "(79) 522819",
%%%     FAX             = "(70) 3432395",
%%%     checksum        = "08898 922 4995 39163",
%%%     email           = "J.L.Braams@research.ptt.nl (Internet)",
%%%     codetable       = "ISO/ASCII",
%%%     keywords        = "",
%%%     supported       = "yes",
%%%     abstract        = "",
%%%     docstring       = "The article on babel that apeared in
%%%                        TuGboat Volume 12, No. 2.
%%%
%%%                        The checksum field above contains a CRC-16
%%%                        checksum as the first value, followed by the
%%%                        equivalent of the standard UNIX wc (word
%%%                        count) utility output of lines, words, and
%%%                        characters.  This is produced by Robert
%%%                        Solovay's checksum utility.",
%%%  }
%%% ====================================================================
%               tb32braa.ltx

\NeedsTeXFormat{LaTeX2e}
\documentclass{ltugboat}
\usepackage{shortvrb}
\MakeShortVerb{\|}
\def\rtitlex{TUGboat, Volume 12 (1991), No. 2}
\def\midrtitle{}
\def\PrelimDraftfooter{}
\let\TBmaketitle\maketitle
\SelfDocumenting
\setcounter{page}{291}
\newcommand{\file}[1]{\texttt{#1}}
\newcommand{\Lopt}[1]{\textsf{#1}}
\newcommand{\babel}{\textsf{babel}}
\makeatletter
%  -------------------------------------------------
%  Definitions copied from doc.sty by FMi
%  -------------------------------------------------
{\obeyspaces%
\gdef\meta{\begingroup\obeyspaces%
\def {\egroup\space\hbox\bgroup\it}\m@ta}}
\def\m@ta#1{\hbox\bgroup$\langle$\it#1\/$\rangle$\egroup\endgroup}

%  -------------------------------------------------
%  define \dlqq so that we can give an example of "'
%  -------------------------------------------------
\gdef\dlqq{{\setbox\tw@=\hbox{,}\setbox\z@=\hbox{''}%
  \dimen\z@=\ht\z@ \advance\dimen\z@-\ht\tw@
  \setbox\z@=\hbox{\lower\dimen\z@\box\z@}\ht\z@=\ht\tw@
  \dp\z@=\dp\tw@ \box\z@\kern-.04em}}
%
%  -------------------------------------------------
%  To demonstrate german double quotes (from german.tex)
%  -------------------------------------------------
\def\allowhyphens{\penalty\@M \hskip\z@skip}
\def\set@low@box#1{\setbox\tw@\hbox{,}\setbox\z@\hbox{#1}\dimen\z@\ht\z@
     \advance\dimen\z@ -\ht\tw@
     \setbox\z@\hbox{\lower\dimen\z@ \box\z@}\ht\z@\ht\tw@ \dp\z@\dp\tw@ }
%    (this lowers the german left quotes to the same level as the comma.)

\def\@glqq{{\ifhmode \edef\@SF{\spacefactor\the\spacefactor}\else
     \let\@SF\empty \fi \leavevmode
     \set@low@box{''}\box\z@\kern-.04em\allowhyphens\@SF\relax}}
\def\glqq{\protect\@glqq}
\def\@grqq{\ifhmode \edef\@SF{\spacefactor\the\spacefactor}\else
     \let\@SF\empty \fi \kern-.07em``\kern.07em\@SF\relax}
\def\grqq{\protect\@grqq}
%
%  -------------------------------------------------
%  To demonstrate french double quotes (from german.tex)
%  -------------------------------------------------
\def\@flqq{\ifhmode \edef\@SF{\spacefactor\the\spacefactor}\else
     \let\@SF\empty \fi
     \ifmmode \ll \else \leavevmode
     \raise .2ex \hbox{$\scriptscriptstyle \ll $}\fi \@SF\relax}
\def\flqq{\protect\@flqq}
\def\@frqq{\ifhmode \edef\@SF{\spacefactor\the\spacefactor}\else
     \let\@SF\empty \fi
     \ifmmode \gg \else \leavevmode
     \raise .2ex \hbox{$\scriptscriptstyle \gg $}\fi \@SF\relax}
\def\frqq{\protect\@frqq}
%+
%     Some extra definitions needed in the article
%-
\def\bsl{\char'134}
\makeatother

\begin{document}
\title {Babel, a multilingual style-option system for use
        with \LaTeX's standard document styles%
       \thanks{During the development ideas from Nico Poppelier,
               Piet van Oostrum and many others have been used.}
       }

\author{Johannes Braams}
\address{PTT Research Neher Laboratories\\
         P.O. Box 421\\
         2260 AK Leidschendam}
\netaddress{J.L.Braams@research.ptt.nl}

\date{22 July 1993}

\maketitle

\begin{abstract}
  The standard distribution of \LaTeX\ contains a number of document
  styles that are meant to be used, but also serve as examples for
  other users to create their own document styles.  These styles have
  become very popular among \LaTeX\ users.  But it should be kept in
  mind that they were designed for American tastes and contain a
  number of hard-wired texts. This article describes a set of
  document-style options that can be used in combination with the
  standard styles, which makes the latter adaptable to other
  languages.
\end{abstract}

\section{Introduction}

Although Leslie Lamport has stated~\cite{LLth} that one should not try
and write \emph{one} document-style option to be used with \emph{all}
the standard document styles of \LaTeX, that is exactly what I~have
done with this system of style options. The reasons for this approach
will be explained in section~\ref{why}.

A lot of the ideas incorporated in this set of files come from the
work of Hubert Partl~\cite{HP}, \file{german.tex}. Some parts in the
implementation are different, others are the same. It will be shown
that \file{german.tex} can be modified to fit into this scheme of
style options.

\section{Why \babel?}\label{why}

When I~first started using \LaTeX\ I~was very happy with just the
style files that are distributed with the standard distributions of
\TeX\ and \LaTeX. That means, as long as I~made texts in English I~was
happy. Then as other users found out about \LaTeX\ and its advantages,
they started using it for texts in languages other than English. As
I~was the most experienced \LaTeX\ user at the time, they came to me
and asked me `When I'm writing a report in Dutch I~don't want chapters
to be named ``Chapter'', I~want them to be named ``Hoofdstuk'', how do
you change that?'. At that time I~didn't know, but I soon found out.
The first thing I~found was that Leslie Lamport states~\cite[pages
85--86]{LLbook} that you have to redefine the command |\@chapapp|
to get the desired result. This looked rather promising to me, so
I~had a look at the style files to find out how other such strings as
``Figure'' might be redefined. It was then that I~found out that
|\@chapapp| is the \emph{only} string defined this way, whereas
all others are hard-wired into the style.

My first solution to this problem was to create a new document style
file called \file{artikel.sty} as a ``Dutch'' counterpart to
\file{article.sty}. The same was done for \file{report.sty}. This is
exactly what Leslie Lamport suggests~\cite{LLth}.  This approach has
one major drawback however: you get two copies of basically the same
file to maintain. This was discovered when newer releases of the
styles reached our site.  The standard styles had to be replaced
\emph{and} edited all over again to get the ``Dutch'' versions back.
About the same time, in early 1988, a discussion on this subject
appeared in \TeXhax. One of the persons commenting was Hubert Partl.
The method he suggested was to modify the standard document styles by
replacing the hard-wired texts by macros such as |\@chapapp|.  This
led me to my second attempt at a solution.  I~modified the standard
styles (all four of them) as suggested, but while doing that added an
option, implemented like the option \texttt{draft}, by defining a
command |\ds@dutch|. This command would set a variable to indicate
which language was requested.  This variable I~used later on in a
|\case| statement. In this |\case| statement a choice is made between
English, Dutch and possibly other languages for texts such as
``Figure'' and ``Contents''.  Unfortunately, some of this implied
changing the secondary style files \file{xxx10.sty}, \file{xxx11.sty}
and \file{xxx12.sty}.  This was unfortunate because one of the
research groups in our laboratories complained their document style
didn't work properly.  It turned out that their style was a modified
\file{article.sty} that had been given a different name, but it still
loaded \file{art10.sty} etc.  I~found a temporary solution, but I
still wasn't exactly happy with the situation. Besides this, the
drawback of replacing the document styles with newer versions still
existed.

When after a while a new version of the \LaTeX\ distribution arrived
at our site, I~began to think about a different way to solve the
problem. In the meantime Hubert Partl had his \file{german.sty}
published in \TUB~\cite{HP}. His article pointed the way to a
different solution. Triggered by the discussion in \TeXhax\ in early
1989 about how to detect which is the main (primary) style when
processing a document, I~started work on what is now available as
\file{dutch.sty} version~1.0, dated may 1989\footnote{This file is
  available from \texttt{listserv@hearn.bitnet} as file
  \file{dutch.old}.}.  While working on this style option I~discovered
that some parts could be borrowed from \file{german.sty}. This
`discovery' and some discussions I had with others at Euro\TeX89, the
fourth European \TeX\ Conference, held in september 1989 in Karlsruhe,
led me towards a more universal approach. The basic idea behind it
was, starting from the algorithm to detect the main style, to design
an approach with one common file that contained macro definitions
needed by a number of language-specific style options. Users specify
the name of any of these language-specific options as an option to the
|\documentstyle| command, and internally the common file is read.

%This is the situation as it stands now. The rest of this article is
%devoted to a description of how the system of style options described
%above is implemented and what the possibilities are. I~realize that
%with the new \TeX\ 3.0 around and work being done on a new version of
%\LaTeX\ as well some of this work may become obsolete in the near
%future. But, taking into consideration the widespread use of the
%current versions of \TeX\ and \LaTeX\, I~think it may take some time
%before {\em everybody} is working with the new releases.

%------------------------------------------------------------
% All text above here was copied from babel.doc, the following
% section is new for this version of the article
%------------------------------------------------------------
\section{\LaTeX\ and document-style files}\label{lat-docstyle}

Before the I discuss some of the code in the \babel{} system I would
like to discuss the document-style mechanism used by \LaTeX.  Every
\LaTeX\ document should start with a line like:
\begin{verbatim}
\documentstyle[opt1,opt2,...]{docstyle}
\end{verbatim}
This line of code instructs \LaTeX\ to first load the file
\file{docstyle.sty}.  When that is done the `options' are processed
\emph{in the order specified}, by reading the files
\file{opt1.sty}\footnote{Except when the documentstyle defines the
  control sequence {\tt\bsl ds@\meta{opt1}}; in that case this control
  sequence will be executed.}, \file{opt2.sty}, etc. This implies that
definitions, made in the file \file{docstyle.sty} can be overridden in
one of the option files. It is even possible to redefine code from the
very kernel of \LaTeX, but you have to know what you are doing.

Some care has to be taken in writing document-style options, because a
number of problems can occur. First of all, if a document-style option
should be modest in size; if it tries to redefine most of the code in
\file{docstyle.sty} I think you should write (and maintain) your own,
complete, document style.  Next, as it was possible to override
definitions from the main file in an option file, it is of course also
possible to override definitions made in another option file. When
this happens, your document might depend on the order in which you
have specified your document-style options.

This mechanism of overriding definitions from the main document style
is exploited in the \babel{} system. The macros that contain the
hard-wired texts are redefined in the common part of \babel{},
replacing each of these texts with a unique macro. These macros have
to be defined in the language-specific files.

\section{\LaTeX\ and multilingual documents}\label{lat-lingual}

In a european environment it sometimes happens that one wants to write
a document that contains more than \emph{one} language. I have an
example of a document, published by the {\sc eec}, that contains 9
(nine) different languages. Also in linguistics one can find documents
written in more than one language, i.e. to compare two languages.

If you have to write such a multilingual document you should try to
conform to the typographical conventions in use for each language. A
well known example is the type of quotation marks used. \TeX\ supplies
the user with ``quoted text'', but a Dutch user might want to have
\dlqq quoted text'', whereas a German text should contain \glqq quoted
text\grqq\ and a frenchman would perhaps like to see something like
\flqq quoted text\frqq. These language specific conventions should be
implemented in a document-style option file for each language. These
files should then be useable with \emph{all} document styles.

In such a multilingual document a user would specify the languages
used as options to the |\documentstyle| command. He would also want a
mechanism to be able to switch between these languages in a simple
way. When he would use \TeX\ version 3.0 for the processing of his
document, he would also want the hyphenation to come out right for the
different languages.

\section{Overview of the \babel{} solution}

\subsection{The core of the system}

The problems described in sections~\ref{lat-docstyle}
and~\ref{lat-lingual} can be solved using the \babel{} system of
document-style options.

The core of this system currently performs three functions.
\begin{enumerate}

\item\label{switch} It defines a user interface for switching between
  languages;

\item\label{hyphs} It contains code to dynamically load several sets
  of hyphenation patterns;

\item\label{repair} It `repairs' the document styles provided in the
  standard distribution of \LaTeX.

\end{enumerate}

Obviously part~\ref{hyphs} can only be used while running ini\TeX\ to
create a new format, whereas part~\ref{repair} should \emph{not} be
read by ini\TeX.  Part~\ref{repair} should even disappear when \LaTeX\
version 3.0 arrives, as the style files supplied with the new \LaTeX\
will no longer be language specific. Part~\ref{switch} can either be
loaded into the format with multiple hyphenation patterns, or it can
be read while processing a document.

For this reason the core of the \babel{} system is stored in two
separate files, \file{babel.switch}, containing parts~\ref{switch}
and~\ref{hyphs}, and \file{babel.sty} which contains
part~\ref{repair}. The file \file{babel.sty} will instruct \LaTeX\ to
load \file{babel.switch} if necessary, the file \file{babel.switch}
checks the format to see if hyphenation patterns \emph{can} be loaded.

\subsection{Language specifics}

The language switching mechanism contains a couple of hooks for the
developers of language-specific document-style options.

First of all the macro |\originalTeX| should be defined. Its function
is to disable special definitions made for a language to bring \TeX\
into a `defined' state. A language-specific document-style option
might, for example, introduce an extra active character. It would then
also modify the definitions of |\dospecials| and |\@sanitize|. Such an
option would then define a macro to restore the original definitions
of these macros and restore the extra active character to its normal
category code. It would then |\let \originalTeX| to this `restoration'
macro.

To enable the language-specific definitions three macros are provided
in the switching mechanism, |\captions|\meta{language},
|\date|\meta{language} and |\extras|\meta{language}.

The macro |\captions|\meta{language} should provide definitions for
the macros that replaced the hard-wired texts in the document style
and the macro |\date|\meta{language} should provide a definition for
|\today|.  The real fun starts with the macro
|\extras|\meta{language}. This macro should activate all definitions
needed for \meta{language}.

\section{The user interface}

The user interface to the \babel{} system is quite simple. He should
specify the languages he wants to use in his document in the list of
document-style options. For instance, for a document in which both the
English and the Dutch language are used, the first line could read:
\begin{verbatim}
\documentstyle[a4,dutch,english]{artikel1}
\end{verbatim}
Please note that in this case the Dutch-specific definitions are
inactive when \LaTeX\ has finished processing document-style option
files.

If the user then wants to switch from English to Dutch he would
include the command
\begin{verbatim}
\selectlanguage{dutch}
\end{verbatim}
before starting to write Dutch.

If a user wants to write a document-style option of his own he might
like to define a macro that checks which language is in use at the
time the macro is executed. For this purpose the macro
|\iflanguage{|\meta{language}|}{|\meta{then-clause}|}{|%
\meta{else-clause}|}| is available.

\section{Implementation of the core of the system}

In this section I would like to discuss some parts of the
implementation of the \babel{} system. Not all code will be shown,
because some parts of it are just series of slightly modified code
from the standard document styles. The files are fully documented and
interested readers can print them if they have access to the
\texttt{doc} option, described by Frank Mittelbach.

The description of the macros that follows is based on an environment
using \TeX~3.x, together with a version of \file{lplain.tex} based on
\file{plain.tex} version~3.x. The actual implementation allows for
other situations as well, i.e a version of \file{babel.sty} for
\TeX~2.x will be available.

\subsection{Switching languages}\label{lang-switch}

For each language to be used in a document a control sequence of the
form |\l@|\meta{language} has to be defined. This will either be done
while loading hyphenation patterns or while loading the
language-specific file. The implementation of
\hbox{|\selectlanguage{|\meta{language}|}|} and\\
\hbox{|\iflanguage{|\meta{language}|}{|\meta{then case}|}{|\meta{else
      case}|}|} is based on the existence of
\hbox{|\l@|\meta{language}}.

\begin{figure*}[htb]
\begin{verbatim}
\def\selectlanguage#1{%
  \@ifundefined{l@#1}
       {\@nolanerr{#1}}
       {\originalTeX
        \language=\expandafter\csname l@#1\endcsname\relax
        \expandafter\csname captions#1\endcsname
        \expandafter\csname date#1\endcsname
        \expandafter\csname extras#1\endcsname
        \gdef\originalTeX{\expandafter\csname noextras#1\endcsname}
       }
}
\end{verbatim}
\caption{The definition of {\tt\bsl selectlanguage}.}
\label{select}
\end{figure*}

To switch from one language to another the macro |\selectlanguage| is
available. Its definition can be seen in figure~\ref{select}.  The
first action it takes is to check whether the \meta{language} is
known, if it is not an error is signalled.  If the language is known
|\originalTeX| is called upon to reset any previously set
language-specific definitions.  Next the register |\language| is
updated and the three macros that should activate all
language-specific definitions are executed.  Finally the macro
|\originalTeX| receives a new replacement text in order to be able to
deactivate the definitions just activated.

\begin{figure*}[htb]
\begin{verbatim}
\def\iflanguage#1#2#3{%
  \@ifundefined{l@#1}
    {\@nolanerr{#1}}
    {\ifnum\language=\expandafter\csname l@#1\endcsname\relax
       #2 \else #3
     \fi}
}
\end{verbatim}
\caption{The definition of {\tt\bsl iflanguage}}
\label{if}
\end{figure*}

The macro |\iflanguage| (see figure~\ref{if}) will issue a warning
when its argument is an `unkown' language. It then goes on to compare
the value of |\language| and |\l@|\meta{language} and executes either
its secon or third argument.

\subsection{Dynamically loading patterns}

With the advent of \TeX~3.0 it has become possible to build a format
with more than one hyphenation pattern preloaded. The core of the
\babel\ system provides code, to be executed by ini\TeX\ \emph{only},
to dynamically load hyphenation patterns. The only restriction is that
the implementation of \TeX\ that you use has to have rather high
settings of \texttt{trie\_size} and \texttt{trie\_op\_size} to
actually load several hyphenation patterns.

For the purpose of dynamically loading hyphenation patterns a
`configuration file' has to be introduced. This file will be read by
ini\TeX. Each line should contain either a comment, nothing or the
name of a language and the name of the file that contains the
hyphenation patterns for that language. In figure~\ref{config} an
example of such a file, instructing ini\TeX\ to load patterns for
three languages, English, Dutch and German.

\begin{figure*}[htb]
\begin{verbatim}
% File    : language.dat
% Purpose : tell iniTeX what files with patterns to load.
english    english.hyphenations

dutch      hyphen.dutch % Nederlands
german hyphen.ger
\end{verbatim}
\caption{An example configuration file}\label{config}
\end{figure*}

The configurationfile will be read line by line using \TeX's |\read|
primitive. Because the name of a file might be followed by a
space-token and comment (as in the example) a macro to process each
line is needed. The definition of this macro, |\process@language|, can
be found in figure~\ref{process}. As can be seen in the definition of
this macro, its second argument \emph{always} has to be followed by a
space-token. The effect of this is that any trailing spaces are
removed.
\begin{figure*}[htb]
\begin{verbatim}
\def\process@language#1 #2 {%
     \expandafter\addlanguage\csname l@#1\endcsname
     \expandafter\language\csname l@#1\endcsname
     \input #2}
\end{verbatim}
\caption{The definition of {\tt\bsl process@language}.}
\label{process}
\end{figure*}
The macro strips all spaces following its arguments. It's first
argument is used to define |\l@|\meta{language}. The macro
|\addlanguage| is basically a non-outer version of the plain \TeX\
macro |\newlanguage|. The second argument of |\process@language| is
the name of the file containing the hyphenation patterns. Before the
file can be read, the register |\language| has to updated.

The configuration file is read in a |\loop| (see figure~\ref{loop}).
When a record is read from the input file a check is done whether the
record was empty. If it was not, a space token is added to the end of
the string of tokens read. The reason for this is that we have to be
sure there always is at least \emph{one} space token present. When
that has been taken care of the data just read can be processed. The
last thing to do is to check the status of the input file, in order to
decide whether \TeX\ has to continue processing the |\loop|. When all
patterns have been processed the value of |\language| is restored.

\begin{figure*}[htb]
\begin{verbatim}
\loop
  \read1 to \@config@line
  \ifx\@config@line\empty
  \else
    \edef\@config@line{\@config@line\space}
    \expandafter\process@language\@config@line
  \fi
  \ifeof1 \@morefalse \fi
  \if@more\repeat
\language=0
\end{verbatim}
\caption{Reading the configuration file line by line}
\label{loop}
\end{figure*}

\subsection{`Repairing' \LaTeX's standard document styles}

A large part of the core of the \babel{} system is dedicated to
`repair' the standard document styles. This means redefining the
macros in table~\ref{macros}.

\begin{table*}[htb]
\DeleteShortVerb{\|}
\begin{center}
\begin{tabular}{l | c c c c}
macro & article & report & book & letter \\
\hline
\tt\bsl fnum@figure      & $\times$ & $\times$ & $\times$ & $\times$ \\
\tt\bsl fnum@table       & $\times$ & $\times$ & $\times$ & $\times$ \\
\tt\bsl tableofcontents  & $\times$ & $\times$ & $\times$ & \\
\tt\bsl listoffigures    & $\times$ & $\times$ & $\times$ & \\
\tt\bsl listoftables     & $\times$ & $\times$ & $\times$ & \\
\tt\bsl thebibliography  & $\times$ & $\times$ & $\times$ & \\
\tt\bsl theindex         & $\times$ & $\times$ & $\times$ & \\
\tt\bsl abstract         & $\times$ & $\times$ & $\times$ & \\
\tt\bsl part             & $\times$ & $\times$ & $\times$ & \\
\tt\bsl chapter          &          & $\times$ & $\times$ & \\
\tt\bsl appendix         &          & $\times$ & $\times$ & \\
\tt\bsl cc               & & & & $\times$ \\
\tt\bsl encl             & & & & $\times$ \\
\tt\bsl ps@headings      & & & & $\times$ \\
\end{tabular}
\caption{macros that need to be redefined for the four standard document
  styles.}
\label{macros}
\end{center}
\MakeShortVerb{\|}
\end{table*}

As an example of the way the macros have to be redefined, the
redefinition of |\tableofcontents| is shown in figure~\ref{content}.
\begin{figure*}[htb]
\begin{verbatim}
\@ifundefined{contentsname}
  {\def\tableofcontents
    {\section*{\contentsname
      \@mkboth{\uppercase\expandafter{\contentsname}}
              {\uppercase\expandafter{\contentsname}}
              }
      \@starttoc{toc}
    }
  }
  {}
\end{verbatim}
\caption{An example of redefining a command}\label{content}
\end{figure*}

The standard styles can be distinguished by checking the existence of
the macros |\chapter| (not in \file{article} and \texttt{letter}) and
|\opening| (only in \texttt{letter}). The result of these checks is
stored in the macro |\doc@style|. When |\doc@style| already exists
(which is the case when for instance \file{artikel1.sty} is
used~\cite{BEP}) it is not superseded (see figure~\ref{maindoc}).

\begin{figure*}[htb]
\begin{verbatim}
\@ifundefined{doc@style}
    {\def\doc@style{0}
        \@ifundefined{opening}
            {\@ifundefined{chapter}
                {\def\doc@style{1}}
                {\def\doc@style{2}}
            }{\def\doc@style{3}}
    }{\relax}
\end{verbatim}
\caption{Determining the main document style}
\label{maindoc}
\end{figure*}

\section{Implementing a language specific document-style option file}

To illustrate the way a language specific file can be implemented the
file \file{dutch.sty} is discussed here. Note that not all of the code
containde in the file \file{dutch.sty} is shown here, only those parts
that are of interest for the scope of this article are included. If
the reader would like to see the complete code, he can print all files
in the \babel{} system, using the file \file{doc.sty}, described by
Frank Mittelbach in~\cite{docsty}.

\subsection{Compatibilty with plain \TeX}
The file \file{german.tex}~\cite{HP} was written in such a way that it
can be used by both plain \TeX\ users and \LaTeX\ users. This seemed a
good idea, so all files in the \babel{} system can be processed by
both plain \TeX\ and \LaTeX. But some of the ``useful hacks'' from
\LaTeX\ are used, so for a plain \TeX\ user they have to be defined.
For this purpose the format is checked at the start of a language
specific file. If the format is \texttt{plain} an extra file, called
\file{latexhax.sty} is read.
\begin{figure*}[htb]
\begin{verbatim}
{\def\format{plain}
\ifx\fmtname\format
    \expandafter\ifx\csname @ifundefined\endcsname\relax
      \gdef\next{latexhax.sty}
      \aftergroup\input\aftergroup\next
    \fi
\fi}
\end{verbatim}
\caption{Conditonal loading of \file{latexhax.sty}}
\end{figure*}

This file should be read only once, so another check is done on the
existence of one of the commands defined there.

A new group is started to keep the definition of the macro
|\format|, which is used in the following if statement, local.
When the current format turns out to be plain \TeX\ the file
\file{latexhax.sty} has to be read. But the definitions in that file
should remain valid after the group is closed. This could be
accomplished by making all definitions \texttt{global}, but another
solution is to tell \TeX\ to process the file \file{latexhax.sty}
\emph{after} the current group has been closed. The command
|\aftergroup| puts the next token on a list to be processed after
the group.

\subsection{Switching to the Dutch language}
In section~\ref{lang-switch} the names of macros needed to switch to a
language have been described. In figure~\ref{switchdutch} these macros
and their definition are shown for the Dutch language.
\begin{figure*}[htb]
\begin{verbatim}
\def\captionsdutch{\gdef\refname{Referenties}%
                   \gdef\abstractname{Samenvatting}%
                   \gdef\bibname{Bibliografie}%
                   ...
                   \gdef\pagename{Pagina}}

\def\datedutch{%
    \gdef\today{\number\day~\ifcase\month\or
                januari\or februari\or maart\or april\or
                mei\or juni\or juli\or augustus\or
                september\or oktober\or november\or december\fi
                \space \number\year}}
\begingroup \catcode`\"\active

\gdef\extrasdutch{%
    \global\let\dospecials\dutch@dospecials
    \global\let\@sanitize\dutch@sanitize
    \catcode`\"\active
    \gdef"{\protect\dutch@active@dq}
    \gdef\"{\protect\@umlaut}
}\endgroup

\def\noextrasdutch{%
    \catcode`\"12
    \global\let\dospecials\original@dospecials
    \global\let\@sanitize\original@sanitize
    \global\let\"\dieresis
}
\end{verbatim}
\caption{The macros needed to switch to the Dutch language}
\label{switchdutch}
\end{figure*}

The definitions of |\captionsdutch| and |\datedutch| are pretty
straightforward and need not be discussed. The macro |\extrasdutch|
will be discussed in some more detail.

First, because for Dutch (as well as for German) the \texttt{"}
character is made active, the \LaTeX\ macros |\dospecials| and
|\@sanitize| have to be redefined to include this character as well.
The new definitions are implemented as two special commands, so we
globally |\let| the originals to their new versions. Then the
\texttt{"} character is made active and is defined. Then, to prevent
an error when |\"| appears in a moving argument, the macro |\"| is
redefined and made robust. All this is done inside a group to keep the
category code change for the \texttt{"} character local.

The macro |\extrasdutch| has a counterpart, |\noextrasdutch|, that
cancels the extra definitions made by |\extrasdutch|. It changes the
|\catcode| of the \texttt{"} character back to `other' and globally
|\let|s the macros |\dospecials| and |\@sanitize| to their original
definitions. The original definition of |\"| is restored as well.

In figure~\ref{specials} the code needed to redefine |\dospecials| and
|\@makeother| is shown.
\begin{figure*}[htb]
\begin{verbatim}
\begingroup
  \def\do{\noexpand\do\noexpand}%
  \xdef\dutch@dospecials{\dospecials\do\"}%
  \expandafter\ifx\csname @sanitize\endcsname\relax
% do nothing if \@sanitize is undefined...
  \else
     \def\@makeother{\noexpand\@makeother\noexpand}%
     \xdef\dutch@sanitize{\@sanitize\@makeother\"}%
  \fi
\endgroup

\global\let\original@dospecials\dospecials
\global\let\original@sanitize\@sanitize
\end{verbatim}
\caption{Code needed for the redefinition of {\tt\bsl dospecials} and
  {\tt\bsl @makeother}.}
\label{specials}
\end{figure*}

\subsection{An extra active character}

All the code disccussed sofar is necessary because we need an extra
active character. This character is then used as indicated in
table~\ref{dutch-quote}.  One of the reasons for this is that in the
Dutch language a word with an umlaut can be hyphenated just before the
letter with the umlaut, but the umlaut has to disappear if the word is
broken between the previous letter and the accented letter.

\begin{table*}[htb]
\centering
\begin{tabular}{lp{8cm}}
\verb="a= & \verb=\"a= which hyphenates as \verb=-a=;
            also implemented for the other letters.        \\
\verb="|= & disable ligature at this position.             \\
\verb="-= & an explicit hyphen sign, allowing hyphenation
            in the rest of the word.                       \\
\verb="`= & lowered double left quotes (see example below).\\
\verb="'= & normal double right quotes.                    \\
\verb=\-= & like the old \verb=\-=, but allowing hyphenation
            in the rest of the word.
\end{tabular}
\caption{The extra definitions made by \file{dutch.sty}}
\label{dutch-quote}
\end{table*}

In~\cite{treebus} the quoting conventions for the Dutch language are
discussed. The preferred convention is the single-quote Anglo-American
convention, i.e. `This is a quote'.  An alternative is the slightly
old-fashioned Dutch method with initial double quotes lowered to the
baseline, \dlqq This is a quote'', which should be typed as
\texttt{"`This is a quote"'}.

\subsubsection{Supporting macro definitions}

The definition of the active \texttt{"} character needs a couple of
support macros. The macro |\allowhyphens| is used make hyphenation of
word possible where it otherwise would be inhibited by \TeX. Basically
its definition is nothing more than |\nobreak \hskip 0pt plus 0pt|.
\begin{verbatim}
\gdef\allowhyphens{\penalty\@M \hskip\z@skip}
\end{verbatim}

Then a macro is defined to lower the Dutch left double quote to the
same level as the comma. It prepares a low double opening quote in box
register~0.  This macro was copied form \file{german.tex}.
\begin{verbatim}
\gdef\set@low@box#1{%
    \setbox\tw@\hbox{,} \setbox\z@\hbox{#1}
    \dimen\z@\ht\z@ \advance\dimen\z@ -\ht\tw@
    \setbox\z@\hbox{\lower\dimen\z@ \box\z@}
    \ht\z@\ht\tw@ \dp\z@\dp\tw@}
\end{verbatim}
The macro |\set@low@box| is used to define low opening quotes.  Since
it may be used in arguments to other macros it needs to be protected.
\begin{verbatim}
\gdef\dlqq{\protect\@dlqq}
\gdef\@dlqq{{%
  \ifhmode
    \edef\@SF{\spacefactor\the\spacefactor}
  \else
    \let\@SF\empty
  \fi
  \leavevmode\set@low@box{''}
  \box\z@\kern-.04em\allowhyphens\@SF\relax}}
\end{verbatim}
For reasons of symmetry we also define |"'|. This command is defined
similar to |\dlqq|, except that the quotes aren't lowered to the
baseline.
\begin{verbatim}
\gdef\@drqq{{%
  \ifhmode
    \edef\@SF{\spacefactor\the\spacefactor}
  \else
    \let\@SF\empty
  \fi
  ''\@SF\relax}}
\end{verbatim}
The original double quote character is saved in the macro |\dq| to
keep it available.
\begin{verbatim}
\begingroup \catcode`\"12
  \gdef\dq{"}
\endgroup
\end{verbatim}
The original definition of |\"| is stored as |\dieresis|. The resason
for this is that if a font with a different encoding scheme is used
the definition of |\"| might not be the plain \TeX\ one.
\begin{verbatim}
\global\let\dieresis\"
\end{verbatim}

In the Dutch language vowels with a dieresis or umlaut accent are
treated specially. If a hyphenation occurs before a vowel-plus-umlaut,
the umlaut should disappear. To be able to do this, the hyphenation
break behaviour for the five vowels, both lowercase and uppercase,
could be defined first in terms of |\discretionary|. But this results
in a large |\if|-construct in the definition of the active |"|.

As both Knuth and Lamport have pointed out, a user should not use |"|
when he really means something like |''|. For this reason no
distinction is made between vowels and consonants. Therefore one
macro, |\@umlaut|, specifies the hyphenation break behaviour for all
letters.
\begin{verbatim}
\def\@umlaut#1{%
    \allowhyphens%
    \discretionary{-}{#1}{\dieresis #1}%
    \allowhyphens}
\end{verbatim}

The last support macro to be defined is |\dutch@active@dq|.
\begin{verbatim}
\gdef\dutch@active@dq#1{%
     \if\string#1`\dlqq{}%
\else\if\string#1'\drqq{}%
\else\if\string#1-\allowhyphens-\allowhyphens%
\else\if\string#1|\discretionary{-}{}{\kern.03em}%
\else\if\string#1i\allowhyphens\discretionary{-}{i}{\dieresis\i}%
                  \allowhyphens%
\else\if\string#1j\allowhyphens\discretionary{-}{j}{\dieresis\j}%
                  \allowhyphens%
\else \@umlaut{#1}\fi\fi\fi\fi\fi\fi}
\end{verbatim}
The macro reads the next token and performs some appropriate action.
If no special action is defined, it will produce an umlaut accent on
top of argument~1.

The last definition needed is a replacement for |\-|. The new version
of |\-| should indicate an extra hyphenation position, while allowing
other hyphenation positions to be generated automatically. The
standard behaviour of \TeX\ in this respect is very unfortunate for
languages such as Dutch and German, where long compound words are
quite normal and all one needs is a means to indicate an extra
hyphenation position on top of the ones that \TeX\ can generate from
the hyphenation patterns.
\begin{verbatim}
\def\-{\allowhyphens\discretionary{-}{}{}\allowhyphens}
\end{verbatim}

\subsection{Activating the definitions}

The last action that should be performed by a language specific file,
is activating it's definitions. Before doing that the macro
|\originalTeX| should be definined.
\begin{verbatim}
\@ifundefined{originalTeX}{\let\originalTeX\relax}{}
\end{verbatim}

Also, the macro |\l@|\meta{language} should be defined. If it hasn't
already been defined, this means that no hyphenation patterns were
loaded for this language.
% This means that \emph{no} hyphenation will take place
\begin{verbatim}
\@ifundefined{l@dutch}{\addlanguage{dutch}}{}
\selectlanguage{dutch}
\end{verbatim}

\section{Conclusion}

In this article a system of document-style option files has been
presented that supports the multilingual use of \LaTeX. Some of the
code involved has been discussed. The actual files will me made
available through the international networks. They will be stored in
the fileserver in the Netherlands (address:
\texttt{LISTSERV@HEARN.BITNET}), the file \file{babel readme} will
explain what you need to get to be able to use the system. The system
was developed using the \Lopt{doc} option, so the files available are
fully documented.

\begin{thebibliography}{9}
 \bibitem{DEK} Donald E. Knuth,
   \emph{The \TeX book}, Addison-Wesley, 1986.
 \bibitem{LLbook} Leslie Lamport,
   \emph{\LaTeX, A document preparation System}, Addison-Wesley, 1986.
  \bibitem{treebus} K.F. Treebus.
  \emph{Tekstwijzer, een gids voor het grafisch verwerken van tekst.}
  SDU Uitgeverij ('s-Gravenhage, 1988). A Dutch book on layout
  design and typography.
 \bibitem{HP} Hubert Partl,
   \emph{German \TeX}, \emph{TUGboat} 9 (1988) \#1, p.~70--72.
 \bibitem{LLth} Leslie Lamport,
   in: \TeXhax\ Digest, Volume 89, \#13, 17 februari 1989.
 \bibitem{docsty}Frank Mittelbach,
  \emph{The \Lopt{doc}-option},
  \emph{TUGboat} 10 (1989) \#2, p.~245--273.
 \bibitem{BEP}Johannes Braams, Victor Eijkhout and Nico Poppelier,
  \emph{The development of national \LaTeX\ styles},
  \emph{TUGboat} 10 (1989) \#3, p.~401--406.
 \bibitem{ilatex}Joachim Schrod,
  \emph{International \LaTeX\ is ready to use},
  \emph{TUGboat} 11 (1990) \#1, p.~87--90.
\end{thebibliography}

\end{document}
