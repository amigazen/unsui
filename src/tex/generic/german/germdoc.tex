%% File ``germdoc.tex'', V. 2.5c <96/05/01>
%%
%% Copyright (C) 1992, 1995, 1996 by B.Raichle and DANTE e.V.
%%                                All rights reserved.
%%
%%
%% "Anderungen:
%% -- Erste Fassung: Tagungsbericht (ca. 3 Seiten) --
%% <87/11/27> Hubert Partl
%%     Bericht "uber die Einigung auf einen "`Minimal Subset von
%%     einheitlichen deutschen \TeX-Befehlen"' auf dem 6. Treffen der
%%     deutschen \TeX-Interessenten in M"unster (Oktober 1987).
%% <90/04/??> Hubert Partl
%%     Korrektur \selectlanguage statt \setlanguage, f"ur v2.3
%% <92/04/12> br
%%     erste Erweiterungen und Erg"anzungen des Berichts mit einer
%%     kurzen Installationsanleitung und den "Anderungen im
%%     German-Style, f"ur v2.4a
%% --- Neufassung: Benutzerhandbuch (ca. 20-22 Seiten) --
%% <95/01/01> br
%%     komplett "uberarbeitete, restrukturierte und stark erweiterte
%%     Version mit ausf"uhrlicherer Benutzungs- und Installations-
%%     anleitung, f"ur v2.5a 
%% <95/01/20> br
%%     vier(!) kleinere Korrekturen, f"ur v2.5b
%% <96/05/01> br
%%     "uberarbeite und korrigierte Version mit Neuerungen aus v2.5c
%%
%%
\ifx\documentclass\undefined  % LaTeX2e?
  \documentstyle[11pt,a4,german]{article}
\else
  \NeedsTeXFormat{LaTeX2e}
  \documentclass[11pt,a4paper]{article}
  \usepackage{german}
\fi

\title{Kurzbeschreibung -- {\tt german.sty} (Version~2.5)}
\author{Bernd Raichle \\
  Koordinator "`german.sty"' \\
  DANTE, Deutschsprachige Anwendervereinigung
         \TeX\ e.V\negthinspace.}
\date{1.~Mai 1996\\ (f"ur {\tt german.sty} Version~2.5c)}

\ifx\LaTeXe\undefined
  \newcommand{\pLaTeXe}{%
    \mbox{\LaTeX\kern.15em$2_{\textstyle\varepsilon}$}}
  \newcommand{\LaTeXe}{\protect\pLaTeXe}\fi
\ifx\emergencystretch\undefined \else
  \setlength{\emergencystretch}{1em}\fi
\newenvironment{beispiel}{%
  \begin{quote}\small \begin{tabbing}%
  {\tt "`\glq Ja, bitte!\grq!"'}\qquad \=ergibt:\qquad\=\kill
}{\end{tabbing}\end{quote}}
\hyphenation{Datei-name Datei-namen}

\begin{document}
\maketitle

\begin{abstract}
  Beim 6.~Treffen der deutschen \TeX-""Interessenten in M"unster
  (Oktober 1987) wurde Einigung "uber ein "`Minimal Subset von
  einheitlichen deutschen \TeX-Befeh\-len"' erzielt, das seitdem an
  allen Installationen von \TeX\ und \LaTeX\ durch die Style-Option
  "`german"' zur Ver"-f"u"-gung stehen und f"ur Texte in deutscher
  Sprache verwendet werden soll.  Damit wird erreicht, da"s alle
  \TeX- und \LaTeX-Dokumente, die diese Befehle enthalten, problemlos
  von einem Rechner zum anderen "uber"-tragen werden k"onnen.
\end{abstract}

\tableofcontents

\section{Allgemeines}

Die Style-Option "`german"' dient mehreren Zwecken:
\begin{itemize}\tolerance=2000
\item Der deutsche Schriftsatz weist einige Besonderheiten auf, die
  in \TeX{} durch neue Makros und verschiedene "Anderungen
  unterst"utzt und dem Benutzer angeboten werden k"onnen.  Beispiele
  in "`german"' sind die Makros f"ur die unterschiedlichen
  Anf"uhrungszeichen und die "Anderung des Umlautmakros f"ur die
  Schriften der Com\-pu\-ter-Modern-Familie.
\item Zur Vereinfachung und Standardisierung der Eingabe von Umlauten
  und des Buchstabens~"s wird die Kurznotation~"`\verb:"x:"' f"ur
  Rechner einge"-f"uhrt, die die Eingabe dieser Zeichen nicht direkt
  erlauben.  Heutzutage unter"-st"utzen die meisten Rechnersysteme die
  direkte Eingabe und Anzeige von Umlauten, jedoch ist die
  Kurznotation weiterhin als Quasi-Standard zum Austausch von \TeX-
  und \LaTeX-Dokumenten sinnvoll und f"ur deutschsprachige Texte weit
  verbreitet.
\item "`german"' unterst"utzt au"serdem das Einbinden
  deutschsprachiger Texte in ein Dokument einer anderen Sprache --
  momentan mit der Beschr"ankung auf Englisch und Franz"osisch.
\end{itemize}


\section{Verwendung}

\subsection{Laden der Style-Option}

Vor der Verwendung der durch die Style-Option "`german"'
zu"-s"atzlich zur Ver"-f"u"-gung gestellten Befehle mu"s die
Style-Option geladen werden.

\paragraph{\LaTeX:} Mit \LaTeX\ (genauer: \LaTeXe) wird die
Style-Option als Paket mit
\begin{quote}
  \verb:\usepackage{german}:
\end{quote}
nach der Deklaration der Dokumentenklasse mit \verb:\documentclass:
geladen.  Momentan hat das Paket "`german"' keine eigenen
Paket"-optionen und verwendet keine Dokumenten"-klassen"-optionen.

\paragraph{\LaTeX~2.09:} Mit der alten \LaTeX-Version oder im
\LaTeX~2.09-Kompatibi\-li\-t"ats"-modus von \LaTeXe{} wird die
Style-Option innerhalb des optionalen Arguments des
\verb:\documentstyle:-Befehls, beispielsweise mit
\begin{quote}
  \verb:\documentstyle[11pt,german]{article}:
\end{quote}
angegeben.

\paragraph{plain-\protect\TeX:} Unter {\tt plain}-\TeX\ wird die
Style-Option als gew"ohnliche Makro"-datei mit dem \TeX-Befehl
\begin{quote}
  \verb:\input german.sty:
\end{quote}
dazugeladen.


\subsection{Befehle}

Der beim 6.~Treffen der deutschen \TeX-Interessenten in M"unster
festgelegte Befehlssatz wurde nach"-tr"aglich um einige Befehle
erweitert.  Diese Erweiterungen werden zur Kenntlichmachung in der
folgenden Liste mit einem~$\dagger$ versehen.


\subsubsection{Umlaute, Tremata und der Buchstabe "s}

\begin{itemize}
\item \verb:\"a: ergibt mit Schriften aus der
  Com\-pu\-ter-Modern-Familie, mit denen ein Umlaut nur aus zwei
  Zeichen zusammengesetzt werden kann, ein "`a"' mit einem Trema.
  Die Style-Option "`german"' definiert dieses Makro deshalb so um,
  da"s die Umlautpunkte etwas nach unten verschoben werden (Original:
  {\originalTeX \"a}, mit "`german"': \"a),
\item \verb:"a: als Kurzform f"ur \verb:\"a: (Umlaute, wie~"a) --
  ebenso f"ur die Vokale o und~u,
\item \verb:"e: und \verb:"i: f"ur ein e und i mit Trema,
\item \verb:"s: als Kurzform f"ur \verb:\ss: (scharfes~s: "s),
  \verb:"S:$^\dagger$ ergibt "`SS"' (\verb:"z:$^\dagger$ und
  \verb:"Z:$^\dagger$ kann f"ur~\ss\ bzw.\ SZ verwendet werden, falls
  man Mi"sverst"andnisse bei der Verwendung in Gro"sbuchstaben
  vermeiden m"ochte).
\end{itemize}

\paragraph{Verwendungsbeispiele}
\begin{beispiel}
\verb:sch"on:  \> ergibt: \> sch"on \\
\verb:sch\"on: \> ergibt: \>
   sch\"on (statt: {\originalTeX sch\"on})\\[3pt]
\verb:Citro"en: \> ergibt: \> Citro"en \\[3pt]
\verb:Stra"se Ma"ze: \> ergibt: \> Stra"se Ma"ze \\
\verb:STRA"SE MA"ZE: \> ergibt: \> STRA"SE MA"ZE \\
\end{beispiel}
%
Die Befehle f"ur Umlaute und scharfes~s sind so definiert, da"s auch
in Silben {\em vor\/} und {\em nach\/} dem Befehl die automatische
Silbentrennung funktioniert. Dabei kann \TeX\ jedoch nicht mehr alle
oder eventuell falsche Trennstellen finden (Beispiel: {\tt
  "ubert-ra-gen} statt {\tt "uber-tra-gen}).  Diese fehlerhaften
Trennungen treten bei der Verwendung von {\tt T1}-kodierten Schriften,
wie beispielsweise den {\tt DC}-Schriften, die man unter \LaTeXe{}
ohne gro"se Anpassungen verwenden kann, nicht auf.  \TeX{}s
Trennalgorithmus kann Trennstellen prinzip"-bedingt nicht 100\%ig
fehlerfrei finden.  Deshalb erhalten Sie, unabh"angig von der
verwendeten Schrift, immer einige, wenn auch nur sehr wenige
fehlerhaften Wort"-trennungen.


\subsubsection{Zusammentreffen von drei gleichen Konsonanten und
  \protect\mbox{"`ck"'} bei Silbentrennung}

\begin{itemize}
\item \verb:"ck: f"ur \mbox{"`ck"'}, das als \mbox{"`k-k"'} getrennt
  wird,
\item \verb:"ff: f"ur \mbox{"`ff"'}, das als \mbox{"`ff-f"'} getrennt
  wird -- auch f"ur die anderen relevanten Konsonanten l, m, n, p, r
  und~t.
\end{itemize}
%
Regel~$179$ der "`Richtlinien zur
Rechtschreibung~\ldots"'~\cite{Duden} bestimmt: {\it Treffen bei
  Wortbildungen drei gleiche Konsonanten zusammen, dann setzt man nur
  zwei, wenn ein Vokal folgt.  Bei Silbentrennung tritt der dritte
  Konsonant wieder ein.} {\it \mbox{ck} wird\/} nach Regel~$204$ {\it
  bei der Silbentrennung in \mbox{k-k} aufgel"ost.}

\paragraph{Verwendungsbeispiele:}
\begin{beispiel}
\verb:Dru"cker:  \> ergibt: \> Drucker bzw.\ Druk-ker \\
\verb:Ro"lladen: \> ergibt: \> Rolladen bzw.\ Roll-laden \\
\end{beispiel}
%
Da man diese besonderen F"alle der Silbentrennung nur durch den
\TeX-Befehl \verb:\discretionary: als Trennausnahmen realisieren kann,
wird dadurch leider die Trennung in den restlichen Wortteilen und die
Ligaturbildung um diese Konsonanten beeinflu"st.


\subsubsection{Anf"uhrungszeichen}

\begin{itemize}
\item \verb:"`: oder \verb:\glqq: f"ur untere und \verb:"': oder
  \verb:\grqq: f"ur obere "`deutsche An\-f"uh\-rungs"-zeichen"'
  (\glqq G"anse"-f"u"schen\grqq),
\item \verb:\glq: f"ur untere und \verb:\grq: f"ur obere \glq
  halbierte An\-f"uh\-rungs"-zeichen\grq,
\item \verb:"<: oder \verb:\flqq: f"ur linke und \verb:">: oder
  \verb:\frqq: f"ur rechte "<An\-f"uh\-rungs"-zeichen"> in der
  franz"osischen Form (\flqq guillemets\frqq),
\item \verb:\flq: f"ur linke und \verb:\frq: f"ur rechte \flq
  halbierte An\-f"uh\-rungs"-zeichen\frq\ in der franz"osischen Form,
\item \verb:\dq: zum Ausdrucken des Doublequote-Zeichens~(\verb:":).
\end{itemize}
%
In~\cite{Duden} findet man: {\it Im deutschen Schriftsatz werden
  vornehmlich die An"-f"uhrungszeichen "`\ldots"' und ">\ldots"<
  angewendet.  Die franz"osische Form "<\ldots"> ist im Deutschen
  weniger gebr"auchlich; in der Schweiz hat sie sich f"ur den
  Antiquasatz eingeb"urgert.} Die Regeln~$10$ff erg"anzen diese
Aussagen durch: {\it Eine Anf"uhrung in einer Anf"uhrung wird durch
  halbe Anf"uhrungszeichen deutlich gemacht.}

\paragraph{Verwendungsbeispiele:}
\begin{beispiel}
\verb:"`Ja, bitte!"': \> ergibt: \> "`Ja, bitte!"' \\[3pt]
\verb:"`Sag' doch nicht immer \glq Ja, bitte!\grq!"': \\
  \> ergibt: \>
     "`Sag' doch nicht immer \glq Ja, bitte!\grq!"' \\[3pt]
\verb:">Ja, bitte!"<: \> ergibt: \> ">Ja, bitte!"< \\[3pt]
\verb:">Sag' doch nicht immer \frq Ja, bitte!\flq!"<: \\
  \> ergibt: \>
     ">Sag' doch nicht immer \frq Ja, bitte!\flq!"< \\[3pt]
\verb:"<Merci bien!">: \> ergibt: \> "<Merci bien!"> \\
\end{beispiel}
%
F"ur die Realisierung der Anf"uhrungszeichen existiert keine L"osung,
die uneingeschr"ankt f"ur alle Schriften verwendbar w"are.  Bei der
momentanen Realisierung ist zu beachten, da"s bei Verwendung von
nicht-{\tt T1}-kodierten Schriften (z.\,B.\ die
Com\-pu\-ter-Modern-Familie) kein {\it Kerning\/} zwischen den
An\-f"uh\-rungs"-zeichen und den anderen Zeichen einge"-f"ugt wird.
Bei einigen Buch\-staben-Anf"uhrungs\-zeichen-Kombi"-nationen k"onnen
daher zu gro"-"se bzw.\ zu kleine Ab"-st"an"-de auftreten (Beispiel:
\mbox{"`\kern0ptV} statt~\mbox{"`\negthinspace V}).

Relativ h"aufig ist in Dokumenten folgender Fehler zu beobachten:
Statt mit Hilfe von \verb:"`: und \verb:"': die Anf"uhrungszeichen
"`\ldots"' bzw.\ mit \verb:``: und \verb:'': die im englischsprachigen
Raum gebr"auchlichen Anf"uhrungszeichen ``\ldots'' zu erzeugen, wird
einfach der direkt auf der Tastatur zu findende Double\-quote~\verb:":
verwendet, der das falsche Ergebnis ''\ldots'' erzeugt.  Dieser Fehler
ist leider in vielen mit \LaTeX\ erstellten deutschsprachigen Texte zu
beobachten, obwohl das Fehlen der unteren, "offnenden
Anf"uhrungszeichen auf"|fallen m"u"ste!  Bei Verwendung der
Style-Option "`german"' mit der dann ge"anderten Bedeutung des
Double\-quote kann dies au"serdem zu verschiedenen Fehlermeldungen
f"uhren.

Ben"otigt man das Double\-quote-Zeichen, so mu"s man mit der
Style-Option "`german"' die Befehle \verb:\dq: oder~\verb:\verb+"+:
verwenden. Aus Kompatibilit"ats"-gr"unden mit alten Versionen der
Style-Option wird auch noch~\verb:"{}: unter"-st"utzt.


\subsubsection{Trennhilfen f"ur die automatische Silbentrennung}

\begin{itemize}
\item \verb:\-: f"ur eine Silbentrennstelle, wobei vor und nach
  dieser Trennstelle die Silbentrennung unter"-dr"uckt wird (dies ist
  \TeX{}s Original"-befehl zur Kennzeichnung von Trennstellen),
\item \verb:"-: f"ur eine Silbentrennstelle "ahnlich wie bei
  \verb:\-:, bei der aber die automatische Silbentrennung vor und
  nach dieser Trennstelle erhalten bleibt; im Gegensatz zu \verb:\-:
  kann man mit \verb:"-: eine Trennstelle ein"-f"ugen und alle
  weiteren Trennstellen werden von \TeX\ selbst bestimmt,
\item \verb:"": f"ur eine Silbentrennstelle, bei der aber im Fall der
  Trennung {\it kein\/} Bindestrich hinzugef"ugt wird,
\item \verb:"|: zur Verhinderung von Ligaturen, "ahnlich wie bei
  \verb:"-:, bei der aber noch zu"-s"atzlich ein kleiner Zwischenraum
  zur besseren Trennung der Einzelzeichen einer Ligatur eingef"ugt
  wird,
%\item \verb:\allowhyphens: dient zum Einf"ugen einer Wortfuge, wobei
%  an dieser Stellen {\it keine\/} Silbentrennstelle eingef"ugt wird,
%  aber die automatische Silbentrennung vor und nach dieser Stelle
%  weiterhin erhalten bleibt,
\item \verb:-: f"ur einen Bindestrich (Divis) bei Zusammensetzungen
  von W"ortern und Wort"-abk"urzungen; \TeX\ f"ugt nach dem
  Bindestrich {\it immer\/} implizit eine Trennstelle ein und
  verhindert die Trennungen der Wortteile vor und nach dem
  Bindestrich,
\item \verb:"~:$^\dagger$ f"ur einen Bindestrich ohne Trennstelle,
\item \verb:"=:$^\dagger$ f"ur einen Bindestrich mit Trennstelle,
  wobei die automatische Silbentrennung vor und nach dieser Stelle
  weiterhin erhalten bleibt.
\end{itemize}

\paragraph{Verwendungsbeispiele:}\hspace{0pt plus 1em} Die m"oglichen
Silbentrennstellen sind in den Beispielen durch das Zeichen~$|$
gekennzeichnet.
\begin{beispiel}
\verb:Auf"|lage: \> ergibt: \> Auf"|lage (statt Auflage) \\
                 \> \> mit den Trennstellen: Auf$|$la$|$ge \\[3pt]

\verb:"ubertragen:   \> ergibt:
              \> "ubert$|$ra$|$gen (falsche Trennstelle!) \\
\verb:"uber\-tragen: \> ergibt: \> "uber$|$tragen \\
\verb:"uber"-tragen: \> ergibt: \> "uber$|$tra$|$gen \\[3pt]

\verb:bergauf und -ab:  \> ergibt:
               \> berg$|$auf und -$|$ab (falsche Trennstelle)\\
\verb:bergauf und "~ab: \> ergibt: \> berg$|$auf und "~ab \\
\verb:I-Punkt:          \> ergibt:
               \> I-$|$Punkt (schlechte Trennstelle)\\
\verb:I"~Punkt:         \> ergibt: \> I"~Punkt \\
\verb:Arbeiter"=Unfall: \> ergibt:
   \> Ar$|$bei$|$ter-$|$Un$|$fall$|$ver$|$si$|$che$|$rungs- \\
\verb:versicherungsgesetz: \>      \> ge$|$setz \\
\end{beispiel}

\paragraph{Tips:}
\begin{itemize}
\item Bei zusammengesetzten W"ortern, die falsche Trennstellen
  aufweisen, sollte man zuerst die Trennhilfe~\verb:"-: in die
  Wortfuge der Zusammensetzung ein"-f"ugen und nochmals testen, ob
  danach richtig getrennt wird, bevor man weitere Trennhilfen
  ein"-f"ugt.
\item Bei der Verwendung von Bindestrichen zur Er"-g"anzung ("`bergauf
  und \mbox{-ab}"', "`\mbox{ein-}, zwei- oder dreimal"'), vor
  Schr"ag"-strichen ("`\mbox{Ein-}\slash Aus\-gang"') und Klammern
  ("`Prim"ar-\mbox{(Haupt-)}""Strom"') sollte man mit den Befehlen
  \verb:\mbox:, \verb:"": und \verb:"~: vor den Satzzeichen die
  Trennung unterbinden und eventuell nach den Satzzeichen wieder
  erlauben.
\item Den Befehl~\verb:"=: sollte man nur f"ur automatisch erstellte
  Texte verwenden, da er in den mit dem Bindestrich verbundenen
  Wortteilen auch nahe am Bindestrich liegende Trennstellen erlaubt,
  die man vermeiden sollte.  Zusammengesetzte W"orter sollten
  grund"-s"atzlich ohne Bindestrich geschrieben werden.  Wenn der
  Bindestrich dennoch zur Vermeidung von Mi"sverst"andnissen oder zur
  Verdeutlichung notwendig ist, und man neben der Trennung am
  Bindestrich weitere Trennstellen erlauben will, sollte man mit
  \verb:\-: und \verb:"-: wenige, wohl"-"uber"-legte Trennhilfen
  ein"-f"ugen.
\end{itemize}


\subsubsection{Erstellen deutschsprachiger Dokumente und Einbinden
  von deutschsprachigen Texten in fremdsprachige Dokumente}

\begin{itemize}
\item \verb:\selectlanguage{:\mbox{\(\langle\)\it
      Sprache\/\(\rangle\)}\verb:}: zum Umschalten zwischen deutschen,
  "oster"-reichischen, englischen, amerikanischen und franz"osischen
  Datumsangaben und "Uber"-schriften.  {\tt\string\select\-language}
  aktivert zu"-s"atz"-lich die der Sprache zugeordneten Trennmuster
  (siehe auch~\ref{sec:languageundef},
  S.~\pageref{sec:languageundef}).  F"ur~\mbox{\(\langle\)\it
    Sprache\/\(\rangle\)} ist einer der folgenden Namen zu verwenden:
  \verb:german:, \verb:austrian:, \verb:english:, \verb:USenglish:
  oder \verb:french:,
\item \verb:\germanTeX: zum Einschalten der deutschen \TeX-Befehle:
  aktiviert alle \verb:"x:-Befehle, "andert das \verb:\"x:-Makro und
  aktiviert durch einen {\tt\string\select\-language}-Aufruf die
  deutschen Trennmuster, "andert die Datumsangabe und die in den
  \LaTeX-Styles verwendeten "Uber"-schriften,
\item \verb:\originalTeX: zum Zur"uckschalten auf Original-\TeX\ bzw.\ 
  \mbox{-\LaTeX}: inaktiviert alle durch \verb:\germanTeX: aktivierten
  Befehle.
\end{itemize}

\paragraph{Verwendungsbeispiele:} Der Befehl \verb:\selectlanguage:
pa"st die Datumsangabe und die in den
\LaTeX-Styles\footnote{Angepa"ste Styles sind seit Dezember~1991
  Bestandteil der offiziellen \LaTeX-Verteilung.} verwendeten
"Uber"-schriften an die Sprache an.

\begin{quote}\small
\day=1 \month=1 \year=1995 % Beispieldatum (J\"anner!)
\begin{tabular}{lll}
\verb:\selectlanguage{:\mbox{\(\langle\)\it
    Sprache\/\(\rangle\)}\verb:}:&
\verb:\today:&\verb:\chaptername:\\[3pt]
\verb:german: & \selectlanguage{german}\today
              & \selectlanguage{german}\chaptername \\
\verb:austrian: & \selectlanguage{austrian}\today
                & \selectlanguage{austrian}\chaptername \\
\verb:english: & \selectlanguage{english}\today
               & \selectlanguage{english}\chaptername \\
\verb:USenglish:& \selectlanguage{USenglish}\today
                & \selectlanguage{USenglish}\chaptername \\
\verb:french: & \selectlanguage{french}\today
              & \selectlanguage{french}\chaptername
\end{tabular}
\end{quote}
%
\verb:\selectlanguage: ist nicht dazu geeignet, innerhalb eines
Dokuments zwischen mehreren Sprachen umzuschalten.  Mit ihm bestimmt
man die "`Haupt\-sprache"' des Dokuments, deshalb sollte dieser
Befehl nur ein einziges Mal in einem Dokument in der Pr"a"-ambel
verwendet werden.  Dadurch k"onnen die Befehle aus der Style-Option
"`german"' auch innerhalb eines fremdsprachigen Dokuments benutzt
werden.

\begin{quote}\small
\begin{verbatim}
\NeedsTeXFormat{latex2e}
\documentclass{article}
\usepackage{german}
%\germanTeX  % nicht notwendig, wird durch das Laden von
             % `german' implizit aufgerufen
\selectlanguage{USenglish}
\begin{document}
An english text with some german words, e.g., "Au"serung.
\end{document}
\end{verbatim}
\end{quote}

\begin{figure}
\setlength{\partopsep}{0pt}
\begin{center}
\rule[.3\baselineskip]{.3\textwidth}{.1pt}
\renewcommand{\arraystretch}{1.1}
%\setlength{\tabcolsep}{.5\tabcolsep}
\begin{tabular}[b]{ll@{}}
\multicolumn{2}{@{\strut}l}{\it
  Befehle mit aktivem Doublequote:}\\[4pt]
\verb:"a:, \verb:"A:, \verb:"o:, \verb:"O:, \verb:"u:, \verb:"U:
  & Umlaute: "a, "A, "o, "O, "u, "U \\
\verb:"s:, \verb:"S:, \verb:"z:, \verb:"Z:
  & Buchstabe~"s, "S, "z und "Z \\ 
\verb:"e:, \verb:"E:, \verb:"i:, \verb:"I:
  & Buchstaben mit Trema: "e, "E, "i, "I \\[3pt]
\verb:"c:, \verb:"C: & ck $\rightarrow$ k-k \\
\verb:"f:, \verb:"l:, \verb:"m:, \verb:"n:,
  \verb:"p:, \verb:"r:, \verb:"t:,\\
\verb:"F:, \verb:"L:, \verb:"M:, \verb:"N:,
  \verb:"P:, \verb:"R:, \verb:"T:
  & ausfallender dritter Konsonant \\[3pt]
\verb:"`:, \verb:"': & "` "' \\
\verb:">:, \verb:"<: & "> "< \\[3pt]
\verb:"-:, \verb:"|:, \verb:"": & Trennhilfen \\
\verb:"~:, \verb:"=: & Bindestriche mit besonderem Verhalten \\[5pt]
\multicolumn{2}{p{.95\textwidth}}{%
  Alle anderen {\tt\string"x}-Befehle, die mit keiner Bedeutung
  belegt sind, erzeugen eine Fehlermeldung, um auf eine fehlerhafte
  Eingabe hinzuweisen.}\\ 
 \\
\multicolumn{2}{@{}l}{\it Befehle mit ge"andertem Verhalten:}\\[4pt]
\verb:\": & Umlautmakro mit tieferem Akzentzeichen, \\
          & ohne Unterdr"uckung der Silbentrennung \\
 \\[-3pt]
\multicolumn{2}{@{}l}{\it Neue Befehle -- Textzeichen:}\\[4pt]
\verb:\glqq:, \verb:\grqq:, \verb:\glq:, \verb:\grq:
 & Anf"uhrungszeichen: \glqq\ \grqq\ \glq\ \grq \\
\verb:\flqq:, \verb:\frqq:, \verb:\flq:, \verb:\frq:
 & dto., franz"osische Form: \flqq\ \frqq\ \flq\ \frq \\[3pt]
\verb:\dq: & Doublequote-Zeichen \\
 \\
\multicolumn{2}{@{}l}{\it Neue Befehle -- Sprachumschaltung:}\\[4pt]
\verb:\selectlanguage: & Trennmuster $+$ "Uberschriften wechseln\\
\verb:\germanTeX:, \verb:\originalTeX:
  & (de-)aktiviere  deutsche \TeX-Befehle \\
 \\
\multicolumn{2}{@{}l}{\it Obsolete Befehle:
 (Nicht mehr verwenden!)}\\[4pt]
\verb:\3:, \verb:\ck:
 & alte Form von \ss\ und ck\,$\rightarrow$\,k-k\\
\verb:\setlanguage:   & ersetzt durch \verb:\selectlanguage:\\
 \\
\multicolumn{2}{@{}l}{\it Low-Level-Befehle:}\\[4pt]
\verb:\allowhyphens: & k"unstliche, nicht trennbare Wortfuge \\
\verb:\mdqon:, \verb:\mdqoff:
  & (de-)aktiviere Doublequote-Befehle
\end{tabular}
\rule{.3\textwidth}{.1pt}
\end{center}
\caption{Befehls"ubersicht}
\end{figure}


\subsection{Beschr"ankungen und bekannte Fehler}\label{sec:fehler}

\paragraph{Allgemein:} Da die Style-Option "`german"' das
Double\-quote-Zeichen aktiviert und dessen urspr"ungliche Bedeutung
"andert, k"onnen Probleme mit anderen Style-Optionen und fremden
Dokumenten auftreten:
\begin{itemize}
\item Ganzzahlige Konstanten k"onnen in \TeX\ auch in hexadezimaler
  Notation mit einem voranstehenden Double\-quote eingegeben werden
  (Beispiel: \verb:"FF:).  Beginnt die Hexadezimalzahl mit einer
  Ziffer, so sollten mit neueren Versionen der Style-Option
  "`german"' keine Probleme auftreten.  Beginnt die Zahl stattdessen
  mit einem Buchstaben \mbox{A--F}, so tritt durch die ge"-"anderte
  Bedeutung des Double\-quote meist der Fehler "`{\tt Missing number,
    treated as zero}"' auf.

  Abhilfe: Hexadezimalzahlen, die mit einem Buchstaben beginnen, die
  Ziffer~$0$ voranstellen -- statt \verb:"FF: sollte man demnach
  \verb:"0FF: verwenden.  Eine andere M"oglichkeit besteht darin, den
  Double\-quote mit Hilfe des \TeX-Primitivs \verb:\string+: (Bsp:
  \verb:\string"FF:) oder mit einem vorangestellten
  {\tt\string\original\-TeX} bzw.\ {\tt\string\mdqoff} zu
  deaktivieren.
\item Leerschritte nach einem aktivierten Double\-quote werden
  ignoriert, d.\,h.\ sowohl \verb:" a: als auch~\verb:"a: ergeben
  dasselbe Ergebnis.  Da dies eine Eigenschaft von \TeX\ selbst ist,
  kann es nicht ohne andere Nachteile verhindert werden.
\item Nach einem aktivierten Double\-quote sollten keine geschweiften
  Klammern folgen.  Sowohl die Kombination mit einer "offnenden
  Klammer~\verb:"{: als auch mit einer schlie"senden
    Klammer~\verb:"}: f"uhrt zu Fehlern.
\end{itemize}
%
Treten die in den letzten beiden Punkten beschriebenen Fehler auf,
ist dies meist auf die falsche Eingabe von An"-f"uh\-rungs"-zeichen
zur"uckzuf"uhren.  An"-f"uh\-rungs"-zeichen sind als
\verb:"`:\ldots\verb:"': (bzw.\ als \verb:``:\ldots\verb:'':)
einzugeben, die Verwendung eines einfachen Double\-quote f"uhrt zu
falschen Ergebnissen!

Bei Verwendung von Schriften mit der Kodierung~{\tt OT1}, z.\,B.\ der
Com\-pu\-ter-Modern-Familie, gibt es zu"-s"atzlich folgende
Einschr"ankungen:
%
\begin{itemize}
\item Es werden nicht alle oder falsche Silbentrennstellen in W"ortern
  mit Umlauten gefunden, insbesondere wird nicht direkt um den Umlaut
  getrennt.  Hier mu"s man eventuell mit dem Befehl~\verb:"-:
  nachhelfen.
\item Da \TeX\ mit Hilfe der Makros in der Style-Option "`german"' die
  Umlaute und die An"-f"uh\-rungs"-zeichen aus mehreren Zeichen
  aufbauen mu"s, findet {\it kein\/} Kerning und {\it keine\/}
  Ligaturbildung mit den umgebenden Zeichen statt.  Dies f"allt
  insbesondere bei den Kombinationen \mbox{"`\kern0pt V},
  \mbox{"`\kern0pt W} und \mbox{f\kern0pt "'} auf und wird bei
  schr"ag"-gestellten Schriften noch ver"-st"arkt.  Eine Abhilfe ist
  auf Makroebene nur sehr schwer realisierbar, so da"s man von Hand
  mit den Befehlen \verb:\negthinspace: und \verb:\/: f"ur korrektere
  Ab"-st"ande sorgen mu"s.
\item Die Anf"uhrungszeichen in der franz"osischen Form "> und "<
  k"onnten, auch vom Aussehen, verbessert werden.  Au"serdem wird
  nach diesen An"-f"uh\-rungs"-zeichen nicht umgebrochen, selbst wenn
  ein Wortzwischenraum folgen sollte.
\item Die Typewriter-Schriften {\tt cmtt} besitzen keine Ligatur f"ur
  die doppelten An"-f"uhrungszeichen ( {\tt `\kern0pt`} statt~`` und
  {\tt '\kern0pt'} statt~''), sondern nur ein Zeichen f"ur den
  Double\-quote~{\tt\dq}.  Daher er"-h"alt man mit \LaTeXe{} f"ur die
  Eingabe \verb:"`..."': das unerwartete Ergebnis~"`{\tt
    \begingroup\setbox2=\hbox{,}\setbox0=\hbox{\char34}%
    \setbox0=\hbox{\dimen0=\ht0\advance\dimen0-\ht2\lower\dimen0\box0}%
    \ht0=\ht2\dp0=\dp2\box0\endgroup...\char92}"'.

  Wer dies vermeiden will, mu"s die \LaTeXe-Deklarationen f"ur die
  beiden Symbole {\tt\string\text\-quote\-dblleft} und
  {\tt\string\text\-quote\-dblright} f"ur die OT1-Kodierung
  entsprechend "andern.
\end{itemize}
%
Diese Einschr"ankungen sollten f"ur Schriften in der Kodierung~{\tt
  T1} nicht mehr existieren, da die verwendeten Zeichen in diesen
Schriften existieren und mit korrektem Kerning verwendet werden.
Diese Einschr"ankungen gelten f"ur die An"-f"uh\-rungs"-zeichen in der
franz"osischen Form durch die momentane Realisierung der Style-Option
auch noch weiterhin.

Zur Verwendung von Post\-Script-Schriften sind unterschiedliche
Versionen zur Einbindung dieser Schriften in \TeX{} und \LaTeX{} im
Gebrauch.  Verwendet man eine zu alte Version mit \LaTeXe{}, so
erh"alt man fehlerhafte Akzent- und Anf"uhrungszeichen.  Bitte
probieren Sie mit folgendem Testdokument aus, ob sie ein zu altes
Paket zur Verwendung von Post\-Script-Schriften installiert haben.
%
\begin{quote}\small
\begin{verbatim}
\NeedsTeXFormat{LaTeX2e}
\documentclass{article}
\usepackage{times}
\begin{document}
Test: \"a \"A \ss.
\end{document}
\end{verbatim}
\end{quote}
%
Erhalten Sie fehlerhafte Umlaute oder kein scharfes~s, so besorgen Sie
sich bitte eine neuere Version des Post\-Script-Schrift\-pakets und
installieren Sie dieses.  (Hinweis: Die Style-Option "`german"' wird
in diesem Testdokument {\em nicht\/} verwendet, da der Test
unabh"angig von "`german"' ist und er auch durch"-gef"uhrt werden
kann, ohne "`german"' installieren zu m"ussen.)

\paragraph{\LaTeXe{} (1.~Dezember~1994 und neuer):} \LaTeXe{} benutzt
als Standard"-font"-auswahlschema die Version~$2$ des~\mbox{NFSS}, das
den neuen Schriftparameter {\tt\string\font\-encoding} zur Verf"ugung
stellt.  Dadurch k"onnen sehr einfach Schriften mit unterschiedlicher
Kodierung verwendet werden.  So kann jetzt innerhalb eines Dokuments
beliebig zwischen {\tt OT1}- und {\tt T1}-kodierten Schriften
gewechselt werden.

Sollen Schriften mit weiteren \mbox{\it font-encoding\/}-Werten
verwendet werden, so m"ussen in einer entsprechenden
Definitions"-datei f"ur das \LaTeXe-Package "`fontenc"' nur mit den
entsprechenden \LaTeXe-Deklarationen die "`Low-Level"'-Makros
\verb:\":, \verb:\ss:, \verb:\textquote...: angepa"st werden.

Leider verhindert die Realisierung der Style-Option "`german"' auch
noch f"ur Schriften mit der Kodierung~{\tt T1} ein korrektes Kerning
und Ligaturbildung f"ur einige Zeichen, wie beispielsweise die
An"-f"uh\-rungs"-zeichen in der franz"osischen Form.

\paragraph{\LaTeXe{} (1.~Juni~1994):} Hier gilt das zuvor gesagte,
jedoch hatte diese \LaTeXe-Version weitere Einschr"ankungen und es
fehlten einige von "`german"' verwendete Eigenschaften.  Daher konnten
beispielsweise bei Verwendung von Schriften mit ge"-"anderten
\mbox{\it font-encoding\/}-Werten Fehler oder fehlerhafte Ergebnisse
auftreten.

Da diese allererste \LaTeXe-Version mittlerweile einige Jahre alt ist,
wird sie von "`german"' in kommenden Versionen nicht mehr
unter"-st"utzt.  Deshalb ist ein \mbox{Update} auf die neueste
\LaTeXe-Version ratsam!

\paragraph{\LaTeX~2.09:} Mit der mittlerweile nicht mehr
offiziellen \LaTeX-Version k"on\-nen weitere Schriften nur mit relativ
gro"sen Aufwand integriert werden.  Man kann auch die mittlerweile von
den Entwicklern nicht mehr gewartete Version~$1$ des
Font"-auswahlschemas~\mbox{NFSS} verwenden, wobei NFSS~1 die
Verwendung von Schriften in einer anderen Kodierung im Unterschied zu
NFSS~2 noch nicht unter"-st"utzt hat.

Die Style-Option "`german"' unterst"utzt daher mit \LaTeX~2.09 nur
Schriften mit der Kodierung~{\tt OT1}, beispielsweise die
Com\-pu\-ter-Modern-Familie.  Sollen Schriften mit einer anderen
Kodierung verwendet werden, so m"us\-sen die Makros \verb:\":,
\verb:\ss:, \verb:\flqq:,\,\ldots\,\verb:\grq: entsprechend
ge"-"andert werden.  Weitere "Anderungen sind dann nicht notwendig.

\paragraph{plain-\protect\TeX:} Hier gilt das zuvor f"ur \LaTeX~2.09
gesagte.  Es ist geplant, \mbox{NFSS} Version~$2$ auch f"ur
plain-\TeX\ verf"ugbar zu machen, so da"s man auch die unter \LaTeXe{}
verf"ugbaren Vorteile nutzen kann.

\subsection{Wohin melde ich Fehler?}

Wenn Sie einen Fehler in der Style-Option "`german"' gefunden haben
oder Vorschl"age zu deren Verbesserung haben, schreiben Sie an die in
der Style-Option-Datei selbst angegebene Adresse oder an
\begin{quote}
 DANTE, Deutschsprachige Anwendervereinigung
        \TeX\ e.V\negthinspace. \\
 Koordinator "`german.sty"' \\
 Postfach 10\,18\,40 \\
 69008 Heidelberg \\
 Federal Republic of Germany\\[5pt]
 E-mail: {\tt german@dante.de}
\end{quote}
Ein Kontakt per {\it electronic mail\/} wird bevorzugt.

Der Fehlermeldung sollten Sie die komplette Eingabedatei beilegen, mit
der dieser Fehler erzeugt werden kann.  In Ihrem eigenen Interesse
sollte diese Eingabedatei m"og"-lichst klein sein und nur die wirklich
notwendigen Style-Optionen, Pakete und Makrodateien laden.
Andernfalls gestaltet sich die Fehlersuche zu aufwendig und/oder es
fehlen verwendete Makrodateien, um den Fehler nachvollziehen zu
k"onnen.  F"ugen Sie auch die Protokolldatei f"ur diese Eingabedatei
mit an, da diese weitere Hinweise auf die verwendeten Makros geben.
Um diese Protokolldatei klein zu halten, brechen Sie bitte bei einem
Fehler mit~{\tt x} und nicht mit~{\tt q} ab.

Bevor Sie einen Fehler melden, vergewissern Sie sich bitte, da"s der
Fehler tats"achlich durch die Style-Option "`german"' und nicht durch
eine fehlerhafte Eingabe Ihrerseits oder einen Folgefehler eines zuvor
aufgetretenen Fehlers verursacht wird.  Sehen Sie bitte auch in der
Protokolldatei nach, ob diese eventuell Hinweise auf die Fehlerursache
gibt.  Sehr h"aufig gibt es Fehler in Zusammenhang mit Konstanten in
hexadezimaler Schreibweise, die zur Fehlermeldung "`{\tt Missing
  number, treated as zero}"' f"uhren.  Dieser "`Fehler"' ist bekannt
und ist in dieser Anleitung in Abschnitt~\ref{sec:fehler}
(S.~\pageref{sec:fehler}) beschrieben.


\section{Installation}

F"ur die Benutzung der Style-Option "`german"' ben"otigen Sie
folgende Dateien:
%
\begin{enumerate}
\item {\tt german.sty}, die Style-Option selbst, die Sie aus den
  Dateien
  \begin{itemize}
  \item {\tt german.ins} und
  \item {\tt german.dtx}
  \end{itemize}
  entpacken k"onnen und
\item {\tt ghyph31.tex}, die deutschen Trennmuster {\tt ghyphen} in
  der z.\,Z.\ aktuellen Version~3.1a vom 13.~Februar 1994.
\end{enumerate}

\subsection{Vorbereitungen f"ur Benutzer}\label{sec:benutzerinit}

Als gew"ohnlicher \TeX- bzw.\ \LaTeX-Benutzer m"ussen Sie nur daf"ur
sorgen, da"s die Datei {\tt german.sty} in der neuesten Version
vorhanden ist und geladen werden kann.

Arbeiten Sie noch mit einer "alteren Version der Style-Option
"`german"', sollten Sie sich eine neuere Version besorgen und
installieren oder, falls Sie sich in der gl"ucklichen Lage befinden,
da"s sich ein spezieller \TeX-Administrator um diese Dinge k"ummert,
diesen auf die neuere Version und diese Anleitung aufmerksam machen.

Hat Ihre plain-\TeX- oder \LaTeX-Version schon die Style-Option
vorgeladen -- d.\,h.\ Sie m"ussen "`german"' nicht explizit als
Style-Option bzw.\ \LaTeXe-Paket deklarieren bzw.\ die Makro-Datei
laden -- sollten Sie dennoch die Style-Option explizit in Ihrem
Dokument deklarieren bzw.\ laden!  Wenn Sie Ihre Dokumente
weitergeben, vermeiden Sie dadurch un"-n"otige Probleme und
R"uck"-fragen.  Am besten "andern Sie Ihre Installation so ab, da"s
die Style-Option "`german"' {\it nicht\/} vorgeladen wird, da Sie
damit sonst zu allen anderen \TeX-Installationen inkompatibel sind.

Erscheint beim Laden der Style-Option die
Meldung\label{sec:languageundef}
%
\begin{quote}\small
\verb:\language number for German undefined, default 255 used.:
\end{quote}
%
so haben Sie die deutschen Trennmuster nicht korrekt installiert und
erhalten eventuell fehlerhafte Trennungen oder es wird nicht getrennt.
Wie Sie die Trennmuster installieren m"ussen, erfahren Sie im
n"achsten Abschnitt.

\subsection{Vorbereitungen f"ur \protect\TeX-Administratoren}

Damit Sie die Style-Option "`german"' den \TeX-Nutzern zur Verf"ugung
stellen, mu"s diese, falls noch nicht getan, ausgepackt und
installiert werden.  Au"serdem m"ussen Sie alle Format"-dateien mit
den deutschen Trennmustern so erstellen, da"s die Style-Option
erkennen kann, wie auf diese Trennmuster umgeschaltet werden kann.
Andernfalls wird nicht mehr oder falsch getrennt!

\subsubsection{Installation der Style-Option}

Die Style-Option "`german"' wird als {\it docstrip\/}-Archiv in zwei
Dateien verteilt:
%
\begin{itemize}
\item {\tt german.dtx} ent"-h"alt den dokumentierten Code f"ur {\tt
    german.sty} und
\item {\tt german.ins} ist das zugeh"orige Installationsskript mit
  Anweisungen f"ur {\it docstrip}, wie die Datei {\tt german.dtx}
  auszupacken ist.
\end{itemize}
%
Neben diesen beiden Dateien ben"otigen Sie noch {\it docstrip}, das
Sie in der Datei {\tt docstrip.tex} oder {\tt docstrip.dtx} (und {\tt
  docstrip.ins}) in der aktuellen \LaTeXe-Release finden k"onnen.
Sie k"onnen {\it docstrip\/} auch mit plain-\TeX\ oder \LaTeX\ 
verwenden, ohne da"s Sie zuvor \LaTeXe{} installiert haben --
entnehmen Sie einfach die Datei {\tt docstrip.tex} einer schon
entpackten Verteilung oder entpacken Sie diese Datei mit den beiden
Dateien {\tt docstrip.dtx} und {\tt docstrip.ins}, indem Sie
%
\begin{quote}\small
  \verb:tex docstrip.ins:
\end{quote}
%
aufrufen.

Ist {\it docstrip\/} vorhanden, entpacken Sie die Style-Option
"`german"' mit
%
\begin{quote}\small
  \verb:tex german.ins:
\end{quote}
%
Die Datei {\tt german.sty}, die dabei erzeugt wird, sollten Sie in ein
Verzeichnis kopieren, in dem \TeX\ nach Makrodateien sucht (Bsp:
{\tt{}.../texmf/""tex/""generic/""misc/}).  Haben Sie noch "altere
Versionen der Datei {\tt german.sty} in weiteren Verzeichnissen, die
\TeX\ durchsucht, sollten Sie diese Dateien l"oschen oder umbenennen.

Haben Sie die neue Version der Style-Option "`german"' installiert,
sollten Sie auf alle F"alle den n"achsten Abschnitt lesen, da es sehr
wahrscheinlich ist, da"s die Trennmuster in Ihren bisher installierten
Formatdateien nicht "`korrekt"' installiert wurden.  Ist dies der
Fall, wird in allen Dokumenten, die die Style-Option verwenden, nicht
mehr getrennt!

\subsubsection{Installation der Trennmuster}

\TeX\ kann ab Version~3.0 in einem Dokument verschiedene Trennmuster
verwenden, so da"s jetzt mehrsprachige Dokumente mit korrekter
Trennung m"oglich sind.  \TeX\ benutzt zur Auswahl der Trennmuster
einer Sprache eine ganze Zahl von 0--255, die der Sprache vor dem
Laden der Trennmuster zugeordnet werden mu"s.

Beim Laden eines Satzes von Trennmustern erfolgt (beim
Ini\TeX-Durch\-lauf) die Zuordnung zu einer Sprache durch Zuweisung
der ge"-w"ahlten Nummer an das spezielle \TeX-Register
\verb:\language:.  Ebenso werden sp"ater beim "Uber"-setzen eines
Dokuments die zu verwendenden Trennmuster durch Zuweisung an
\verb:\language: wieder aktiviert.  Hat \verb:\language: beim
"Uber"-setzen eines Dokuments einen Wert, f"ur den keine Trennmuster
geladen wurden, so findet {\it keine\/} Silbentrennung statt.

Wie werden nun die Nummern von 0--255 an die einzelnen Sprachen
vergeben?  Als Quasi-Standard f"ur die Vergabe der Nummern hat sich
mittlerweile das im {\it
  Babel-System\/}~\cite{Braams91,Goosens94a,Goosens94b} verwendete
Schema herausgebildet.
% Hierbei befindet sich in der Datei
% {\tt language.dat} eine Liste der Sprache und zugeordneten
% Trennmusterdateien, bei der Erstellung der Formatdatei f"ur
% plain-\TeX, \LaTeX\ oder f"ur ein anderes Format gelesen wird.  Dabei
% werden diesen Sprachen Nummern aufsteigend und von~0 beginnend
% vergeben.
Dabei wird an jede neu deklarierte Sprache einfach eine Nummer in
aufsteigender Folge vergeben.  Wichtig dabei ist, da"s die Zuordnung
dieser Nummer an die Sprache in der {\it control sequence\/}
{\tt\string\l@}$\langle$Sprache$\rangle$ abgespeichert wird, so da"s
man sp"ater wieder auf diese zugewiesene Nummer zugreifen kann.  Die
Style-Option "`german"' benutzt auch diese {\it control sequences},
daher sollten Sie beim Installieren der Trennmuster folgenderma"sen
vorgehen:

\paragraph{Schritt 1$^*$:}  Sie verwenden das Babel-System (aktuell
ist z.\,Z.\ die Version~3.5c vom 21.~Juni 1995).  Dann sollten Sie das
Paket nach der im Babel-Paket enthaltenen Installationsanleitung
auspacken und installieren.  Sie k"onnen die Datei {\tt hyphen.cfg}
und das nach der Installation entstandene \LaTeXe-Format problemlos
mit der Style-Option "`german"' verwenden.  Jedoch sind die im
Babel-System enthaltenen Style-Optionen zur Sprachanpassung
inkompatibel zur Style-Option "`german"', so da"s die {\em
  gleichzeitige\/} Verwendung dieser Style-Optionen zusammen mit
"`german"' in einem Dokument zu Problemen f"uhren kann.

Wenn Sie das Babel-System verwenden und die darin enthaltene Datei
{\tt hyphen.cfg} nach Anleitung installiert haben, sind Sie fertig.
Wenn Sie das Babel-System {\it nicht\/} verwenden, dann lesen Sie
bitte weiter.

\paragraph{Schritt 1:}  Sie ben"otigen zum Laden der
Trennmusterdateien f"ur verschiedene Sprachen eine
Konfigurationsdatei, die (leider) je nach verwendetem Format einen
unterschiedlichen Dateinamen besitzt.  Den Namen dieser Datei k"onnen
Sie der Tabelle in Schritt~$2$ entnehmen, die Datei selbst sollte
folgenden Inhalt haben:
%
\begin{quote}\small
\begin{verbatim}
%% Konfigurationsdatei -- Trennmuster: `hyphen.cfg'
\message{== Loading hyphenation patterns:}

\message{us-english}
\chardef\l@USenglish=\language
%% british english als "Dialekt"
\chardef\l@english=\l@USenglish
\input hyphen

\message{german}
\newlanguage\l@german \language=\l@german
\chardef\l@austrian=\l@german
\input ghyph31

%% weitere Sprachen nach folgendem Schema:
% \message{SPRACHE}
% \newlanguage\l@SPRACHE \language=\l@SPRACHE
% \chardef\l@DIALEKT=\l@SPRACHE
% \input SPRACHhyphen

%% Default-Trennmuster: USenglish
\language=\l@USenglish \lefthyphenmin=2 \righthyphenmin=3
\message{done.}
\endinput
\end{verbatim}
\end{quote}
%
In dem gezeigten Beispiel einer Konfigurationsdatei werden die
US-eng\-li\-schen Trennmuster und die deutschen Trennmuster geladen,
wobei zwei "`Aliases"' f"ur die britisch-englische und f"ur die
"oster"-reichische "`Sprache"' definiert werden.  Wollen Sie f"ur
weitere Sprachen Trennmuster laden -- beispielsweise existieren
separate Trennmuster f"ur britisch-englische Silbentrennung -- sollten
Sie in der Lage sein, die dazu notwendigen Zeilen anhand des
auskommentierten Schemas im gezeigten Beispiel zu vervoll"-st"andigen.

Es ist wichtig, da"s Sie die beiden Parameter
{\tt\string\left\-hyphen\-min} und {\tt\string\right\-hyphen\-min},
die die Mindest"-l"ange der nicht trennbaren Wortpr"afixe und
\mbox{-suffixe} angeben, am Ende der Datei mit passenden Werten
vorbesetzen.  Ansonsten erhalten Sie eventuell Trennungen, bei denen
einzelne Zeichen am Wortanfang oder \mbox{-ende} abgetrennt wird.  Es
ist jedoch {\em nicht\/} notwendig, diese beiden Parameter f"ur jede
zu ladende Trennmusterdatei zu setzen, da \TeX\ die Parameterwerte
nicht mit den Trennmustern abspeichert.

Wollen Sie in einem Format nur Trennmuster f"ur eine einzige Sprache
laden und soll diese Sprache nicht mehr gewechselt -- und damit
ausgeschaltet -- werden k"onnen, verwenden Sie einfach die folgende
Konfigurationsdatei:
%
\begin{quote}\small
\begin{verbatim}
%% Konfigurationsdatei -- Trennmuster: `hyphen.cfg'
%% !! NUR EIN SATZ TRENNMUSTER LADEN !!
\message{== Loading hyphenation patterns:}

\chardef\l@german=\language
\newcount\language   %% <<== "neues" \language-Register 

\message{german}
\chardef\l@austrian=\l@german
\input ghyph31

%% Parameter fuer deutsche Trennmuster setzen:
\lefthyphenmin=2 \righthyphenmin=2
\message{done.}
\endinput
\end{verbatim}
\end{quote}

\paragraph{Schritt 2:}  Entscheiden Sie sich, f"ur welches
\TeX-Format Sie eine neue Formatdatei erstellen wollen.  Je nach zu
erzeugendem Format hat die in Schritt~$1$ gezeigte
Konfigurationsdatei einen anderen Namen:
%
\begin{quote}
\begin{tabular}{@{}ll@{}}
\qquad {\it Format} & {\it Dateiname} \\[3pt]
plain-\TeX                      & {\tt hyphen.tex} \\
\LaTeXe{} (ab 1.~Dezember 1994) & {\tt hyphen.cfg}\\
\LaTeXe{} (1.~Juni 1994)        & {\tt lthyphen.cfg} \\
\LaTeX 2.09 (ab Dezember 1991)  & {\tt lhyphen.tex} \\
\LaTeX 2.09 (vor Dezember 1991) & {\tt hyphen.tex}
\end{tabular}
\end{quote}
%
Geben Sie der in Schritt~$1$ erstellten Datei den entsprechenden
Dateinamen aus dieser Tabelle.

{\it Zu plain-\TeX{} und den "alteren \LaTeX-Versionen:\/} Da die
US-englischen Trennmuster, die als Mindestausstattung jeder
Installation beiliegen sollten, auch den von Donald E.\ Knuth
vergebenen Namen {\tt hyphen.tex} haben, m"us\-sen Sie diese
Trennmusterdatei f"ur plain-\TeX\ und die "alteren
\LaTeX~2.09-Versionen in {\tt ushyphen.tex} umbenennen, um
Namens"-kollisionen zu vermeiden.  Vergessen Sie in diesem Falle
nicht, die Zeile mit dem Namen dieser Datei in der Konfigurationsdatei
entsprechend zu "andern, falls Sie die Trennmusterdatei umbenennen!

\paragraph{Schritt 3:}  Bevor Sie nun mit Ini\TeX\ eine Formatdatei
erstellen, sollten Sie sich vergewissern, da"s Ihre \TeX-Version
gen"ugend Platz f"ur die Trennmuster reserviert hat.  F"ur die
deutschen Trennmuster {\tt ghyph31.tex} be"-n"otigen Sie einen
"`Trie"'-Wert, der mindestens $9733$ {\tt trie\_size} und $207$ {\tt
  trie\_op\_size} Platz bietet.  Wollen Sie Trennmuster f"ur weitere
Sprachen, wie beispielsweise US-englisch, dazuladen, m"ussen diese
Werte er"-h"oht werden.  Vergessen Sie sp"ater nicht, da"s Sie auch
bei der "Uber"-setzung der Dokumente entsprechend mehr Platz f"ur die
Trennmuster be"-n"otigen.  Ob und wie Sie diese Werte in Ihrer
\TeX-Installation "andern k"onnen, entnehmen Sie bitte der
(hoffentlich) mit der Implementierung mitgelieferten Dokumentation.
H"aufig kann man die Gr"o"se der Trennmustertabelle "uber
Umgebungsvariablen, spezielle Konfigurationsdateien oder Optionen in
der Kommandozeile bestimmen.

\paragraph{Schritt 4:}  Erzeugen Sie durch Aufruf von Ini\TeX\ aus
der Datei, die die Makros f"ur das \TeX-Format ent"-h"alt, die
Formatdatei.  Zum Beispiel wird die \LaTeXe-Formatdatei durch den
Aufruf
%
\begin{quote}\small
  \verb:initex latex.ltx:
\end{quote}
%
erzeugt.  Genaueres dar"uber, wie Sie mit Ini\TeX\ eine Formatdatei
erstellen, entnehmen Sie bitte der Ihrer \TeX-Installation
beigelegten Dokumentation.  Welche Datei Sie zum Erstellen eines
speziellen \TeX-Formats laden m"ussen, entnehmen Sie der
Dokumentation des entsprechenden Formats.

W"ahrend Ini\TeX\ die angegebene Makrodatei und eventuell noch weitere
Dateien l"adt, sollten Sie darauf achten, da"s auch die oben
angegebene Konfigurationsdatei und die Trennmuster geladen werden.
Wenn dies nicht geschieht, haben Sie den falschen Namen f"ur die
Konfigurationsdatei ge"-w"ahlt.  "Uber"-pr"ufen Sie nochmals den
Dateinamen und ziehen Sie die Dokumentation des Formats zu Rate.

Bricht Ini\TeX\ beim Laden der Trennmusterdateien mit der
Fehlermeldung
%
\begin{quote}\small
  {\tt \TeX\ capacity exceeded, sorry
  [pattern memory=}{\it x}{\tt ]}
\end{quote}
%
ab, so haben Sie {\tt trie\_size} zu klein gew"ahlt, bei
%
\begin{quote}\small
  {\tt \TeX\ capacity exceeded, sorry
  [pattern memory ops=}{\it x}{\tt ]}
\end{quote}
%
wurde {\tt trie\_op\_size} zu klein gew"ahlt.  Falls dies bei Ihrer
\TeX-Implemen"-tierung m"og"-lich ist, sollten Sie in diesen F"allen
die Tabellen entsprechend ver"-gr"o"sern (siehe Schritt~3).
Ansonsten k"onnen Sie nicht alle in Ihrer Konfigurationsdatei
angegebenen Trennmusterdateien in einer einzigen Formatdatei
verwenden.

\paragraph{Schritt~5:}  Wenn alles ohne Probleme durchlief, sollte
zum Schlu"s eine Datei mit Endung~{\tt .fmt} und eine Protokolldatei
mit Endung~{\tt .log} erstellt worden sein.  Kopieren Sie beide
Dateien in das Verzeichnis, in denen Ihre \TeX-Implementierung nach
Formatdateien sucht.  Die Protokolldatei sollten Sie nicht l"oschen,
da Sie anhand dieser Datei noch Monate sp"ater nachvollziehen
k"onnen, mit welchen Makros Sie diese Formatdatei erstellt haben.

Als Kontrolle, welche Trennmuster f"ur welche Sprachen geladen wurden,
k"onnen Sie das Ende der Protokolldatei betrachten.  Sie sollten beim
Erstellen eines \LaTeXe-Formates ungef"ahr folgende Zeilen erhalten:
%
\begin{quote}\small
\begin{verbatim}
This is TeX, C Version 3.14159 (INITEX)
**latex.ltx
(latex.ltx (texsys.cfg)
...
(hyphen.cfg == Loading hyphenation patterns:
us-english (hyphen.tex) german
(ghyph31.tex German Hyphenation Patterns `ghyphen'
Version 3.1a <94/02/13>) done.)
...
 )
Beginning to dump on file latex.fmt
 (format=latex 96.5.1)
...
14 hyphenation exceptions
Hyphenation trie of length 13987 has 388 ops out of 750
  207 for language 1
  181 for language 0
No pages of output.
\end{verbatim}
\end{quote}
%
Wie Sie in der vorletzten und drittletzten Zeile sehen, enth"alt die
Formatdatei Trennmuster f"ur zwei Sprachen, denen die Nummern $0$
und~$1$ zugewiesen wurden.  Zusammen be"-n"o"-tigen Sie {\tt
  trie\_size}${}=13987$ und {\tt trie\_op\_size}${}=388$ von
$750$~vorhandenen Pl"atzen im "`Trie"'-Bereich.  Au"serdem wurden mit
\verb:\hyphenation: zus"atzlich $14$~Trennungsausnahmen geladen.
Laden Sie andere Trennmuster, unterscheiden sich Ihre Werte nat"urlich
von den hier gezeigten Werten.

\vspace{\baselineskip}

In der Konfigurationsdatei, in der die Trennmuster geladen werden,
werden die den Sprachen zugeordneten Nummern in den {\it control
  sequences\/} {\tt\string\l@}$\langle$Sprache$\rangle$ abgespeichert.
Wird sp"ater beim "Uber"-setzen eines Dokuments auf eine Sprache
gewechselt, f"ur die keine {\it control sequence\/} definiert wurde,
so verwendet die Style-Option "`german"' die
folgenden Default-Werte:
%
\begin{quote}\small
\begin{tabular}{@{}lcl@{}}
\multicolumn{1}{c}{\it Sprache~}&\it Wert&\it oder,
  falls gesetzt, Wert der Sprache\\[3pt]
 \tt USenglish & 255 & \tt english\\
 \tt english   & 255 & \tt USenglish\\
 \tt german    & 255 & \tt austrian\\
 \tt austrian  & 255 & \tt german\\
 \tt french    & 255
\end{tabular}
\end{quote}
%
In "alteren Versionen der Style-Option "`german"' wurden den Sprachen
Werte zwischen $0$ und~$5$ zugewiesen.  Ab Version~2.5a wird der
Wert~$255$ verwendet, um die Trennung f"ur alle undefinierten Sprachen
zu unterbinden.  Dies funktioniert in den meisten F"allen, da im
allgemeinen f"ur den Wert~$255$ keine Trennmuster geladen werden.


\section{Sonstiges}

\subsection{Geschichtliches}

Die Style-Option "`german"', die das Erstellen deutschsprachiger
Texte mit \TeX\ vereinfacht, wurde von Dr.\ H.\ Partl (Technische
Universit"at Wien) realisiert und zusammengestellt.  Sie war als
"`rasche L"osung"' entstanden, die den Vorteil hat, da"s sie keine
"Anderungen an der \TeX-Software, den Schriftdateien und den
Trennmustern erfordert, sondern direkt auf die Originalversion von
\TeX\ aufgesetzt werden kann~\cite{Partl87,Partl88}.  Diese "`rasche
L"osung"' hat sich be"-w"ahrt und besitzt heutzutage ein weites
Verbreitungsgebiet im deutschsprachigen Raum und dar"uberhinaus.

Folgende Personen haben durch Ideen und Code-Beispiele zum Erfolg der
Style-Option "`german"' beigetragen (die Liste erhebt keinen Anspruch
auf Voll"-st"andigkeit): W.~Appelt, F.~Hommes und andere
(Gesellschaft f"ur Mathematik und Datenverarbeitung St.~Augustin),
T.~Hofmann ({\sc Ciba-Geigy} Basel), N.~Schwarz (Universit"at
Bochum), J.~Schrod (TH Darmstadt), D.~Armbruster (Universit"at
Stuttgart), R.~Sch\"opf (Universit"at Mainz, Universit"at Heidelberg,
Zuse-Zentrum f"ur Informationsverarbeitung Berlin), F.~Mittelbach
(Universit"at Mainz, EDS R"usselsheim), J.~Knappen (Universit"at
Mainz), P.~Breitenlohner (Max-Planck-Institut M\"unchen) und viele
andere.

Ab Version~2.3e vom 31.~Juli 1991 wird die Style-Option "`german"'
von B.~Raichle (Universit"at Stuttgart) gepflegt und
weiterentwickelt.


\subsection{Verf"ugbarkeit der Style-Option}

Die Style-Option "`german"' kann man, wie fast jede andere
\TeX-Soft\-ware, vom {\it Comprehensive \TeX\ Archive Network}, kurz
\mbox{CTAN}, erhalten.  Dieses Netzwerk besteht z.\,Z.\ aus den drei
ftp-Servern {\tt ftp.dante.de}, {\tt ftp.tex.ac.uk} und {\tt
  ftp.shsu.edu} wobei auf jedem dieser drei Ser\-ver der gleiche
Inhalt zu finden ist.  \mbox{CTAN} dient als sogenannter
"`Back\-bone"', d.\,h.\ das komplette Archiv oder Teile davon werden
von diesen drei Servern auf vielen weiteren Ser\-ver "`gespiegelt"'
vorgehalten, so da"s Sie die f"ur Sie g"unstigste
Zugriffs"-m"oglichkeit w"ahlen k"onnen.  Au"serdem ist der Inhalt von
\mbox{CTAN} von mehreren Anbietern auch auf CD-ROM er"-h"altlich.

Die Style-Option "`german"', die zugeh"orige Dokumentation und die
deutschen Trennmuster finden Sie auf \mbox{CTAN} in den
Verzeichnissen
\begin{quote}
\begin{verbatim}
tex-archive/languages/german/
tex-archive/languages/hyphenation/
\end{verbatim}
\end{quote}

Die Mitgliedschaft in DANTE~e.V\negthinspace.\ bietet eine weitere
M"og"-lichkeit an die Style-Option "`german"', weitere \TeX-Soft\-ware
und viele Kontakte zu anderen \TeX-Anwendern zu gelangen.
Informationen "uber DANTE~e.V\negthinspace.\ erhalten Sie von
%
\begin{quote}
 DANTE, Deutschsprachige Anwendervereinigung
        \TeX\ e.V\negthinspace. \\
 Postfach 10\,18\,40 \\
 69008 Heidelberg \\
 Federal Republic of Germany\\[4pt]
 Tel.: +49 6221 2\,97\,66\\
 Fax: +49 6221 16\,79\,06\\
 E-mail: {\tt dante@dante.de}\\[4pt]
 {\small WWW}: {\tt http://www.dante.de/}\\
 ftp: {\tt ftp://ftp.dante.de/tex-archive/usergrps/dante/}
\end{quote}


\subsection{\protect\TeX~2.x vs.\ \protect\TeX~3.x und die
  Style-Option "`german"'}

Seit Oktober~1987 ist \TeX\ in der Version~$3.x$ verf"ugbar.  Da
diese Version gegen"-"uber \TeX~$2.x$ einige Erweiterungen besitzt,
soll hier kurz auf die wichtigsten "Anderungen in der Style-Option
"`german"' eingegangen werden.

\TeX\ Version~3 f"uhrte neue {\it control sequences\/} f"ur neue
Primitive und interne Register ein.  Darunter f"allt das in "alteren
Versionen der Style-Option "`german"' verwendete Makro
{\tt\string\set\-language}, das ab Version~2.3 in
{\tt\string\select\-language} umbenannt wurde.  Da
{\tt\string\set\-language} ein \TeX~3-Primitiv ist und deshalb keine
auf"-w"arts"-kompatible Definition angeboten werden kann, mu"s in
"alteren Texten der alte Makroname durch den neuen Namen ersetzt
werden.

Zu \TeX{}s internen Registern kamen u.\,a.\ \verb:\language:,
{\tt\string\left\-hyphen\-min} und {\tt\string\right\-hyphen\-min}
hinzu, um Trennmuster f"ur mehr als eine Sprache unter"-st"utzen zu
k"on\-nen.  Diese drei Register bestimmen die zu verwendenden
Trennmuster und die Mindest"-l"an"-ge der nicht trennbaren
Wort"-pr"a"-fixe und \mbox{-suffixe}.  Der "`normale"' \TeX-Benutzer
sollte diese Register zum Wechsel der Trennmuster nie direkt "andern,
sondern sollte dazu das {\tt\string\select\-language}-Makro
verwenden.  Ab Version~2.4a der Style-Option "`german"' wird f"ur die
Sprachen {\tt german} und {\tt austrian} zu"-s"atzlich
{\tt\string\french\-spacing} und die Werte f"ur
{\tt\string\left\-hyphen\-min} und {\tt\string\right\-hyphen\-min}
auf zwei gesetzt.


\subsection{"Anderungen seit Version~2.0 (Oktober 1987)}

Zus"atzlich zu den in fr"uheren Abschnitten erw"ahnten "Anderungen
kommen folgende hinzu:

\begin{itemize}\tolerance=9999\hbadness=2500
\item In Versionen vor~2.2 fehlen die Befehle \verb:"S:, \verb:"CK:,
  \verb:"FF: f"ur Gro"sbuchstaben und die entsprechenden Befehle f"ur
  die Konsonanten L, M, N, P und~T.
\item In Versionen bis~2.2 gab es die undokumentierten Makros
  {\tt\string\original\-@dospecials} und
  {\tt\string\original\-@sanitize}, die die urspr"unglichen
  Definitionen von \verb:\dospecial: und \verb:\@sanitize:
  enthielten, und {\tt\string\german\-@dospecials},
  {\tt\string\german\-@sanitize}, die zu"-s"atzlich das
  Double\-quote~(\verb:":) enthielten. Diese Makros werden von
  einigen "`fremden"' Makros benutzt, obwohl sie undokumentiert und
  nur zur internen Verwendung bestimmt waren.
\item Ab Version~2.3e werden die etwas tieferen Umlautakzente durch
  ein ge"-"an"-der"-tes Makro erzeugt, das schneller ist und zu
  kleineren \verb:dvi:-Dateien f"uhrt.  Au"serdem werden jetzt alle
  Definitionen\slash Zuweisungen lokal aus\-ge\-f"uhrt.  Ausnahmen
  hiervon sind alle Z"ah\-ler"-allokationen.
\item Bis Version~2.3e wurden bei Verwendung von~\verb:"|: zur
  Verhinderung von Ligaturen keine weiteren Trennstellen im Wort
  gefunden.
\item In Version~2.4a wurden durch zu"-s"atzliche~\verb:\/:
  "Uberschneidungen der "offnenden An"-f"uh\-rungs"-zeichen mit
  nachfolgenden Zeichen bei Verwendung der
  Com\-pu\-ter-Modern-Schriften f"ur viele F"alle verhindert.
  Undefinierte \verb:"x:-Befehle f"uhren jetzt zu einer Fehlermeldung
  (dies f"uhrt oft dazu, da"s viele einfache Tippfehler schon beim
  "Ubersetzen aufgedeckt werden).  Die Befehle \verb:"z:, \verb:"Z:,
  \verb:"~: und \verb:"=: kamen neu dazu.
\item Version~2.5a ent"-h"alt Anpassungen an \LaTeXe{} zur Verwendung
  von Schriften mit Kodierung {\tt OT1} und~{\tt T1}.  Die Befehle
  \verb:"r: und~\verb:"R: kamen hinzu.  Die Defaults f"ur
  "`undefinierten"' Sprachen sind jetzt~$255$ mit entsprechenden
  Warnungen, um die Benutzer zu zwingen, beim Laden der Trennmuster
  gleich die verwendete \verb:\language:-Nummer zu sichern (z.\,B.\ 
  mit dem Babel-System oder einer von Hand geschriebenen
  Konfigurationsdatei).  Intern wurden, neben der Verwendung von {\it
    docstrip}, kleinere Optimierungen ausgef"uhrt.
\item Bis Version~2.5b wird bei \verb:"ff: die ff-Ligatur verhindert,
  obwohl diese bei Ausfall des dritten~"`\mbox{f}"' stehen sollte.
  Diese "Anderung ab Version~2.5c kann zu gering"-f"ugig anderen
  Trennungen im direkt auf \verb:"ff: folgenden Wortteil f"uhren.
\item Version~2.5c f"ugt das explizite Kerning der oberen deutschen
  Anf"uhrungszeichen~"' f"ur T1-kodierte Schriften nicht mehr ein, da
  ab Version~1.2 der DC-Schriften diese Anf"uhrungszeichen etwas
  weiter nach links plaziert werden, so da"s es zu "Uber"-schneidungen
  kommen w"urde.
\end{itemize}


\begin{thebibliography}{99}\tolerance=2000\hbadness=2000
\bibitem[Duden, Bd.~1]{Duden} {\sc Duden},
  Rechtschreibung der deutschen Sprache und der
  Fremd"-w"orter.  Hrsg.\ von der {\sc Duden}-Redaktion.
  Auf der Grundlage der amtlichen Rechtschreibregeln. 
  Mannheim; Wien; Z"urich: Bibliographisches Institut,
  19.~Auf"|lage, 1986.
\bibitem[Partl87]{Partl87} Hubert Partl,
  {\it Ein "`Minimal Subset"' f"ur einheitliche deutsche
    \TeX-Befehle}, Vortrag und Diskussion beim 6.~Treffen der
  deutschen \TeX-""Interessenten in M"unster (1987).
  Anm.: ver"offentlicht in einer Datei namens {\tt germdoc.tex},
  mittlerweile ersetzt durch das gleichnamige, Ihnen vorliegende
  Dokument "`Kurzbeschreibung -- {\tt german.sty}"'.
\bibitem[Partl88]{Partl88} Hubert Partl,
  German \TeX, {\em TUGboat\/} 9(1):70--72, 1988.
\bibitem[Braams91]{Braams91} Johannes Braams,
  Babel, a multilingual style-option system for use with \LaTeX's
  standard document styles,
  {\it TUGboat\/} 12(2):291--301, Juni 1991.
\bibitem[Goosens94a]{Goosens94a} Michel Goosens, Frank Mittelbach und
  Alexander Samarin, {\it The \LaTeX\ Companion},
  Addison-Wesley, Reading, 2.~Auf"|lage, 1994.
\bibitem[Goosens94b]{Goosens94b} Michel Goosens, Frank Mittelbach und
  Alexander Samarin, {\it Der \LaTeX-Begleiter},
  Addison-Wesley, Bonn, 1.~Auf"|lage, 1994.
  Anm.: "uber"-arbeitete, deutsche "Uber"-setzung
  von~\cite{Goosens94a}.
\end{thebibliography}

\end{document}
