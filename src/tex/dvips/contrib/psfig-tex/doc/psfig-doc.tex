%\documentstyle[psfig]{article}
\documentstyle{article}
%\input ../psfig.sty
% this is for AmigaDOS
\input /psfig.sty
\begin{document}
\pssilent
\psfigurepath{figs}
\def\Ps{Post\-Script}
%
\title{Psfig/\TeX\ 1.10 Users Guide}
\author{Trevor Darrell \\ {\it trevor@media.mit.edu}}
\date{\ }
\maketitle

\section{Introduction}
Psfig/\TeX\ is a macro package for \TeX\ that facilitates the
inclusion of  \Ps\ figures into \TeX\ documents. With the
help of a compatible postprocessor,
\footnote{The {\tt dvips} program developed by T. Rokicki 
has full psfig support; it is available via anonymous FTP from {\tt labrea.stanford.edu}. OzTeX also supports psfig and is available from various ftp sites. Psfig is available from {\tt whitechapel.media.mit.edu}.}
\Ps\ figures are automatically
scaled and positioned on the page, and the proper amount of space is
reserved.  
Custom characters such as
`\psfig{figure=pretzel.ps,height=8pt,silent=}' and 
`\psfig{figure=cm.ps,height=8pt,silent=}'
may be created and used freely throughout a document, or figures can
be presented as traditional broken-out displays:
\par
\hbox{
\hspace{.3in}
\vbox{\psfig{figure=zip.ps}\vspace{.5in}}
\psfig{figure=piechart.ps,height=1.5in}
\vbox{\psfig{figure=starlines.ps}\vspace{.6in}}
}

\section{Simple figures}
To include a \Ps\ figure with psfig, include the psfig style at the top
of your document:
\begin{quote}
{\tt\verb+\documentstyle[psfig,...]{article}+}
\end{quote}
and then, when you wish to include a figure, invoke the macro
\begin{quote}
{\tt\verb+\+psfig\{figure={\it input}\}}
\end{quote}
where {\it input} is the name of a \Ps\ file.  Psfig will
automatically position the figure at the current point on the page,
and reserve the proper amount of space in \TeX\ so that it doesn't
block any other objects on the page.  For example, if we have a file
called `piechart.ps' which contains the \Ps\ code to draw the chart in
the introduction, we could use the commands
\begin{quote}
\tt
\verb+\+par \\
\verb+\+centerline\{\verb+\+psfig\{figure=piechart.ps\}\} \\
\verb+\+par
\end{quote}
to include it as a centered paragraph.
Since no mention of size was made in the above example, psfig draws the figure 
at its natural size (as if it was printed directly on 
a \Ps\ printer.) The pie's natural size is several inches across, which
is a little large; the pie in the introduction was produced with:
\begin{quote}
\tt\verb+\+centerline\{\verb+\+psfig\{figure=piechart.ps,height=1.5in\}\}
\end{quote}
The {\tt height} option specifies how tall the figure should be on the
page. Since no {\tt width} was specified, the figure was scaled
equally in both dimensions. (This will also happen with a {\tt width}
but no {\tt height} option.)  By listing both a {\tt height} and {\tt
width}, figures can be scaled disproportionately, with interesting
results:
\par
\centerline{\hbox{
\psfig{figure=rosette.ps,height=.8in,width=.15in}
\psfig{figure=rosette.ps,height=.8in,width=.35in}
\psfig{figure=rosette.ps,height=.8in}
\psfig{figure=rosette.ps,height=.8in,width=1.2in}
\psfig{figure=rosette.ps,height=.8in,width=1.5in}
}}
\par
\noindent This figure was produced with:
\begin{quote}
\tt\verb+\+centerline\{\verb+\+hbox\{ \\
\verb+\+psfig\{figure=rosette.ps,height=.8in,width=.15in\}\\
\verb+\+psfig\{figure=rosette.ps,height=.8in,width=.35in\}\\
\verb+\+psfig\{figure=rosette.ps,height=.8in\} \\
\verb+\+psfig\{figure=rosette.ps,height=.8in,width=1.2in\}\\
\verb+\+psfig\{figure=rosette.ps,height=.8in,width=1.5in\}\\
\}\} 
\end{quote}
The \verb+\psfig+ macro is (unfortunately) sensitive to whitespace, and will be
confused by any extra spaces or newlines in its argument.
\section{Sources of figures}
\hbox{
\vbox{\psfig{figure=trevor.ps,height=1in}}
\hspace{.15in}
\vbox{\parbox{3.5in} {Any program that produces \Ps\ as output
can be used for psfig figures
as long as it adheres to the bounding box comment convention (see below).
For instance, the bitmap image
of the author was included as a figure at left.}\vspace{.2in}}}

Since the Macintosh drawing applications produce \Ps, they can be used
to create figures.  However, the \Ps\ produced by most Macintosh
applications is often not well suited to be included directly as
a psfig figure, unless it is saved as an ``EPSF compliant'' file. If
the file is not EPSF compliant then the postscript may have to be
edited before being used as an included figure.

\section{Draft figures and Silent mode}
\begin{figure}
\centerline{
\psfig{figure=rosette.ps,height=2in}
\psdraft
\psfig{figure=rosette.ps,height=2in}
\psfull
}
\vspace{.25in}
\centerline{Figure 2: The rosette.ps figure, in non-draft and draft mode}
\end{figure}
Normally, psfig will print advisory messages to remind you that it is
including figures as TeX processes a document. This behavior can
be disabled with {\tt \verb+\+pssilent}, and re-enabled with
{\tt \verb+\+psnoisy}.
\par
Some \Ps\ figures can take quite a long time to transmit and print;
for these figures a draft mode is available to speed printing of draft
versions of the document.  A figure printed in draft mode will appear
as an outlined box (Figure 2).  The macro {\tt
\verb+\+psdraft} will switch into draft mode, and all subsequent psfig
macros will produce draft figures until reaching the macro {\tt
\verb+\+psfull}, which switches out of draft mode.

No {\tt
\verb+\+special} commands are used in draft mode, so a draft document
can be previewed using any Dvi viewer.  Psfig uses the \LaTeX\ {\tt
\verb+\+fbox} command to produce the draft box; thus draft boxes will
not work in plain \TeX. The printing of boxes in draft mode can be
disabled/enabled with {\tt
\verb+\+psnodraftbox} and {\tt
\verb+\+psdraftbox}. 

\section{Bounding boxes}
To properly translate and scale a figure psfig must know its `natural'
position on the page; this information is present
in what is called the {\it bounding box} of a \Ps\ program. The 
bounding box is an outer limit to the marks created by a program,
and is specified as four coordinates of a rectangle: the lower-left $x$ coordinate
(bbllx), the lower-left $y$ coordinate (bblly), the upper-right
$x$ coordinate (bburx), and the upper-right $y$ coordinate (bbury).
Adobe has defined a convention whereby the bounding box of a program
is contained in a `bounding box comment'.
\footnote{See `Appendix J: \Ps\ File Structuring Conventions' in
{\it The \Ps\ Language Reference Manual}}
This comment, which must be present in any file to be used as a psfig figure,
is a line of the form
\begin{quote}
\tt \verb+%%+BoundingBox: \it bbllx bblly bburx bbury
\end{quote}
All values are in \Ps\ points, relative to the {\it default}
transformation matrix. The only mandatory \Ps\ convention is
that the first line of the file should begin with the characters
`\verb+%+!' (a `\verb+%+' begins a comment in \Ps.) A good place for the
bounding box comment is as the second line of the file.
\par
If a bounding box comment is present in the figure file, psfig will
extract its values.  The bounding box values may instead be specified
directly in the {\tt \verb+\+psfig} argument, using clauses of the
form {\tt bbllx=\it bbllx},{\tt bblly=\it bblly},..., in which case
the figure file is not searched for the bounding box.

\section{Reserved size}
\par
\psfig{figure=box.ps,rheight=0bp,rwidth=0bp,height=1.25in,width=\textwidth,silent=}
\par
There are two sizes associated with each psfig figure: the size
at which it is to be printed on the page
and the size it reserves in \TeX. This latter size is appropriately
termed the {\it reserved size}, and is expressed as clauses of the form
``{\tt rheight={\it dimen}}''
and ``{\tt rwidth={\it dimen}}''. If omitted, the reserved size defaults
to the real size. Some special effects need to be transparent
to \TeX\ and thus have a zero reserved size, such as the grey
box enclosing
this paragraph.

\section{Clipping}
Normally a \Ps\ program can be expected to not mark the page 
outside its bounding box. If this is not the case, or if you
want to specify a bounding box so as to isolate part of a larger figure,
there is an option that sets the \Ps\ clip path so that
no marks will show up outside the declared bounding box. Currently
this is invoked by adding a clause of the form ``{\tt clip=}''.
Here a slice has been taken out of the pie chart in the example by
specifying a smaller bounding box and adding the clip option.
\par
\centerline{\protect\fbox{\psfig{figure=piechart.ps,height=2in,bbllx=306bp,bblly=396bp,bburx=486bp,bbury=546bp,clip=}}}
\centerline{A piece of the pie.}
\vspace{.2in}
\par
Some \Ps\ programs use the clipping path to position their output on
the page; if a figure is being drawn at its natural size and position
despite psfig commands to the contrary, it may need the clip option.

\newpage
\section{Rotating figures}
Figures can be rotated by psfig using the {\tt angle={\it degrees}}
clause. For example, here is the rosette and its 90 degree rotation:

\centerline{\hbox{
\psfig{figure=rosette.ps,height=1.0in}
\psfig{figure=rosette.ps,height=1.0in,angle=90}
}} 

By default psfig scales the figure so that its rotated bounding box
fits within the desired size. This can lead to counterintuitive results
when rotating to angles which are not multiples of 90 degrees. Here
is the rosette rotated to 0,20,40, and 60 degrees.

\centerline{\hbox{
\psfig{figure=rosette.ps,height=1.0in}
\psfig{figure=rosette.ps,height=1.0in,angle=20}
\psfig{figure=rosette.ps,height=1.0in,angle=40}
\psfig{figure=rosette.ps,height=1.0in,angle=60}
}} 

With autoscaling, some rotated figures come out smaller because the
diagonal of their bounding box is of course longer than their height
or width alone.  This behavior can be disabled with {\tt
\verb+\+psscalefirst}, and re-enabled with {\tt
\verb+\+psrotatefirst}. With {\tt \verb+\+psscalefirst} a new
height and width is computed after the bounding box;
the previous figure would now look like:

\psscalefirst
\centerline{\hbox{
\psfig{figure=rosette.ps,height=1.0in}
\psfig{figure=rosette.ps,height=1.0in,angle=20}
\psfig{figure=rosette.ps,height=1.0in,angle=40}
\psfig{figure=rosette.ps,height=1.0in,angle=60}
}} 

While the rotated figures will all come out at the same
size their reserved sizes will be different, thus
they may not be aligned correctly.

\section{Compressed Figures}
Psfig allows the inclusion of compressed \Ps files when using 
the {\tt dvips} dvi processor.
The shell script {\tt pscompress} is used to compress figures, and 
produces two files: {\it filename}{\tt .bb} and {\it filename}{\tt .Z}.
The first file contains the bounding box comment only, while the second
file contains the actual compressed PostScript file. Usage:
\begin{quote}
\tt \verb+%+ pscompress \it filename
\end{quote}
When psfig searches for a figure, if it fails to find 
{\it filename}, it then searches for {\it filename}{\tt .bb}.
If that file exists, it is used for bounding box processing,
and a command to  decompress and include the file {\it filename}{\tt .Z}
is issued to dvips.

\section{Figure search path}

Psfig first searches in the current directory for a figure (or
in the specified directory if given an absolute path). If it fails
to find the figure in the current directory, it optionally searches
a search path of figure directories to see if the figure is
present. To specify the figure search path, use
\begin{quote}
\tt \verb+\+psfigurepath\verb+{+\it dir1 \tt :\it dir2\tt :\it ...\tt :\it dirn\tt\verb+}+
\end{quote}
where {\it dir1...dirn} are the directories figures are to be found in.
\newpage
\section{Acknowledgements}
\par
This work was done while the author was with the Department of
Computer and Information Science, University of Pennsylvania.  Ned
Batchelder co-developed the original {\it troff} version of this
program with the author, and was responsible for much of the overall
design.  For more detailed information on the original version of \Ps\
see {\it Psfig -- A Ditroff Preprocessor for PostScript Figures} in
the USENIX 87 proceedings, or {\it Bringing troff up to speed} in the
July 1987 issue of Unix Review.

Greg Hager provided the initial pure-\TeX\ implementation of psfig.
J. Daniel Smith of Schlumberger CAD/CAM implemented the rotation
feature and improved the file scanning routines, using certain code
fragments from Tom Rokicki's dvips program.
\par 
\end{document}

