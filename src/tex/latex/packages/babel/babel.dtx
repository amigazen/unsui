% \iffalse meta-comment
%
% Copyright 1989-1995 Johannes L. Braams and any individual authors
% listed elsewhere in this file.  All rights reserved.
% 
% For further copyright information any other copyright notices in this
% file.
% 
% This file is part of the Babel system release 3.5.
% --------------------------------------------------
%   This system is distributed in the hope that it will be useful,
%   but WITHOUT ANY WARRANTY; without even the implied warranty of
%   MERCHANTABILITY or FITNESS FOR A PARTICULAR PURPOSE.
% 
%   For error reports concerning UNCHANGED versions of this file no more
%   than one year old, see bugs.txt.
% 
%   Please do not request updates from me directly.  Primary
%   distribution is through the CTAN archives.
% 
% 
% IMPORTANT COPYRIGHT NOTICE:
% 
% You are NOT ALLOWED to distribute this file alone.
% 
% You are allowed to distribute this file under the condition that it is
% distributed together with all the files listed in manifest.txt.
% 
% If you receive only some of these files from someone, complain!
% 
% Permission is granted to copy this file to another file with a clearly
% different name and to customize the declarations in that copy to serve
% the needs of your installation, provided that you comply with
% the conditions in the file legal.txt from the LaTeX2e distribution.
% 
% However, NO PERMISSION is granted to produce or to distribute a
% modified version of this file under its original name.
%  
% You are NOT ALLOWED to change this file.
% 
% 
% \fi
% \CheckSum{2575}
%%%\iffalse
%%%  @LaTeX-file{
%%%     author          = "Johannes L. Braams",
%%%     version         = "3.5c",
%%%     date            = "21 june 1995",
%%%     time            = "23:55:00 MET",
%%%     filename        = "babel.dtx",
%%%     address         = "Kooienswater 62
%%%                        The Netherlands",
%%%     telephone       = "(70) 3432037",
%%%     FAX             = "(70) 3432395",
%%%     checksum        = "54332 2092 9869 81235",
%%%     email           = "JLBraams@cistron.nl (Internet)",
%%%     codetable       = "ISO/ASCII",
%%%     keywords        = "babel",
%%%     supported       = "yes",
%%%     docstring       = "This file, babel.dtx contains the core of
%%%                        the babel system as it was described in
%%%                        TuGboat Volume 12 no 2 and Volume 14 no 1.
%%%
%%%                        The checksum field above contains a CRC-16
%%%                        checksum as the first value, followed by the
%%%                        equivalent of the standard UNIX wc (word
%%%                        count) utility output of lines, words, and
%%%                        characters.  This is produced by Robert
%%%                        Solovay's checksum utility.",
%%%      }
%%% \fi
%%
% \def\filename{babel.dtx}
% \let\thisfilename\filename
%
%\iffalse
%<*dtx>
\ProvidesFile{babel.dtx}
%</dtx>
%
% Babel DOCUMENT-STYLE option for LaTeX version 2.09 or plain TeX;
% Babel package for LaTeX2e.
%
%% Copyright (C) 1989 -- 1995 by Johannes Braams,
%%                            TeXniek
%%                            all rights reserved.
%
%% Please report errors to: J.L. Braams
%%                          JLBraams@cistron.nl
%<*filedriver>
\documentclass{ltxdoc}
\font\manual=logo10 % font used for the METAFONT logo, etc.
\newcommand*\MF{{\manual META}\-{\manual FONT}}
\newcommand*\TeXhax{\TeX hax}
\newcommand*\babel{\textsf{babel}}
\newcommand*\m[1]{\mbox{$\langle$\it#1\/$\rangle$}}
\newcommand*\langvar{\m{lang}}
\newcommand*\note[1]{}
\newcommand*\bsl{\protect\bslash}
\newcommand*\Lopt[1]{\textsf{#1}}
\newcommand*\file[1]{\texttt{#1}}
\newcommand*\pkg[1]{\texttt{#1}}
\begin{document}
 \DocInput{babel.dtx}
\end{document}
%</filedriver>
%\fi
%
% \GetFileInfo{babel.dtx}
%
% \changes{babel~2.0a}{1990/04/02}{Added text about \file{german.sty}}
% \changes{babel~2.0b}{1990/04/18}{Changed order of code to prevent
%    plain \TeX from seeing all of it}
% \changes{babel~2.1}{1990/04/24}{Modified user interface,
%    \cs{langTeX} no longer necessary}
% \changes{babel~2.1a}{1990/05/01}{Incorporated Nico's comments}
% \changes{babel~2.1b}{1990/05/01}{rename \cs{language} to
%    \cs{current@language}}
% \changes{babel~2.1c}{1990/05/22}{abstract for report fixed, missing
%    \texttt{\}}, found by Nicolas Brouard BROUARD@FRINED51.BITNET}
% \changes{babel~2.1d}{1990/07/04}{Missing right brace in definition of
%    abstract environment, found by Werenfried Spit}
% \changes{babel~2.1e}{1990/07/16}{Incorporated more comments from
%    Nico}
% \changes{babel~2.2}{1990/07/17}{Renamed \cs{newlanguage} to
%    \cs{addlanguage}}
% \changes{babel~2.2a}{1990/08/27}{Modified the documentation
%    somewhat}
% \changes{babel~3.0}{1991/04/23}{Moved part of the code to hyphen.doc
%    in preparation for \TeX~3.0}
% \changes{babel~3.0a}{1991/05/21}{Updated comments in various places}
% \changes{babel~3.0b}{1991/05/25}{Removed some problems in change log}
% \changes{babel~3.0c}{1991/07/15}{Renamed \file{babel.sty} and
%    \file{latexhax.sty} to \file{.com}}
% \changes{babel~3.1}{1991/10/31}{Added the support for active
%    characters and for extending a macro}
% \changes{babel~3.1}{1991/11/05}{Removed the need for
%    \file{latexhax}}
% \changes{babel~3.2}{1991/11/10}{Some Changes by br}
% \changes{babel~3.2a}{1992/02/15}{Fixups of the code and
%    documentation}
% \changes{babel~3.3}{1993/07/06}{Included driver file, and prepared
%    for dsitribution}
% \changes{babel~3.4}{1994/01/30}{Updated for \LaTeXe}
% \changes{babel~3.4}{1994/02/28}{Added language definition file for
%    bahasa}
% \changes{babel~3.4b}{1994/05/18}{Added a small driver to be able to
%    process just this file}
% \changes{babel~3.5a}{1995/02/03}{Provided common code to handle the
%    active double quote}
% \changes{babel~3.5c}{1995/06/14}{corrected a few typos (PR1652)}
% \changes{babel~3.5d}{1995/07/02}{Merged glyphs.dtx into this file}
%
% \title {Babel, a multilingual package for use with \LaTeX's standard
%    document classes\thanks{During the development ideas from Nico
%    Poppelier, Piet van Oostrum and many others have been used.
%    Bernd Raichle has provided many helpful suggestions.}}
%
% \author{Johannes Braams\\
%         Kooienswater 62\\
%         2715 AJ Zoetermeer\\
%         The Netherlands\\
%         \texttt{J.L.Braams@cistron.nl}}
%
% \date{Printed \today}
%
% \maketitle
%
%  \begin{abstract}
%    The standard distribution of \LaTeX\ contains a number of
%    document classes that are meant to be used, but also serve as
%    examples for other users to create their own document classes.
%    These document classes have become very popular among \LaTeX\
%    users. But it should be kept in mind that they were designed for
%    American tastes and typography. At one time they contained a
%    number of hard-wired texts. This report describes \babel{}, a
%    package that makes use of the new capabilities of \TeX\ version 3
%    to provide an environment in which multilingual documents can be
%    written.
%  \end{abstract}
%
%  \begin{multicols}{2}
%    \tableofcontents
%  \end{multicols}
%
% \section{The user interface}\label{U-I}
%
%    The user interface of this package is quite simple. It basially
%    consists of only two commands. These commands can be used to
%    select another language or to find out what the current language
%    is.
%
% \DescribeEnv{language}
%    The environment \textsf{language} does basically the same as
%    |\selectlanguage|, except the language change is local to the
%    environment. For mixing left-to-right typesetting with
%    right-to-left typesetting the use of this environment is a
%    prerequisite.
%
% \DescribeMacro\foreignlanguage
%    The command |\foreignlanguage| takes two arguments, the second
%    argument is a phrase to be typeset according to the rules of the
%    language named in its first argument.
%
% \DescribeMacro\selectlanguage
%    When a user wants to switch from one language to another he can
%    do so using the macro |\selectlanguage|. This macro takes the
%    language, defined previously by a language definition file, as
%    its argument. It calls several macros that should be defined in
%    the language definition files to activate the special definitions
%    for the language chosen.
% 
% \DescribeMacro\languagename
%    The name of the current language is stored in the control
%    sequence |\languagename|.
%
% \DescribeMacro\iflanguage
%    If more than one language is used it might be necessary to know
%    which language is active at a specific time. This can be checked
%    by a call to |\iflanguage|. This macro takes three arguments.
%    The first argument is the name of a language, the second and
%    third arguments are the actions to take if the result of the test
%    is \texttt{true} or \texttt{false} respectively.
%
% \DescribeMacro\useshorthands
%     A command with one argument, the character to make active in
%     order to define personal shorthands.
%
% \DescribeMacro\defineshorthand
%     The command |\defineshorthand| takes two arguments, the first of
%     which is a one or two character sequence, the second argument is
%     the code the shorthand should expand to.
%
% \DescribeMacro\languageshorthands
%     The command |\languageshorthands| can be used to switch the
%     shorthands on the language level. It takes one argument, the
%     name of a language. Note that for this to work the language
%     should have been specified as an option when loading the \babel\
%     package. 
%
% \section{Changes for \LaTeXe}
%
%    With the advent of \LaTeXe\ the interface to \babel\ in the
%    preamble of the doument has changed. With \LaTeX2.09 one used to
%    call up the \babel\ system with a line such as:
%
%\begin{verbatim}
%\documentstyle[dutch,english]{article}
%\end{verbatim}
%
%    which would tell \LaTeX\ that the document would be written in
%    two languages, dutch and english and that english would be the
%    first language in use.
%
%    The \LaTeX2e\ way of providing the same information is:
%
%\begin{verbatim}
%\documentclass{article}
%\usepackage[dutch,english]{babel}
%\end{verbatim}
%
%    or, making \Lopt{dutch} and \Lopt{english} global options in
%    order to let other packages detect and use them:
%
%\begin{verbatim}
%\documentclass[dutch,english]{article}
%\usepackage{babel}
%\usepackage{varioref}
%\end{verbatim}
%
%    In this last example the package \texttt{varioref} will also see
%    the options and will be able to use them.
%
% \section{Changes in Babel version 3.5}
%
%    In Babel version 3.5 a lot of changes have been made when
%    compared with the previous release. Here is a list of the most
%    important ones:
%    \begin{itemize}
%    \item \babel\ now has a \textsf{language} environment and a new
%      command |\foreignlanguage|;
%    \item the way active characters are dealt with is completely
%      changed. They are called `shorthands'; one can have three
%      levels of shorthands: on the user level, the language level and
%      on `system level'. A consequence of the new way of handling
%      active characters is that they are now written to auxiliary
%      files `verbatim';
%    \item A language change now also writes information in the
%      \file{.aux} file as the change might also affect typesetting
%      the table of contents. The consequence is that an .aux file
%      generated by a LaTeX format with babel preloaded gives errors
%      when read with a LaTeX format without babel, but I think this
%      problaly doesn't occur;
%    \item \babel\ is now compatible with the \pkg{inputenc} and
%      \pkg{fontenc} packages;
%    \item the language definition files now have a new extension,
%      \file{ldf};
%    \item the syntax of the file \file{language.dat} is extended to
%      be compatible with the \pkg{french} package by Bernard Gaulle;
%    \item each language definition file looks for a configuration
%      file which has the same name, but the extension \file{.cfg}.
%    \end{itemize}
%
% \section{The interface between the core of \babel{} and the language
%    definition files}
%
%    In the core of the \babel{} system two macros are defined that
%    are to be used in language definition files. Their purpose is to
%    make a new language known.
%
%  \DescribeMacro\addlanguage
%    The macro |\addlanguage| is a non-outer version of the macro
%    |\newlanguage|, defined in \file{plain.tex} version~3.x. For
%    older versions of \file{plain.tex} and \file{lplain.tex} a
%    substitute definition is used.
%
%  \DescribeMacro\adddialect 
%    The macro |\adddialect| can be used in the case where two
%    languages can (or have to) use the same hyphenation
%    patterns. This can be useful when a user wants to use a language
%    for which no patterns are preloaded in the format. In such a case
%    the default behaviour of the \babel{} system is to define this
%    language as a `dialect' of the language for which the patterns
%    were loaded as |\language0|.
%
%    The language definition files have to conform to a number of
%    conventions. The reason for this is that these files have to fill
%    in the gaps left by the common code in \file{babel.def}, i.\,e.,
%    the definitions of the macros that produce texts.  Also the
%    language-switching possibility which has been built into the
%    \babel{} system has its implications.
%
%    The following assumptions are made:
%   \begin{itemize}
%    \item Some of the language-specific definitions might be used by
%    plain \TeX\ users, so the files have to be coded such that they
%    can be read by \LaTeX\ as well as by plain \TeX. This can be
%    checked by looking at the value of the macro |\fmtname|.
%
%    \item The common part of the \babel{} system redefines a number
%    of macros and environments (defined previously in the document
%    style) to put in the names of macros that replace the previously
%    hard-wired texts.  These macros have to be defined in the
%    language definition files.
%
%    \item The language definition files define five macros, used to
%    activate and deactivate the language-specific definitions.  These
%    macros are |\|\langvar|hyphenmins|, |\captions|\langvar,
%    |\date|\langvar, |\extras|\langvar and |\noextras|\langvar. These
%    macros and their functions are discussed below.
%
%    \item When a language definition file is loaded, it can define
%    |\l@|\langvar to be a dialect of |\language0| when |\l@|\langvar
%    is undefined.
%
%    \item The languagedefinition files can be read in the preamble of
%    the document, but also in the middle of document processing. This
%    means that they have to function independently of the current
%    |\catcode| of the \texttt{@}~sign.
%   \end{itemize}
%
%
%  \DescribeMacro\langhyphenmin
%    The macro |\|\langvar|hyphenmins| is used to store the values of
%    the |\lefthyphenmin| and |\righthyphenmin|.
%
%  \DescribeMacro\captionslang
%    The macro |\captions|\langvar defines the macros that
%    hold the texts to replace the original hard-wired texts.
%
%  \DescribeMacro\datelang
%    The macro |\date|\langvar defines |\today| and
%
%  \DescribeMacro\extraslang
%    The macro |\extras|\langvar contains all the extra definitions
%    needed for a specific language.
%
%  \DescribeMacro\noextraslang
%    Because we want to offer the user the possibility to switch
%    between languages and we do not know in what state \TeX\ might be
%    after the execution of |\extras|\langvar, a macro that brings
%    \TeX\ into a predefined state is needed. It will be no surprise
%    that the name of this macro is |\noextras|\langvar.
%
%  \DescribeMacro\main@language
%    To postpone the activation of the definitions needed for a
%    language untill the beginning of a document, all language
%    definition files should use |\main@language| instead of
%    |\selectlanguage|. This will just store the name of the language
%    and the proper language will be activated at the start of the
%    document.
%
%  \DescribeMacro\loadlocalcfg
%    At the end of the processing of a language definition file
%    \LaTeX\ can be instructed to load a local configuration
%    file. This file can for instance be used to add strings to
%    |\captions|\langvar\ in order to support local document
%    classes. The user will be informed of the fact that this
%    configuration file is loaded.
%
% \subsection{Support for active characters}
%
%    In quite a few language definition files, active characters are
%    introduced. To facilitate this, some support macros are provided.
%
% \DescribeMacro{\initiate@active@char}
%    The internal macro |\initiate@active@char| is used in language
%    definition files to instruct \LaTeX\ to give a character the
%    category code `active'. When a character has been made active it
%    will remain that way untill the end of then document. Its
%    definition may vary.
%
% \DescribeMacro{\bbl@activate}
% \DescribeMacro{\bbl@deactivate}
%    The command |\bbl@activate| is used to change the way an active
%    character expands. |\bbl@activate| `switches on' the active
%    behaviour of the character. |\bbl@deactive| lets the active
%    character expand to its former (mostly) non-active self.
%
% \DescribeMacro{\declare@shorthand}
%    The macro |\declare@shorthand| is used to define the various
%    shorthands. It takes three arguments, the name for the collection
%    of shorthands this definition belongs to; the character
%    (sequence) that makes up the shorthand i.i.\ |~| or |"a| and the
%    code to be executed when the shorthand is encountered.
%
% \DescribeMacro{\bbl@add@special}
% \DescribeMacro{\bbl@remove@special}
%    ``Plain \TeX\ includes a macro called |\dospecials| that is
%    essentially a set macro, representing the set of all characters
%    that have a special category code.'' \cite[p.~380]{DEK} It is
%    used to set text `verbatim'.  To make this work if more
%    characters get a special category code, you have to add this
%    character to the macro |\dospecial|.  \LaTeX\ adds another macro
%    called |\@sanitize| representing the same character set, but
%    without the curly braces.  The macros
%    |\bbl@add@special|\meta{char} and
%    |\bbl@remove@special|\meta{char} add and remove the character 
%    \meta{char} to these two sets.
%
% \subsection{Support for saving macro definitions}
%
%    Language definition files may want to \emph{re}define macros that
%    already exist. Therefore a mechanism for saving (and restoring)
%    the original definition of those macros is provided. We provide
%    two macros for this\footnote{This mechanism was introduced by
%    Bernd Raichle.}.
%
% \DescribeMacro{\babel@save} To save the current meaning of any
%    control sequence the macro |\babel@save| is provided. It takes
%    one argument, \meta{csname}, the control sequence for which the
%    meaning has to be saved.
%
% \DescribeMacro{\babel@savevariable} A second macro is provided to
%    save the current value of a variable.  In this context anything
%    that is allowed after the |\the| primitive is considered to be a
%    variable. The macro takes one argument, the \meta{variable}.
%
%    The effect of the aforementioned macros is that a piece of code
%    is appended to the current definition of |\originalTeX|. When
%    |\originalTeX| is expanded this code restores the previous
%    definition of the control sequence or the previous value of the
%    variable.
%
% \subsection{Support for extending macros}
%
% \DescribeMacro{\addto}
%    The macro |\addto{|\meta{control sequence}|}{|\meta{\TeX\
%    code}|}| can be used to extend the definition of a macro. The
%    macro need not be defined. This macro can, for instance, be used
%    in adding instructions to a macro like |\extrasenglish|.
%
% \subsection{Macros common to a number of languages}
%
% \DescribeMacro{\allowhyphens}
%    In a couple of european languages compound words are used. This
%    means that when \TeX\ has to hyphenate such a compound word it
%    only does that at the `\texttt{-}' that is used in such words. To
%    allow hyphenation in the rest of such a compound word the macro
%    |\allowhyphens| can be used.
%
% \DescribeMacro{\set@low@box}
%    For some languages quotes need to be lowered to the baseline. For
%    this purpose the macro |\set@low@box| is available. It takes one
%    argument and puts that argument in an |\hbox|, at the
%    baseline. The result is available in |\box0| for further
%    processing.
%
% \DescribeMacro{\save@sf@q}
%    Sometimes it is necessary to preserve the |\spacefactor|.  For
%    this purpose the macro |\save@sf@q| is available. It takes one
%    argument, saves the current spacefactor, executes the argument
%    and restores the spacefactor.
%
% \DescribeMacro{\bbl@frenchspacing}
% \DescribeMacro{\bbl@nonfrenchspacing}
%    The commands |\bbl@frenchspacing| and |\bbl@nonfrenchspacing| can
%    be used to properly switch french spacing on and off.
%
% \section{Compatibility with \file{german.sty}}\label{l-h}
%
%    As has been discussed before, the file \file{german.sty} has been
%    one of the sources of inspiration for the \babel{}
%    system. Because of this I wanted to include \file{german.sty} in
%    the \babel{} system.  To be able to do that I had to allow for
%    one incompatibility: in the definition of the macro
%    |\selectlanguage| in \file{german.sty} the argument is used as the
%    {$\langle \it number \rangle$} for an |\ifcase|. So in this case
%    a call to |\selectlanguage| might look like
%    |\selectlanguage{\german}|.
%
%    In the definition of the macro |\selectlanguage| in
%    \file{babel.def} the argument is used as a part of other
%    macronames, so a call to |\selectlanguage| now looks like
%    |\selectlanguage{german}|.  Notice the absence of the escape
%    character.  As of version~3.1a of \babel{} both syntaxes are
%    allowed.
%
%    All other features of the original \file{german.sty} have been
%    copied into a new file, called \file{germanb.sty}\footnote{The
%    `b' is added to the name to distinguish the file from Partls'
%    file.}.
%
%    Although the \babel{} system was developed to be used with
%    \LaTeX, some of the features implemented in the language
%    definition files might be needed by plain \TeX\ users. Care has
%    been taken that all files in the system can be processed by plain
%    \TeX.
%
%
%\StopEventually{%
% \clearpage
% \let\filename\thisfilename
% \section{Conclusion}
%
%    A system of document options has been presented that enable the
%    user of \LaTeX\ to adapt the standard document classes of \LaTeX\
%    to the language he or she prefers to use. These options offer the
%    possibility to switch between languages in one document. The
%    basic interface consists of using ones option, which is the same
%    for \emph{all} standard document classes.
%
%    In some cases the language definition files provide macros that
%    can be of use to plain \TeX\ users as well as to \LaTeX\ users.
%    The \babel{} system has been implemented in such a way that it
%    can be used by both groups of users.
%
% \section{Acknowledgements}
%
%    I would like to thank all who volunteered as $\beta$-testers for
%    their time. I would like to mention Julio Sanchez who supplied
%    the option file for the Spanish language and Maurizio Codogno who
%    supplied the option file for the Italian language. Werenfried
%    Spit supplied the files for the Russian language. Michel Goossens
%    supplied contributions for most of the other languages.  Nico
%    Poppelier helped polishing the text of the documentation and
%    supplied parts of the macros for the Dutch language.  Paul
%    Wackers and Werenfried Spit helped finding and repairing bugs.
%
%    During the further development of the babel system I received
%    much help from Bernd Raichle, for which I am grateful.
%
%  \begin{thebibliography}{9}
%  \bibitem{DEK} Donald E. Knuth,
%    \emph{The \TeX book}, Addison-Wesley, 1986.
%  \bibitem{LLbook} Leslie Lamport,
%    \emph{\LaTeX, A document preparation System}, Addison-Wesley, 1986.
%   \bibitem{treebus} K.F. Treebus.
%   \emph{Tekstwijzer, een gids voor het grafisch verwerken van tekst.}
%   SDU Uitgeverij ('s-Gravenhage, 1988). A Dutch book on layout
%   design and typography.
%  \bibitem{HP} Hubert Partl,
%    \emph{German \TeX}, \emph{TUGboat} 9 (1988) \#1, p.~70--72.
%  \bibitem{LLth} Leslie Lamport,
%    in: \TeXhax\ Digest, Volume 89, \#13, 17 februari 1989.
%  \bibitem{BEP} Johannes Braams, Victor Eijkhout and Nico Poppelier,
%   \emph{The development of national \LaTeX\ styles},
%   \emph{TUGboat} 10 (1989) \#3, p.~401--406.
%  \bibitem{ilatex} Joachim Schrod,
%   \emph{International \LaTeX\ is ready to use},
%   \emph{TUGboat} 11 (1990) \#1, p.~87--90.
%  \end{thebibliography}
% }
%
% \section{Identification}
%
%    The file \file{babel.sty}\footnote{The file described in this
%    section is called \texttt{\filename}, has version
%    number~\fileversion\ and was last revised on~\filedate.} is meant
%    for \LaTeXe, therefore we make sure that the format file used is
%    the right one.
%
%    \begin{macrocode}
%<+package>\NeedsTeXFormat{LaTeX2e}
%    \end{macrocode}
%
%    The identification code for each file is something that was
%    introduced in \LaTeXe. When the command |\ProvidesFile| does not
%    exist, a dummy definition is provided.
% \changes{babel~3.4e}{1994/06/24}{Redid the identification code,
%    provided dummy definition of \cs{ProvidesFile} for plain \TeX}
%    \begin{macrocode}
%<*!package>
\ifx\ProvidesFile\undefined
  \def\ProvidesFile#1[#2 #3 #4]{\wlog{#4 #3 <#2>}}\fi
%</!package>
%    \end{macrocode}
%    Identify each file that is produced from this source file.
% \changes{babel~3.4c}{1995/04/28}{lhyphen.cfg has become
%    lthyphen.cfg}
% \changes{babel~3.5b}{1995/01/25}{lthyphen.cfg has become hyphen.cfg}
%    \begin{macrocode}
%<+package>\ProvidesPackage{babel}
%<+core>\ProvidesFile{babel.def}
%<+kernel&patterns>\ProvidesFile{hyphen.cfg}
%<+kernel&!patterns>\ProvidesFile{switch.def}
%<+driver&!user>\ProvidesFile{babel.drv}
%<+driver&user>\ProvidesFile{user.drv}
                [1995/07/11 v3.5e 
%<+package>     The Babel package]
%<+core>         Babel common definitions]
%<+kernel>      Babel language switching mechanism]
%<+driver>]
%    \end{macrocode}
%
% \section{The Package File}
%
%    In order to make use of the new features of \LaTeXe, a new file
%    is introdued to the \babel\ system, \file{babel.sty}. This file
%    is loaded by the |\usepackage| command and defines all the
%    language options known in the \babel system.
%
%    For all the languages supported we need to declare an option.
%
%    `American' is a version of `English' which can have its own
%    hyphenation patterns. The default english patterns are in fact
%    for  american english. We allow or the patterns to be loaded as
%    `english' `american' or `USenglish'.
% \changes{babel~3.5a}{1995/03/14}{Changed extension of language
%    definition files to \texttt{ldf}}
% \changes{babel~3.5d}{1995/07/02}{Load language definition files
%    \emph{after} the check for the hyphenation patterns}
%    \begin{macrocode}
%<*package>
\DeclareOption{american}{%
  \ifx\l@USenglish\undefined
  \else
    \let\l@american\l@USenglish
  \fi
  % \iffalse meta-comment
%
% Copyright 1989-1995 Johannes L. Braams and any individual authors
% listed elsewhere in this file.  All rights reserved.
% 
% For further copyright information any other copyright notices in this
% file.
% 
% This file is part of the Babel system release 3.5.
% --------------------------------------------------
%   This system is distributed in the hope that it will be useful,
%   but WITHOUT ANY WARRANTY; without even the implied warranty of
%   MERCHANTABILITY or FITNESS FOR A PARTICULAR PURPOSE.
% 
%   For error reports concerning UNCHANGED versions of this file no more
%   than one year old, see bugs.txt.
% 
%   Please do not request updates from me directly.  Primary
%   distribution is through the CTAN archives.
% 
% 
% IMPORTANT COPYRIGHT NOTICE:
% 
% You are NOT ALLOWED to distribute this file alone.
% 
% You are allowed to distribute this file under the condition that it is
% distributed together with all the files listed in manifest.txt.
% 
% If you receive only some of these files from someone, complain!
% 
% Permission is granted to copy this file to another file with a clearly
% different name and to customize the declarations in that copy to serve
% the needs of your installation, provided that you comply with
% the conditions in the file legal.txt from the LaTeX2e distribution.
% 
% However, NO PERMISSION is granted to produce or to distribute a
% modified version of this file under its original name.
%  
% You are NOT ALLOWED to change this file.
% 
% 
% \fi
% \CheckSum{194}
% \iffalse
%    Tell the \LaTeX\ system who we are and write an entry on the
%    transcript.
%<*dtx>
\ProvidesFile{english.dtx}
%</dtx>
%<code>\ProvidesFile{english.ldf}
        [1995/07/04 v3.3e English support from the babel system]
%
% Babel package for LaTeX version 2e
% Copyright (C) 1989 - 1995
%           by Johannes Braams, TeXniek
%
% Please report errors to: J.L. Braams
%                          JLBraams@cistron.nl
%
%    This file is part of the babel system, it provides the source
%    code for the English language definition file.
%<*filedriver>
\documentclass{ltxdoc}
\newcommand*\TeXhax{\TeX hax}
\newcommand*\babel{\textsf{babel}}
\newcommand*\langvar{$\langle \mathit lang \rangle$}
\newcommand*\note[1]{}
\newcommand*\Lopt[1]{\textsf{#1}}
\newcommand*\file[1]{\texttt{#1}}
\begin{document}
 \DocInput{english.dtx}
\end{document}
%</filedriver>
%\fi
% \GetFileInfo{english.dtx}
%
% \changes{english-2.0a}{1990/04/02}{Added checking of format}
% \changes{english-2.1}{1990/04/24}{Reflect changes in babel 2.1}
% \changes{english-2.1a}{1990/05/14}{Incorporated Nico's comments}
% \changes{english-2.1b}{1990/05/14}{merged \file{USenglish.sty} into
%    this file}
% \changes{english-2.1c}{1990/05/22}{fixed typo in definition for
%    american language found by Werenfried Spit (nspit@fys.ruu.nl)}
% \changes{english-2.1d}{1990/07/16}{Fixed some typos}
% \changes{english-3.0}{1991/04/23}{Modified for babel 3.0}
% \changes{english-3.0a}{1991/05/29}{Removed bug found by van der Meer}
% \changes{english-3.0c}{1991/07/15}{Renamed \file{babel.sty} in
%    \file{babel.com}}
% \changes{english-3.1}{1991/11/05}{Rewrote parts of the code to use
%    the new features of babel version 3.1}
% \changes{english-3.3}{1994/02/08}{Update or \LaTeXe}
% \changes{english-3.3c}{1994/06/26}{Removed the use of \cs{filedate}
%    and moved the identification after the loading of
%    \file{babel.def}}
%
%  \section{The English language}
%
%    The file \file{\filename}\footnote{The file described in this
%    section has version number \fileversion\ and was last revised on
%    \filedate.} defines all the language definition macros for the
%    English language as well as for the American version of this
%    language.
%
%    For this language currently no special definitions are needed or
%    available.
%
% \StopEventually{}
%
% \changes{english-3.0d}{1991/10/22}{Removed code to load
%    \file{latexhax.com}}
%
%    As this file needs to be read only once, we check whether it was
%    read before. If it was, the command |\captionsenglish| is already
%    defined, so we can stop processing. If this command is undefined
%    we proceed with the various definitions and first show the
%    current version of this file.
%
% \changes{english-3.0c}{1991/07/15}{Added reset of catcode of @
%    before \cs{endinput}.}
% \changes{english-3.0d}{1991/10/22}{removed use of \cs{@ifundefined}}
% \changes{english-3.1a}{1991/11/11}{Moved code to the beginning of
%    the file and added \cs{selectlanguage} call}
%    \begin{macrocode}
%<*code>
\ifx\undefined\captionsenglish
\else
  \selectlanguage{english}
  \expandafter\endinput
\fi
%    \end{macrocode}
%
% \begin{macro}{\atcatcode}
%    This file, \file{english.ldf}, may have been read while \TeX\ is
%    in the middle of processing a document, so we have to make sure
%    the category code of \texttt{@} is `letter' while this file is
%    being read. We save the category code of the @-sign in
%    |\atcatcode| and make it `letter'. Later the category code can be
%    restored to whatever it was before.
% \changes{english-3.0b}{1991/06/06}{Made test of catcode of @ more
%    robust}
% \changes{english-3.0c}{1991/07/15}{Modified handling of catcode of @
%    again.}
% \changes{english-3.0d}{1991/10/22}{Removed use of \cs{makeatletter}
%    and hence the need to load \file{latexhax.com}}
%    \begin{macrocode}
\chardef\atcatcode=\catcode`\@
\catcode`\@=11\relax
%    \end{macrocode}
% \end{macro}
%
%    Now we determine whether the common macros from the file
%    \file{babel.def} need to be read. We can be in one of two
%    situations: either another language option has been read earlier
%    on, in which case that other option has already read
%    \file{babel.def}, or \texttt{english} is the first language
%    option to be processed. In that case we need to read
%    \file{babel.def} right here before we continue.
%
% \changes{english-3.0}{1991/04/23}{New check before loading
%    \file{babel.com}}
% \changes{english-3.1c}{1992/02/15}{Added \cs{relax} after the
%    argument of \cs{input}}
%    \begin{macrocode}
\ifx\undefined\babel@core@loaded\input babel.def\relax\fi
%    \end{macrocode}
%
% \changes{english-3.0a}{1991/05/29}{Add a check for existence
%    \cs{originalTeX}}
%
%    Another check that has to be made, is if another language
%    definition file has been read already. In that case its
%    definitions have been activated. This might interfere with
%    definitions this file tries to make. Therefore we make sure that
%    we cancel any special definitions. This can be done by checking
%    the existence of the macro |\originalTeX|. If it exists we simply
%    execute it, otherwise it is |\let| to |\empty|.
% \changes{english-3.0c}{1991/07/15}{Added
%    \cs{let}\cs{originalTeX}\cs{relax} to test for existence}
% \changes{english-3.1b}{1992/01/26}{\cs{originalTeX} should be
%    expandable, \cs{let} it to \cs{empty}}
%    \begin{macrocode}
\ifx\undefined\originalTeX \let\originalTeX\empty\fi
\originalTeX
%    \end{macrocode}
%
%    When this file is read as an option, i.e. by the |\usepackage|
%    command, \texttt{english} could be an `unknown' language in which
%    case we have to make it known.  So we check for the existence of
%    |\l@english| to see whether we have to do something here.
%
% \changes{english-3.0}{1991/04/23}{Now use \cs{adddialect} if
%    language undefined}
% \changes{english-3.0d}{1991/10/22}{removed use of \cs{@ifundefined}}
% \changes{english-3.3c}{1994/06/26}{Now use \cs{@nopatterns} to
%    produce the warning}
%    \begin{macrocode}
\ifx\undefined\l@english
  \ifx\undefined\l@UKenglish
    \@nopatterns{English}
    \adddialect\l@english0
  \else
    \let\l@english\l@UKenglish
  \fi
\fi
%    \end{macrocode}
%    For the American version of these definitions we just add a
%    ``dialect''. Also, the macros |\captionsamerican| and
%    |\extrasamerican| are |\let| to their English counterparts when
%    these parts are defined.
% \changes{english-3.0}{1990/04/23}{Now use \cs{adddialect} for
%    american}
% \changes{english-3.0b}{1991/06/06}{Removed \cs{global} definitions}
% \changes{english-v3.3d}{1995/02/01}{Only define american as a
%    dialect when no separate patterns have been loaded}
%    \begin{macrocode}
\ifx\l@american\undefined
  \adddialect\l@american\l@english
\fi
%    \end{macrocode}
%
%    The next step consists of defining commands to switch to (and
%    from) the English language.
%
% \begin{macro}{\captionsenglish}
%    The macro |\captionsenglish| defines all strings used
%    in the four standard document classes provided with \LaTeX.
% \changes{english-3.0b}{1991/06/06}{Removed \cs{global} definitions}
% \changes{english-3.0b}{1991/06/06}{\cs{pagename} should be
%    \cs{headpagename}}
% \changes{english-3.1a}{1991/11/11}{added \cs{seename} and
%    \cs{alsoname}}
% \changes{english-3.1b}{1992/01/26}{added \cs{prefacename}}
% \changes{english-3.2}{1993/07/15}{\cs{headpagename} should be
%    \cs{pagename}}
% \changes{english-3.3e}{1995/07/04}{Added \cs{proofname} for
%    AMS-\LaTeX}
%    \begin{macrocode}
\addto\captionsenglish{%
  \def\prefacename{Preface}%
  \def\refname{References}%
  \def\abstractname{Abstract}%
  \def\bibname{Bibliography}%
  \def\chaptername{Chapter}%
  \def\appendixname{Appendix}%
  \def\contentsname{Contents}%
  \def\listfigurename{List of Figures}%
  \def\listtablename{List of Tables}%
  \def\indexname{Index}%
  \def\figurename{Figure}%
  \def\tablename{Table}%
  \def\partname{Part}%
  \def\enclname{encl}%
  \def\ccname{cc}%
  \def\headtoname{To}%
  \def\pagename{Page}%
  \def\seename{see}%
  \def\alsoname{see also}%
  \def\proofname{Proof}%
  }
%    \end{macrocode}
% \end{macro}
%
% \begin{macro}{\captionsamerican}
%    The `captions' are the same for both versions of the language, so
%    we can |\let| the macro |\captionsamerican| be equal to
%    |\captionsenglish|.
%    \begin{macrocode}
\let\captionsamerican\captionsenglish
%    \end{macrocode}
% \end{macro}
%
% \begin{macro}{\dateenglish}
%    The macro |\dateenglish| redefines the command |\today| to
%    produce English dates.
% \changes{english-3.0b}{1991/06/06}{Removed \cs{global} definitions}
%    \begin{macrocode}
\def\dateenglish{%
\def\today{\ifcase\day\or
  1st\or 2nd\or 3rd\or 4th\or 5th\or
  6th\or 7th\or 8th\or 9th\or 10th\or
  11th\or 12th\or 13th\or 14th\or 15th\or
  16th\or 17th\or 18th\or 19th\or 20th\or
  21st\or 22nd\or 23rd\or 24th\or 25th\or
  26th\or 27th\or 28th\or 29th\or 30th\or
  31st\fi~\ifcase\month\or
  January\or February\or March\or April\or May\or June\or
  July\or August\or September\or October\or November\or December\fi
  \space \number\year}}
%    \end{macrocode}
% \end{macro}
%
% \begin{macro}{\dateamerican}
%    The macro |\dateamerican| redefines the command |\today| to
%    produce American dates.
% \changes{english-3.0b}{1991/06/06}{Removed \cs{global} definitions}
%    \begin{macrocode}
\def\dateamerican{%
\def\today{\ifcase\month\or
  January\or February\or March\or April\or May\or June\or
  July\or August\or September\or October\or November\or December\fi
  \space\number\day, \number\year}}
%    \end{macrocode}
% \end{macro}
%
% \begin{macro}{\extrasenglish}
% \begin{macro}{\noextrasenglish}
%    The macro |\extrasenglish| will perform all the extra definitions
%    needed for the English language. The macro |\extrasenglish| is
%    used to cancel the actions of |\extrasenglish|.  For the moment
%    these macros are empty but they are defined for compatibility
%    with the other language definition files.
%
%    \begin{macrocode}
\addto\extrasenglish{}
\addto\noextrasenglish{}
%    \end{macrocode}
% \end{macro}
% \end{macro}
%
% \begin{macro}{\extrasamerican}
% \begin{macro}{\noextrasamerican}
%    Also for the ``american'' variant no extra definitions are needed
%    at the moment.
%    \begin{macrocode}
\let\extrasamerican\extrasenglish
\let\noextrasamerican\noextrasenglish
%    \end{macrocode}
% \end{macro}
% \end{macro}
%
%    It is possible that a site might need to add some extra code to
%    the babel macros. To enable this we load a local configuration
%    file, \file{english.cfg} if it is found on \TeX' search path.
% \changes{english-3.3e}{1995/07/02}{Added loading of configuration
%    file}
%    \begin{macrocode}
\loadlocalcfg{english}
%    \end{macrocode}
%
%    Our last action is to make a note that the commands we have just
%    defined, will be executed by calling the macro |\selectlanguage|
%    at the beginning of the document.
%    \begin{macrocode}
\main@language{english}
%    \end{macrocode}
%    Finally, the category code of \texttt{@} is reset to its original
%    value. The macrospace used by |\atcatcode| is freed.
% \changes{english-3.0c}{1991/07/15}{Modified handling of catcode of
%    @-sign.}
%    \begin{macrocode}
\catcode`\@=\atcatcode \let\atcatcode\relax
%</code>
%    \end{macrocode}
%
% \Finale
%%
%% \CharacterTable
%%  {Upper-case    \A\B\C\D\E\F\G\H\I\J\K\L\M\N\O\P\Q\R\S\T\U\V\W\X\Y\Z
%%   Lower-case    \a\b\c\d\e\f\g\h\i\j\k\l\m\n\o\p\q\r\s\t\u\v\w\x\y\z
%%   Digits        \0\1\2\3\4\5\6\7\8\9
%%   Exclamation   \!     Double quote  \"     Hash (number) \#
%%   Dollar        \$     Percent       \%     Ampersand     \&
%%   Acute accent  \'     Left paren    \(     Right paren   \)
%%   Asterisk      \*     Plus          \+     Comma         \,
%%   Minus         \-     Point         \.     Solidus       \/
%%   Colon         \:     Semicolon     \;     Less than     \<
%%   Equals        \=     Greater than  \>     Question mark \?
%%   Commercial at \@     Left bracket  \[     Backslash     \\
%%   Right bracket \]     Circumflex    \^     Underscore    \_
%%   Grave accent  \`     Left brace    \{     Vertical bar  \|
%%   Right brace   \}     Tilde         \~}
%%
\endinput
%
  \main@language{american}}
%    \end{macrocode}
%    Austrian is really a dialect of German.
%    \begin{macrocode}
\DeclareOption{austrian}{%
  % \iffalse meta-comment
%
% Copyright 1989-1995 Johannes L. Braams and any individual authors
% listed elsewhere in this file.  All rights reserved.
% 
% For further copyright information any other copyright notices in this
% file.
% 
% This file is part of the Babel system release 3.5.
% --------------------------------------------------
%   This system is distributed in the hope that it will be useful,
%   but WITHOUT ANY WARRANTY; without even the implied warranty of
%   MERCHANTABILITY or FITNESS FOR A PARTICULAR PURPOSE.
% 
%   For error reports concerning UNCHANGED versions of this file no more
%   than one year old, see bugs.txt.
% 
%   Please do not request updates from me directly.  Primary
%   distribution is through the CTAN archives.
% 
% 
% IMPORTANT COPYRIGHT NOTICE:
% 
% You are NOT ALLOWED to distribute this file alone.
% 
% You are allowed to distribute this file under the condition that it is
% distributed together with all the files listed in manifest.txt.
% 
% If you receive only some of these files from someone, complain!
% 
% Permission is granted to copy this file to another file with a clearly
% different name and to customize the declarations in that copy to serve
% the needs of your installation, provided that you comply with
% the conditions in the file legal.txt from the LaTeX2e distribution.
% 
% However, NO PERMISSION is granted to produce or to distribute a
% modified version of this file under its original name.
%  
% You are NOT ALLOWED to change this file.
% 
% 
% \fi
% \CheckSum{341}
%
% \iffalse
%    Tell the \LaTeX\ system who we are and write an entry on the
%    transcript.
%<*dtx>
\ProvidesFile{germanb.dtx}
%</dtx>
%<code>\ProvidesFile{germanb.ldf}
        [1995/07/04 v2.6b German support from the babel system]
%
% Babel package for LaTeX version 2e
% Copyright (C) 1989 - 1995
%           by Johannes Braams, TeXniek
%
% Germanb Language Definition File
% Copyright (C) 1989 - 1995
%           by Bernd Raichle <raichle@azu.Informatik.Uni-Stuttgart.de>
%              Johannes Braams, TeXniek
% This file is based on german.tex version 2.5b,
%                       by Bernd Raichle, Hubert Partl et.al.
%
% Please report errors to: J.L. Braams
%                          JSLBraams@cistron.nl
%
%<*filedriver>
\documentclass{ltxdoc}
\font\manual=logo10 % font used for the METAFONT logo, etc.
\newcommand*\MF{{\manual META}\-{\manual FONT}}
\newcommand*\TeXhax{\TeX hax}
\newcommand*\babel{\textsf{babel}}
\newcommand*\langvar{$\langle \it lang \rangle$}
\newcommand*\note[1]{}
\newcommand*\Lopt[1]{\textsf{#1}}
\newcommand*\file[1]{\texttt{#1}}
\begin{document}
 \DocInput{germanb.dtx}
\end{document}
%</filedriver>
%\fi
% \GetFileInfo{germanb.dtx}
%
% \changes{germanb-1.0a}{1990/05/14}{Incorporated Nico's comments}
% \changes{germanb-1.0b}{1990/05/22}{fixed typo in definition for
%    austrian language found by Werenfried Spit
%    \texttt{nspit@fys.ruu.nl}}
% \changes{germanb-1.0c}{1990/07/16}{Fixed some typos}
% \changes{germanb-1.1}{1990/07/30}{When using PostScript fonts with
%    the Adobe fontencoding, the dieresis-accent is located elsewhere,
%    modified code}
% \changes{germanb-1.1a}{1990/08/27}{Modified the documentation
%    somewhat}
% \changes{germanb-2.0}{1991/04/23}{Modified for babel 3.0}
% \changes{germanb-2.0a}{1991/05/25}{Removed some problems in change
%    log}
% \changes{germanb-2.1}{1991/05/29}{Removed bug found by van der Meer}
% \changes{germanb-2.2}{1991/06/11}{Removed global assignments,
%    brought uptodate with \file{german.tex} v2.3d}
% \changes{germanb-2.2a}{1991/07/15}{Renamed \file{babel.sty} in
%    \file{babel.com}}
% \changes{germanb-2.3}{1991/11/05}{Rewritten parts of the code to use
%    the new features of babel version 3.1}
% \changes{germanb-2.3e}{1991/11/10}{Brought up-to-date with
%    \file{german.tex} v2.3e (plus some bug fixes) [br]}
% \changes{germanb-2.5}{1994/02/08}{Update or \LaTeXe}
% \changes{germanb-2.5c}{1994/06/26}{Removed the use of \cs{filedate}
%    and moved the identification after the loading of
%    \file{babel.def}}
% \changes{germanb-2.6a}{1995/02/15}{Moved the identification to the
%    top of the file}
% \changes{germanb-2.6a}{1995/02/15}{Rewrote the code that handles the
%    active double quote character}
%
%  \section{The German language}
%
%    The file \file{\filename}\footnote{The file described in this
%    section has version number \fileversion\ and was last revised on
%    \filedate.}  defines all the language definition macros for the
%    German language as well as for the Austrian dialect of this
%    language\footnote{This file is a re-implementation of Hubert
%    Partl's \file{german.sty} version 2.5b, see~\cite{HP}.}.
%
%    For this language the character |"| is made active. In
%    table~\ref{tab:german-quote} an overview is given of its
%    purpose. One of the reasons for this is that in the German
%    language some character combinations change when a word is broken
%    between the combination. Also the vertical placement of the
%    umlaut can be controlled this way.
%    \begin{table}[htb]
%     \begin{center}
%     \begin{tabular}{lp{8cm}}
%      |"a| & |\"a|, also implemented for the other
%                  lowercase and uppercase vowels.                 \\
%      |"s| & to produce the German \ss{} (like |\ss{}|).          \\
%      |"z| & to produce the German \ss{} (like |\ss{}|).          \\
%      |"ck|& for |ck| to be hyphenated as |k-k|.                  \\
%      |"ff|& for |ff| to be hyphenated as |ff-f|,
%                  this is also implemented for l, m, n, p, r and t\\
%      |"S| & for |SS| to be |\uppercase{"s}|.                     \\
%      |"Z| & for |SZ| to be |\uppercase{"z}|.                     \\
%      \verb="|= & disable ligature at this position.              \\
%      |"-| & an explicit hyphen sign, allowing hyphenation
%             in the rest of the word.                             \\
%      |""| & like |"-|, but producing no hyphen sign
%             (for compund words with hyphen, e.g.\ |x-""y|).      \\
%      |"~| & for a compound word mark without a breakpoint.       \\
%      |"=| & for a compound word mark with a breakpoint, allowing
%             hyphenation in the composing words.                  \\
%      |"`| & for German left double quotes (looks like ,,).       \\
%      |"'| & for German right double quotes.                      \\
%      |"<| & for French left double quotes (similar to $<<$).     \\
%      |">| & for French right double quotes (similar to $>>$).    \\
%     \end{tabular}
%     \caption{The extra definitions made
%              by \file{german.ldf}}\label{tab:german-quote}
%     \end{center}
%    \end{table}
%    The quotes in table~\ref{tab:german-quote} can also be typeset by
%    using the commands in table~\ref{tab:more-quote}.
%    \begin{table}[htb]
%     \begin{center}
%     \begin{tabular}{lp{8cm}}
%      |\glqq| & for German left double quotes (looks like ,,).   \\
%      |\grqq| & for German right double quotes (looks like ``).  \\
%      |\glq|  & for German left single quotes (looks like ,).    \\
%      |\grq|  & for German right single quotes (looks like `).   \\
%      |\flqq| & for French left double quotes (similar to $<<$). \\
%      |\frqq| & for French right double quotes (similar to $>>$).\\
%      |\flq|  & for (French) left single quotes (similar to $<$).  \\
%      |\frq|  & for (French) right single quotes (similar to $>$). \\
%      |\dq|   & the original quotes character (|"|).        \\
%     \end{tabular}
%     \caption{More commands which produce quotes, defined
%              by \file{german.ldf}}\label{tab:more-quote}
%     \end{center}
%    \end{table}
%
% \StopEventually{}
%
% \changes{germanb-2.2d}{1991/10/27}{Removed code to load
%    \file{latexhax.com}}
%
%    As this file, \file{germanb.ldf}, needs to be read only once, we
%    check whether it was read before.  If it was, the command
%    |\captionsgerman| is already defined, so we can stop
%    processing. If this command is undefined we proceed with the
%    various definitions and first show the current version of this
%    file.
%
% \changes{germanb-2.2a}{1991/07/15}{Added reset of catcode of @ before
%                                  \cs{endinput}.}
% \changes{germanb-2.2d}{1991/10/27}{Removed use of \cs{@ifundefined}}
% \changes{germanb-2.3e}{1991/11/10}{Moved code to the beginning of
%    the file and added \cs{selectlanguage} call}
%    \begin{macrocode}
%<*code>
\ifx\undefined\captionsgerman
\else
  \selectlanguage{german}
  \expandafter\endinput
\fi
%    \end{macrocode}
%
%  \begin{macro}{\atcatcode}
%    This file, \file{germanb.ldf}, may have been read while \TeX\ is
%    in the middle of processing a document, so we have to make sure
%    the category code of \texttt{@} is `letter' while this file is
%    being read. We save the category code of the @-sign in
%    |\atcatcode| and make it `letter'. Later the category code can be
%    restored to whatever it was before.
%
% \changes{germanb-2.2}{1991/06/11}{Made test of catcode of @ more
%    robust}
% \changes{germanb-2.2a}{1991/07/15}{Modified handling of catcode of @
%    again.}
% \changes{germanb-2.2d}{1991/10/27}{Removed use of \cs{makeatletter}
%    and hence the need to load \file{latexhax.com}}
%    \begin{macrocode}
\chardef\atcatcode=\catcode`\@
\catcode`\@=11\relax
%    \end{macrocode}
%  \end{macro}
%
%
%    Now we determine whether the common macros from the file
%    \file{babel.def} need to be read. We can be in one of two
%    situations: either another language option has been read earlier
%    on, in which case that other option has already read
%    \file{babel.def}, or \file{germanb} is the first language option
%    to be processed. In that case we need to read \file{babel.def}
%    right here before we continue.
%
% \changes{germanb-2.0}{1991/04/23}{New check before loading
%    \file{babel.com}}
% \changes{germanb-2.3g}{1992/02/15}{Added \cs{relax} after the
%    argument of \cs{input}}
%    \begin{macrocode}
\ifx\undefined\babel@core@loaded\input babel.def\relax\fi
%    \end{macrocode}
%
% \changes{germanb-2.1}{1991/05/29}{Add a check for existence of
%    \cs{originalTeX}}
%
%    Another check that has to be made, is if another language
%    definition file has been read already. In that case its
%    definitions have been activated. This might interfere with
%    definitions this file tries to make. Therefore we make sure that
%    we cancel any special definitions. This can be done by checking
%    the existence of the macro |\originalTeX|. If it exists we simply
%    execute it, otherwise it is |\let| to |\empty|.
% \changes{germanb-2.2a}{1991/07/15}{Added
%    \cs{let}\cs{originalTeX}\cs{relax} to test for existence}
% \changes{germanb-2.3f}{1992/01/25}{Set \cs{originalTeX} to \cs{bsl
%    empty}, because it should be expandable.}
%    \begin{macrocode}
\ifx\undefined\originalTeX \let\originalTeX\empty\fi
\originalTeX
%    \end{macrocode}
%
%    When this file is read as an option, i.e., by the |\usaepackage|
%    command, \texttt{german} will be an `unknown' language, so we
%    have to make it known.  So we check for the existence of
%    |\l@german| to see whether we have to do something here.
%
% \changes{germanb-2.0}{1991/04/23}{Now use \cs{adddialect} if
%    language undefined}
% \changes{germanb-2.2d}{1991/10/27}{Removed use of \cs{@ifundefined}}
% \changes{germanb-2.3e}{1991/11/10}{Added warning, if no german
%    patterns loaded}
% \changes{germanb-2.5c}{1994/06/26}{Now use \cs{@nopatterns} to
%    produce the warning}
%    \begin{macrocode}
\ifx\undefined\l@german
  \@nopatterns{German}
  \adddialect\l@german0
\fi
%    \end{macrocode}
%
%    For the Austrian version of these definitions we just add another
%    language. Also, the macros |\captionsaustrian| and
%    |\extrasaustrian| are |\let| to their German counterparts if
%    these parts are defined.
% \changes{germanb-2.0}{1991/04/23}{Now use \cs{adddialect} for
%    austrian}
%    \begin{macrocode}
\adddialect\l@austrian\l@german
%    \end{macrocode}
%
%
%    The next step consists of defining commands to switch to (and
%    from) the German language.
%
%  \begin{macro}{\captionsgerman}
%    The macro |\captionsgerman| defines all strings used in the four
%    standard document classes provided with \LaTeX.
%
% \changes{germanb-2.2}{1991/06/06}{Removed \cs{global} definitions}
% \changes{germanb-2.2}{1991/06/06}{\cs{pagename} should be
%    \cs{headpagename}}
% \changes{germanb-2.3e}{1991/11/10}{Added \cs{prefacename},
%    \cs{seename} and \cs{alsoname}}
% \changes{germanb-2.4}{1993/07/15}{\cs{headpagename} should be
%    \cs{pagename}}
% \changes{german-2.6b}{1995/07/04}{Added \cs{proofname} for
%    AMS-\LaTeX}
%    \begin{macrocode}
\addto\captionsgerman{%
  \def\prefacename{Vorwort}%
  \def\refname{Literatur}%
  \def\abstractname{Zusammenfassung}%
  \def\bibname{Literaturverzeichnis}%
  \def\chaptername{Kapitel}%
  \def\appendixname{Anhang}%
  \def\contentsname{Inhaltsverzeichnis}%    % oder nur: Inhalt
  \def\listfigurename{Abbildungsverzeichnis}%
  \def\listtablename{Tabellenverzeichnis}%
  \def\indexname{Index}%
  \def\figurename{Abbildung}%
  \def\tablename{Tabelle}%                  % oder: Tafel
  \def\partname{Teil}%
  \def\enclname{Anlage(n)}%                 % oder: Beilage(n)
  \def\ccname{Verteiler}%                   % oder: Kopien an
  \def\headtoname{An}%
  \def\pagename{Seite}%
  \def\seename{siehe}%
  \def\alsoname{siehe auch}%
  \def\proofname{Beweis}%
  }
%    \end{macrocode}
%  \end{macro}
%
% \begin{macro}{\captionsgerman}
%    The `captions' are the same for both version of the language, so
%    we can |\let| the macro |\captionsaustrian| be equal to
%    |\captionsgerman|.
%    \begin{macrocode}
\let\captionsaustrian\captionsgerman
%    \end{macrocode}
%  \end{macro}
%
%  \begin{macro}{\dategerman}
%    The macro |\dategerman| redefines the command
%    |\today| to produce German dates.
% \changes{germanb-2.3e}{1991/11/10}{Added \cs{month@german}}
%    \begin{macrocode}
\def\month@german{\ifcase\month\or
  Januar\or Februar\or M\"arz\or April\or Mai\or Juni\or
  Juli\or August\or September\or Oktober\or November\or Dezember\fi}
\def\dategerman{\def\today{\number\day.~\month@german
  \space\number\year}}
%    \end{macrocode}
%  \end{macro}
%
%  \begin{macro}{\dateaustrian}
%    The macro |\dateaustrian| redefines the command
%    |\today| to produce Austrian version of the German dates.
%    \begin{macrocode}
\def\dateaustrian{\def\today{\number\day.~\ifnum1=\month
  J\"anner\else \month@german\fi \space\number\year}}
%    \end{macrocode}
%  \end{macro}
%
%
%  \begin{macro}{\extrasgerman}
% \changes{germanb-2.0b}{1991/05/29}{added some comment chars to
%    prevent white space}
% \changes{germanb-2.2}{1991/06/11}{Save all redefined macros}
%  \begin{macro}{\noextrasgerman}
% \changes{germanb-1.1}{1990/07/30}{Added \cs{dieresis}}
% \changes{germanb-2.0b}{1991/05/29}{added some comment chars to
%    prevent white space}
% \changes{germanb-2.2}{1991/06/11}{Try to restore everything to its
%    former state}
%
%    The macro |\extrasgerman| will perform all the extra definitions
%    needed for the German language. The macro |\noextrasgerman|
%    is used to cancel the actions of |\extrasgerman|.
%
%    For German (as well as for Dutch) the \texttt{"} character is
%    made active. This is done once, later on its definition may vary.
%    \begin{macrocode}
\initiate@active@char{"}
\addto\extrasgerman{\languageshorthands{german}}
\addto\extrasgerman{\bbl@activate{"}}
%\addto\noextrasgerman{\bbl@deactivate{"}}
%    \end{macrocode}
%
% \changes{germanb-2.6a}{1995/02/15}{All the code to handle the active
%    double quote has been moved to \file{babel.def}}
%
%    In order for \TeX\ to be able to hyphenate German words which
%    contain `\ss' (in the \texttt{OT1} position |^^Y|) we have to
%    give the character a nonzero |\lccode| (see Appendix H, the \TeX
%    book).
%    \begin{macrocode}
\addto\extrasgerman{%
  \babel@savevariable{\lccode`\^^Y}%
  \lccode`\^^Y`\^^Y}
%    \end{macrocode}
% \changes{germanb-2.6a}{1995/02/15}{Removeed \cs{3} as it is no
%    longer in \file{german.ldf}}
%
%    The umlaut accent macro |\"| is changed to lower the umlaut dots.
%    The redefinition is done with the help of |\umlautlow|.
%    \begin{macrocode}
\addto\extrasgerman{\babel@save\"\umlautlow}
\addto\noextrasgerman{\umlauthigh}
%    \end{macrocode}
%    The german hyphenation patterns can be used with |\lefthyphenmin|
%    and |\righthyphenmin| set to~2.
% \changes{germanb-2.6a}{1995/05/13}{use \cs{germanhyphenmins} to store
%    the correct values}
%    \begin{macrocode}
\def\germanhyphenmins{\tw@\tw@}
%    \end{macrocode}
%  \end{macro}
%  \end{macro}
%
%  \begin{macro}{\extrasaustrian}
%  \begin{macro}{\noextrasaustrian}
%    For both versions of the language the same special macros are
%    used, so we can |\let| the austrian macros be equal to their
%    german counterparts.
%    \begin{macrocode}
\let\extrasaustrian\extrasgerman
\let\noextrasaustrian\noextrasgerman
%    \end{macrocode}
%  \end{macro}
%  \end{macro}
%
% \changes{germanb-2.6a}{1995/02/15}{\cs{umlautlow} and
%    \cs{umlauthigh} moved to \file{glyphs.dtx}, as well as
%    \cs{newumlaut} (now \cs{lower@umlaut}}
%
%    The code above is necessary because we need an extra active
%    character. This character is then used as indicated in
%    table~\ref{tab:german-quote}.
%
%    To be able to define the function of |"|, we first define a
%    couple of `support' macros.
%
% \changes{germanb-2.3e}{1991/11/10}{Added \cs{save@sf@q} macro and
%    rewrote all quote macros to use it}
% \changes{germanb-2.3h}{1991/02/16}{moved definition of
%    \cs{allowhyphens}, \cs{set@low@box} and \cs{save@sf@q} to
%    \file{babel.com}}
% \changes{german-2.6a}{1995/02/15}{Moved all quotation characters to
%    \file{glyphs.dtx}}
%
%  \begin{macro}{\dq}
%    We save the original double quote character in |\dq| to keep
%    it available, the math accent |\"| can now be typed as |"|.
%    \begin{macrocode}
\begingroup \catcode`\"12
\def\x{\endgroup
  \def\@SS{\mathchar"7019 }
  \def\dq{"}}
\x
%    \end{macrocode}
%  \end{macro}
%
%  \begin{macro}{\german@dq@disc}
%    For the discretionary macros we use this macro:
%    \begin{macrocode}
\def\german@dq@disc#1#2{%
  \penalty\@M\discretionary{#2-}{}{#1}\allowhyphens}
%    \end{macrocode}
%  \end{macro}
%
% \changes{german-2.6a}{1995/02/15}{Use \cs{ddot} instead of
%    \cs{@MATHUMLAUT}}
%
%    Now we can define the doublequote macros: the umlauts,
%    \begin{macrocode}
\declare@shorthand{german}{"a}{\textormath{\"{a}}{\ddot a}}
\declare@shorthand{german}{"o}{\textormath{\"{o}}{\ddot o}}
\declare@shorthand{german}{"u}{\textormath{\"{u}}{\ddot u}}
\declare@shorthand{german}{"A}{\textormath{\"{A}}{\ddot A}}
\declare@shorthand{german}{"O}{\textormath{\"{O}}{\ddot O}}
\declare@shorthand{german}{"U}{\textormath{\"{U}}{\ddot U}}
%    \end{macrocode}
%    tremas,
%    \begin{macrocode}
\declare@shorthand{german}{"e}{\textormath{\"{e}}{\ddot e}}
\declare@shorthand{german}{"E}{\textormath{\"{E}}{\ddot E}}
\declare@shorthand{german}{"i}{\textormath{\"{\i}}{\ddot\imath}}
\declare@shorthand{german}{"I}{\textormath{\"{I}}{\ddot I}}
%    \end{macrocode}
%    german es-zet (sharp s),
%    \begin{macrocode}
\declare@shorthand{german}{"s}{\textormath{\ss{}}{\@SS{}}}
\declare@shorthand{german}{"S}{SS}
\declare@shorthand{german}{"z}{\textormath{\ss{}}{\@SS{}}}
\declare@shorthand{german}{"Z}{SZ}
%    \end{macrocode}
%    german and french quotes,
%    \begin{macrocode}
\declare@shorthand{german}{"`}{%
  \textormath{\quotedblbase{}}{\mbox{\quotedblbase}}}
\declare@shorthand{german}{"'}{%
  \textormath{\textquotedblleft{}}{\mbox{\textquotedblleft}}}
\declare@shorthand{german}{"<}{%
  \textormath{\guillemotleft{}}{\mbox{\guillemotleft}}}
\declare@shorthand{german}{">}{%
  \textormath{\guillemotright{}}{\mbox{\guillemotright}}}
%    \end{macrocode}
%    discretionary commands
%    \begin{macrocode}
\declare@shorthand{german}{"c}{\textormath{\german@dq@disc ck}{c}}
\declare@shorthand{german}{"C}{\textormath{\german@dq@disc CK}{C}}
\declare@shorthand{german}{"f}{\textormath{\german@dq@disc f{ff}}{f}}
\declare@shorthand{german}{"F}{\textormath{\german@dq@disc F{FF}}{F}}
\declare@shorthand{german}{"l}{\textormath{\german@dq@disc l{ll}}{l}}
\declare@shorthand{german}{"L}{\textormath{\german@dq@disc L{LL}}{L}}
\declare@shorthand{german}{"m}{\textormath{\german@dq@disc m{mm}}{m}}
\declare@shorthand{german}{"M}{\textormath{\german@dq@disc M{MM}}{M}}
\declare@shorthand{german}{"n}{\textormath{\german@dq@disc n{nn}}{n}}
\declare@shorthand{german}{"N}{\textormath{\german@dq@disc N{NN}}{N}}
\declare@shorthand{german}{"p}{\textormath{\german@dq@disc p{pp}}{p}}
\declare@shorthand{german}{"P}{\textormath{\german@dq@disc P{PP}}{P}}
\declare@shorthand{german}{"r}{\textormath{\german@dq@disc r{rr}}{r}}
\declare@shorthand{german}{"R}{\textormath{\german@dq@disc R{RR}}{R}}
\declare@shorthand{german}{"t}{\textormath{\german@dq@disc t{tt}}{t}}
\declare@shorthand{german}{"T}{\textormath{\german@dq@disc T{TT}}{T}}
%    \end{macrocode}
%    and some additional commands:
%    \begin{macrocode}
\declare@shorthand{german}{"-}{\penalty\@M\-\allowhyphens}
\declare@shorthand{german}{"|}{%
  \textormath{\penalty\@M\discretionary{-}{}{\kern.03em}%
              \allowhyphens}{}}
\declare@shorthand{german}{""}{\hskip\z@skip}
\declare@shorthand{german}{"~}{\textormath{\leavevmode\hbox{-}}{-}}
\declare@shorthand{german}{"=}{\penalty\@M-\hskip\z@skip}
%    \end{macrocode}
%
%  \begin{macro}{\mdqon}
%  \begin{macro}{\mdqoff}
%  \begin{macro}{\ck}
%    All that's left to do now is to  define a couple of commands
%    for reasons of compatibility with \file{german.sty}.
%    \begin{macrocode}
\def\mdqon{\bbl@activate{"}}
\def\mdqoff{\bbl@deactivate{"}}
\def\ck{\allowhyphens\discretionary{k-}{k}{ck}\allowhyphens}
%    \end{macrocode}
%  \end{macro}
%  \end{macro}
%  \end{macro}
%
%    It is possible that a site might need to add some extra code to
%    the babel macros. To enable this we load a local configuration
%    file, \file{germanb.cfg} if it is found on \TeX' search path.
% \changes{german-2.6b}{1995/07/02}{Added loading of configuration
%    file}
%    \begin{macrocode}
\loadlocalcfg{germanb}
%    \end{macrocode}
%
%    Our last action is to make a note that the commands we have just
%    defined, will be executed by calling the macro |\selectlanguage|
%    at the beginning of the document.
%    \begin{macrocode}
\main@language{german}
%    \end{macrocode}
%    Finally, the category code of \texttt{@} is reset to its original
%    value.
% \changes{germanb-2.2a}{1991/07/15}{Modified handling of catcode of
%    @-sign.}
%    \begin{macrocode}
\catcode`\@=\atcatcode
%</code>
%    \end{macrocode}
%
% \Finale
%%
%% \CharacterTable
%%  {Upper-case    \A\B\C\D\E\F\G\H\I\J\K\L\M\N\O\P\Q\R\S\T\U\V\W\X\Y\Z
%%   Lower-case    \a\b\c\d\e\f\g\h\i\j\k\l\m\n\o\p\q\r\s\t\u\v\w\x\y\z
%%   Digits        \0\1\2\3\4\5\6\7\8\9
%%   Exclamation   \!     Double quote  \"     Hash (number) \#
%%   Dollar        \$     Percent       \%     Ampersand     \&
%%   Acute accent  \'     Left paren    \(     Right paren   \)
%%   Asterisk      \*     Plus          \+     Comma         \,
%%   Minus         \-     Point         \.     Solidus       \/
%%   Colon         \:     Semicolon     \;     Less than     \<
%%   Equals        \=     Greater than  \>     Question mark \?
%%   Commercial at \@     Left bracket  \[     Backslash     \\
%%   Right bracket \]     Circumflex    \^     Underscore    \_
%%   Grave accent  \`     Left brace    \{     Vertical bar  \|
%%   Right brace   \}     Tilde         \~}
%%
\endinput
%
  \main@language{austrian}}
\DeclareOption{bahasa}{% \iffalse meta-comment
%
% Copyright 1989-1995 Johannes L. Braams and any individual authors
% listed elsewhere in this file.  All rights reserved.
% 
% For further copyright information any other copyright notices in this
% file.
% 
% This file is part of the Babel system release 3.5.
% --------------------------------------------------
%   This system is distributed in the hope that it will be useful,
%   but WITHOUT ANY WARRANTY; without even the implied warranty of
%   MERCHANTABILITY or FITNESS FOR A PARTICULAR PURPOSE.
% 
%   For error reports concerning UNCHANGED versions of this file no more
%   than one year old, see bugs.txt.
% 
%   Please do not request updates from me directly.  Primary
%   distribution is through the CTAN archives.
% 
% 
% IMPORTANT COPYRIGHT NOTICE:
% 
% You are NOT ALLOWED to distribute this file alone.
% 
% You are allowed to distribute this file under the condition that it is
% distributed together with all the files listed in manifest.txt.
% 
% If you receive only some of these files from someone, complain!
% 
% Permission is granted to copy this file to another file with a clearly
% different name and to customize the declarations in that copy to serve
% the needs of your installation, provided that you comply with
% the conditions in the file legal.txt from the LaTeX2e distribution.
% 
% However, NO PERMISSION is granted to produce or to distribute a
% modified version of this file under its original name.
%  
% You are NOT ALLOWED to change this file.
% 
% 
% \fi
% \CheckSum{114}
%\iffalse
%    Tell the \LaTeX\ system who we are and write an entry on the
%    transcript.
%<*dtx>
\ProvidesFile{bahasa.dtx}
%</dtx>
%<code>\ProvidesFile{bahasa.ldf}
       [1995/07/04 v1.0b Bahasa support from the babel system]
%
% Babel package for LaTeX version 2e
% Copyright (C) 1989 - 1995
%           by Johannes Braams, TeXniek
%
% Bahasa Language Definition File
% Copyright (C) 1994 - 1995
%           by J"org Knappen, (knappen@vkpmzd.kph.uni-mainz.de)
%              Terry Mart (mart@vkpmzd.kph.uni-mainz.de)
%              Institut f\"ur Kernphysik
%              Johannes Gutenberg-Universit\"at Mainz
%              D-55099 Mainz
%              Germany
%
% Please report errors to: J.L. Braams
%                          JLBraams@cistron.nl
%
%    This file is part of the babel system, it provides the source
%    code for the bahasa indonesia / bahasa melayu language definition
%    file.  The original version of this file was written by Terry
%    Mart (mart@vkpmzd.kph.uni-mainz.de) and J"org Knappen
%    (knappen@vkpmzd.kph.uni-mainz.de).
%<*filedriver>
\documentclass{ltxdoc}
\newcommand*\TeXhax{\TeX hax}
\newcommand*\babel{\textsf{babel}}
\newcommand*\langvar{$\langle \it lang \rangle$}
\newcommand*\note[1]{}
\newcommand*\Lopt[1]{\textsf{#1}}
\newcommand*\file[1]{\texttt{#1}}
\begin{document}
 \DocInput{bahasa.dtx}
\end{document}
%</filedriver>
%\fi
% \GetFileInfo{bahasa.dtx}
%
% \changes{bahasa-0.9c}{1994/06/26}{Removed the use of \cs{filedate}
%    and moved identification after the loading of \file{babel.def}}
%
%  \section{The Bahasa language}
%
%    The file \file{\filename}\footnote{The file described in this
%    section has version number \fileversion\ and was last revised on
%    \filedate.}  defines all the language definition macros for the
%    bahasa indonesia / bahasa melayu language. Bahasa just means
%    `language' in bahasa indonesia / bahasa melayu. Since both
%    national versions of the language use the same writing, although
%    differing in pronounciation, this file can be used for both
%    languages.
%
%    For this language currently no special definitions are needed or
%    available.
%
% \StopEventually{}
%
%    As this file needs to be read only once, we check whether it was
%    read before. If it was, the command |\captionsbahasa| is already
%    defined, so we can stop processing. If this command is undefined
%    we proceed with the various definitions and first show the
%    current version of this file.
%
%    \begin{macrocode}
%<*code>
\ifx\undefined\captionsbahasa
\else
  \selectlanguage{bahasa}
  \expandafter\endinput
\fi
%    \end{macrocode}
%
% \begin{macro}{\atcatcode}
%    This file, \file{bahasa.sty}, may have been read while \TeX\ is
%    in the middle of processing a document, so we have to make sure
%    the category code of \texttt{@} is `letter' while this file is
%    being read.  We save the category code of the @-sign in
%    |\atcatcode| and make it `letter'. Later the category code can be
%    restored to whatever it was before.
%    \begin{macrocode}
\chardef\atcatcode=\catcode`\@
\catcode`\@=11\relax
%    \end{macrocode}
% \end{macro}
%
%    Now we determine whether the common macros from the file
%    \file{babel.def} need to be read. We can be in one of two
%    situations: either another language option has been read earlier
%    on, in which case that other option has already read
%    \file{babel.def}, or \texttt{bahasa} is the first language option
%    to be processed. In that case we need to read \file{babel.def}
%    right here before we continue.
%
%    \begin{macrocode}
\ifx\undefined\babel@core@loaded\input babel.def\relax\fi
%    \end{macrocode}
%
%    Another check that has to be made, is if another language
%    definition file has been read already. In that case its
%    definitions have been activated. This might interfere with
%    definitions this file tries to make. Therefore we make sure that
%    we cancel any special definitions. This can be done by checking
%    the existence of the macro |\originalTeX|. If it exists we simply
%    execute it, otherwise it is |\let| to |\empty|.
%    \begin{macrocode}
\ifx\undefined\originalTeX \let\originalTeX\empty\fi
\originalTeX
%    \end{macrocode}
%
%    When this file is read as an option, i.e. by the |\usepackage|
%    command, \texttt{bahasa} could be an `unknown' language in which
%    case we have to make it known. So we check for the existence of
%    |\l@bahasa| to see whether we have to do something here.
%
% \changes{bahasa-0.9c}{1994/06/26}{Now use \cs{@patterns} to produce
%    the warning}
%    \begin{macrocode}
\ifx\undefined\l@bahasa
  \@nopatterns{Bahasa}
  \adddialect\l@bahasa0\fi
%    \end{macrocode}
%
%    The next step consists of defining commands to switch to (and
%    from) the Bahasa language.
%
% \begin{macro}{\captionsbahasa}
%    The macro |\captionsbahasa| defines all strings used in the four
%    standard documentclasses provided with \LaTeX.
% \changes{bahasa-1.0b}{1995/07/04}{Added \cs{proofname} for
%    AMS-\LaTeX}
%    \begin{macrocode}
\addto\captionsbahasa{%
  \def\prefacename{Pendahuluan}%
  \def\refname{Pustaka}%
  \def\abstractname{Ringkasan}% (sometime it's called 'intisari'
                              %  or 'ikhtisar')
  \def\bibname{Bibliografi}%
  \def\chaptername{Bab}%
  \def\appendixname{Lampiran}%
  \def\contentsname{Daftar Isi}%
  \def\listfigurename{Daftar Gambar}%
  \def\listtablename{Daftar Tabel}%
% Glossary: Daftar Istilah
  \def\indexname{Indeks}%
  \def\figurename{Gambar}%
  \def\tablename{Tabel}%
  \def\partname{Bagian}%
%  Subject:  Subyek
%  From:  Dari
  \def\enclname{Lampiran}%
  \def\ccname{cc}%
  \def\headtoname{Kepada}%
  \def\pagename{Halaman}%
%  Notes (Endnotes): Catatan
  \def\seename{lihat}%
  \def\alsoname{lihat juga}%
  \def\proofname{Proof}%  <-- needs translation
  }
%    \end{macrocode}
% \end{macro}
%
% \begin{macro}{\datebahasa}
%    The macro |\datebahasa| redefines the command |\today| to produce
%    Bahasa dates.
%    \begin{macrocode}
\def\datebahasa{%
  \def\today{\number\day~\ifcase\month\or
    Januari\or Februari\or Maret\or April\or Mei\or Juni\or
    Juli\or Agustus\or September\or Oktober\or Nopember\or Desember\fi
    \space \number\year}}
%    \end{macrocode}
% \end{macro}
%
%
% \begin{macro}{\extrasbahasa}
% \begin{macro}{\noextrasbahasa}
%    The macro |\extrasbahasa| will perform all the extra definitions
%    needed for the Bahasa language. The macro |\extrasbahasa| is used
%    to cancel the actions of |\extrasbahasa|.  For the moment these
%    macros are empty but they are defined for compatibility with the
%    other language definition files.
%
%    \begin{macrocode}
\addto\extrasbahasa{}
\addto\noextrasbahasa{}
%    \end{macrocode}
% \end{macro}
% \end{macro}
%
%    It is possible that a site might need to add some extra code to
%    the babel macros. To enable this we load a local configuration
%    file, \file{bahasa.cfg} if it is found on \TeX' search path.
% \changes{bahasa-1.0b}{1995/07/02}{Added loading of configuration
%    file}
%    \begin{macrocode}
\loadlocalcfg{bahasa}
%    \end{macrocode}
%
%    Our last action is to make a note that the commands we have just
%    defined, will be executed by calling the macro |\selectlanguage|
%    at the beginning of the document.
% \changes{bahasa-0.9d}{1995/05/05}{Use \cs{main@language} instead
%    of \cs{selectlanguage}}
%    \begin{macrocode}
\main@language{bahasa}
%    \end{macrocode}
%    Finally, the category code of \texttt{@} is reset to its original
%    value. The macrospace used by |\atcatcode| is freed.
%    \begin{macrocode}
\catcode`\@=\atcatcode \let\atcatcode\relax
%</code>
%    \end{macrocode}
%
% \Finale
%%
%% \CharacterTable
%%  {Upper-case    \A\B\C\D\E\F\G\H\I\J\K\L\M\N\O\P\Q\R\S\T\U\V\W\X\Y\Z
%%   Lower-case    \a\b\c\d\e\f\g\h\i\j\k\l\m\n\o\p\q\r\s\t\u\v\w\x\y\z
%%   Digits        \0\1\2\3\4\5\6\7\8\9
%%   Exclamation   \!     Double quote  \"     Hash (number) \#
%%   Dollar        \$     Percent       \%     Ampersand     \&
%%   Acute accent  \'     Left paren    \(     Right paren   \)
%%   Asterisk      \*     Plus          \+     Comma         \,
%%   Minus         \-     Point         \.     Solidus       \/
%%   Colon         \:     Semicolon     \;     Less than     \<
%%   Equals        \=     Greater than  \>     Question mark \?
%%   Commercial at \@     Left bracket  \[     Backslash     \\
%%   Right bracket \]     Circumflex    \^     Underscore    \_
%%   Grave accent  \`     Left brace    \{     Vertical bar  \|
%%   Right brace   \}     Tilde         \~}
%%
\endinput
}
\DeclareOption{brazil}{% \iffalse meta-comment
%
% Copyright 1989-1995 Johannes L. Braams and any individual authors
% listed elsewhere in this file.  All rights reserved.
% 
% For further copyright information any other copyright notices in this
% file.
% 
% This file is part of the Babel system release 3.5.
% --------------------------------------------------
%   This system is distributed in the hope that it will be useful,
%   but WITHOUT ANY WARRANTY; without even the implied warranty of
%   MERCHANTABILITY or FITNESS FOR A PARTICULAR PURPOSE.
% 
%   For error reports concerning UNCHANGED versions of this file no more
%   than one year old, see bugs.txt.
% 
%   Please do not request updates from me directly.  Primary
%   distribution is through the CTAN archives.
% 
% 
% IMPORTANT COPYRIGHT NOTICE:
% 
% You are NOT ALLOWED to distribute this file alone.
% 
% You are allowed to distribute this file under the condition that it is
% distributed together with all the files listed in manifest.txt.
% 
% If you receive only some of these files from someone, complain!
% 
% Permission is granted to copy this file to another file with a clearly
% different name and to customize the declarations in that copy to serve
% the needs of your installation, provided that you comply with
% the conditions in the file legal.txt from the LaTeX2e distribution.
% 
% However, NO PERMISSION is granted to produce or to distribute a
% modified version of this file under its original name.
%  
% You are NOT ALLOWED to change this file.
% 
% 
% \fi
% \CheckSum{275}
% \iffalse
%    Tell the \LaTeX\ system who we are and write an entry on the
%    transcript.
%<*dtx>
\ProvidesFile{portuges.dtx}
%</dtx>
%<code>\ProvidesFile{portuges.ldf}
        [1995/07/04 v1.2h Portuguese support from the babel system]
%
% Babel package for LaTeX version 2e
% Copyright (C) 1989 - 1995
%           by Johannes Braams, TeXniek
%
% Portuguese Language Definition File
% Copyright (C) 1989 - 1995
%           by Johannes Braams, TeXniek
%
% Please report errors to: J.L. Braams
%                          JLBraams@cistron.nl
%
%    This file is part of the babel system, it provides the source
%    code for the Portuguese language definition file.  The Portuguese
%    words were contributed by Jose Pedro Ramalhete, (JRAMALHE@CERNVM
%    or Jose-Pedro_Ramalhete@MACMAIL).
%
%    Arnaldo Viegas de Lima <arnaldo@VNET.IBM.COM> contributed
%    brazilian translations and suggestions for enhancements.
%<*filedriver>
\documentclass{ltxdoc}
\newcommand*\TeXhax{\TeX hax}
\newcommand*\babel{\textsf{babel}}
\newcommand*\langvar{$\langle \it lang \rangle$}
\newcommand*\note[1]{}
\newcommand*\Lopt[1]{\textsf{#1}}
\newcommand*\file[1]{\texttt{#1}}
\begin{document}
 \DocInput{portuges.dtx}
\end{document}
%</filedriver>
%\fi
%
% \GetFileInfo{portuges.dtx}
%
% \changes{portuges-1.0a}{1991/07/15}{Renamed \file{babel.sty} in
%    \file{babel.com}}
% \changes{portuges-1.1}{1992/02/16}{Brought up-to-date with babel 3.2a}
% \changes{portuges-1.2}{1994/02/26}{Update for \LaTeXe}
% \changes{portuges-1.2d}{1994/06/26}{Removed the use of \cs{filedate}
%    and moved identification after the loading of \file{babel.def}}
% \changes{portuges-1.2g}{1995/06/04}{Enhanced support for brasilian}
%
%  \section{The Portuguese language}
%
%    The file \file{\filename}\footnote{The file described in this
%    section has version number \fileversion\ and was last revised on
%    \filedate.  Contributions were made by Jose Pedro Ramalhete
%    (\texttt{JRAMALHE@CERNVM} or
%    \texttt{Jose-Pedro\_Ramalhete@MACMAIL}) and Arnaldo Viegas de
%    Lima \texttt{arnaldo@VNET.IBM.COM}.}  defines all the
%    language-specific macros for the Portuguese language as well as
%    for the Brasilian version of this language.
%
%    For this language the character |"| is made active. In
%    table~\ref{tab:port-quote} an overview is given of its purpose.
%
%    \begin{table}[htb]
%     \centering
%     \begin{tabular}{lp{8cm}}
%       \verb="|= & disable ligature at this position.\\
%        |"-| & an explicit hyphen sign, allowing hyphenation
%               in the rest of the word.\\
%        |""| & like \verb="-=, but producing no hyphen sign (for
%              words that should break at some sign such as
%              ``entrada/salida.''\\
%        |"<| & for French left double quotes (similar to $<<$).\\
%        |">| & for French right double quotes (similar to $>>$).\\
%        |\-| & like the old |\-|, but allowing hyphenation
%               in the rest of the word. \\
%     \end{tabular}
%     \caption{The extra definitions made by \file{portuges.ldf}}
%     \label{tab:port-quote}
%    \end{table}
%
% \StopEventually{}

%    As this file needs to be read only once, we check whether it was
%    read before. If it was, the command |\captionsportuges| is
%    already defined, so we can stop processing. If this command is
%    undefined we proceed with the various definitions and first show
%    the current version of this file.
%
% \changes{portuges-1.0a}{1991/07/15}{Added reset of catcode of @
%    before \cs{endinput}.}
% \changes{portuges-1.0b}{1991/10/29}{Removed use of
%    \cs{@ifundefined}}
%    \begin{macrocode}
%<*code>
\ifx\undefined\captionsportuges
\else
  \selectlanguage{portuges}
  \expandafter\endinput
\fi
%    \end{macrocode}
%
% \changes{portuges-1.0b}{1991/10/29}{Removed code to load
%    \file{latexhax.com}}
%
% \begin{macro}{\atcatcode}
%    This file, \file{portuges.ldf}, may have been read while \TeX\ is
%    in the middle of processing a document, so we have to make sure
%    the category code of \texttt{@} is `letter' while this file is
%    being read.  We save the category code of the @-sign in
%    |\atcatcode| and make it `letter'. Later the category code can be
%    restored to whatever it was before.
%
% \changes{portuges-1.0a}{1991/07/15}{Modified handling of catcode of
%    @ again.}
% \changes{portuges-1.0b}{1991/10/ 29}{Removed use of
%    \cs{makeatletter} and hence the need to load \file{latexhax.com}}
%    \begin{macrocode}
\chardef\atcatcode=\catcode`\@
\catcode`\@=11\relax
%    \end{macrocode}
% \end{macro}
%
%    Now we determine whether the the common macros from the file
%    \file{babel.def} need to be read. We can be in one of two
%    situations: either another language option has been read earlier
%    on, in which case that other option has already read
%    \file{babel.def}, or \texttt{portuges} is the first language
%    option to be processed. In that case we need to read
%    \file{babel.def} right here before we continue.
%
% \changes{portuges-1.1}{1992/02/16}{Added \cs{relax} after the
%    argument of \cs{input}}
%    \begin{macrocode}
\ifx\undefined\babel@core@loaded\input babel.def\relax\fi
%    \end{macrocode}
%
%    Another check that has to be made, is if another language
%    definition file has been read already. In that case its
%    definitions have been activated. This might interfere with
%    definitions this file tries to make. Therefore we make sure that
%    we cancel any special definitions. This can be done by checking
%    the existence of the macro |\originalTeX|. If it exists we simply
%    execute it, otherwise it is |\let| to |\empty|.
% \changes{portuges-1.0a}{1991/07/ 15}{Added
%    \cs{let}\cs{originalTeX}\cs{relax} to test for existence}
% \changes{portuges-1.1}{1992/02/16}{\cs{originalTeX} should be
%    expandable, \cs{let} it to \cs{empty}}
%    \begin{macrocode}
\ifx\undefined\originalTeX \let\originalTeX\empty \else\originalTeX\fi
%    \end{macrocode}
%
%    When this file is read as an option, i.e. by the |\usepackage|
%    command, \texttt{portuges} will be an `unknown' language in which
%    case we have to make it known. So we check for the existence of
%    |\l@portuges| to see whether we have to do something here.
%
% \changes{portuges-1.0b}{1991/10/29}{Removed use of cs{@ifundefined}}
% \changes{portuges-1.1}{1992/02/16}{Added a warning when no
%    hyphenation patterns were loaded.}
% \changes{portuges-1.2d}{1994/06/26}{Now use \cs{@nopatterns} to
%    produce the warning}
%    \begin{macrocode}
\ifx\undefined\l@portuges
    \@nopatterns{Portuges}
    \adddialect\l@portuges0\fi
%    \end{macrocode}
%
%    For the Brasilian version of these definitions we just add a
%    ``dialect''. Also, the macros |\captionsbrazil| and
%    |\extrasbrazil| are |\let| to their Portuguese counterparts when
%    these parts are defined.
%    \begin{macrocode}
\adddialect\l@brazil\l@portuges
%    \end{macrocode}
%
%    The next step consists of defining commands to switch to (and from)
%    the Portuguese language.
%
% \begin{macro}{\captionsportuges}
%    The macro |\captionsportuges| defines all strings used
%    in the four standard documentclasses provided with \LaTeX.
% \changes{portuges-1.1}{1992/02/16}{Added \cs{seename}, \cs{alsoname}
%    and \cs{prefacename}}
% \changes{portuges-1.1}{1993/07/15}{\cs{headpagename} should be
%    \cs{pagename}}
% \changes{portuges-1.2e}{1994/11/09}{Added a few missing
%    translations}
% \changes{portuges-1.2h}{1995/07/04}{Added \cs{proofname} for
%    AMS-\LaTeX}
%    \begin{macrocode}
\addto\captionsportuges{%
  \def\prefacename{Pref\'acio}%
  \def\refname{Refer\^encias}%
  \def\abstractname{Resumo}%
  \def\bibname{Bibliografia}%
  \def\chaptername{Cap\'{\i}tulo}%
  \def\appendixname{Ap\^endice}%
  \def\contentsname{\'Indice}%
  \def\listfigurename{Lista de Figuras}%
  \def\listtablename{Lista de Tabelas}%
  \def\indexname{\'Indice Remissivo}%
  \def\figurename{Figura}%
  \def\tablename{Tabela}%
  \def\partname{Parte}%
  \def\enclname{Anexos}%
  \def\ccname{C\'opia a}%
  \def\headtoname{Para}%
  \def\pagename{P\'agina}%
  \def\seename{ver}%
  \def\alsoname{ver tamb\'em}%
  \def\proofname{Proof}%  <-- needs translation
  }%
%    \end{macrocode}
% \end{macro}
%
% \begin{macro}{\captionsbrazil}
% \changes{portuges-1.2g}{1995/06/04}{The coptions for brazilian and
%    portuguese are different now}
%
%    The ``captions'' are different for both versions of the language,
%    so we define the macro |\captionsbrazil| here.
%    \begin{macrocode}
\addto\captionsbrazil{%
  \def\prefacename{Pref\'acio}%
  \def\refname{Refer\^encias}%
  \def\abstractname{Resumo}%
  \def\bibname{Refer\^encias Bibliogr\'aficas}%
  \def\chaptername{Cap\'{\i}tulo}%
  \def\appendixname{Ap\^endice}%
  \def\contentsname{Sum\'ario}%
  \def\listfigurename{Lista de Figuras}%
  \def\listtablename{Lista de Tabelas}%
  \def\indexname{\'Indice}%
  \def\figurename{Figura}%
  \def\tablename{Tabela}%
  \def\partname{Parte}%
  \def\enclname{Anexo}%
  \def\ccname{C\'opia para}%
  \def\headtoname{Para}%
  \def\pagename{P\'agina}%
  \def\seename{veja}%
  \def\alsoname{veja tamb\'em}%
  }
%    \end{macrocode}
% \end{macro}
%
% \begin{macro}{\dateportuges}
%    The macro |\dateportuges| redefines the command |\today| to
%    produce Portuguese dates.
%    \begin{macrocode}
\def\dateportuges{%
\def\today{\number\day\space de\space\ifcase\month\or
  Janeiro\or Fevereiro\or Mar\c{c}o\or Abril\or Maio\or Junho\or
  Julho\or Agosto\or Setembro\or Outubro\or Novembro\or Dezembro\fi
  \space de\space\number\year}}
%    \end{macrocode}
% \end{macro}
%
% \begin{macro}{\datebrazil}
%    The macro |\datebrazil| redefines the command
%    |\today| to produce Brasilian dates, for which the names
%    of the months are not capitalized.
%    \begin{macrocode}
\def\datebrazil{%
\def\today{\number\day\space de\space\ifcase\month\or
  janeiro\or fevereiro\or mar\c{c}o\or abril\or maio\or junho\or
  julho\or agosto\or setembro\or outubro\or novembro\or dezembro\fi
  \space de\space\number\year}}
%    \end{macrocode}
% \end{macro}
%
%  \begin{macro}{\portugeshyphenmins}
%  \begin{macro}{\brasilhyphenmins}
% \changes{portuges-1.2g}{1995/06/04}{Added setting of hyphenmin
%    values}
%    Set correct values for |\lefthyphenmin| and |\righthyphenmin|.
%    \begin{macrocode}
\def\portugeshyphenmins{\tw@\tw@}
\def\brazilhyphenmins{\tw@\tw@}
%    \end{macrocode}
%  \end{macro}
%  \end{macro}
%
% \begin{macro}{\extrasportuges}
% \changes{portuges-1.2g}{1995/06/04}{Added using some \texttt{"}
%    shorthands}
% \begin{macro}{\noextrasportuges}
%    The macro |\extrasportuges| will perform all the extra
%    definitions needed for the Portuguese language. The macro
%    |\noextrasportuges| is used to cancel the actions of
%    |\extrasportuges|.
%
%    For Portuguese the \texttt{"} character is made active. This is
%    done once, later on its definition may vary. Other languages in
%    the same document may also use the \texttt{"} character for
%    shorthands; we specify that the portuguese group of shorthands
%    should be used.
%
%    \begin{macrocode}
\initiate@active@char{"}
\addto\extrasportuges{\languageshorthands{portuges}}
\addto\extrasportuges{\bbl@activate{"}}
%\addto\noextrasportuges{\bbl@deactivate{"}}
%    \end{macrocode}
%    First we define access to the guillemets for quotations,
%    \begin{macrocode}
\declare@shorthand{portuges}{"<}{%
  \textormath{\guillemotleft{}}{\mbox{\guillemotleft}}}
\declare@shorthand{portuges}{">}{%
  \textormath{\guillemotright{}}{\mbox{\guillemotright}}}
%    \end{macrocode}
%    then we define two shorthands to be able to specify hyphenation
%    breakpoints that behavew a little different from |\-|.
%    \begin{macrocode}
\declare@shorthand{portuges}{"-}{\allowhyphens-\allowhyphens}
\declare@shorthand{portuges}{""}{\hskip\z@skip}
%    \end{macrocode}
%    And we want to have a shorthand for disabling a ligature.
%    \begin{macrocode}
\declare@shorthand{portuges}{"|}{%
  \textormath{\discretionary{-}{}{\kern.03em}}{}}
%    \end{macrocode}
% \end{macro}
% \end{macro}
%
%  \begin{macro}{\-}
%
%    All that is left now is the redefinition of |\-|. The new version
%    of |\-| should indicate an extra hyphenation position, while
%    allowing other hyphenation positions to be generated
%    automatically. The standard behaviour of \TeX\ in this respect is
%    very unfortunate for languages such as Dutch and German, where
%    long compound words are quite normal and all one needs is a means
%    to indicate an extra hyphenation position on top of the ones that
%    \TeX\ can generate from the hyphenation patterns.
%    \begin{macrocode}
\addto\extrasportuges{\babel@save\-}
\addto\extrasportuges{\def\-{\allowhyphens
                          \discretionary{-}{}{}\allowhyphens}}
%    \end{macrocode}
%  \end{macro}
%
%  \begin{macro}{\ord}
% \changes{portuges-1.2g}{1995/06/04}{Added macro}
%  \begin{macro}{\ro}
% \changes{portuges-1.2g}{1995/06/04}{Added macro}
%  \begin{macro}{\orda}
% \changes{portuges-1.2g}{1995/06/04}{Added macro}
%  \begin{macro}{\ra}
% \changes{portuges-1.2g}{1995/06/04}{Added macro}
%    We also provide an easy way to typeset ordinals, both in the male
%    (|\ord| or |\ro|) and the female (|orda| or |\ra|) form.
%    \begin{macrocode}
\def\ord{$^{\rm o}$}
\def\orda{$^{\rm a}$}
\let\ro\ord\let\ra\orda
%    \end{macrocode}
%  \end{macro}
%  \end{macro}
%  \end{macro}
%  \end{macro}
%
% \begin{macro}{\extrasbrazil}
% \begin{macro}{\noextrasbrazil}
%    Also for the ``brazil'' variant no extra definitions are needed
%    at the moment.
%    \begin{macrocode}
\let\extrasbrazil\extrasportuges
\let\noextrasbrazil\noextrasportuges
%    \end{macrocode}
% \end{macro}
% \end{macro}
%
%    It is possible that a site might need to add some extra code to
%    the babel macros. To enable this we load a local configuration
%    file, \file{portuges.cfg} if it is found on \TeX' search path.
% \changes{portuges-1.2h}{1995/07/02}{Added loading of configuration
%    file}
%    \begin{macrocode}
\loadlocalcfg{portuges}
%    \end{macrocode}
%
%    Our last action is to make a note that the commands we have just
%    defined, will be executed by calling the macro |\selectlanguage|
%    at the beginning of the document.
% \changes{portuges-1.2f}{1995/03/14}{Use \cs{main@language} instead
%    of \cs{selectlanguage}}
%    \begin{macrocode}
\main@language{portuges}
%    \end{macrocode}
%    Finally, the category code of \texttt{@} is reset to its original
%    value. The macrospace used by |\atcatcode| is freed.
% \changes{portuges-1.0a}{1991/07/15}{Modified handling of catcode of
%    @-sign.}
%    \begin{macrocode}
\catcode`\@=\atcatcode \let\atcatcode\relax
%</code>
%    \end{macrocode}
%
% \Finale
%%
%% \CharacterTable
%%  {Upper-case    \A\B\C\D\E\F\G\H\I\J\K\L\M\N\O\P\Q\R\S\T\U\V\W\X\Y\Z
%%   Lower-case    \a\b\c\d\e\f\g\h\i\j\k\l\m\n\o\p\q\r\s\t\u\v\w\x\y\z
%%   Digits        \0\1\2\3\4\5\6\7\8\9
%%   Exclamation   \!     Double quote  \"     Hash (number) \#
%%   Dollar        \$     Percent       \%     Ampersand     \&
%%   Acute accent  \'     Left paren    \(     Right paren   \)
%%   Asterisk      \*     Plus          \+     Comma         \,
%%   Minus         \-     Point         \.     Solidus       \/
%%   Colon         \:     Semicolon     \;     Less than     \<
%%   Equals        \=     Greater than  \>     Question mark \?
%%   Commercial at \@     Left bracket  \[     Backslash     \\
%%   Right bracket \]     Circumflex    \^     Underscore    \_
%%   Grave accent  \`     Left brace    \{     Vertical bar  \|
%%   Right brace   \}     Tilde         \~}
%%
\endinput
\main@language{brazil}}
%    \end{macrocode}
% \changes{babel~3.5b}{1995/05/25}{Added brazilian as alternative for
%    brazil}
%    \begin{macrocode}
\DeclareOption{brazilian}{%
  % \iffalse meta-comment
%
% Copyright 1989-1995 Johannes L. Braams and any individual authors
% listed elsewhere in this file.  All rights reserved.
% 
% For further copyright information any other copyright notices in this
% file.
% 
% This file is part of the Babel system release 3.5.
% --------------------------------------------------
%   This system is distributed in the hope that it will be useful,
%   but WITHOUT ANY WARRANTY; without even the implied warranty of
%   MERCHANTABILITY or FITNESS FOR A PARTICULAR PURPOSE.
% 
%   For error reports concerning UNCHANGED versions of this file no more
%   than one year old, see bugs.txt.
% 
%   Please do not request updates from me directly.  Primary
%   distribution is through the CTAN archives.
% 
% 
% IMPORTANT COPYRIGHT NOTICE:
% 
% You are NOT ALLOWED to distribute this file alone.
% 
% You are allowed to distribute this file under the condition that it is
% distributed together with all the files listed in manifest.txt.
% 
% If you receive only some of these files from someone, complain!
% 
% Permission is granted to copy this file to another file with a clearly
% different name and to customize the declarations in that copy to serve
% the needs of your installation, provided that you comply with
% the conditions in the file legal.txt from the LaTeX2e distribution.
% 
% However, NO PERMISSION is granted to produce or to distribute a
% modified version of this file under its original name.
%  
% You are NOT ALLOWED to change this file.
% 
% 
% \fi
% \CheckSum{275}
% \iffalse
%    Tell the \LaTeX\ system who we are and write an entry on the
%    transcript.
%<*dtx>
\ProvidesFile{portuges.dtx}
%</dtx>
%<code>\ProvidesFile{portuges.ldf}
        [1995/07/04 v1.2h Portuguese support from the babel system]
%
% Babel package for LaTeX version 2e
% Copyright (C) 1989 - 1995
%           by Johannes Braams, TeXniek
%
% Portuguese Language Definition File
% Copyright (C) 1989 - 1995
%           by Johannes Braams, TeXniek
%
% Please report errors to: J.L. Braams
%                          JLBraams@cistron.nl
%
%    This file is part of the babel system, it provides the source
%    code for the Portuguese language definition file.  The Portuguese
%    words were contributed by Jose Pedro Ramalhete, (JRAMALHE@CERNVM
%    or Jose-Pedro_Ramalhete@MACMAIL).
%
%    Arnaldo Viegas de Lima <arnaldo@VNET.IBM.COM> contributed
%    brazilian translations and suggestions for enhancements.
%<*filedriver>
\documentclass{ltxdoc}
\newcommand*\TeXhax{\TeX hax}
\newcommand*\babel{\textsf{babel}}
\newcommand*\langvar{$\langle \it lang \rangle$}
\newcommand*\note[1]{}
\newcommand*\Lopt[1]{\textsf{#1}}
\newcommand*\file[1]{\texttt{#1}}
\begin{document}
 \DocInput{portuges.dtx}
\end{document}
%</filedriver>
%\fi
%
% \GetFileInfo{portuges.dtx}
%
% \changes{portuges-1.0a}{1991/07/15}{Renamed \file{babel.sty} in
%    \file{babel.com}}
% \changes{portuges-1.1}{1992/02/16}{Brought up-to-date with babel 3.2a}
% \changes{portuges-1.2}{1994/02/26}{Update for \LaTeXe}
% \changes{portuges-1.2d}{1994/06/26}{Removed the use of \cs{filedate}
%    and moved identification after the loading of \file{babel.def}}
% \changes{portuges-1.2g}{1995/06/04}{Enhanced support for brasilian}
%
%  \section{The Portuguese language}
%
%    The file \file{\filename}\footnote{The file described in this
%    section has version number \fileversion\ and was last revised on
%    \filedate.  Contributions were made by Jose Pedro Ramalhete
%    (\texttt{JRAMALHE@CERNVM} or
%    \texttt{Jose-Pedro\_Ramalhete@MACMAIL}) and Arnaldo Viegas de
%    Lima \texttt{arnaldo@VNET.IBM.COM}.}  defines all the
%    language-specific macros for the Portuguese language as well as
%    for the Brasilian version of this language.
%
%    For this language the character |"| is made active. In
%    table~\ref{tab:port-quote} an overview is given of its purpose.
%
%    \begin{table}[htb]
%     \centering
%     \begin{tabular}{lp{8cm}}
%       \verb="|= & disable ligature at this position.\\
%        |"-| & an explicit hyphen sign, allowing hyphenation
%               in the rest of the word.\\
%        |""| & like \verb="-=, but producing no hyphen sign (for
%              words that should break at some sign such as
%              ``entrada/salida.''\\
%        |"<| & for French left double quotes (similar to $<<$).\\
%        |">| & for French right double quotes (similar to $>>$).\\
%        |\-| & like the old |\-|, but allowing hyphenation
%               in the rest of the word. \\
%     \end{tabular}
%     \caption{The extra definitions made by \file{portuges.ldf}}
%     \label{tab:port-quote}
%    \end{table}
%
% \StopEventually{}

%    As this file needs to be read only once, we check whether it was
%    read before. If it was, the command |\captionsportuges| is
%    already defined, so we can stop processing. If this command is
%    undefined we proceed with the various definitions and first show
%    the current version of this file.
%
% \changes{portuges-1.0a}{1991/07/15}{Added reset of catcode of @
%    before \cs{endinput}.}
% \changes{portuges-1.0b}{1991/10/29}{Removed use of
%    \cs{@ifundefined}}
%    \begin{macrocode}
%<*code>
\ifx\undefined\captionsportuges
\else
  \selectlanguage{portuges}
  \expandafter\endinput
\fi
%    \end{macrocode}
%
% \changes{portuges-1.0b}{1991/10/29}{Removed code to load
%    \file{latexhax.com}}
%
% \begin{macro}{\atcatcode}
%    This file, \file{portuges.ldf}, may have been read while \TeX\ is
%    in the middle of processing a document, so we have to make sure
%    the category code of \texttt{@} is `letter' while this file is
%    being read.  We save the category code of the @-sign in
%    |\atcatcode| and make it `letter'. Later the category code can be
%    restored to whatever it was before.
%
% \changes{portuges-1.0a}{1991/07/15}{Modified handling of catcode of
%    @ again.}
% \changes{portuges-1.0b}{1991/10/ 29}{Removed use of
%    \cs{makeatletter} and hence the need to load \file{latexhax.com}}
%    \begin{macrocode}
\chardef\atcatcode=\catcode`\@
\catcode`\@=11\relax
%    \end{macrocode}
% \end{macro}
%
%    Now we determine whether the the common macros from the file
%    \file{babel.def} need to be read. We can be in one of two
%    situations: either another language option has been read earlier
%    on, in which case that other option has already read
%    \file{babel.def}, or \texttt{portuges} is the first language
%    option to be processed. In that case we need to read
%    \file{babel.def} right here before we continue.
%
% \changes{portuges-1.1}{1992/02/16}{Added \cs{relax} after the
%    argument of \cs{input}}
%    \begin{macrocode}
\ifx\undefined\babel@core@loaded\input babel.def\relax\fi
%    \end{macrocode}
%
%    Another check that has to be made, is if another language
%    definition file has been read already. In that case its
%    definitions have been activated. This might interfere with
%    definitions this file tries to make. Therefore we make sure that
%    we cancel any special definitions. This can be done by checking
%    the existence of the macro |\originalTeX|. If it exists we simply
%    execute it, otherwise it is |\let| to |\empty|.
% \changes{portuges-1.0a}{1991/07/ 15}{Added
%    \cs{let}\cs{originalTeX}\cs{relax} to test for existence}
% \changes{portuges-1.1}{1992/02/16}{\cs{originalTeX} should be
%    expandable, \cs{let} it to \cs{empty}}
%    \begin{macrocode}
\ifx\undefined\originalTeX \let\originalTeX\empty \else\originalTeX\fi
%    \end{macrocode}
%
%    When this file is read as an option, i.e. by the |\usepackage|
%    command, \texttt{portuges} will be an `unknown' language in which
%    case we have to make it known. So we check for the existence of
%    |\l@portuges| to see whether we have to do something here.
%
% \changes{portuges-1.0b}{1991/10/29}{Removed use of cs{@ifundefined}}
% \changes{portuges-1.1}{1992/02/16}{Added a warning when no
%    hyphenation patterns were loaded.}
% \changes{portuges-1.2d}{1994/06/26}{Now use \cs{@nopatterns} to
%    produce the warning}
%    \begin{macrocode}
\ifx\undefined\l@portuges
    \@nopatterns{Portuges}
    \adddialect\l@portuges0\fi
%    \end{macrocode}
%
%    For the Brasilian version of these definitions we just add a
%    ``dialect''. Also, the macros |\captionsbrazil| and
%    |\extrasbrazil| are |\let| to their Portuguese counterparts when
%    these parts are defined.
%    \begin{macrocode}
\adddialect\l@brazil\l@portuges
%    \end{macrocode}
%
%    The next step consists of defining commands to switch to (and from)
%    the Portuguese language.
%
% \begin{macro}{\captionsportuges}
%    The macro |\captionsportuges| defines all strings used
%    in the four standard documentclasses provided with \LaTeX.
% \changes{portuges-1.1}{1992/02/16}{Added \cs{seename}, \cs{alsoname}
%    and \cs{prefacename}}
% \changes{portuges-1.1}{1993/07/15}{\cs{headpagename} should be
%    \cs{pagename}}
% \changes{portuges-1.2e}{1994/11/09}{Added a few missing
%    translations}
% \changes{portuges-1.2h}{1995/07/04}{Added \cs{proofname} for
%    AMS-\LaTeX}
%    \begin{macrocode}
\addto\captionsportuges{%
  \def\prefacename{Pref\'acio}%
  \def\refname{Refer\^encias}%
  \def\abstractname{Resumo}%
  \def\bibname{Bibliografia}%
  \def\chaptername{Cap\'{\i}tulo}%
  \def\appendixname{Ap\^endice}%
  \def\contentsname{\'Indice}%
  \def\listfigurename{Lista de Figuras}%
  \def\listtablename{Lista de Tabelas}%
  \def\indexname{\'Indice Remissivo}%
  \def\figurename{Figura}%
  \def\tablename{Tabela}%
  \def\partname{Parte}%
  \def\enclname{Anexos}%
  \def\ccname{C\'opia a}%
  \def\headtoname{Para}%
  \def\pagename{P\'agina}%
  \def\seename{ver}%
  \def\alsoname{ver tamb\'em}%
  \def\proofname{Proof}%  <-- needs translation
  }%
%    \end{macrocode}
% \end{macro}
%
% \begin{macro}{\captionsbrazil}
% \changes{portuges-1.2g}{1995/06/04}{The coptions for brazilian and
%    portuguese are different now}
%
%    The ``captions'' are different for both versions of the language,
%    so we define the macro |\captionsbrazil| here.
%    \begin{macrocode}
\addto\captionsbrazil{%
  \def\prefacename{Pref\'acio}%
  \def\refname{Refer\^encias}%
  \def\abstractname{Resumo}%
  \def\bibname{Refer\^encias Bibliogr\'aficas}%
  \def\chaptername{Cap\'{\i}tulo}%
  \def\appendixname{Ap\^endice}%
  \def\contentsname{Sum\'ario}%
  \def\listfigurename{Lista de Figuras}%
  \def\listtablename{Lista de Tabelas}%
  \def\indexname{\'Indice}%
  \def\figurename{Figura}%
  \def\tablename{Tabela}%
  \def\partname{Parte}%
  \def\enclname{Anexo}%
  \def\ccname{C\'opia para}%
  \def\headtoname{Para}%
  \def\pagename{P\'agina}%
  \def\seename{veja}%
  \def\alsoname{veja tamb\'em}%
  }
%    \end{macrocode}
% \end{macro}
%
% \begin{macro}{\dateportuges}
%    The macro |\dateportuges| redefines the command |\today| to
%    produce Portuguese dates.
%    \begin{macrocode}
\def\dateportuges{%
\def\today{\number\day\space de\space\ifcase\month\or
  Janeiro\or Fevereiro\or Mar\c{c}o\or Abril\or Maio\or Junho\or
  Julho\or Agosto\or Setembro\or Outubro\or Novembro\or Dezembro\fi
  \space de\space\number\year}}
%    \end{macrocode}
% \end{macro}
%
% \begin{macro}{\datebrazil}
%    The macro |\datebrazil| redefines the command
%    |\today| to produce Brasilian dates, for which the names
%    of the months are not capitalized.
%    \begin{macrocode}
\def\datebrazil{%
\def\today{\number\day\space de\space\ifcase\month\or
  janeiro\or fevereiro\or mar\c{c}o\or abril\or maio\or junho\or
  julho\or agosto\or setembro\or outubro\or novembro\or dezembro\fi
  \space de\space\number\year}}
%    \end{macrocode}
% \end{macro}
%
%  \begin{macro}{\portugeshyphenmins}
%  \begin{macro}{\brasilhyphenmins}
% \changes{portuges-1.2g}{1995/06/04}{Added setting of hyphenmin
%    values}
%    Set correct values for |\lefthyphenmin| and |\righthyphenmin|.
%    \begin{macrocode}
\def\portugeshyphenmins{\tw@\tw@}
\def\brazilhyphenmins{\tw@\tw@}
%    \end{macrocode}
%  \end{macro}
%  \end{macro}
%
% \begin{macro}{\extrasportuges}
% \changes{portuges-1.2g}{1995/06/04}{Added using some \texttt{"}
%    shorthands}
% \begin{macro}{\noextrasportuges}
%    The macro |\extrasportuges| will perform all the extra
%    definitions needed for the Portuguese language. The macro
%    |\noextrasportuges| is used to cancel the actions of
%    |\extrasportuges|.
%
%    For Portuguese the \texttt{"} character is made active. This is
%    done once, later on its definition may vary. Other languages in
%    the same document may also use the \texttt{"} character for
%    shorthands; we specify that the portuguese group of shorthands
%    should be used.
%
%    \begin{macrocode}
\initiate@active@char{"}
\addto\extrasportuges{\languageshorthands{portuges}}
\addto\extrasportuges{\bbl@activate{"}}
%\addto\noextrasportuges{\bbl@deactivate{"}}
%    \end{macrocode}
%    First we define access to the guillemets for quotations,
%    \begin{macrocode}
\declare@shorthand{portuges}{"<}{%
  \textormath{\guillemotleft{}}{\mbox{\guillemotleft}}}
\declare@shorthand{portuges}{">}{%
  \textormath{\guillemotright{}}{\mbox{\guillemotright}}}
%    \end{macrocode}
%    then we define two shorthands to be able to specify hyphenation
%    breakpoints that behavew a little different from |\-|.
%    \begin{macrocode}
\declare@shorthand{portuges}{"-}{\allowhyphens-\allowhyphens}
\declare@shorthand{portuges}{""}{\hskip\z@skip}
%    \end{macrocode}
%    And we want to have a shorthand for disabling a ligature.
%    \begin{macrocode}
\declare@shorthand{portuges}{"|}{%
  \textormath{\discretionary{-}{}{\kern.03em}}{}}
%    \end{macrocode}
% \end{macro}
% \end{macro}
%
%  \begin{macro}{\-}
%
%    All that is left now is the redefinition of |\-|. The new version
%    of |\-| should indicate an extra hyphenation position, while
%    allowing other hyphenation positions to be generated
%    automatically. The standard behaviour of \TeX\ in this respect is
%    very unfortunate for languages such as Dutch and German, where
%    long compound words are quite normal and all one needs is a means
%    to indicate an extra hyphenation position on top of the ones that
%    \TeX\ can generate from the hyphenation patterns.
%    \begin{macrocode}
\addto\extrasportuges{\babel@save\-}
\addto\extrasportuges{\def\-{\allowhyphens
                          \discretionary{-}{}{}\allowhyphens}}
%    \end{macrocode}
%  \end{macro}
%
%  \begin{macro}{\ord}
% \changes{portuges-1.2g}{1995/06/04}{Added macro}
%  \begin{macro}{\ro}
% \changes{portuges-1.2g}{1995/06/04}{Added macro}
%  \begin{macro}{\orda}
% \changes{portuges-1.2g}{1995/06/04}{Added macro}
%  \begin{macro}{\ra}
% \changes{portuges-1.2g}{1995/06/04}{Added macro}
%    We also provide an easy way to typeset ordinals, both in the male
%    (|\ord| or |\ro|) and the female (|orda| or |\ra|) form.
%    \begin{macrocode}
\def\ord{$^{\rm o}$}
\def\orda{$^{\rm a}$}
\let\ro\ord\let\ra\orda
%    \end{macrocode}
%  \end{macro}
%  \end{macro}
%  \end{macro}
%  \end{macro}
%
% \begin{macro}{\extrasbrazil}
% \begin{macro}{\noextrasbrazil}
%    Also for the ``brazil'' variant no extra definitions are needed
%    at the moment.
%    \begin{macrocode}
\let\extrasbrazil\extrasportuges
\let\noextrasbrazil\noextrasportuges
%    \end{macrocode}
% \end{macro}
% \end{macro}
%
%    It is possible that a site might need to add some extra code to
%    the babel macros. To enable this we load a local configuration
%    file, \file{portuges.cfg} if it is found on \TeX' search path.
% \changes{portuges-1.2h}{1995/07/02}{Added loading of configuration
%    file}
%    \begin{macrocode}
\loadlocalcfg{portuges}
%    \end{macrocode}
%
%    Our last action is to make a note that the commands we have just
%    defined, will be executed by calling the macro |\selectlanguage|
%    at the beginning of the document.
% \changes{portuges-1.2f}{1995/03/14}{Use \cs{main@language} instead
%    of \cs{selectlanguage}}
%    \begin{macrocode}
\main@language{portuges}
%    \end{macrocode}
%    Finally, the category code of \texttt{@} is reset to its original
%    value. The macrospace used by |\atcatcode| is freed.
% \changes{portuges-1.0a}{1991/07/15}{Modified handling of catcode of
%    @-sign.}
%    \begin{macrocode}
\catcode`\@=\atcatcode \let\atcatcode\relax
%</code>
%    \end{macrocode}
%
% \Finale
%%
%% \CharacterTable
%%  {Upper-case    \A\B\C\D\E\F\G\H\I\J\K\L\M\N\O\P\Q\R\S\T\U\V\W\X\Y\Z
%%   Lower-case    \a\b\c\d\e\f\g\h\i\j\k\l\m\n\o\p\q\r\s\t\u\v\w\x\y\z
%%   Digits        \0\1\2\3\4\5\6\7\8\9
%%   Exclamation   \!     Double quote  \"     Hash (number) \#
%%   Dollar        \$     Percent       \%     Ampersand     \&
%%   Acute accent  \'     Left paren    \(     Right paren   \)
%%   Asterisk      \*     Plus          \+     Comma         \,
%%   Minus         \-     Point         \.     Solidus       \/
%%   Colon         \:     Semicolon     \;     Less than     \<
%%   Equals        \=     Greater than  \>     Question mark \?
%%   Commercial at \@     Left bracket  \[     Backslash     \\
%%   Right bracket \]     Circumflex    \^     Underscore    \_
%%   Grave accent  \`     Left brace    \{     Vertical bar  \|
%%   Right brace   \}     Tilde         \~}
%%
\endinput
%
  \let\captionsbrazilian\captionsbrazil
  \let\datebrazilian\datebrazil
  \let\extrasbrazilian\extrasbrazil
  \let\noextrasbrazilian\noextrasbrazil
  \let\brazilianhyphenmins\brazilhyphenmins
  \main@language{brazilian}}
\DeclareOption{breton}{% \iffalse meta-comment
%
% Copyright 1989-1995 Johannes L. Braams and any individual authors
% listed elsewhere in this file.  All rights reserved.
% 
% For further copyright information any other copyright notices in this
% file.
% 
% This file is part of the Babel system release 3.5.
% --------------------------------------------------
%   This system is distributed in the hope that it will be useful,
%   but WITHOUT ANY WARRANTY; without even the implied warranty of
%   MERCHANTABILITY or FITNESS FOR A PARTICULAR PURPOSE.
% 
%   For error reports concerning UNCHANGED versions of this file no more
%   than one year old, see bugs.txt.
% 
%   Please do not request updates from me directly.  Primary
%   distribution is through the CTAN archives.
% 
% 
% IMPORTANT COPYRIGHT NOTICE:
% 
% You are NOT ALLOWED to distribute this file alone.
% 
% You are allowed to distribute this file under the condition that it is
% distributed together with all the files listed in manifest.txt.
% 
% If you receive only some of these files from someone, complain!
% 
% Permission is granted to copy this file to another file with a clearly
% different name and to customize the declarations in that copy to serve
% the needs of your installation, provided that you comply with
% the conditions in the file legal.txt from the LaTeX2e distribution.
% 
% However, NO PERMISSION is granted to produce or to distribute a
% modified version of this file under its original name.
%  
% You are NOT ALLOWED to change this file.
% 
% 
% \fi
% \CheckSum{295}
%
% \iffalse
%    Tell the \LaTeX\ system who we are and write an entry on the
%    transcript.
%    \begin{macrocode}
%<*dtx>
\ProvidesFile{breton.dtx}
%</dtx>
%<code>\ProvidesFile{breton.ldf}
        [1995/07/09 v1.0c Breton support from the babel system]
%    \end{macrocode}
%
% Babel package for LaTeX version 2e
% Copyright (C) 1989 - 1995
%           by Johannes Braams, TeXniek
%
% Breton Language Definition File
% Copyright (C) 1994 - 1995
%           by Christian Rolland
%              Universite de Bretagne occidentale
%              Departement d'informatique
%              6, avenue Le Gorgeu
%              BP 452
%              29275 Brest Cedex -- FRANCE
%              Christian.Rolland@univ-brest.fr (Internet)
%
%              Johannes Braams, TeXniek
%              Kooienswater 62
%              2715 AJ Zoetermeer
%              The Netherlands
%
% Please report errors to: J.L. Braams
%                          JLBraams@cistron.nl
%
%    This file is part of the babel system, it provides the source
%    code for the Breton language definition file. It is based on the
%    language definition file for french, version 4.5c
%<*filedriver>
\documentclass{ltxdoc}
\newcommand*\TeXhax{\TeX hax}
\newcommand*\babel{\textsf{babel}}
\newcommand*\langvar{$\langle \it lang \rangle$}
\newcommand*\note[1]{}
\newcommand*\Lopt[1]{\textsf{#1}}
\newcommand*\file[1]{\texttt{#1}}
\begin{document}
 \DocInput{breton.dtx}
\end{document}
%</filedriver>
%\fi
% \GetFileInfo{breton.dtx}
%
% \changes{breton-1.0}{1994/09/21}{First release}
%
%  \section{The Breton language}
%
%    The file \file{\filename}\footnote{The file described in this
%    section has version number \fileversion\ and was last revised on
%    \filedate.} defines all the language-specific macros for the Breton
%    language. 
%
%    There are not really typographic rules for the Breton
%    language. It is a local language (it's one of the celtic
%    languages) which is spoken in Brittany (West of France). So we
%    have a synthesis between french typographic rules and english
%    typographic rules. The characters \texttt{:}, \texttt{;},
%    \texttt{!} and \texttt{?} are made active in order to get a
%    whitespace automatically before these characters.
%
% \StopEventually{}
%
%    As this file needs to be read only once, we check whether it was
%    read before. If it was, the |\captionsbreton| is already
%    defined, so we can stop processing. If this command is undefined
%    we proceed with the various definitions and first show the
%    current version of this file.
%
%    \begin{macrocode}
%<*code>
\ifx\undefined\captionsbreton
\else
  \selectlanguage{breton}
  \expandafter\endinput
\fi
%    \end{macrocode}
%
% \begin{macro}{\atcatcode}
%    This file, \file{breton.ldf}, may have been read while \TeX\ is
%    in the middle of processing a document, so we have to make sure
%    the category code of \texttt{@} is `letter' while this file is
%    being read. We save the category code of the @-sign in
%    |\atcatcode| and make it `letter'. Later the category code can be
%    restored to whatever it was before.
%    \begin{macrocode}
\chardef\atcatcode=\catcode`\@
\catcode`\@=11\relax
%    \end{macrocode}
% \end{macro}
%
%    Now we determine whether the the common macros from the file
%    \file{babel.def} need to be read. We can be in one of two
%    situations: either another language option has been read earlier
%    on, in which case that other option has already read
%    \file{babel.def}, or \texttt{breton} is the first language option
%    to be processed. In that case we need to read \file{babel.def}
%    right here before we continue.
%    \begin{macrocode}
\ifx\undefined\babel@core@loaded\input babel.def\relax\fi
%    \end{macrocode}
%
%    Another check that has to be made, is if another language
%    definition file has been read already. In that case its
%    definitions have been activated. This might interfere with
%    definitions this file tries to make. Therefore we make sure that
%    we cancel any special definitions. This can be done by checking
%    the existence of the macro |\originalTeX|. If it exists we simply
%    execute it, otherwise it is |\let| to |\empty|.
%    \begin{macrocode}
\ifx\undefined\originalTeX \let\originalTeX\empty \fi
\originalTeX
%    \end{macrocode}
%
%    When this file is read as an option, i.e. by the |\usepackage|
%    command, \texttt{breton} will be an `unknown' language in which
%    case we have to make it known.  So we check for the existence of
%    |\l@breton| to see whether we have to do something here.
%
%    \begin{macrocode}
\ifx\undefined\l@breton
    \@nopatterns{Breton}
    \adddialect\l@breton0\fi
%    \end{macrocode}
%    The next step consists of defining commands to switch to the
%    English language. The reason for this is that a user might want
%    to switch back and forth between languages.
%
% \begin{macro}{\captionsbreton}
%    The macro |\captionsbreton| defines all strings used in the
%    four standard document classes provided with \LaTeX.
% \changes{breton-1.0b}{1995/07/04}{Added \cs{proofname} for
%    AMS-\LaTeX}
%    \begin{macrocode}
\addto\captionsbreton{%
  \def\prefacename{Rakskrid}%
  \def\refname{Daveenno\`u}%
  \def\abstractname{Dvierra\~n}%
  \def\bibname{Lennadurezh}%
  \def\chaptername{Pennad}%
  \def\appendixname{Stagadenn}%
  \def\contentsname{Taolenn}%
  \def\listfigurename{Listenn ar Figurenno\`u}%
  \def\listtablename{Listenn an taolenno\`u}%
  \def\indexname{Meneger}%
  \def\figurename{Figurenn}%
  \def\tablename{Taolenn}%
  \def\partname{Lodenn}%
  \def\enclname{Diello\`u kevret}%
  \def\ccname{Eilskrid da}%
  \def\headtoname{evit}
  \def\pagename{Pajenn}%
  \def\seename{Gwelout}%
  \def\alsoname{Gwelout ivez}%
  \def\proofname{Proof}%  <-- needs translation
}
%    \end{macrocode}
% \end{macro}
%
% \begin{macro}{\datebreton}
%    The macro |\datebreton| redefines the command
%    |\today| to produce Breton dates.
%    \begin{macrocode}
\def\datebreton{%
\def\today{\ifnum\day=1\relax 1\/$^{\rm a\tilde{n}}$\else
  \number\day\fi \space a\space viz\space\ifcase\month\or
  Genver\or C'hwevrer\or Meurzh\or Ebrel\or Mae\or Mezheven\or
  Gouere\or Eost\or Gwengolo\or Here\or Du\or Kerzu\fi
  \space\number\year}}
%    \end{macrocode}
% \end{macro}
%
% \begin{macro}{\extrasbreton}
% \begin{macro}{\noextrasbreton}
%    The macro |\extrasbreton| will perform all the extra
%    definitions needed for the Breton language. The macro
%    |\noextrasbreton| is used to cancel the actions of
%    |\extrasbreton|.
%
%    The category code of the characters \texttt{:}, \texttt{;},
%    \texttt{!} and \texttt{?} is made |\active| to insert a little
%    white space.
% \changes{breton-4.6}{1995/03/07}{Use the new mechanism for dealing
%    with active chars}
%    \begin{macrocode}
\initiate@active@char{:}
\initiate@active@char{;}
\initiate@active@char{!}
\initiate@active@char{?}
%    \end{macrocode}
%    We specify that the breton group of shorthands should be used.
%    \begin{macrocode}
\addto\extrasbreton{\languageshorthands{breton}}
%    \end{macrocode}
%    These characters are `turned on' once, later their definition may
%    vary. 
%    \begin{macrocode}
\addto\extrasbreton{%
  \bbl@activate{:}\bbl@activate{;}%
  \bbl@activate{!}\bbl@activate{?}}
%\addto\noextrasbreton{%
%  \bbl@deactivate{:}\bbl@deactivate{;}%
%  \bbl@deactivate{!}\bbl@deactivate{?}}
%    \end{macrocode}
%
%    The last thing |\extrasbreton| needs to do is to make sure that
%    |\frenchspacing| is in effect.  If this is not the case the
%    execution of |\noextrasbreton| will switch it of again.
%    \begin{macrocode}
\addto\extrasbreton{\bbl@frenchspacing}
\addto\noextrasbreton{\bbl@nonfrenchspacing}
%    \end{macrocode}
% \end{macro}
% \end{macro}
%
% \begin{macro}{\breton@sh@;@}
%    We have to reduce the amount of white space before \texttt{;},
%    \texttt{:} and \texttt{!} when the user types a space in front of
%    these characters. This should only happen outside mathmode, hence
%    the test with |\ifmmode|.
%
%    \begin{macrocode}
\declare@shorthand{breton}{;}{%
    \ifmmode
      \string;\space
    \else\relax
%    \end{macrocode}
%    In horizontal mode we check for the presence of a `space' and
%    replace it by a |\thinspace|.
%    \begin{macrocode}
      \ifhmode
        \ifdim\lastskip>\z@
          \unskip\penalty\@M\thinspace
        \fi
      \fi
      \string;\space
    \fi}%
%    \end{macrocode}
% \end{macro}
%
% \begin{macro}{\breton@sh@:@}
% \begin{macro}{\breton@sh@!@}
%    Because these definitions are very similar only one is displayed
%    in a way that the definition can be easily checked.
%    \begin{macrocode}
\declare@shorthand{breton}{:}{%
  \ifmmode\string:\space
  \else\relax
    \ifhmode
      \ifdim\lastskip>\z@\unskip\penalty\@M\thinspace\fi
    \fi
    \string:\space
  \fi}
\declare@shorthand{breton}{!}{%
  \ifmmode\string!\space
  \else\relax
    \ifhmode
      \ifdim\lastskip>\z@\unskip\penalty\@M\thinspace\fi
    \fi
    \string!\space
  \fi}
%    \end{macrocode}
% \end{macro}
% \end{macro}
%
% \begin{macro}{\breton@sh@?@}
%    For the question mark something different has to be done. In this
%    case the amount of white space that replaces the space character
%    depends on the dimensions of the font.
%    \begin{macrocode}
\declare@shorthand{breton}{?}{%
  \ifmmode
    \string?\space
  \else\relax
    \ifhmode
      \ifdim\lastskip>\z@
        \unskip
        \kern\fontdimen2\font
        \kern-1.4\fontdimen3\font
      \fi
    \fi
    \string?\space
  \fi}
%    \end{macrocode}
% \end{macro}
%
%    All that is left to do now is provide the breton user with some
%    extra utilities.
%
%    Some definitions for special characters.
%    \begin{macrocode}
\DeclareTextSymbol{\at}{OT1}{64}
\DeclareTextSymbol{\at}{T1}{64}
\DeclareTextSymbolDefault{\at}{OT1}
\DeclareTextSymbol{\boi}{OT1}{92}
\DeclareTextSymbol{\boi}{T1}{16}
\DeclareTextSymbolDefault{\boi}{OT1}
\DeclareTextSymbol{\circonflexe}{OT1}{94}
\DeclareTextSymbol{\circonflexe}{T1}{2}
\DeclareTextSymbolDefault{\circonflexe}{OT1}
\DeclareTextSymbol{\tild}{OT1}{126}
\DeclareTextSymbol{\tild}{T1}{3}
\DeclareTextSymbolDefault{\tild}{OT1}
\DeclareTextSymbol{\degre}{OT1}{23}
\DeclareTextSymbol{\degre}{T1}{6}
\DeclareTextSymbolDefault{\degre}{OT1}
%    \end{macrocode}
%
%    The following macros are used in the redefinition of |\^| and
%    |\"| to handle the letter i.
% \changes{breton-1.0c}{1995/07/07}{Postpone the declaration of the
%    TextCompositeCommands untill \cs{AtBeginDocument}}
%
%    \begin{macrocode}
\AtBeginDocument{%
  \DeclareTextCompositeCommand{\^}{OT1}{i}{\^\i}
  \DeclareTextCompositeCommand{\"}{OT1}{i}{\"\i}} 
%    \end{macrocode}
%
%    And some more macros for numbering.
%    \begin{macrocode}
\def\kentan{1\/${}^{\rm a\tilde{n}}$}
\def\eil{2\/${}^{\rm l}$}
\def\re{\/${}^{\rm re}$}
\def\trede{3\re}
\def\pevare{4\re}
\def\vet{\/${}^{\rm vet}$}
\def\pempvet{5\vet}
%    \end{macrocode}
%
%    It is possible that a site might need to add some extra code to
%    the babel macros. To enable this we load a local configuration
%    file, \file{breton.cfg} if it is found on \TeX' search path.
% \changes{breton-1.0b}{1995/07/02}{Added loading of configuration
%    file}
%    \begin{macrocode}
\loadlocalcfg{breton}
%    \end{macrocode}
%
%    Our last action is to make a note that the commands we have just
%    defined, will be executed by calling the macro |\selectlanguage|
%    at the beginning of the document.
%    \begin{macrocode}
\main@language{breton}
%    \end{macrocode}
%    Finally, the category code of \texttt{@} is reset to its original
%    value. The macrospace used by |\atcatcode| is freed.
%    \begin{macrocode}
\catcode`\@=\atcatcode \let\atcatcode\relax
%</code>
%    \end{macrocode}
%
% \Finale
%%
%% \CharacterTable
%%  {Upper-case    \A\B\C\D\E\F\G\H\I\J\K\L\M\N\O\P\Q\R\S\T\U\V\W\X\Y\Z
%%   Lower-case    \a\b\c\d\e\f\g\h\i\j\k\l\m\n\o\p\q\r\s\t\u\v\w\x\y\z
%%   Digits        \0\1\2\3\4\5\6\7\8\9
%%   Exclamation   \!     Double quote  \"     Hash (number) \#
%%   Dollar        \$     Percent       \%     Ampersand     \&
%%   Acute accent  \'     Left paren    \(     Right paren   \)
%%   Asterisk      \*     Plus          \+     Comma         \,
%%   Minus         \-     Point         \.     Solidus       \/
%%   Colon         \:     Semicolon     \;     Less than     \<
%%   Equals        \=     Greater than  \>     Question mark \?
%%   Commercial at \@     Left bracket  \[     Backslash     \\
%%   Right bracket \]     Circumflex    \^     Underscore    \_
%%   Grave accent  \`     Left brace    \{     Vertical bar  \|
%%   Right brace   \}     Tilde         \~}
%%
\endinput
}
%    \end{macrocode}
% \changes{babel~3.5d}{1995/07/02}{Added british as an alternative for
%    `english' with a preference for british hyphenation}
%    \begin{macrocode}
\DeclareOption{british}{%
  \ifx\l@british\undefined
    \ifx\l@UKenglish\undefined
    \else
      \let\l@english\l@UKenglish
    \fi
  \else
    \let\l@english\l@british
  \fi
  % \iffalse meta-comment
%
% Copyright 1989-1995 Johannes L. Braams and any individual authors
% listed elsewhere in this file.  All rights reserved.
% 
% For further copyright information any other copyright notices in this
% file.
% 
% This file is part of the Babel system release 3.5.
% --------------------------------------------------
%   This system is distributed in the hope that it will be useful,
%   but WITHOUT ANY WARRANTY; without even the implied warranty of
%   MERCHANTABILITY or FITNESS FOR A PARTICULAR PURPOSE.
% 
%   For error reports concerning UNCHANGED versions of this file no more
%   than one year old, see bugs.txt.
% 
%   Please do not request updates from me directly.  Primary
%   distribution is through the CTAN archives.
% 
% 
% IMPORTANT COPYRIGHT NOTICE:
% 
% You are NOT ALLOWED to distribute this file alone.
% 
% You are allowed to distribute this file under the condition that it is
% distributed together with all the files listed in manifest.txt.
% 
% If you receive only some of these files from someone, complain!
% 
% Permission is granted to copy this file to another file with a clearly
% different name and to customize the declarations in that copy to serve
% the needs of your installation, provided that you comply with
% the conditions in the file legal.txt from the LaTeX2e distribution.
% 
% However, NO PERMISSION is granted to produce or to distribute a
% modified version of this file under its original name.
%  
% You are NOT ALLOWED to change this file.
% 
% 
% \fi
% \CheckSum{194}
% \iffalse
%    Tell the \LaTeX\ system who we are and write an entry on the
%    transcript.
%<*dtx>
\ProvidesFile{english.dtx}
%</dtx>
%<code>\ProvidesFile{english.ldf}
        [1995/07/04 v3.3e English support from the babel system]
%
% Babel package for LaTeX version 2e
% Copyright (C) 1989 - 1995
%           by Johannes Braams, TeXniek
%
% Please report errors to: J.L. Braams
%                          JLBraams@cistron.nl
%
%    This file is part of the babel system, it provides the source
%    code for the English language definition file.
%<*filedriver>
\documentclass{ltxdoc}
\newcommand*\TeXhax{\TeX hax}
\newcommand*\babel{\textsf{babel}}
\newcommand*\langvar{$\langle \mathit lang \rangle$}
\newcommand*\note[1]{}
\newcommand*\Lopt[1]{\textsf{#1}}
\newcommand*\file[1]{\texttt{#1}}
\begin{document}
 \DocInput{english.dtx}
\end{document}
%</filedriver>
%\fi
% \GetFileInfo{english.dtx}
%
% \changes{english-2.0a}{1990/04/02}{Added checking of format}
% \changes{english-2.1}{1990/04/24}{Reflect changes in babel 2.1}
% \changes{english-2.1a}{1990/05/14}{Incorporated Nico's comments}
% \changes{english-2.1b}{1990/05/14}{merged \file{USenglish.sty} into
%    this file}
% \changes{english-2.1c}{1990/05/22}{fixed typo in definition for
%    american language found by Werenfried Spit (nspit@fys.ruu.nl)}
% \changes{english-2.1d}{1990/07/16}{Fixed some typos}
% \changes{english-3.0}{1991/04/23}{Modified for babel 3.0}
% \changes{english-3.0a}{1991/05/29}{Removed bug found by van der Meer}
% \changes{english-3.0c}{1991/07/15}{Renamed \file{babel.sty} in
%    \file{babel.com}}
% \changes{english-3.1}{1991/11/05}{Rewrote parts of the code to use
%    the new features of babel version 3.1}
% \changes{english-3.3}{1994/02/08}{Update or \LaTeXe}
% \changes{english-3.3c}{1994/06/26}{Removed the use of \cs{filedate}
%    and moved the identification after the loading of
%    \file{babel.def}}
%
%  \section{The English language}
%
%    The file \file{\filename}\footnote{The file described in this
%    section has version number \fileversion\ and was last revised on
%    \filedate.} defines all the language definition macros for the
%    English language as well as for the American version of this
%    language.
%
%    For this language currently no special definitions are needed or
%    available.
%
% \StopEventually{}
%
% \changes{english-3.0d}{1991/10/22}{Removed code to load
%    \file{latexhax.com}}
%
%    As this file needs to be read only once, we check whether it was
%    read before. If it was, the command |\captionsenglish| is already
%    defined, so we can stop processing. If this command is undefined
%    we proceed with the various definitions and first show the
%    current version of this file.
%
% \changes{english-3.0c}{1991/07/15}{Added reset of catcode of @
%    before \cs{endinput}.}
% \changes{english-3.0d}{1991/10/22}{removed use of \cs{@ifundefined}}
% \changes{english-3.1a}{1991/11/11}{Moved code to the beginning of
%    the file and added \cs{selectlanguage} call}
%    \begin{macrocode}
%<*code>
\ifx\undefined\captionsenglish
\else
  \selectlanguage{english}
  \expandafter\endinput
\fi
%    \end{macrocode}
%
% \begin{macro}{\atcatcode}
%    This file, \file{english.ldf}, may have been read while \TeX\ is
%    in the middle of processing a document, so we have to make sure
%    the category code of \texttt{@} is `letter' while this file is
%    being read. We save the category code of the @-sign in
%    |\atcatcode| and make it `letter'. Later the category code can be
%    restored to whatever it was before.
% \changes{english-3.0b}{1991/06/06}{Made test of catcode of @ more
%    robust}
% \changes{english-3.0c}{1991/07/15}{Modified handling of catcode of @
%    again.}
% \changes{english-3.0d}{1991/10/22}{Removed use of \cs{makeatletter}
%    and hence the need to load \file{latexhax.com}}
%    \begin{macrocode}
\chardef\atcatcode=\catcode`\@
\catcode`\@=11\relax
%    \end{macrocode}
% \end{macro}
%
%    Now we determine whether the common macros from the file
%    \file{babel.def} need to be read. We can be in one of two
%    situations: either another language option has been read earlier
%    on, in which case that other option has already read
%    \file{babel.def}, or \texttt{english} is the first language
%    option to be processed. In that case we need to read
%    \file{babel.def} right here before we continue.
%
% \changes{english-3.0}{1991/04/23}{New check before loading
%    \file{babel.com}}
% \changes{english-3.1c}{1992/02/15}{Added \cs{relax} after the
%    argument of \cs{input}}
%    \begin{macrocode}
\ifx\undefined\babel@core@loaded\input babel.def\relax\fi
%    \end{macrocode}
%
% \changes{english-3.0a}{1991/05/29}{Add a check for existence
%    \cs{originalTeX}}
%
%    Another check that has to be made, is if another language
%    definition file has been read already. In that case its
%    definitions have been activated. This might interfere with
%    definitions this file tries to make. Therefore we make sure that
%    we cancel any special definitions. This can be done by checking
%    the existence of the macro |\originalTeX|. If it exists we simply
%    execute it, otherwise it is |\let| to |\empty|.
% \changes{english-3.0c}{1991/07/15}{Added
%    \cs{let}\cs{originalTeX}\cs{relax} to test for existence}
% \changes{english-3.1b}{1992/01/26}{\cs{originalTeX} should be
%    expandable, \cs{let} it to \cs{empty}}
%    \begin{macrocode}
\ifx\undefined\originalTeX \let\originalTeX\empty\fi
\originalTeX
%    \end{macrocode}
%
%    When this file is read as an option, i.e. by the |\usepackage|
%    command, \texttt{english} could be an `unknown' language in which
%    case we have to make it known.  So we check for the existence of
%    |\l@english| to see whether we have to do something here.
%
% \changes{english-3.0}{1991/04/23}{Now use \cs{adddialect} if
%    language undefined}
% \changes{english-3.0d}{1991/10/22}{removed use of \cs{@ifundefined}}
% \changes{english-3.3c}{1994/06/26}{Now use \cs{@nopatterns} to
%    produce the warning}
%    \begin{macrocode}
\ifx\undefined\l@english
  \ifx\undefined\l@UKenglish
    \@nopatterns{English}
    \adddialect\l@english0
  \else
    \let\l@english\l@UKenglish
  \fi
\fi
%    \end{macrocode}
%    For the American version of these definitions we just add a
%    ``dialect''. Also, the macros |\captionsamerican| and
%    |\extrasamerican| are |\let| to their English counterparts when
%    these parts are defined.
% \changes{english-3.0}{1990/04/23}{Now use \cs{adddialect} for
%    american}
% \changes{english-3.0b}{1991/06/06}{Removed \cs{global} definitions}
% \changes{english-v3.3d}{1995/02/01}{Only define american as a
%    dialect when no separate patterns have been loaded}
%    \begin{macrocode}
\ifx\l@american\undefined
  \adddialect\l@american\l@english
\fi
%    \end{macrocode}
%
%    The next step consists of defining commands to switch to (and
%    from) the English language.
%
% \begin{macro}{\captionsenglish}
%    The macro |\captionsenglish| defines all strings used
%    in the four standard document classes provided with \LaTeX.
% \changes{english-3.0b}{1991/06/06}{Removed \cs{global} definitions}
% \changes{english-3.0b}{1991/06/06}{\cs{pagename} should be
%    \cs{headpagename}}
% \changes{english-3.1a}{1991/11/11}{added \cs{seename} and
%    \cs{alsoname}}
% \changes{english-3.1b}{1992/01/26}{added \cs{prefacename}}
% \changes{english-3.2}{1993/07/15}{\cs{headpagename} should be
%    \cs{pagename}}
% \changes{english-3.3e}{1995/07/04}{Added \cs{proofname} for
%    AMS-\LaTeX}
%    \begin{macrocode}
\addto\captionsenglish{%
  \def\prefacename{Preface}%
  \def\refname{References}%
  \def\abstractname{Abstract}%
  \def\bibname{Bibliography}%
  \def\chaptername{Chapter}%
  \def\appendixname{Appendix}%
  \def\contentsname{Contents}%
  \def\listfigurename{List of Figures}%
  \def\listtablename{List of Tables}%
  \def\indexname{Index}%
  \def\figurename{Figure}%
  \def\tablename{Table}%
  \def\partname{Part}%
  \def\enclname{encl}%
  \def\ccname{cc}%
  \def\headtoname{To}%
  \def\pagename{Page}%
  \def\seename{see}%
  \def\alsoname{see also}%
  \def\proofname{Proof}%
  }
%    \end{macrocode}
% \end{macro}
%
% \begin{macro}{\captionsamerican}
%    The `captions' are the same for both versions of the language, so
%    we can |\let| the macro |\captionsamerican| be equal to
%    |\captionsenglish|.
%    \begin{macrocode}
\let\captionsamerican\captionsenglish
%    \end{macrocode}
% \end{macro}
%
% \begin{macro}{\dateenglish}
%    The macro |\dateenglish| redefines the command |\today| to
%    produce English dates.
% \changes{english-3.0b}{1991/06/06}{Removed \cs{global} definitions}
%    \begin{macrocode}
\def\dateenglish{%
\def\today{\ifcase\day\or
  1st\or 2nd\or 3rd\or 4th\or 5th\or
  6th\or 7th\or 8th\or 9th\or 10th\or
  11th\or 12th\or 13th\or 14th\or 15th\or
  16th\or 17th\or 18th\or 19th\or 20th\or
  21st\or 22nd\or 23rd\or 24th\or 25th\or
  26th\or 27th\or 28th\or 29th\or 30th\or
  31st\fi~\ifcase\month\or
  January\or February\or March\or April\or May\or June\or
  July\or August\or September\or October\or November\or December\fi
  \space \number\year}}
%    \end{macrocode}
% \end{macro}
%
% \begin{macro}{\dateamerican}
%    The macro |\dateamerican| redefines the command |\today| to
%    produce American dates.
% \changes{english-3.0b}{1991/06/06}{Removed \cs{global} definitions}
%    \begin{macrocode}
\def\dateamerican{%
\def\today{\ifcase\month\or
  January\or February\or March\or April\or May\or June\or
  July\or August\or September\or October\or November\or December\fi
  \space\number\day, \number\year}}
%    \end{macrocode}
% \end{macro}
%
% \begin{macro}{\extrasenglish}
% \begin{macro}{\noextrasenglish}
%    The macro |\extrasenglish| will perform all the extra definitions
%    needed for the English language. The macro |\extrasenglish| is
%    used to cancel the actions of |\extrasenglish|.  For the moment
%    these macros are empty but they are defined for compatibility
%    with the other language definition files.
%
%    \begin{macrocode}
\addto\extrasenglish{}
\addto\noextrasenglish{}
%    \end{macrocode}
% \end{macro}
% \end{macro}
%
% \begin{macro}{\extrasamerican}
% \begin{macro}{\noextrasamerican}
%    Also for the ``american'' variant no extra definitions are needed
%    at the moment.
%    \begin{macrocode}
\let\extrasamerican\extrasenglish
\let\noextrasamerican\noextrasenglish
%    \end{macrocode}
% \end{macro}
% \end{macro}
%
%    It is possible that a site might need to add some extra code to
%    the babel macros. To enable this we load a local configuration
%    file, \file{english.cfg} if it is found on \TeX' search path.
% \changes{english-3.3e}{1995/07/02}{Added loading of configuration
%    file}
%    \begin{macrocode}
\loadlocalcfg{english}
%    \end{macrocode}
%
%    Our last action is to make a note that the commands we have just
%    defined, will be executed by calling the macro |\selectlanguage|
%    at the beginning of the document.
%    \begin{macrocode}
\main@language{english}
%    \end{macrocode}
%    Finally, the category code of \texttt{@} is reset to its original
%    value. The macrospace used by |\atcatcode| is freed.
% \changes{english-3.0c}{1991/07/15}{Modified handling of catcode of
%    @-sign.}
%    \begin{macrocode}
\catcode`\@=\atcatcode \let\atcatcode\relax
%</code>
%    \end{macrocode}
%
% \Finale
%%
%% \CharacterTable
%%  {Upper-case    \A\B\C\D\E\F\G\H\I\J\K\L\M\N\O\P\Q\R\S\T\U\V\W\X\Y\Z
%%   Lower-case    \a\b\c\d\e\f\g\h\i\j\k\l\m\n\o\p\q\r\s\t\u\v\w\x\y\z
%%   Digits        \0\1\2\3\4\5\6\7\8\9
%%   Exclamation   \!     Double quote  \"     Hash (number) \#
%%   Dollar        \$     Percent       \%     Ampersand     \&
%%   Acute accent  \'     Left paren    \(     Right paren   \)
%%   Asterisk      \*     Plus          \+     Comma         \,
%%   Minus         \-     Point         \.     Solidus       \/
%%   Colon         \:     Semicolon     \;     Less than     \<
%%   Equals        \=     Greater than  \>     Question mark \?
%%   Commercial at \@     Left bracket  \[     Backslash     \\
%%   Right bracket \]     Circumflex    \^     Underscore    \_
%%   Grave accent  \`     Left brace    \{     Vertical bar  \|
%%   Right brace   \}     Tilde         \~}
%%
\endinput
%
  \ifx\l@british\undefined
    \let\l@british\l@english
  \fi
  \let\captionsbritish\captionsenglish
  \let\datebritish\dateenglish
  \let\extrasbritish\extrasenglish
  \let\britishhyphenmins\englishhyphenmins
  \main@language{british}
  }
\DeclareOption{catalan}{% \iffalse meta-comment
%
% Copyright 1989-1995 Johannes L. Braams and any individual authors
% listed elsewhere in this file.  All rights reserved.
% 
% For further copyright information any other copyright notices in this
% file.
% 
% This file is part of the Babel system release 3.5.
% --------------------------------------------------
%   This system is distributed in the hope that it will be useful,
%   but WITHOUT ANY WARRANTY; without even the implied warranty of
%   MERCHANTABILITY or FITNESS FOR A PARTICULAR PURPOSE.
% 
%   For error reports concerning UNCHANGED versions of this file no more
%   than one year old, see bugs.txt.
% 
%   Please do not request updates from me directly.  Primary
%   distribution is through the CTAN archives.
% 
% 
% IMPORTANT COPYRIGHT NOTICE:
% 
% You are NOT ALLOWED to distribute this file alone.
% 
% You are allowed to distribute this file under the condition that it is
% distributed together with all the files listed in manifest.txt.
% 
% If you receive only some of these files from someone, complain!
% 
% Permission is granted to copy this file to another file with a clearly
% different name and to customize the declarations in that copy to serve
% the needs of your installation, provided that you comply with
% the conditions in the file legal.txt from the LaTeX2e distribution.
% 
% However, NO PERMISSION is granted to produce or to distribute a
% modified version of this file under its original name.
%  
% You are NOT ALLOWED to change this file.
% 
% 
% \fi
% \CheckSum{491}
%
% \iffalse
%    Tell the \LaTeX\ system who we are and write an entry on the
%    transcript.
%<*dtx>
\ProvidesFile{catalan.dtx}
%</dtx>
%<code>\ProvidesFile{catalan.ldf}
        [1995/07/10 v2.2d Catalan support from the babel system]
%
% Babel package for LaTeX version 2e
% Copyright (C) 1989 - 1995
%           by Johannes Braams, TeXniek
%
% Catalan Language Definition File
% Copyright (C) 1991 - 1995
%           by Goncal Badenes <badenes@imec.be>
%              Johannes Braams, TeXniek
%
% Please report errors to: J.L. Braams <JLBraams@cistron.nl>
%
%    This file is part of the babel system, it provides the source
%    code for the Catalan language definition file.
%    This file was developped out of spanish.sty and suggestions by
%    Goncal Badenes <badenes@imec.be> and Joerg Knappen
%    <knappen@vkpmzd.kph.uni-mainz.de>.
%
%    The file spanish.sty was written by Julio Sanchez,
%    (jsanchez@gmv.es) The code for the catalan language has been
%    removed and now is in this file.
%<*filedriver>
\documentclass{ltxdoc}
\newcommand*\TeXhax{\TeX hax}
\newcommand*\babel{\textsf{babel}}
\newcommand*\langvar{$\langle \it lang \rangle$}
\newcommand*\note[1]{}
\newcommand*\Lopt[1]{\textsf{#1}}
\newcommand*\file[1]{\texttt{#1}}
\begin{document}
 \DocInput{catalan.dtx}
\end{document}
%</filedriver>
%\fi
%
% \GetFileInfo{catalan.dtx}
%
% \changes{catalan-2.0b}{1993/09/23}{Incorporated the changes from
%    \file{spanish.sty}}
% \changes{catalan-2.1}{1994/02/27}{Update for \LaTeXe}
% \changes{catalan-2.1d}{1994/06/26}{Removed the use of \cs{filedate}
%    and moved identification after the loading of \file{babel.def}}
% \changes{catalan-2.2b}{1995/07/04}{Made the activation of the grave
%    and acute accents optional}
% \changes{catalan-2.2c}{1995/07/08}{Removed the use of the tilde for
%    catalan}
%
%  \section{The Catalan language}
%
%    The file \file{\filename}\footnote{The file described in this
%    section has version number \fileversion\ and was last revised on
%    \filedate.}  defines all the language-specific macro's for the
%    Catalan language.
%
%    For this language only the double quote character (|"|) is made
%    active by default. In table~\ref{tab:catalan-quote-def} an
%    overview is given of the new macros defined and the new meanings of
%    |"|. Additionally to that, the user can explicitly activate the
%    acute accent or apostrophe (|'|) and/or the grave accent (|`|)
%    characters by using the \Lopt{activeacute} and \Lopt{activegrave}
%    options. In that case, the definitions shown in
%    table~\ref{tab:catalan-quote-opt} become also
%    available\footnote{Please note that if the acute accent
%    character is active, it is necessary to take special care of coding
%    apostrophes in a way which cannot be confounded with
%    accents. Therefore, it is necessary, for example, to type
%    \texttt{l'\{\}\{\}astre} instead of \texttt{l'astre}.}.
%
%    \begin{table}[htb]
%     \centering
%     \begin{tabular}{lp{8cm}}
%      |\lgem|  & geminated-l digraph (similar to
%               l$\cdot$l). |\Lgem| produces the uppercase version.\\
%      |\up| & Macro to help typing raised ordinals, like {1\raise
%               1ex\hbox{\small er}}. Takes one argument.\\
%      |\-| & like the old |\-|, but allowing hyphenation
%               in the rest of the word. \\
%      |"i| & i with diaeresis, allowing hyphenation
%             in the rest of the word. Valid for the following vowels:
%               i, u (both lowercase and uppercase).\\
%      |"c| & c-cedilla (\c{c}). Valid for both uppercase and
%               lowercase c.\\
%      |"l| & geminated-l digraph (similar to
%               l$\cdot$l). Valid for both uppercase and lowercase l.\\
%      |"<| & French left double quotes (similar to $<<$).\\
%      |">| & French right double quotes (similar to $>>$).\\
%      |"-| & explicit hyphen sign, allowing hyphenation
%               in the rest of the word.\\
%      \verb="|= & disable ligature at this position.
%     \end{tabular}
%     \caption{Extra definitions made by file \file{catalan.ldf}
%       (activated by default)}
%     \label{tab:catalan-quote-def}
%    \end{table}
%
%    \begin{table}[htb]
%     \centering
%     \begin{tabular}{lp{8cm}}
%      |'e| & acute accented a, allowing hyphenation
%             in the rest of the word. Valid for the following
%             vowels: e, i, o, u (both lowercase and uppercase).\\
%      |`a| & grave accented a, allowing hyphenation
%             in the rest of the word. Valid for the following
%             vowels: a, e, o (both lowercase and uppercase).
%     \end{tabular}
%     \caption{Extra definitions made by file \file{catalan.ldf}
%       (activated only when using the options \Lopt{activeacute} and
%       \Lopt{activegrave})}
%     \label{tab:catalan-quote-opt}
%    \end{table}
%    These active accents characters behave according to their original
%    definitions if not followed by one of the characters indicated in
%    that table.
%
% \StopEventually{}
%
% \changes{catalan-2.0}{1993/07/11}{Removed code to load
%    \file{latexhax.com}}
%
%    As this file needs to be read only once, we check whether it was
%    read before. If it was, the command |\captionscatalan| is already
%    defined, so we can stop processing. If this command is undefined
%    we proceed with the various definitions and first show the
%    current version of this file.
%
%    \begin{macrocode}
%<*code>
\ifx\undefined\captionscatalan
\else
  \selectlanguage{catalan}
  \expandafter\endinput
\fi
%    \end{macrocode}
%
% \begin{macro}{\atcatcode}
%    This file, \file{catalan.ldf}, may have been read while \TeX\ is
%    in the middle of processing a document, so we have to make sure
%    the category code of \texttt{@} is `letter' while this file is
%    being read.  We save the category code of the @-sign in
%    |\atcatcode| and make it `letter'. Later the category code can be
%    restored to whatever it was before.
%
%    \begin{macrocode}
\chardef\atcatcode=\catcode`\@
\catcode`\@=11\relax
%    \end{macrocode}
% \end{macro}
%
%    Now we determine whether the common macros from the file
%    \file{babel.def} need to be read. We can be in one of two
%    situations: either another language option has been read earlier
%    on, in which case that other option has already read
%    \file{babel.def}, or \texttt{catalan} is the first language
%    option to be processed. In that case we need to read
%    \file{babel.def} right here before we continue.
%
%    \begin{macrocode}
\ifx\undefined\babel@core@loaded\input babel.def\fi
%    \end{macrocode}
%
%    Another check that has to be made, is if another language
%    definition file has been read already. In that case its
%    definitions have been activated. This might interfere with
%    definitions this file tries to make. Therefore we make sure that
%    we cancel any special definitions. This can be done by checking
%    the existence of the macro |\originalTeX|. If it exists we simply
%    execute it, otherwise it is |\let| to |\empty|.
%    \begin{macrocode}
\ifx\undefined\originalTeX
  \let\originalTeX\empty
\fi
\originalTeX
%    \end{macrocode}
%
%    When this file is read as an option, i.e. by the |\usepackage|
%    command, \texttt{catalan} could be an `unknown' language in which
%    case we have to make it known.  So we check for the existence of
%    |\l@catalan| to see whether we have to do something here.
%
% \changes{catalan-2.1d}{1994/06/26}{Now use \cs{@nopatterns} to
%    produce the warning}
%    \begin{macrocode}
\ifx\undefined\l@catalan
  \@nopatterns{Catalan}
  \adddialect\l@catalan0
\fi
%    \end{macrocode}
%
%    The next step consists of defining commands to switch to (and
%    from) the Catalan language.
%
% \begin{macro}{\captionscatalan}
%    The macro |\captionscatalan| defines all strings used
%    in the four standard documentclasses provided with \LaTeX.
% \changes{catalan-1.1}{1993/07/11}{\cs{headpagename} should be
%    \cs{pagename}}
% \changes{catalan-2.0}{1993/07/11}{Added some names}
% \changes{catalan-2.1d}{1994/11/09}{Added a few missing translations}
% \changes{catalan-2.2b}{1995/07/03}{Added \cs{proofname} for
%    AMS-\LaTeX}
% \changes{catalan-2.2d}{1995/07/10}{added translation of Proof}
%    \begin{macrocode}
\addto\captionscatalan{%
  \def\prefacename{Prefaci}%
  \def\refname{Refer\`encies}%
  \def\abstractname{Resum}%
  \def\bibname{Bibliografia}%
  \def\chaptername{Cap\'{\i}tol}%
  \def\appendixname{Ap\`endix}%
  \def\contentsname{\'Index}%
  \def\listfigurename{\'Index de figures}%
  \def\listtablename{\'Index de taules}%
  \def\indexname{\'Index de mat\`eries}%
  \def\figurename{Figura}%
  \def\tablename{Taula}%
  \def\partname{Part}%
  \def\enclname{Adjunt}%
  \def\ccname{C\`opia a}%
  \def\headtoname{A}%
  \def\pagename{P\`agina}%
  \def\seename{veure}%
  \def\alsoname{veure tamb\'e}%
  \def\proofname{Demostraci\'o}%
}
%    \end{macrocode}
% \end{macro}
%
% \begin{macro}{\datecatalan}
%    The macro |\datecatalan| redefines the command |\today| to
%    produce Catalan dates. Months are written in
%    lowercase\footnote{This seems to be the common practice. See for
%    example: E.~Coromina, \emph{El 9 Nou: Manual de redacci\'o i
%    estil}, Ed.~Eumo, Vic, 1993}.
% \changes{catalan-2.2b}{1995/06/18}{Month names in lowercase}
%    \begin{macrocode}
\def\datecatalan{%
  \def\today{\number\day~\ifcase\month\or
    de gener\or de febrer\or de mar\c{c}\or d'abril\or de maig\or
    de juny\or de juliol\or d'agost\or de setembre\or d'octubre\or
    de novembre\or de desembre\fi
    \space de~\number\year}}
%    \end{macrocode}
% \end{macro}
%
% \begin{macro}{\extrascatalan}
% \changes{catalan-2.0}{1993/07/11}{Macro completely rewritten}
% \changes{catalan-2.2a}{1995/03/11}{Handling of active characters
%    completely rewritten}
%
% \begin{macro}{\noextrascatalan}
% \changes{catalan-2.0}{1993/07/11}{Macro completely rewritten}
%
%    The macro |\extrascatalan| will perform all the extra definitions
%    needed for the Catalan language.  The macro |\noextrascatalan| is
%    used to cancel the actions of |\extrascatalan|.
%
%    For Catalan, some characters are made active or are redefined. In
%    particular, the \texttt{"} character receives a new meaning; this
%    can also happen for the \texttt{'} character and the \texttt{`}
%    character when the options \Lopt{activegrave} and/or
%    \Lopt{activeacute} are specified.
%
% \changes{catalan-2.2b}{1995/07/07}{Make activating the accent
%    characters optional}
%    \begin{macrocode}
\addto\extrascatalan{\languageshorthands{catalan}}
\initiate@active@char{"}
\addto\extrascatalan{\bbl@activate{"}}
\@ifpackagewith{babel}{activegrave}{%
  \initiate@active@char{`}
  \addto\extrascatalan{\bbl@activate{`}}}{}
\@ifpackagewith{babel}{activeacute}{%
  \initiate@active@char{'}
  \addto\extrascatalan{\bbl@activate{'}}}{}
%\addto\noextrascatalan{%
%  \bbl@deactivate{"}
%  \bbl@deactivate{`}\bbl@deactivate{'}}
%    \end{macrocode}
%
% \changes{catalan-2.2a}{1995/03/11}{All the code for handling active
%    characters is now moved to \file{babel.def}}
%
%    Apart from the active characters some other macros get a new
%    definition. Therefore we store the current ones to be able
%    to restore them later.
%    When their current meanings are saved, we can safely redefine
%    them.
%
%    We provide new definitions for the accent macros when one or
%    boith of the options \Lopt{activegrave} or \Lopt{activeacute}
%    were specified.
%
%    \begin{macrocode}
\addto\extrascatalan{%
  \babel@save\"
  \def\"{\protect\@umlaut}}%
\@ifpackagewith{babel}{activegrave}{%
  \babel@save\`
  \addto\extrascatalan{\def\`{\protect\@grave}}
  }{}
\@ifpackagewith{babel}{activeacute}{%
  \babel@save\'
  \addto\extrascatalan{\def\'{\protect\@acute}}
  }{}
%    \end{macrocode}
% \end{macro}
% \end{macro}
%
%    All the code above is necessary because we need a few extra
%    active characters. These characters are then used as indicated in
%    tables~\ref{tab:catalan-quote-def}
%    and~\ref{tab:catalan-quote-opt}.
%
%  \begin{macro}{\dieresis}
%  \begin{macro}{\textacute}
% \changes{catalan-2.1d}{1994/06/26}{Renamed from \cs{acute} as that
%    is a \cs{mathaccent}}
%  \begin{macro}{\textgrave}
%
%    The original definition of |\"| is stored as |\dieresis|, because
%    the definition of |\"| might not be the default plain \TeX\
%    one. If the user uses \textsc{PostScript} fonts with the Adobe
%    font encoding the \texttt{"} character is not in the same
%    position as in Knuth's font encoding. In this case |\"| will not
%    be defined as |\accent"7F 1|, but as |\accent'310 #1|. Something
%    similar happens when using fonts that follow the Cork
%    encoding. For this reason we save the definition of |\"| and use
%    that in the definition of other macros. We do likewise for |\`|,
%    and |\'|.
%    \begin{macrocode}
\let\dieresis\"
\@ifpackagewith{babel}{activegrave}{\let\textgrave\`}{}
\@ifpackagewith{babel}{activeacute}{\let\textacute\'}{}
%    \end{macrocode}
%  \end{macro}
%  \end{macro}
%  \end{macro}
%
%  \begin{macro}{\@umlaut}
%  \begin{macro}{\@acute}
%  \begin{macro}{\@grave}
%    We check the encoding and if not using T1, we make the accents
%    expand but enabling hyphenation beyond the accent. If this is the
%    case, not all break positions will be found in words that contain
%    accents, but this is a limitation in \TeX. An unsolved problem
%    here is that the encoding can change at any time. The definitions
%    below are made in such a way that a change between two 256-char
%    encodings are supported, but changes between a 128-char and a
%    256-char encoding are not properly supported. We check if T1 is
%    in use. If not, we will give a warning and proceed redefining the
%    accent macros so that \TeX{} at least finds the breaks that are
%    not too close to the accent. The warning will only be printed to
%    the log file.
%
%    \begin{macrocode}
\ifx\DeclareFontShape\undefined
  \wlog{Warning: You are using an old LaTeX}
  \wlog{Some word breaks will not be found.}
  \def\@umlaut#1{\allowhyphens\dieresis{#1}\allowhyphens}
  \@ifpackagewith{babel}{activeacute}{%
    \def\@acute#1{\allowhyphens\textacute{#1}\allowhyphens}}{}
  \@ifpackagewith{babel}{activegrave}{%
    \def\@grave#1{\allowhyphens\textgrave{#1}\allowhyphens}}{}
\else
  \edef\next{T1}
  \ifx\f@encoding\next
    \let\@umlaut\dieresis
    \@ifpackagewith{babel}{activeacute}{%
      \let\@acute\textacute}{}
    \@ifpackagewith{babel}{activegrave}{%
      \let\@grave\textgrave}{}
  \else
    \wlog{Warning: You are using encoding \f@encoding\space
      instead of T1.}
    \wlog{Some word breaks will not be found.}
    \def\@umlaut#1{\allowhyphens\dieresis{#1}\allowhyphens}
    \@ifpackagewith{babel}{activeacute}{%
      \def\@acute#1{\allowhyphens\textacute{#1}\allowhyphens}}{}
    \@ifpackagewith{babel}{activegrave}{%
      \def\@grave#1{\allowhyphens\textgrave{#1}\allowhyphens}}{}
  \fi
\fi
%    \end{macrocode}
%    If the user setup has extended fonts, the Ferguson macros are
%    required to be defined. We check for their existance and, if
%    defined, expand to whatever they are defined to. For instance,
%    |\'a| would check for the existance of a |\@ac@a| macro. It is
%    assumed to expand to the code of the accented letter.  If it is
%    not defined, we assume that no extended codes are available and
%    expand to the original definition but enabling hyphenation beyond
%    the accent. This is as best as we can do. It is better if you
%    have extended fonts or ML-\TeX{} because the hyphenation
%    algorithm can work on the whole word. The following macros are
%    directly derived from ML-\TeX{}.\footnote{A problem is perceived
%    here with these macros when used in a multilingual environment
%    where extended hyphenation patterns are available for some but
%    not all languages. Assume that no extended patterns exist at some
%    site for French and that \file{french.sty} would adopt this
%    scheme too. In that case, \mbox{\texttt{'e}} in French would
%    produce the combined accented letter, but hyphenation around it
%    would be suppressed. Both language options would need an
%    independent method to know whether they have extended patterns
%    available. The precise impact of this problem and the possible
%    solutions are under study.}
%  \end{macro}
%  \end{macro}
%  \end{macro}
%
% \changes{catalan-2.2a}{1995/03/14}{All the code to deal with active
%    characters is now in \file{babel.def}} 
%
%    Now we can define our shorthands: the diaeresis and ``ela
%    geminada'' support,
%    \begin{macrocode}
\declare@shorthand{catalan}{"i}{\textormath{\@umlaut\i}{\ddot\imath}}
\declare@shorthand{catalan}{"l}{\lgem{}}
\declare@shorthand{catalan}{"u}{\textormath{\@umlaut u}{\ddot u}}
\declare@shorthand{catalan}{"I}{\textormath{\@umlaut I}{\ddot I}}
\declare@shorthand{catalan}{"L}{\Lgem{}}
\declare@shorthand{catalan}{"U}{\textormath{\@umlaut U}{\ddot U}}
%    \end{macrocode}
%    cedille,
% \changes{catalan-2.2c}{1995/07/08}{cedile now produced by double
%    quote shorthand}
%    \begin{macrocode}
\declare@shorthand{catalan}{"c}{\textormath{\c c}{^{\prime} c}}
\declare@shorthand{catalan}{"C}{\textormath{\c C}{^{\prime} C}}
%    \end{macrocode}
%    `french' quote characters,
% \changes{catalan-2.2c}{1995/07/08}{Added shorthands for guillemets}
%    \begin{macrocode}
\declare@shorthand{catalan}{"<}{%
  \textormath{\guillemotleft{}}{\mbox{\guillemotleft}}}
\declare@shorthand{catalan}{">}{%
  \textormath{\guillemotright{}}{\mbox{\guillemotright}}}
%    \end{macrocode}
%     grave accents,
%    \begin{macrocode}
\@ifpackagewith{babel}{activegrave}{%
  \declare@shorthand{catalan}{`a}{\textormath{\@grave a}{\grave a}}
  \declare@shorthand{catalan}{`e}{\textormath{\@grave e}{\grave e}}
  \declare@shorthand{catalan}{`o}{\textormath{\@grave o}{\grave o}}
  \declare@shorthand{catalan}{`A}{\textormath{\@grave A}{\grave A}}
  \declare@shorthand{catalan}{`E}{\textormath{\@grave E}{\grave E}}
  \declare@shorthand{catalan}{`O}{\textormath{\@grave O}{\grave O}}
  }{}
%    \end{macrocode}
%     acute accents,
% \changes{catalan-2.2b}{1995/07/03}{Changed mathmode definition of
%    acute shorthands to expand to a single prime followed by the next
%    character in the input}
%    \begin{macrocode}
\@ifpackagewith{babel}{activeacute}{%
  \declare@shorthand{catalan}{'a}{\textormath{\@acute a}{^{\prime} a}}
  \declare@shorthand{catalan}{'e}{\textormath{\@acute e}{^{\prime} e}}
  \declare@shorthand{catalan}{'i}{\textormath{\@acute\i{}}{^{\prime} i}}
  \declare@shorthand{catalan}{'o}{\textormath{\@acute o}{^{\prime} o}}
  \declare@shorthand{catalan}{'u}{\textormath{\@acute u}{^{\prime} u}}
  \declare@shorthand{catalan}{'A}{\textormath{\@acute A}{^{\prime} A}}
  \declare@shorthand{catalan}{'E}{\textormath{\@acute E}{^{\prime} E}}
  \declare@shorthand{catalan}{'I}{\textormath{\@acute I}{^{\prime} I}}
  \declare@shorthand{catalan}{'O}{\textormath{\@acute O}{^{\prime} O}}
  \declare@shorthand{catalan}{'U}{\textormath{\@acute U}{^{\prime} U}}
%    \end{macrocode}
%         the acute accent,
% \changes{catalan-2.2c}{1995/07/08}{Added '{}' as an axtra shorthand,
%    removed 'n as a shorthand}
%    \begin{macrocode}
  \declare@shorthand{catalan}{''}{%
    \textormath{\textquotedblright}{\sp\bgroup\prim@s'}}
  }{}
%    \end{macrocode}
%    and finally, some support definitions
%    \begin{macrocode}
\declare@shorthand{catalan}{"-}{\allowhyphens-\allowhyphens}
\declare@shorthand{catalan}{"|}{%
  \textormath{\penalty\@M\discretionary{-}{}{\kern.03em}%
              \allowhyphens}{}}
%    \end{macrocode}
%
%  \begin{macro}{\-}
%
%    All that is left now is the redefinition of |\-|. The new version
%    of |\-| should indicate an extra hyphenation position, while
%    allowing other hyphenation positions to be generated
%    automatically. The standard behaviour of \TeX\ in this respect is
%    unfortunate for Catalan but not as much as for Dutch or German,
%    where long compound words are quite normal and all one needs is a
%    means to indicate an extra hyphenation position on top of the
%    ones that \TeX\ can generate from the hyphenation
%    patterns. However, the average length of words in Catalan makes
%    this desirable and so it is kept here.
%
%    \begin{macrocode}
\addto\extrascatalan{%
  \babel@save{\-}%
  \def\-{\allowhyphens\discretionary{-}{}{}\allowhyphens}}
%    \end{macrocode}
%  \end{macro}
%
%  \begin{macro}{\lgem}
%  \begin{macro}{\Lgem}
% \changes{catalan-2.2b}{1995/06/18}{Added support for typing the
%    catalan ``ela geminada'' with the macros \cs{lgem} and \cs{Lgem}}
%
%     Here we define a macro for typing the catalan ``ela geminada''
%    (geminated l). The macros |\lgem| and |\Lgem| have been chosen
%    for its lowercase and uppercase representation,
%    respectively\footnote{The macro names \cs{ll} and \cs{LL} were
%    not taken because of the fact that \cs{ll} is already used in
%    mathematical mode.}.
%
%    The code used in the actual macro used is a combination of the
%    one proposed by Feruglio and Fuster\footnote{G.V.~Valiente and
%    R.~Fuster, Typesetting Catalan Texts with \TeX, \emph{TUGboat}
%    \textbf{14}(3), 1993.} and the proposal from Valiente to appear
%    in the next \TeX\ Users Group Annual Meeting (1995). This last
%    proposal has not been fully implemented due to its limitation to
%    CM fonts.
%    \begin{macrocode}
\newskip\zzz
\newdimen\leftllkern \newdimen\rightllkern \newdimen\raiselldim
\def\lgem{\relax\ifmmode\orig@ll
  \else
    \leftllkern=0pt\rightllkern=0pt\raiselldim=0pt%
    \setbox0\hbox{l}\setbox1\hbox{l\/}\setbox2\hbox{.}%
    \advance\raiselldim by \the\fontdimen5\the\font
    \advance\raiselldim by -\ht2%
    \leftllkern=-.25\wd0%
    \advance\leftllkern by \wd1%
    \advance\leftllkern by -\wd0%
    \rightllkern=-.25\wd0%
    \advance\rightllkern by -\wd1%
    \advance\rightllkern by \wd0%
    \allowhyphens\discretionary{l-}{l}%
      {\hbox{l}\kern\leftllkern\raise\raiselldim\hbox{.}%
      \kern\rightllkern\hbox{l}}\allowhyphens
    }
\def\Lgem{\leftllkern=0pt\rightllkern=0pt\raiselldim=0pt%
  \setbox0\hbox{L}\setbox1\hbox{L\/}\setbox2\hbox{.}%
  \advance\raiselldim by .5\ht0%
  \advance\raiselldim by -.5\ht2%
  \leftllkern=-.125\wd0%
  \advance\leftllkern by \wd1%
  \advance\leftllkern by -\wd0%
  \rightllkern=-\wd0%
  \divide\rightllkern by 6%
  \advance\rightllkern by -\wd1%
  \advance\rightllkern by \wd0%
  \allowhyphens\discretionary{L-}{L}%
    {\hbox{L}\kern\leftllkern\raise\raiselldim\hbox{.}%
    \kern\rightllkern\hbox{L}}\allowhyphens}
%    \end{macrocode}
%  \end{macro}
%  \end{macro}
%
%  \begin{macro}{\up}
%
% \changes{catalan-2.2b}{1995/06/18}{Added definition of macro
%    \cs{up}, which can be used to type ordinals}
%    A macro for typesetting things like 1\raise1ex\hbox{\small er} as
%    proposed by Raymon Seroul\footnote{This macro has been borrowed
%    from francais.dtx}.
%    \begin{macrocode}
\def\up#1{\raise 1ex\hbox{\small#1}}
%    \end{macrocode}
%  \end{macro}
%
%    It is possible that a site might need to add some extra code to
%    the babel macros. To enable this we load a local configuration
%    file, \file{catalan.cfg} if it is found on \TeX' search path.
% \changes{catalan-2.2b}{1995/07/02}{Added loading of configuration
%    file}
%    \begin{macrocode}
\loadlocalcfg{catalan}
%    \end{macrocode}
%
%    Our last action is to make a note that the commands we have just
%    defined, will be executed by calling the macro |\selectlanguage|
%    at the beginning of the document.
%    \begin{macrocode}
\main@language{catalan}
%    \end{macrocode}
%
%    Finally, the category code of \texttt{@} is reset to its original
%    value. The macrospace used by |\atcatcode| is freed.
%
%    \begin{macrocode}
\catcode`\@=\atcatcode \let\atcatcode\relax
%</code>
%    \end{macrocode}
%
% \Finale
%% \CharacterTable
%%  {Upper-case    \A\B\C\D\E\F\G\H\I\J\K\L\M\N\O\P\Q\R\S\T\U\V\W\X\Y\Z
%%   Lower-case    \a\b\c\d\e\f\g\h\i\j\k\l\m\n\o\p\q\r\s\t\u\v\w\x\y\z
%%   Digits        \0\1\2\3\4\5\6\7\8\9
%%   Exclamation   \!     Double quote  \"     Hash (number) \#
%%   Dollar        \$     Percent       \%     Ampersand     \&
%%   Acute accent  \'     Left paren    \(     Right paren   \)
%%   Asterisk      \*     Plus          \+     Comma         \,
%%   Minus         \-     Point         \.     Solidus       \/
%%   Colon         \:     Semicolon     \;     Less than     \<
%%   Equals        \=     Greater than  \>     Question mark \?
%%   Commercial at \@     Left bracket  \[     Backslash     \\
%%   Right bracket \]     Circumflex    \^     Underscore    \_
%%   Grave accent  \`     Left brace    \{     Vertical bar  \|
%%   Right brace   \}     Tilde         \~}
%%
\endinput
}
\DeclareOption{croatian}{% \iffalse meta-comment
%
% Copyright 1989-1995 Johannes L. Braams and any individual authors
% listed elsewhere in this file.  All rights reserved.
% 
% For further copyright information any other copyright notices in this
% file.
% 
% This file is part of the Babel system release 3.5.
% --------------------------------------------------
%   This system is distributed in the hope that it will be useful,
%   but WITHOUT ANY WARRANTY; without even the implied warranty of
%   MERCHANTABILITY or FITNESS FOR A PARTICULAR PURPOSE.
% 
%   For error reports concerning UNCHANGED versions of this file no more
%   than one year old, see bugs.txt.
% 
%   Please do not request updates from me directly.  Primary
%   distribution is through the CTAN archives.
% 
% 
% IMPORTANT COPYRIGHT NOTICE:
% 
% You are NOT ALLOWED to distribute this file alone.
% 
% You are allowed to distribute this file under the condition that it is
% distributed together with all the files listed in manifest.txt.
% 
% If you receive only some of these files from someone, complain!
% 
% Permission is granted to copy this file to another file with a clearly
% different name and to customize the declarations in that copy to serve
% the needs of your installation, provided that you comply with
% the conditions in the file legal.txt from the LaTeX2e distribution.
% 
% However, NO PERMISSION is granted to produce or to distribute a
% modified version of this file under its original name.
%  
% You are NOT ALLOWED to change this file.
% 
% 
% \fi
% \CheckSum{121}
% \iffalse
%    Tell the \LaTeX\ system who we are and write an entry on the
%    transcript.
%<*dtx>
\ProvidesFile{croatian.dtx}
%</dtx>
%<code>\ProvidesFile{croatian.ldf}
       [1995/07/04 v1.3e Croatian support from the babel system]
%
% Babel package for LaTeX version 2e
% Copyright (C) 1989 - 1995
%           by Johannes Braams, TeXniek
%
% Please report errors to: J.L. Braams
%                          JLBraams@cistron.nl
%
%    This file is part of the babel system, it provides the source
%    code for the Croatian language definition file.  A contribution
%    was made by Alan Pai\'{c} (paica@cernvm.cern.ch)
%<*filedriver>
\documentclass{ltxdoc}
\newcommand*\TeXhax{\TeX hax}
\newcommand*\babel{\textsf{babel}}
\newcommand*\langvar{$\langle \it lang \rangle$}
\newcommand*\note[1]{}
\newcommand*\Lopt[1]{\textsf{#1}}
\newcommand*\file[1]{\texttt{#1}}
\begin{document}
 \DocInput{croatian.dtx}
\end{document}
%</filedriver>
%\fi
% \GetFileInfo{croatian.dtx}
%
% \changes{croatian-1.0a}{1991/07/15}{Renamed \file{babel.sty} in
%    \file{babel.com}}
% \changes{croatian-1.0c}{1992/01/25}{Removed some typos}
% \changes{croatian-1.1}{1992/02/15}{Brought up-to-date with babel 3.2a}
% \changes{croatian-1.3}{1994/02/27}{Update for \LaTeXe}
%
%  \section{The Croatian language}
%
%    The file \file{\filename}\footnote{The file described in this
%    section has version number \fileversion\ and was last revised on
%    \filedate.  A contribution was made by Alan Pai\'{c}
%    (\texttt{paica@cernvm.cern.ch}).}  defines all the
%    language definition macros for the Croatian language.
%
%    For this language currently no special definitions are needed or
%    available.
%
% \StopEventually{}
%
%    As this file needs to be read only once, we check whether it was
%    read before. If it was, the command |\captionscroatian| is already
%    defined, so we can stop processing. If this command is undefined
%    we proceed with the various definitions and first show the
%    current version of this file.
%
% \changes{croatian-1.0a}{1991/07/15}{Added reset of catcode of @
%    before \cs{endinput}.}
% \changes{croatian-1.0b}{1991/10/07}{Removed use of
%    \cs{@ifundefined}}
%    \begin{macrocode}
%<*code>
\ifx\undefined\captionscroatian
\else
  \selectlanguage{croatian}
  \expandafter\endinput
\fi
%    \end{macrocode}
%
% \changes{croatian-1.0b}{1991/10/07}{Removed code to load
%    \file{latexhax.com}}
%
%  \begin{macro}{\atcatcode}
%    This file, \file{croatian.sty}, may have been read while \TeX\ is
%    in the middle of processing a document, so we have to make sure the
%    category code of \texttt{@} is `letter' while this file is being
%    read.  We save the category code of the @-sign in |\atcatcode|
%    and make it `letter'. Later the category code can be restored to
%    whatever it was before.
%
% \changes{croatian-1.0a}{1991/07/15}{Modified handling of catcode of
%    @ again.}
% \changes{croatian-1.0b}{1991/10/07}{Removed use of \cs{makeatletter}
%    and hence the need to load \file{latexhax.com}}
%    \begin{macrocode}
\chardef\atcatcode=\catcode`\@
\catcode`\@=11\relax
%    \end{macrocode}
%  \end{macro}
%
%    Now we determine whether the the common macros from the file
%    \file{babel.def} need to be read. We can be in one of two
%    situations: either another language option has been read earlier
%    on, in which case that other option has already read
%    \file{babel.def}, or \texttt{croatian} is the first language option
%    to be processed. In that case we need to read \file{babel.def}
%    right here before we continue.
%
% \changes{croatian-1.1}{1992/02/15}{Added \cs{relax} after the
%    argument of \cs{input}}
%    \begin{macrocode}
\ifx\undefined\babel@core@loaded\input babel.def\relax\fi
%    \end{macrocode}
%
%    Another check that has to be made, is if another language
%    definition file has been read already. In that case its definitions
%    have been activated. This might interfere with definitions this
%    file tries to make. Therefore we make sure that we cancel any
%    special definitions. This can be done by checking the existence
%    of the macro |\originalTeX|. If it exists we simply execute it,
%    otherwise it is |\let| to |\empty|.
% \changes{croatian-1.0a}{1991/07/15}{Added \cs{let}%
%    \cs{originalTeX}\cs{relax} to test for existence}
% \changes{croatian-1.1}{1992/02/15}{\cs{originalTeX} should be
%    expandable, \cs{let} it to \cs{empty}}
%    \begin{macrocode}
\ifx\undefined\originalTeX \let\originalTeX\empty \else\originalTeX\fi
%    \end{macrocode}
%
%    When this file is read as an option, i.e. by the |\usepackage|
%    command, \texttt{croatian} will be an `unknown' language in which
%    case we have to make it known. So we check for the existence of
%    |\l@croatian| to see whether we have to do something here.
%
% \changes{croatian-1.0b}{1991/10/07}{Removed use of
%    \cs{@ifundefined}}
% \changes{croatian-1.1}{1992/02/15}{Added a warning when no
%    hyphenation patterns were loaded.}
%    \begin{macrocode}
\ifx\undefined\l@croatian
    \@nopatterns{Croatian}
    \adddialect\l@croatian0\fi
%    \end{macrocode}
%
%    The next step consists of defining commands to switch to (and
%    from) the Croatian language.
%
%  \begin{macro}{\captionscroatian}
%    The macro |\captionscroatian| defines all strings used
%    in the four standard documentclasses provided with \LaTeX.
% \changes{croatian-1.1}{1992/02/15}{Added \cs{seename}, 
%    \cs{alsoname} and \cs{prefacename}}
% \changes{croatian-1.2}{1993/07/11}{\cs{headpagename} should be
%    \cs{pagename}}
% \changes{croatian-1.3d}{1995/05/08}{Added a few translations}
% \changes{croatian-1.3e}{1995/07/04}{Added \cs{proofname} for
%    AMS-\LaTeX}
%    \begin{macrocode}
\addto\captionscroatian{%
  \def\prefacename{Predgovor}%
  \def\refname{Literatura}%
  \def\abstractname{Sa\v{z}etak}%
  \def\bibname{Bibliografija}%
  \def\chaptername{Glava}%
  \def\appendixname{Dodatak}%
  \def\contentsname{Sadr\v{z}aj}%
  \def\listfigurename{Slike}%
  \def\listtablename{Tablice}%
  \def\indexname{Indeks}%
  \def\figurename{Slika}%
  \def\tablename{Tablica}%
  \def\partname{Dio}%
  \def\enclname{Prilozi}%
  \def\ccname{Kopije}%
  \def\headtoname{Prima}%
  \def\pagename{Strana}%
  \def\seename{Vidi}%
  \def\alsoname{Vidi tako\dj er}%
  \def\proofname{Proof}%  <-- needs translation
  }%
%    \end{macrocode}
%  \end{macro}
%
%  \begin{macro}{\datecroatian}
%    The macro |\datecroatian| redefines the command |\today| to
%    produce Croatian dates.
%    \begin{macrocode}
\def\datecroatian{%
\def\today{\number\day .~\ifcase\month\or
  sije\v{c}anj\or velja\v{c}a\or o\v{z}ujak\or travanj\or svibanj\or
  lipanj\or srpanj\or kolovoz\or rujan\or listopad\or studeni\or
  prosinac\fi
  \space \number\year}}
%    \end{macrocode}
%  \end{macro}
%
%  \begin{macro}{\extrascroatian}
%  \begin{macro}{\noextrascroatian}
%    The macro |\extrascroatian| will perform all the extra
%    definitions needed for the Croatian language. The macro
%    |\noextrascroatian| is used to cancel the actions of
%    |\extrascroatian|.  For the moment these macros are empty but
%    they are defined for compatibility with the other language
%    definition files.
%
%    \begin{macrocode}
\addto\extrascroatian{}
\addto\noextrascroatian{}
%    \end{macrocode}
%  \end{macro}
%  \end{macro}
%
%    It is possible that a site might need to add some extra code to
%    the babel macros. To enable this we load a local configuration
%    file, \file{croatian.cfg} if it is found on \TeX' search path.
% \changes{croatian-1.3e}{1995/07/02}{Added loading of configuration
%    file}
%    \begin{macrocode}
\loadlocalcfg{croatian}
%    \end{macrocode}
%
%    Our last action is to make a note that the commands we have just
%    defined, will be executed by calling the macro |\selectlanguage|
%    at the beginning of the document.
%    \begin{macrocode}
\main@language{croatian}
%    \end{macrocode}
%
%    Finally, the category code of \texttt{@} is reset to its original
%    value. The macrospace used by |\atcatcode| is freed.
% \changes{croatian-1.0a}{1991/07/15}{Modified handling of catcode of
%    @-sign.}
%    \begin{macrocode}
\catcode`\@=\atcatcode \let\atcatcode\relax
%</code>
%    \end{macrocode}
%
% \Finale
%% \CharacterTable
%%  {Upper-case    \A\B\C\D\E\F\G\H\I\J\K\L\M\N\O\P\Q\R\S\T\U\V\W\X\Y\Z
%%   Lower-case    \a\b\c\d\e\f\g\h\i\j\k\l\m\n\o\p\q\r\s\t\u\v\w\x\y\z
%%   Digits        \0\1\2\3\4\5\6\7\8\9
%%   Exclamation   \!     Double quote  \"     Hash (number) \#
%%   Dollar        \$     Percent       \%     Ampersand     \&
%%   Acute accent  \'     Left paren    \(     Right paren   \)
%%   Asterisk      \*     Plus          \+     Comma         \,
%%   Minus         \-     Point         \.     Solidus       \/
%%   Colon         \:     Semicolon     \;     Less than     \<
%%   Equals        \=     Greater than  \>     Question mark \?
%%   Commercial at \@     Left bracket  \[     Backslash     \\
%%   Right bracket \]     Circumflex    \^     Underscore    \_
%%   Grave accent  \`     Left brace    \{     Vertical bar  \|
%%   Right brace   \}     Tilde         \~}
%%
\endinput
}
\DeclareOption{czech}{% \iffalse meta-comment
%
% Copyright 1989-1995 Johannes L. Braams and any individual authors
% listed elsewhere in this file.  All rights reserved.
% 
% For further copyright information any other copyright notices in this
% file.
% 
% This file is part of the Babel system release 3.5.
% --------------------------------------------------
%   This system is distributed in the hope that it will be useful,
%   but WITHOUT ANY WARRANTY; without even the implied warranty of
%   MERCHANTABILITY or FITNESS FOR A PARTICULAR PURPOSE.
% 
%   For error reports concerning UNCHANGED versions of this file no more
%   than one year old, see bugs.txt.
% 
%   Please do not request updates from me directly.  Primary
%   distribution is through the CTAN archives.
% 
% 
% IMPORTANT COPYRIGHT NOTICE:
% 
% You are NOT ALLOWED to distribute this file alone.
% 
% You are allowed to distribute this file under the condition that it is
% distributed together with all the files listed in manifest.txt.
% 
% If you receive only some of these files from someone, complain!
% 
% Permission is granted to copy this file to another file with a clearly
% different name and to customize the declarations in that copy to serve
% the needs of your installation, provided that you comply with
% the conditions in the file legal.txt from the LaTeX2e distribution.
% 
% However, NO PERMISSION is granted to produce or to distribute a
% modified version of this file under its original name.
%  
% You are NOT ALLOWED to change this file.
% 
% 
% \fi
% \CheckSum{217}
%
% \iffalse
%    Tell the \LaTeX\ system who we are and write an entry on the
%    transcript.
%<*dtx>
\ProvidesFile{czech.dtx}
%</dtx>
%<code>\ProvidesFile{czech.ldf}
        [1995/07/04 v1.3f Czech support from the babel system]
%
% Babel package for LaTeX version 2e
% Copyright (C) 1989 - 1995
%           by Johannes Braams, TeXniek
%
% Please report errors to: J.L. Braams
%                          JLBraams@cistron.nl
%
%    This file is part of the babel system, it provides the source
%    code for the Czech language definition file.
%    Contributions were made by Milos Lokajicek (LOKAJICK@CERNVM).
%<*filedriver>
\documentclass{ltxdoc}
\newcommand*\TeXhax{\TeX hax}
\newcommand*\babel{\textsf{babel}}
\newcommand*\langvar{$\langle \it lang \rangle$}
\newcommand*\note[1]{}
\newcommand*\Lopt[1]{\textsf{#1}}
\newcommand*\file[1]{\texttt{#1}}
\begin{document}
 \DocInput{czech.dtx}
\end{document}
%</filedriver>
%\fi
%
% \GetFileInfo{czech.dtx}
%
% \changes{czech-1.0a}{1991/07/15}{Renamed babel.sty in babel.com}
% \changes{czech-1.1}{1992/02/15}{Brought up-to-date with babel 3.2a}
% \changes{czech-1.2}{1993/07/11}{Included some features from Kasal's
%    czech.sty}
% \changes{czech-1.3}{1994/02/27}{Update for \LaTeXe}
% \changes{czech-1.3d}{1994/06/26}{Removed the use of \cs{filedate}
%    and moved identification after the loading of \file{babel.def}}
%
%  \section{The Czech language}
%
%    The file \file{\filename}\footnote{The file described in this
%    section has version number \fileversion\ and was last revised on
%    \filedate.  Contributions were made by Milos Lokajicek
%    (\texttt{LOKAJICK@CERNVM}).}  defines all the language definition
%    macros for the Czech language.
%
%    For this language |\frenchspacing| is set and two macros |\q| and
%    |\w| for easy access to two accents are defined.
%
%    The command |\q| is used with the letters (\texttt{t},
%    \texttt{d}, \texttt{l}, and \texttt{L}) and adds a \texttt{'} to
%    them to simulate a `hook' that should be there.  The result looks
%    like t\kern-2pt\char'47. The command |\w| is used to put the
%    ring-accent which appears in \aa ngstr\o m over the letters
%    \texttt{u} and \texttt{U}.
%
% \StopEventually{}
%
%    As this file needs to be read only once, we check whether it was
%    read before. If it was, the command |\captionsczech| is already
%    defined, so we can stop processing. If this command is undefined
%    we proceed with the various definitions and first show the
%    current version of this file.
%
% \changes{czech-1.0a}{1991/07/15}{Added reset of catcode of @ before
%    \cs{endinput}.}
% \changes{czech-1.0b}{1991/10/27}{Removed use of \cs{@ifundefined}}
%    \begin{macrocode}
%<*code>
\ifx\undefined\captionsczech
\else
  \selectlanguage{czech}
  \expandafter\endinput
\fi
%    \end{macrocode}
%
%  \begin{macro}{\atcatcode}
%    This file, \file{czech.sty}, may have been read while \TeX\ is in
%    the middle of processing a document, so we have to make sure the
%    category code of \texttt{@} is `letter' while this file is being
%    read.  We save the category code of the @-sign in |\atcatcode|
%    and make it `letter'. Later the category code can be restored to
%    whatever it was before.
%
% \changes{czech-1.0a}{1991/07/15}{Modified handling of catcode of @
%    again.}
% \changes{czech-1.0b}{1991/10/27}{Removed use of \cs{makeatletter}
%    and hence the need to load \file{latexhax.com}}
%    \begin{macrocode}
\chardef\atcatcode=\catcode`\@
\catcode`\@=11\relax
%    \end{macrocode}
%  \end{macro}
%
%    Now we determine whether the the common macros from the file
%    \file{babel.def} need to be read. We can be in one of two
%    situations: either another language option has been read earlier
%    on, in which case that other option has already read
%    \file{babel.def}, or \texttt{czech} is the first language option to
%    be processed. In that case we need to read \file{babel.def} right
%    here before we continue.
%
% \changes{czech-1.1}{1992/02/15}{Added \cs{relax} after the argument
%    of \cs{input}}
%    \begin{macrocode}
\ifx\undefined\babel@core@loaded\input babel.def\relax\fi
%    \end{macrocode}
%
%    Another check that has to be made, is if another language
%    definition file has been read already. In that case its
%    definitions have been activated. This might interfere with
%    definitions this file tries to make. Therefore we make sure that
%    we cancel any special definitions. This can be done by checking
%    the existence of the macro |\originalTeX|. If it exists we simply
%    execute it, otherwise it is |\let| to |\empty|.
% \changes{czech-1.0a}{1991/07/15}{Added \cs{let}\cs{originalTeX}%
%    \cs{relax} to test for existence}
% \changes{czech-1.1}{1992/02/15}{\cs{originalTeX} should be
%    expandable, \cs{let} it to \cs{empty}}
%    \begin{macrocode}
\ifx\undefined\originalTeX \let\originalTeX\empty \else\originalTeX\fi
%    \end{macrocode}
%
%    When this file is read as an option, i.e. by the |\usepackage|
%    command, \texttt{czech} will be an `unknown' language in which case
%    we have to make it known. So we check for the existence of
%    |\l@czech| to see whether we have to do something here.
%
% \changes{czech-1.0b}{1991/10/27}{Removed use of \cs{@ifundefined}}
% \changes{czech-1.1}{1992/02/15}{Added a warning when no hyphenation
%    patterns were loaded.}
% \changes{czech-1.3d}{1994/06/26}{Now use \cs{@nopatterns} to produce
%    the warning}
%    \begin{macrocode}
\ifx\undefined\l@czech
    \@nopatterns{Czech}
    \adddialect\l@czech0\fi
%    \end{macrocode}
%
%    The next step consists of defining commands to switch to (and
%    from) the Czech language.
%
%  \begin{macro}{\captionsczech}
%    The macro |\captionsczech| defines all strings used in the four
%    standard documentlasses provided with \LaTeX.
% \changes{czech-1.1}{1992/02/15}{Added \cs{seename}, \cs{alsoname}
%    and \cs{prefacename}}
% \changes{czech-1.3f}{1995/07/04}{Added \cs{proofname} for AMS-\LaTeX}
%    \begin{macrocode}
\addto\captionsczech{%
  \def\prefacename{P\v redmluva}%
  \def\refname{Reference}%
  \def\abstractname{Abstrakt}%
  \def\bibname{Literatura}%
  \def\chaptername{Kapitola}%
  \def\appendixname{Dodatek}%
  \def\contentsname{Obsah}%
  \def\listfigurename{Seznam obr\'azk\r{u}}%
  \def\listtablename{Seznam tabulek}%
  \def\indexname{Index}%
  \def\figurename{Obr\'azek}%
  \def\tablename{Tabulka}%
  \def\partname{\v{C}\'ast}%
  \def\enclname{P\v{r}\'{\i}loha}%
  \def\ccname{cc}%
  \def\headtoname{Komu}%
  \def\pagename{Strana}%
  \def\seename{viz}%
  \def\alsoname{viz tek\'e}%
  \def\proofname{Proof}%   <-- needs translation
  }%
%    \end{macrocode}
%  \end{macro}
%
%  \begin{macro}{\dateczech}
%    The macro |\dateczech| redefines the command |\today| to produce
%    Czech dates.
%    \begin{macrocode}
\def\dateczech{%
\def\today{\number\day.~\ifcase\month\or
  ledna\or \'unora\or b\v{r}ezna\or dubna\or kv\v{e}tna\or \v{c}ervna\or
  \v{c}ervence\or srpna\or z\'a\v{r}\'{\i}\or \v{r}\'{\i}jna\or
  listopadu\or prosince\fi
  \space \number\year}}
%    \end{macrocode}
%  \end{macro}
%
%  \begin{macro}{\extrasczech}
%  \begin{macro}{\noextrasczech}
%    The macro |\extrasczech| will perform all the extra definitions
%    needed for the Czech language. The macro |\noextrasczech| is used
%    to cancel the actions of |\extrasczech|.  This means saving the
%    meaning of two one-letter control sequences before defining them.
%
% \changes{czech-1.1a}{1992/07/07}{Removed typo, \cs{q} was restored
%    twice, once too many.}
% \changes{czech-1.3e}{1995/03/15}{Use \LaTeX's \cs{v} and \cs{r}
%    accent commands}
%    \begin{macrocode}
\addto\extrasczech{\babel@save\q\let\q\v}
\addto\extrasczech{\babel@save\w\let\w\r}
%    \end{macrocode}
%    For Czech texts |\frenchspacing| should be in effect. We make
%    sure this is the case and reset it if necessary.
% \changes{czech-1.3e}{1995/03/14}{now use \cs{bbl@frenchspacing} and
%    \cs{bbl@nonfrenchspacing}}
%    \begin{macrocode}
\addto\extrasczech{\bbl@frenchspacing}
\addto\noextrasczech{\bbl@nonfrenchspacing}
%    \end{macrocode}
%  \end{macro}
%  \end{macro}
%
%  \begin{macro}{\v}
%    \LaTeX's normal |\v| accent places a caron over the letter that
%    follows it (\v{o}). This is not what we want for the letters d,
%    t, l and L; for those the accent should change shape. This is
%    acheived by the following.
%    \begin{macrocode}
\AtBeginDocument{%
  \DeclareTextCompositeCommand{\v}{OT1}{t}{%
    t\kern-.23em\raise.24ex\hbox{'}}
  \DeclareTextCompositeCommand{\v}{OT1}{d}{%
    d\kern-.13em\raise.24ex\hbox{'}}
  \DeclareTextCompositeCommand{\v}{OT1}{l}{\lcaron{}}
  \DeclareTextCompositeCommand{\v}{OT1}{L}{\Lcaron{}}}
%    \end{macrocode}
%
%  \begin{macro}{\lcaron}
%  \begin{macro}{\Lcaron}
%    Fot the letters \texttt{l} and \texttt{L} we want to disinguish
%    between normal fonts and monospaced fonts.
%    \begin{macrocode}
\def\lcaron{%
  \setbox0\hbox{M}\setbox\tw@\hbox{i}%
  \ifdim\wd0>\wd\tw@\relax
    l\kern-.13em\raise.24ex\hbox{'}\kern-.11em%
  \else
    l\raise.45ex\hbox to\z@{\kern-.35em '\hss}%
  \fi}
\def\Lcaron{%
  \setbox0\hbox{M}\setbox\tw@\hbox{i}%
  \ifdim\wd0>\wd\tw@\relax
    L\raise.24ex\hbox to\z@{\kern-.28em'\hss}%
  \else
    L\raise.45ex\hbox to\z@{\kern-.40em '\hss}%
  \fi}
%    \end{macrocode}
%  \end{macro}
%  \end{macro}
%  \end{macro}
%
%    It is possible that a site might need to add some extra code to
%    the babel macros. To enable this we load a local configuration
%    file, \file{czech.cfg} if it is found on \TeX' search path.
% \changes{czech-1.3f}{1995/07/02}{Added loading of configuration
%    file}
%    \begin{macrocode}
\loadlocalcfg{czech}
%    \end{macrocode}
%
%    Our last action is to make a note that the commands we have just
%    defined, will be executed by calling the macro |\selectlanguage|
%    at the beginning of the document.
%    \begin{macrocode}
\main@language{czech}
%    \end{macrocode}
%    Finally, the category code of \texttt{@} is reset to its original
%    value. The macrospace used by |\atcatcode| is freed.
% \changes{czech-1.0a}{1991/07/15}{Modified handling of catcode of
%    @-sign.}
%    \begin{macrocode}
\catcode`\@=\atcatcode \let\atcatcode\relax
%</code>
%    \end{macrocode}
%
% \Finale
%%
%% \CharacterTable
%%  {Upper-case    \A\B\C\D\E\F\G\H\I\J\K\L\M\N\O\P\Q\R\S\T\U\V\W\X\Y\Z
%%   Lower-case    \a\b\c\d\e\f\g\h\i\j\k\l\m\n\o\p\q\r\s\t\u\v\w\x\y\z
%%   Digits        \0\1\2\3\4\5\6\7\8\9
%%   Exclamation   \!     Double quote  \"     Hash (number) \#
%%   Dollar        \$     Percent       \%     Ampersand     \&
%%   Acute accent  \'     Left paren    \(     Right paren   \)
%%   Asterisk      \*     Plus          \+     Comma         \,
%%   Minus         \-     Point         \.     Solidus       \/
%%   Colon         \:     Semicolon     \;     Less than     \<
%%   Equals        \=     Greater than  \>     Question mark \?
%%   Commercial at \@     Left bracket  \[     Backslash     \\
%%   Right bracket \]     Circumflex    \^     Underscore    \_
%%   Grave accent  \`     Left brace    \{     Vertical bar  \|
%%   Right brace   \}     Tilde         \~}
%%
\endinput
}
\DeclareOption{danish}{% \iffalse meta-comment
%
% Copyright 1989-1995 Johannes L. Braams and any individual authors
% listed elsewhere in this file.  All rights reserved.
% 
% For further copyright information any other copyright notices in this
% file.
% 
% This file is part of the Babel system release 3.5.
% --------------------------------------------------
%   This system is distributed in the hope that it will be useful,
%   but WITHOUT ANY WARRANTY; without even the implied warranty of
%   MERCHANTABILITY or FITNESS FOR A PARTICULAR PURPOSE.
% 
%   For error reports concerning UNCHANGED versions of this file no more
%   than one year old, see bugs.txt.
% 
%   Please do not request updates from me directly.  Primary
%   distribution is through the CTAN archives.
% 
% 
% IMPORTANT COPYRIGHT NOTICE:
% 
% You are NOT ALLOWED to distribute this file alone.
% 
% You are allowed to distribute this file under the condition that it is
% distributed together with all the files listed in manifest.txt.
% 
% If you receive only some of these files from someone, complain!
% 
% Permission is granted to copy this file to another file with a clearly
% different name and to customize the declarations in that copy to serve
% the needs of your installation, provided that you comply with
% the conditions in the file legal.txt from the LaTeX2e distribution.
% 
% However, NO PERMISSION is granted to produce or to distribute a
% modified version of this file under its original name.
%  
% You are NOT ALLOWED to change this file.
% 
% 
% \fi
% \CheckSum{159}
% \iffalse
%    Tell the \LaTeX\ system who we are and write an entry on the
%    transcript.
%<*dtx>
\ProvidesFile{danish.dtx}
%</dtx>
%<code>\ProvidesFile{danish.ldf}
        [1995/07/02 v1.3h Danish support from the babel system]
%
% Babel package for LaTeX version 2e
% Copyright (C) 1989 - 1995
%           by Johannes Braams, TeXniek
%
% Please report errors to: J.L. Braams
%                          JLBraams@cistron.nl
%
%    This file is part of the babel system, it provides the source
%    code for the Danish language definition file.
%<*filedriver>
\documentclass{ltxdoc}
\newcommand*\TeXhax{\TeX hax}
\newcommand*\babel{\textsf{babel}}
\newcommand*\langvar{$\langle \it lang \rangle$}
\newcommand*\note[1]{}
\newcommand*\Lopt[1]{\textsf{#1}}
\newcommand*\file[1]{\texttt{#1}}
\begin{document}
 \DocInput{danish.dtx}
\end{document}
%</filedriver>
%    A contribution was made by Henning Larsen (larsen@cernvm.cern.ch)
%\fi
% \GetFileInfo{danish.dtx}
%
% \changes{danish-1.0a}{1991/07/15}{Renamed \file{babel.sty} in
%    \file{babel.com}}
% \changes{danish-1.1}{1992/02/15}{Brought up-to-date with babel 3.2a}
% \changes{danish-1.3}{1994/02/27}{Update for \LaTeXe}
% \changes{danish-1.3f}{1994/06/26}{Removed the use of \cs{filedate}
%    and moved identification after the loading of \file{babel.def}}
% \changes{danish-1.3g}{1995/06/08}{Added teh active double quote
%    character as suggested by Peter Busk Laursen}
%
%  \section{The Danish language}
%
%    The file \file{\filename}\footnote{The file described in this
%    section has version number \fileversion\ and was last revised on
%    \filedate.  A contribution was made by Henning Larsen
%    (\texttt{larsen@cernvm.cern.ch})} defines all the
%    language definition macros for the Danish language.
%
%    For this language the character |"| is made active. In
%    table~\ref{tab:danish-quote} an overview is given of its purpose.
%
%    \begin{table}[htb]
%     \centering
%     \begin{tabular}{lp{8cm}}
%       \verb="|= & disable ligature at this position.\\
%        |"-| & an explicit hyphen sign, allowing hyphenation
%               in the rest of the word.\\
%        |""| & like \verb="-=, but producing no hyphen sign (for
%              words that should break at some sign such as
%              ``entrada/salida.''\\
%        |"`| & lowered double left quotes (looks like ,,)\\
%        |"'| & normal double right quotes\\
%        |"<| & for French left double quotes (similar to $<<$).\\
%        |">| & for French right double quotes (similar to $>>$).\\
%     \end{tabular}
%     \caption{The extra definitions made by \file{danish.ldf}}
%     \label{tab:danish-quote}
%    \end{table}
%
% \StopEventually{}
%
%    As this file needs to be read only once, we check whether it was
%    read before. If it was, the command |\captionsdanish| is already
%    defined, so we can stop processing. If this command is undefined
%    we proceed with the various definitions and first show the
%    current version of this file.
%
% \changes{danish-1.0a}{1991/07/15}{Added reset of catcode of @ before
%    \cs{endinput}.}
% \changes{danish-1.0b}{1991/10/27}{Removed use of \cs{@ifundefined}}
%    \begin{macrocode}
%<*code>
\ifx\undefined\captionsdanish
\else
  \selectlanguage{danish}
  \expandafter\endinput
\fi
%    \end{macrocode}
%
% \changes{danish-1.0b}{1991/10/27}{Removed code to load
%    \file{latexhax.com}}
%
% \begin{macro}{\atcatcode}
%    This file, \file{danish.sty}, may have been read while \TeX\ is
%    in the middle of processing a document, so we have to make sure
%    the category code of \texttt{@} is `letter' while this file is
%    being read.  We save the category code of the @-sign in
%    |\atcatcode| and make it `letter'. Later the category code can be
%    restored to whatever it was before.
%
% \changes{danish-1.0a}{1991/07/15}{Modified handling of catcode of @
%    again.}
% \changes{danish-1.0b}{1991/10/27}{Removed use of \cs{makeatletter}
%    and hence the need to load \file{latexhax.com}}
%    \begin{macrocode}
\chardef\atcatcode=\catcode`\@
\catcode`\@=11\relax
%    \end{macrocode}
% \end{macro}
%
%    Now we determine whether the the common macros from the file
%    \file{babel.def} need to be read. We can be in one of two
%    situations: either another language option has been read earlier
%    on, in which case that other option has already read
%    \file{babel.def}, or \texttt{danish} is the first language option
%    to be processed. In that case we need to read \file{babel.def}
%    right here before we continue.
%
% \changes{danish-1.1}{1992/02/15}{Added \cs{relax} after the argument
%    of \cs{input}}
%    \begin{macrocode}
\ifx\undefined\babel@core@loaded\input babel.def\relax\fi
%    \end{macrocode}
%
%    Another check that has to be made, is if another language
%    definition file has been read already. In that case its
%    definitions have been activated. This might interfere with
%    definitions this file tries to make. Therefore we make sure that
%    we cancel any special definitions. This can be done by checking
%    the existence of the macro |\originalTeX|. If it exists we simply
%    execute it, otherwise it is |\let| to |\empty|.
% \changes{danish-1.0a}{1991/07/15}{Added
%    \cs{let}\cs{originalTeX}\cs{relax} to test for existence}
% \changes{danish-1.1}{1992/02/15}{\cs{originalTeX} should be
%    expandable, \cs{let} it to \cs{empty}}
%    \begin{macrocode}
\ifx\undefined\originalTeX \let\originalTeX\empty \else\originalTeX\fi
%    \end{macrocode}
%
%    When this file is read as an option, i.e. by the |\usepackage|
%    command, \texttt{danish} will be an `unknown' language in which
%    case we have to make it known.  So we check for the existence of
%    |\l@danish| to see whether we have to do something here.
%
% \changes{danish-1.0b}{1991/10/27}{Removed use of \cs{@ifundefined}}
% \changes{danish-1.1}{1992/02/15}{Added a warning when no hyphenation
%    patterns were loaded.}
% \changes{danish-1.3f}{1994/06/26}{Now use \cs{@nopatterns} to
%    produce the warning}
%    \begin{macrocode}
\ifx\undefined\l@danish
    \@nopatterns{Danish}
    \adddialect\l@danish0\fi
%    \end{macrocode}
%
%    The next step consists of defining commands to switch to (and
%    from) the Danish language.
%
% \begin{macro}{\captionsdanish}
%    The macro |\captionsdanish| defines all strings used in the four
%    standard documentclasses provided with \LaTeX.
% \changes{danish-1.1}{1992/02/15}{Added \cs{seename}, \cs{alsoname}
%    and \cs{prefacename}}
% \changes{danish-1.2}{1993/07/11}{\cs{headpagename} should be
%    \cs{pagename}}
% \changes{danish-1.2.2}{1993/10/23}{Added a few translations}
% \changes{danish-1.3c}{1994/06/04}{Included some revisions from Peter
%    Busk Larsen}
% \changes{danish-1.3h}{1995/07/02}{Added \cs{proofname} for
%    AMS-\LaTeX}
%    \begin{macrocode}
\addto\captionsdanish{%
  \def\prefacename{Forord}%
  \def\refname{Litteratur}%
  \def\abstractname{Resum\'e}%
  \def\bibname{Litteratur}%
  \def\chaptername{Kapitel}%
  \def\appendixname{Bilag}%
  \def\contentsname{Indhold}%
  \def\listfigurename{Figurer}%
  \def\listtablename{Tabeller}%
  \def\indexname{Indeks}%
  \def\figurename{Figur}%
  \def\tablename{Tabel}%
  \def\partname{Del}%
  \def\enclname{Vedlagt}%
  \def\ccname{Kopi til}%   or    Kopi sendt til
  \def\headtoname{Til}% in letter
  \def\pagename{Side}%
  \def\seename{Se}%
  \def\alsoname{Se ogs{\aa}}%
  \def\proofname{Proof}%  <-- needs translation!
  }%
%    \end{macrocode}
% \end{macro}
%
% \begin{macro}{\datedanish}
%    The macro |\datedanish| redefines the command |\today| to produce
%    Danish dates.
% \changes{danish-1.3a}{1994/03/23}{Added `.' to definition of
%    \cs{today}}
%    \begin{macrocode}
\def\datedanish{%
\def\today{\number\day.~\ifcase\month\or
  januar\or februar\or marts\or april\or maj\or juni\or
  juli\or august\or september\or oktober\or november\or december\fi
  \space\number\year}}
%    \end{macrocode}
% \end{macro}
%
% \begin{macro}{\extrasdanish}
% \changes{danish-1.3h}{1995/07/02}{Added \cs{bbl@frenchspacing}}
% \begin{macro}{\noextrasdanish}
% \changes{danish-1.3h}{1995/07/02}{Added \cs{bbl@nonfrenchspacing}}
%    The macro |\extrasdanish| will perform all the extra definitions
%    needed for the Danish language. The macro |\noextrasdanish| is
%    used to cancel the actions of |\extrasdanish|.
%
%    Danish typesetting requires |\frencspacing| to be in effect.
%    \begin{macrocode}
\addto\extrasdanish{\bbl@frenchspacing}
\addto\noextrasdanish{\bbl@nonfrenchspacing}
%    \end{macrocode}
%
%    For Danish the \texttt{"} character is made active. This is
%    done once, later on its definition may vary. Other languages in
%    the same document may also use the \texttt{"} character for
%    shorthands; we specify that the danish group of shorthands
%    should be used.
%
%    \begin{macrocode}
\initiate@active@char{"}
\addto\extrasdanish{\languageshorthands{danish}}
\addto\extrasdanish{\bbl@activate{"}}
%\addto\noextrasdanish{\bbl@deactivate{"}}
%    \end{macrocode}
%
%    First we define access to the low opening double quote and
%    guillemets for quotations,
%    \begin{macrocode}
\declare@shorthand{danish}{"`}{%
  \textormath{\quotedblbase{}}{\mbox{\quotedblbase}}}
\declare@shorthand{danish}{"'}{%
  \textormath{\textquotedblright{}}{\mbox{\textquotedblright}}}
\declare@shorthand{danish}{"<}{%
  \textormath{\guillemotleft{}}{\mbox{\guillemotleft}}}
\declare@shorthand{danish}{">}{%
  \textormath{\guillemotright{}}{\mbox{\guillemotright}}}
%    \end{macrocode}
%    then we define two shorthands to be able to specify hyphenation
%    breakpoints that behavew a little different from |\-|.
%    \begin{macrocode}
\declare@shorthand{danish}{"-}{\allowhyphens-\allowhyphens}
\declare@shorthand{danish}{""}{\hskip\z@skip}
%    \end{macrocode}
%    And we want to have a shorthand for disabling a ligature.
%    \begin{macrocode}
\declare@shorthand{danish}{"|}{%
  \textormath{\discretionary{-}{}{\kern.03em}}{}}
%    \end{macrocode}
% \end{macro}
% \end{macro}
%
%    It is possible that a site might need to add some extra code to
%    the babel macros. To enable this we load a local configuration
%    file, \file{danish.cfg} if it is found on \TeX' search path.
% \changes{danish-1.3h}{1995/07/02}{Added loading of configuration
%    file}
%    \begin{macrocode}
\loadlocalcfg{danish}
%    \end{macrocode}
%
%    Our last action is to make a note that the commands we have just
%    defined, will be executed by calling the macro |\selectlanguage|
%    at the beginning of the document.
%    \begin{macrocode}
\main@language{danish}
%    \end{macrocode}
%    Finally, the category code of \texttt{@} is reset to its original
%    value. The macrospace used by |\atcatcode| is freed.
% \changes{danish-1.0a}{1991/07/15}{Modified handling of catcode of
%    @-sign.}
%    \begin{macrocode}
\catcode`\@=\atcatcode \let\atcatcode\relax
%</code>
%    \end{macrocode}
%
% \Finale
%%
%% \CharacterTable
%%  {Upper-case    \A\B\C\D\E\F\G\H\I\J\K\L\M\N\O\P\Q\R\S\T\U\V\W\X\Y\Z
%%   Lower-case    \a\b\c\d\e\f\g\h\i\j\k\l\m\n\o\p\q\r\s\t\u\v\w\x\y\z
%%   Digits        \0\1\2\3\4\5\6\7\8\9
%%   Exclamation   \!     Double quote  \"     Hash (number) \#
%%   Dollar        \$     Percent       \%     Ampersand     \&
%%   Acute accent  \'     Left paren    \(     Right paren   \)
%%   Asterisk      \*     Plus          \+     Comma         \,
%%   Minus         \-     Point         \.     Solidus       \/
%%   Colon         \:     Semicolon     \;     Less than     \<
%%   Equals        \=     Greater than  \>     Question mark \?
%%   Commercial at \@     Left bracket  \[     Backslash     \\
%%   Right bracket \]     Circumflex    \^     Underscore    \_
%%   Grave accent  \`     Left brace    \{     Vertical bar  \|
%%   Right brace   \}     Tilde         \~}
%%
\endinput
}
\DeclareOption{dutch}{% \iffalse meta-comment
%
% Copyright 1989-1995 Johannes L. Braams and any individual authors
% listed elsewhere in this file.  All rights reserved.
% 
% For further copyright information any other copyright notices in this
% file.
% 
% This file is part of the Babel system release 3.5.
% --------------------------------------------------
%   This system is distributed in the hope that it will be useful,
%   but WITHOUT ANY WARRANTY; without even the implied warranty of
%   MERCHANTABILITY or FITNESS FOR A PARTICULAR PURPOSE.
% 
%   For error reports concerning UNCHANGED versions of this file no more
%   than one year old, see bugs.txt.
% 
%   Please do not request updates from me directly.  Primary
%   distribution is through the CTAN archives.
% 
% 
% IMPORTANT COPYRIGHT NOTICE:
% 
% You are NOT ALLOWED to distribute this file alone.
% 
% You are allowed to distribute this file under the condition that it is
% distributed together with all the files listed in manifest.txt.
% 
% If you receive only some of these files from someone, complain!
% 
% Permission is granted to copy this file to another file with a clearly
% different name and to customize the declarations in that copy to serve
% the needs of your installation, provided that you comply with
% the conditions in the file legal.txt from the LaTeX2e distribution.
% 
% However, NO PERMISSION is granted to produce or to distribute a
% modified version of this file under its original name.
%  
% You are NOT ALLOWED to change this file.
% 
% 
% \fi
% \CheckSum{207}
%\iffalse
%    Tell the \LaTeX\ system who we are and write an entry on the
%    transcript.
%<*dtx>
\ProvidesFile{dutch.dtx}
%</dtx>
%<code>\ProvidesFile{dutch.ldf}
        [1995/07/04 v3.7d Dutch support from the babel system]
%
% Babel package for LaTeX version 2e
% Copyright (C) 1989 - 1995
%           by Johannes Braams, TeXniek
%
% Dutch Language Definition File
% Copyright (C) 1989 - 1995
%           by Johannes Braams, TeXniek
%
% Please report errors to: J.L. Braams
%                          JLBraams@cistron.nl
%
%    This file is part of the babel system, it provides the source
%    code for the Dutch language definition file.
%<*filedriver>
\documentclass{ltxdoc}
\makeatletter
\gdef\dlqq{{\setbox\tw@=\hbox{,}\setbox\z@=\hbox{''}%
  \dimen\z@=\ht\z@ \advance\dimen\z@-\ht\tw@
  \setbox\z@=\hbox{\lower\dimen\z@\box\z@}\ht\z@=\ht\tw@
  \dp\z@=\dp\tw@ \box\z@\kern-.04em}}
\makeatother
\font\manual=logo10 % font used for the METAFONT logo, etc.
\newcommand*\MF{{\manual META}\-{\manual FONT}}
\newcommand*\TeXhax{\TeX hax}
\newcommand*\babel{\textsf{babel}}
\newcommand*\langvar{$\langle \it lang \rangle$}
\newcommand*\note[1]{}
\newcommand*\Lopt[1]{\textsf{#1}}
\newcommand*\file[1]{\texttt{#1}}
\begin{document}
 \DocInput{dutch.dtx}
\end{document}
%</filedriver>
% \fi
% \GetFileInfo{dutch.dtx}
%
% \changes{dutch-2.0a}{1990/04/02}{Added checking of format}
% \changes{dutch-2.0b}{1990/04/02}{Added extrasdutch}
% \changes{dutch-2.0c}{1990/04/18}{Added grqq macros}
% \changes{dutch-2.1}{1990/04/24}{reflect change to version 2.1 in
%    babel and changes in german v2.3}
% \changes{dutch-2.1a}{1990/05/01}{Incorporated Nico's comments}
% \changes{dutch-2.1b}{1990/07/04}{Incorporated more comments by Nico}
% \changes{dutch-2.1c}{1990/07/16}{Fixed some typos}
% \changes{dutch-2.2}{1990/07/16}{Fixed problem with the use of
%    \texttt{"} in moving arguments while \texttt{"} is active}
% \changes{dutch-2.3}{1990/07/30}{When using PostScript fonts with the
%    Adobe font-encoding, the dieresis-accent is located elsewhere,
%    modified code}
% \changes{dutch-2.3a}{1990/08/27}{Modified the documentation somewhat}
% \changes{dutch-3.0}{1991/04/23}{Modified for babel 3.0}
% \changes{dutch-3.0a}{1991/05/25}{Removed some problems in change log}
% \changes{dutch-3.1}{1991/05/29}{Removed bug found by van der Meer}
% \changes{dutch-3.2a}{1991/07/15}{Renamed babel.sty in babel.com}
% \changes{dutch-3.3}{1991/10/31}{Rewritten parts of the code to use
%    the new features of babel version 3.1}
% \changes{dutch-3.6}{1994/02/02}{Update or LaTeX2e}
% \changes{dutch-3.6c}{1994/06/26}{Removed the use of \cs{filedate},
%    moved identification after the loading of babel.def}
% \changes{dutch-3.7a}{1995/02/03}{Moved identification code to the
%    top of the file}
% \changes{dutch-3.7a}{1995/02/04}{Rewrote the code with respect to
%    the active double quote character}
%
%  \section{The Dutch language}
%
%    The file \file{\filename}\footnote{The file described in this
%    section has version number \fileversion, and was last revised on
%    \filedate.} defines all the language-specific macros for the Dutch
%    language.
%
%    For this language the character |"| is made active. In
%    table~\ref{tab:dutch-quote} an overview is given of its purpose.
%    One of the reasons for this is that in the Dutch language a word
%    with a dieresis can be hyphenated just before the letter with the
%    umlaut, but the dieresis has to disappear if the word is broken
%    between the previous letter and the accented letter.
%
%    In~\cite{treebus} the quoting conventions for the Dutch language
%    are discussed. The preferred convention is the single-quote
%    Anglo-American convention, i.e. `This is a quote'.  An
%    alternative is the slightly old-fashioned Dutch method with
%    initial double quotes lowered to the baseline, \dlqq This is a
%    quote'', which should be typed as \texttt{"`This is a quote"'}.
%
%    \begin{table}[htb]
%     \centering
%     \begin{tabular}{lp{8cm}}
%      |"a| & |\"a| which hyphenates as |-a|;
%             also implemented for the other letters.        \\
%      |"y| & puts a negative kern between \texttt{i} and \texttt{j}\\
%      |"Y| & puts a negative kern between \texttt{I} and \texttt{J}\\
%      \verb="|= & disable ligature at this position.             \\
%      |"-| & an explicit hyphen sign, allowing hyphenation
%             in the rest of the word.                       \\
%      |"`| & lowered double left quotes (see example below).\\
%      |"'| & normal double right quotes.                    \\
%      |\-| & like the old |\-|, but allowing hyphenation
%             in the rest of the word.
%     \end{tabular}
%     \caption{The extra definitions made by \file{dutch.ldf}}
%     \label{tab:dutch-quote}
%    \end{table}
%
% \StopEventually{}
%
% \changes{dutch-3.2c}{1991/10/22}{Removed code to load
%    \file{latexhax.com}}
%
%    As this file needs to be read only once, we check whether it was
%    read before. If it was, the command |\captionsdutch| is
%    already defined, so we can stop processing. If this command is
%    undefined we proceed with the various definitions and first show
%    the current version of this file.
%
% \changes{dutch-3.2a}{1991/07/15}{Added reset of catcode of @ before
%    \cs{endinput}.}
% \changes{dutch-3.2c}{1991/10/22}{removed use of \cs{@ifundefined}}
% \changes{dutch-3.3a}{1991/11/11}{Moved code to the beginning of the
%    file and added \cs{selectlanguage} call}
%    \begin{macrocode}
%<*code>
\ifx\undefined\captionsdutch
\else
  \selectlanguage{dutch}
  \expandafter\endinput
\fi
%    \end{macrocode}
%
% \begin{macro}{\atcatcode}
%    This file, \file{dutch.ldf}, may have been read while \TeX\ is in
%    the middle of processing a document, so we have to make sure the
%    category code of \texttt{@} is `letter' while this file is being
%    read. We save the category code of the @-sign in |\atcatcode| and
%    make it `letter'. Later the category code can be restored to
%    whatever it was before.
%
% \changes{dutch-3.1a}{1991/06/06}{Made test of catcode of @ more
%    robust}
% \changes{dutch-3.2a}{1991/07/15}{Modified handling of catcode of @
%    again.}
% \changes{dutch-3.2c}{1991/10/22}{Removed use of \cs{makeatletter}
%    and hence the need to load \file{latexhax.com}}
%    \begin{macrocode}
\chardef\atcatcode=\catcode`\@
\catcode`\@=11\relax
%    \end{macrocode}
% \end{macro}
%
%    Now we determine whether the common macros from the file
%    \file{babel.def} need to be read. We can be in one of two
%    situations: either another language option has been read earlier
%    on, in which case that other option has already read
%    \file{babel.def}, or \texttt{dutch} is the first language option
%    to be processed. In that case we need to read \file{babel.def}
%    right here before we continue.
%
% \changes{dutch-3.0}{1991/04/23}{New check before loading babel.com}
% \changes{dutch-3.4a}{1992/02/15}{Added \cs{relax} after the argument
%    of \cs{input}}
%    \begin{macrocode}
\ifx\undefined\babel@core@loaded\input babel.def\relax\fi
%    \end{macrocode}
%
% \changes{dutch-3.1}{1991/05/29}{Add a check for existence
%    \cs{originalTeX}}
%
%    Another check that has to be made, is if another language
%    definition file has been read already. In that case its
%    definitions have been activated. This might interfere with
%    definitions this file tries to make. Therefore we make sure that
%    we cancel any special definitions. This can be done by checking
%    the existence of the macro |\originalTeX|. If it exists we simply
%    execute it, otherwise it is |\let| to |\empty|.
% \changes{dutch-3.2a}{1991/07/15}{Added \cs{let}\cs{originalTeX}%
%    \cs{relax} to test for existence}
% \changes{dutch-3.3b}{1992/01/25}{Set \cs{originalTeX} to \cs{empty},
%    because it should be expandable.}
%    \begin{macrocode}
\ifx\undefined\originalTeX \let\originalTeX\empty \fi
\originalTeX
%    \end{macrocode}
%
%    When this file is read as an option, i.e. by the |\usepackage|
%    command, \texttt{dutch} could be an `unknown' language in which
%    case we have to make it known.  So we check for the existence of
%    |\l@dutch| to see whether we have to do something here.
%
% \changes{dutch-3.0}{1991/04/23}{Now use \cs{adddialect} if language
%    undefined}
% \changes{dutch-3.2c}{1991/10/22}{removed use of \cs{@ifundefined}}
% \changes{dutch-3.3b}{1992/01/25}{Added warning, if no dutch patterns
%    loaded}
% \changes{dutch-3.6c}{1994/06/26}{Now use \cs{@nopatterns} to produce
%    the warning}
%    \begin{macrocode}
\ifx\undefined\l@dutch
  \@nopatterns{Dutch}
  \adddialect\l@dutch0
\fi
%    \end{macrocode}
%
%    The next step consists of defining commands to switch to (and
%    from) the Dutch language.
%
% \begin{macro}{\captionsdutch}
%    The macro |\captionsdutch| defines all strings used
%    in the four standard document classes provided with \LaTeX.
% \changes{dutch-3.1a}{1991/06/06}{Removed \cs{global} definitions}
% \changes{dutch-3.1a}{1991/06/06}{\cs{pagename} should be
%    \cs{headpagename}}
% \changes{dutch-3.3a}{1991/11/11}{added \cs{seename} and
%    \cs{alsoname}}
% \changes{dutch-3.3b}{1992/01/25}{added \cs{prefacename}}
% \changes{dutch-3.5}{1993/07/11}{\cs{headpagename} should be
%    \cs{pagename}}
% \changes{dutch-3.7c}{1995/06/08}{We need the \texttt{"} to be active
%    while defining \cs{captionsdutch}}
% \changes{dutch-3.7d}{1995/07/04}{Added \cs{proofname} for
%    AMS-\LaTeX}
%    \begin{macrocode}
\begingroup
  \catcode`\"\active
  \def\x{\endgroup
    \addto\captionsdutch{%
      \def\prefacename{Voorwoord}%
      \def\refname{Referenties}%
      \def\abstractname{Samenvatting}%
      \def\bibname{Bibliografie}%
      \def\chaptername{Hoofdstuk}%
      \def\appendixname{B"ylage}%
      \def\contentsname{Inhoudsopgave}%
      \def\listfigurename{L"yst van figuren}%
      \def\listtablename{L"yst van tabellen}%
      \def\indexname{Index}%
      \def\figurename{Figuur}%
      \def\tablename{Tabel}%
      \def\partname{Deel}%
      \def\enclname{B"ylage(n)}%
      \def\ccname{cc}%
      \def\headtoname{Aan}%
      \def\pagename{Pagina}%
      \def\seename{zie}%
      \def\alsoname{zie ook}%
      \def\proofname{Bewijs}%
      }
    }\x
%    \end{macrocode}
% \end{macro}
%
% \begin{macro}{\datedutch}
%    The macro |\datedutch| redefines the command |\today| to produce
%    Dutch dates.
% \changes{dutch-3.1a}{1991/06/06}{Removed \cs{global} definitions}
%    \begin{macrocode}
\def\datedutch{%
\def\today{\number\day~\ifcase\month\or
  januari\or februari\or maart\or april\or mei\or juni\or juli\or
  augustus\or september\or oktober\or november\or december\fi
  \space \number\year}}
%    \end{macrocode}
% \end{macro}
%
% \begin{macro}{\extrasdutch}
% \changes{dutch-3.0b}{1991/05/29}{added some comment chars to prevent
%    white space}
% \changes{dutch-3.1a}{1991/06/6}{Removed \cs{global} definitions}
% \changes{dutch-3.2}{1991/07/02}{Save all redefined macros}
% \changes{dutch-3.3}{1991/10/31}{Macro complete rewritten}
% \changes{dutch-3.3b}{1992/01/25}{modified handling of
%    \cs{dospecials} and \cs{@sanitize}}
%
% \begin{macro}{\noextrasdutch}
% \changes{dutch-2.3}{1990/07/30}{Added \cs{dieresis}}
% \changes{dutch-3.0b}{1991/05/29}{added some comment chars to prevent
%    white space}
% \changes{dutch-3.1a}{1991/06/06}{Removed \cs{global} definitions}
% \changes{dutch-3.2}{1991/07/02}{Try to restore everything to its
%    former state}
% \changes{dutch-3.3}{1991/10/31}{Macro complete rewritten}
% \changes{dutch-3.3b}{1992/01/25}{modified handling of \cs{dospecials}
%    and \cs{@sanitize}}
%
%    The macro |\extrasdutch| will perform all the extra definitions
%    needed for the Dutch language. The macro |\noextrasdutch| is used
%    to cancel the actions of |\extrasdutch|.
%
%    For Dutch the \texttt{"} character is made active. This is done
%    once, later on its definition may vary. Other languages in the
%    same document may also use the \texttt{"} character for
%    shorthands; we specify that the dutch group of shorthands should
%    be used.
%    \begin{macrocode}
\initiate@active@char{"}
\addto\extrasdutch{\languageshorthands{dutch}}
\addto\extrasdutch{\bbl@activate{"}}
%\addto\noextrasdutch{\bbl@deactivate{"}}
%    \end{macrocode}
%
%    The `umlaut' character should be positioned lower on \emph{all}
%    vowels in Dutch texts.
%    \begin{macrocode}
\addto\extrasdutch{\umlautlow\umlautelow}
\addto\noextrasdutch{\umlauthigh}
%    \end{macrocode}
% \end{macro}
% \end{macro}
%
%  \begin{macro}{\dutchhyphenmins}
%    The dutch hyphenation patterns can be used with |\lefthyphenmin|
%    set to~2 and |\righthyphenmin| set to~3.
% \changes{dutch-3.7a}{1995/05/13}{use \cs{dutchhyphenmins} to store
%    the correct values}
%    \begin{macrocode}
\def\dutchhyphenmins{\tw@\thr@@}
%    \end{macrocode}
%  \end{macro}
%
% \changes{dutch-3.3a}{1991/11/11}{Added \cs{save@sf@q} macro from
%    germanb and rewrote all quote macros to use it}
% \changes{dutch-3.4b}{1991/02/16}{moved definition of
%    \cs{allowhyphens}, \cs{set@low@box} and \cs{save@sf@q} to
%    \file{babel.com}}
% \changes{dutch-3.7a}{1995/02/04}{Removed \cs{dlqq}, \cs{@dlqq},
%    \cs{drqq}, \cs{@drqq} and \cs{dieresis}}
% \changes{dutch-3.7a}{1995/02/15}{moved the definition of the double
%    quote character at the baseline to \file{glyhps.def}}
%
%  \begin{macro}{\@trema}
%    In the Dutch language vowels with a trema are treated
%    specially. If a hyphenation occurs before a vowel-plus-trema, the
%    trema should disappear. To be able to do this we could first
%    define the hyphenation break behaviour for the five vowels, both
%    lowercase and uppercase, in terms of |\discretionary|. But this
%    results in a large |\if|-construct in the definition of the
%    active |"|. Because we think a user should not use |"| when he
%    really means something like |''| we chose not to distinguish
%    between vowels and consonants. Therefore we have one macro
%    |\@trema| which specifies the hyphenation break behaviour for all
%    letters.
%
% \changes{dutch-2.3}{1990/07/30}{\cs{dieresis} instead of
%    \cs{accent127}}
% \changes{dutch-3.3a}{1991/11/11}{renamed \cs{@umlaut} to
%    \cs{@trema}}
%    \begin{macrocode}
\def\@trema#1{\allowhyphens\discretionary{-}{#1}{\"{#1}}\allowhyphens}
%    \end{macrocode}
%  \end{macro}
%
% \changes{dutch-3.7a}{1995/02/15}{Moved the definition of \cs{ij} and
%    \cs{IJ} to \file{glyphs.def}}
% \changes{dutch-3.7a}{1995/02/03}{The support macros for the active
%    double quote have been moved to \file{babel.def}}
%
%     Now we can define the doublequote macros: the tremas,
%
% \changes{dutch-2.3}{1990/07/30}{\cs{dieresis} instead of
%    \cs{accent127}}
% \changes{dutch-3.2}{1991/07/02}{added case for \texttt{"y} and
%    \texttt{"Y}}
% \changes{dutch-3.2b}{1991/07/16}{removed typo (allowhpyhens)}
% \changes{dutch-3.7a}{1995/02/03}{Now use \cs{Declaredq{dutch}} to
%    define the functions of the active double quote}
% \changes{dutch-3.7a}{1995/02/03}{Use \cs{ddot} instead of
%    \cs{@MATHUMLAUT}}
% \changes{dutch-3.7a}{1995/03/05}{Use more general mechanism of
%    \cs{declare@shorthand}}
%    \begin{macrocode}
\declare@shorthand{dutch}{"a}{\textormath{\@trema a}{\ddot a}}
\declare@shorthand{dutch}{"e}{\textormath{\@trema e}{\ddot e}}
\declare@shorthand{dutch}{"i}{%
  \textormath{\discretionary{-}{i}{\"{\i}}}{\ddot \imath}}
\declare@shorthand{dutch}{"o}{\textormath{\@trema o}{\ddot o}}
\declare@shorthand{dutch}{"u}{\textormath{\@trema u}{\ddot u}}
%    \end{macrocode}
%    dutch quotes,
%    \begin{macrocode}
\declare@shorthand{dutch}{"`}{%
  \textormath{\quotedblbase{}}{\mbox{\quotedblbase}}}
\declare@shorthand{dutch}{"'}{%
  \textormath{\textquotedblright{}}{\mbox{\textquotedblright}}}
%    \end{macrocode}
%    and some additional commands:
% \changes{dutch-3.7b}{1995/06/04}{Added \texttt{""} shorthand}
%    \begin{macrocode}
\declare@shorthand{dutch}{"-}{\allowhyphens-\allowhyphens}
\declare@shorthand{dutch}{"|}{%
  \textormath{\discretionary{-}{}{\kern.03em}}{}}
\declare@shorthand{dutch}{""}{\hskip\z@skip}
\declare@shorthand{dutch}{"y}{\textormath{\ij{}}{\ddot y}}
\declare@shorthand{dutch}{"Y}{\textormath{\IJ{}}{\ddot Y}}
%    \end{macrocode}
%
%  \begin{macro}{\-}
%
%    All that is left now is the redefinition of |\-|. The new version
%    of |\-| should indicate an extra hyphenation position, while
%    allowing other hyphenation positions to be generated
%    automatically. The standard behaviour of \TeX\ in this respect is
%    very unfortunate for languages such as Dutch and German, where
%    long compound words are quite normal and all one needs is a means
%    to indicate an extra hyphenation position on top of the ones that
%    \TeX\ can generate from the hyphenation patterns.
%    \begin{macrocode}
\addto\extrasdutch{\babel@save\-}
\addto\extrasdutch{\def\-{\allowhyphens
                          \discretionary{-}{}{}\allowhyphens}}
%    \end{macrocode}
%  \end{macro}
%
%    It is possible that a site might need to add some extra code to
%    the babel macros. To enable this we load a local configuration
%    file, \file{dutch.cfg} if it is found on \TeX' search path.
% \changes{dutch-3.7d}{1995/07/04}{Added loading of configuration
%    file}
%    \begin{macrocode}
\loadlocalcfg{dutch}
%    \end{macrocode}
%
%    Our last action is to make a note that the commands we have just
%    defined, will be executed by calling the macro |\selectlanguage|
%    at the beginning of the document.
% \changes{dutch-3.7a}{1995/03/14}{Use \cs{main@language} instead
%    of \cs{selectlanguage}}
%    \begin{macrocode}
\main@language{dutch}
%    \end{macrocode}
%    Finally, the category code of \texttt{@} is reset to its original
%    value. The macrospace used by |\atcatcode| is freed.
% \changes{dutch-3.2a}{1991/07/15}{Modified handling of catcode of
%    @-sign.}
%    \begin{macrocode}
\catcode`\@=\atcatcode \let\atcatcode\relax
%</code>
%    \end{macrocode}
%
% \Finale
%%
%% \CharacterTable
%%  {Upper-case    \A\B\C\D\E\F\G\H\I\J\K\L\M\N\O\P\Q\R\S\T\U\V\W\X\Y\Z
%%   Lower-case    \a\b\c\d\e\f\g\h\i\j\k\l\m\n\o\p\q\r\s\t\u\v\w\x\y\z
%%   Digits        \0\1\2\3\4\5\6\7\8\9
%%   Exclamation   \!     Double quote  \"     Hash (number) \#
%%   Dollar        \$     Percent       \%     Ampersand     \&
%%   Acute accent  \'     Left paren    \(     Right paren   \)
%%   Asterisk      \*     Plus          \+     Comma         \,
%%   Minus         \-     Point         \.     Solidus       \/
%%   Colon         \:     Semicolon     \;     Less than     \<
%%   Equals        \=     Greater than  \>     Question mark \?
%%   Commercial at \@     Left bracket  \[     Backslash     \\
%%   Right bracket \]     Circumflex    \^     Underscore    \_
%%   Grave accent  \`     Left brace    \{     Vertical bar  \|
%%   Right brace   \}     Tilde         \~}
%%
\endinput
}
%    \end{macrocode}
%    We allow for the british english patterns to be loaded as either
%    `english' or `UKenglish'
%    \begin{macrocode}
\DeclareOption{english}{%
  \ifx\l@UKenglish\undefined
  \else
    \let\l@english\l@UKenglish
  \fi
  % \iffalse meta-comment
%
% Copyright 1989-1995 Johannes L. Braams and any individual authors
% listed elsewhere in this file.  All rights reserved.
% 
% For further copyright information any other copyright notices in this
% file.
% 
% This file is part of the Babel system release 3.5.
% --------------------------------------------------
%   This system is distributed in the hope that it will be useful,
%   but WITHOUT ANY WARRANTY; without even the implied warranty of
%   MERCHANTABILITY or FITNESS FOR A PARTICULAR PURPOSE.
% 
%   For error reports concerning UNCHANGED versions of this file no more
%   than one year old, see bugs.txt.
% 
%   Please do not request updates from me directly.  Primary
%   distribution is through the CTAN archives.
% 
% 
% IMPORTANT COPYRIGHT NOTICE:
% 
% You are NOT ALLOWED to distribute this file alone.
% 
% You are allowed to distribute this file under the condition that it is
% distributed together with all the files listed in manifest.txt.
% 
% If you receive only some of these files from someone, complain!
% 
% Permission is granted to copy this file to another file with a clearly
% different name and to customize the declarations in that copy to serve
% the needs of your installation, provided that you comply with
% the conditions in the file legal.txt from the LaTeX2e distribution.
% 
% However, NO PERMISSION is granted to produce or to distribute a
% modified version of this file under its original name.
%  
% You are NOT ALLOWED to change this file.
% 
% 
% \fi
% \CheckSum{194}
% \iffalse
%    Tell the \LaTeX\ system who we are and write an entry on the
%    transcript.
%<*dtx>
\ProvidesFile{english.dtx}
%</dtx>
%<code>\ProvidesFile{english.ldf}
        [1995/07/04 v3.3e English support from the babel system]
%
% Babel package for LaTeX version 2e
% Copyright (C) 1989 - 1995
%           by Johannes Braams, TeXniek
%
% Please report errors to: J.L. Braams
%                          JLBraams@cistron.nl
%
%    This file is part of the babel system, it provides the source
%    code for the English language definition file.
%<*filedriver>
\documentclass{ltxdoc}
\newcommand*\TeXhax{\TeX hax}
\newcommand*\babel{\textsf{babel}}
\newcommand*\langvar{$\langle \mathit lang \rangle$}
\newcommand*\note[1]{}
\newcommand*\Lopt[1]{\textsf{#1}}
\newcommand*\file[1]{\texttt{#1}}
\begin{document}
 \DocInput{english.dtx}
\end{document}
%</filedriver>
%\fi
% \GetFileInfo{english.dtx}
%
% \changes{english-2.0a}{1990/04/02}{Added checking of format}
% \changes{english-2.1}{1990/04/24}{Reflect changes in babel 2.1}
% \changes{english-2.1a}{1990/05/14}{Incorporated Nico's comments}
% \changes{english-2.1b}{1990/05/14}{merged \file{USenglish.sty} into
%    this file}
% \changes{english-2.1c}{1990/05/22}{fixed typo in definition for
%    american language found by Werenfried Spit (nspit@fys.ruu.nl)}
% \changes{english-2.1d}{1990/07/16}{Fixed some typos}
% \changes{english-3.0}{1991/04/23}{Modified for babel 3.0}
% \changes{english-3.0a}{1991/05/29}{Removed bug found by van der Meer}
% \changes{english-3.0c}{1991/07/15}{Renamed \file{babel.sty} in
%    \file{babel.com}}
% \changes{english-3.1}{1991/11/05}{Rewrote parts of the code to use
%    the new features of babel version 3.1}
% \changes{english-3.3}{1994/02/08}{Update or \LaTeXe}
% \changes{english-3.3c}{1994/06/26}{Removed the use of \cs{filedate}
%    and moved the identification after the loading of
%    \file{babel.def}}
%
%  \section{The English language}
%
%    The file \file{\filename}\footnote{The file described in this
%    section has version number \fileversion\ and was last revised on
%    \filedate.} defines all the language definition macros for the
%    English language as well as for the American version of this
%    language.
%
%    For this language currently no special definitions are needed or
%    available.
%
% \StopEventually{}
%
% \changes{english-3.0d}{1991/10/22}{Removed code to load
%    \file{latexhax.com}}
%
%    As this file needs to be read only once, we check whether it was
%    read before. If it was, the command |\captionsenglish| is already
%    defined, so we can stop processing. If this command is undefined
%    we proceed with the various definitions and first show the
%    current version of this file.
%
% \changes{english-3.0c}{1991/07/15}{Added reset of catcode of @
%    before \cs{endinput}.}
% \changes{english-3.0d}{1991/10/22}{removed use of \cs{@ifundefined}}
% \changes{english-3.1a}{1991/11/11}{Moved code to the beginning of
%    the file and added \cs{selectlanguage} call}
%    \begin{macrocode}
%<*code>
\ifx\undefined\captionsenglish
\else
  \selectlanguage{english}
  \expandafter\endinput
\fi
%    \end{macrocode}
%
% \begin{macro}{\atcatcode}
%    This file, \file{english.ldf}, may have been read while \TeX\ is
%    in the middle of processing a document, so we have to make sure
%    the category code of \texttt{@} is `letter' while this file is
%    being read. We save the category code of the @-sign in
%    |\atcatcode| and make it `letter'. Later the category code can be
%    restored to whatever it was before.
% \changes{english-3.0b}{1991/06/06}{Made test of catcode of @ more
%    robust}
% \changes{english-3.0c}{1991/07/15}{Modified handling of catcode of @
%    again.}
% \changes{english-3.0d}{1991/10/22}{Removed use of \cs{makeatletter}
%    and hence the need to load \file{latexhax.com}}
%    \begin{macrocode}
\chardef\atcatcode=\catcode`\@
\catcode`\@=11\relax
%    \end{macrocode}
% \end{macro}
%
%    Now we determine whether the common macros from the file
%    \file{babel.def} need to be read. We can be in one of two
%    situations: either another language option has been read earlier
%    on, in which case that other option has already read
%    \file{babel.def}, or \texttt{english} is the first language
%    option to be processed. In that case we need to read
%    \file{babel.def} right here before we continue.
%
% \changes{english-3.0}{1991/04/23}{New check before loading
%    \file{babel.com}}
% \changes{english-3.1c}{1992/02/15}{Added \cs{relax} after the
%    argument of \cs{input}}
%    \begin{macrocode}
\ifx\undefined\babel@core@loaded\input babel.def\relax\fi
%    \end{macrocode}
%
% \changes{english-3.0a}{1991/05/29}{Add a check for existence
%    \cs{originalTeX}}
%
%    Another check that has to be made, is if another language
%    definition file has been read already. In that case its
%    definitions have been activated. This might interfere with
%    definitions this file tries to make. Therefore we make sure that
%    we cancel any special definitions. This can be done by checking
%    the existence of the macro |\originalTeX|. If it exists we simply
%    execute it, otherwise it is |\let| to |\empty|.
% \changes{english-3.0c}{1991/07/15}{Added
%    \cs{let}\cs{originalTeX}\cs{relax} to test for existence}
% \changes{english-3.1b}{1992/01/26}{\cs{originalTeX} should be
%    expandable, \cs{let} it to \cs{empty}}
%    \begin{macrocode}
\ifx\undefined\originalTeX \let\originalTeX\empty\fi
\originalTeX
%    \end{macrocode}
%
%    When this file is read as an option, i.e. by the |\usepackage|
%    command, \texttt{english} could be an `unknown' language in which
%    case we have to make it known.  So we check for the existence of
%    |\l@english| to see whether we have to do something here.
%
% \changes{english-3.0}{1991/04/23}{Now use \cs{adddialect} if
%    language undefined}
% \changes{english-3.0d}{1991/10/22}{removed use of \cs{@ifundefined}}
% \changes{english-3.3c}{1994/06/26}{Now use \cs{@nopatterns} to
%    produce the warning}
%    \begin{macrocode}
\ifx\undefined\l@english
  \ifx\undefined\l@UKenglish
    \@nopatterns{English}
    \adddialect\l@english0
  \else
    \let\l@english\l@UKenglish
  \fi
\fi
%    \end{macrocode}
%    For the American version of these definitions we just add a
%    ``dialect''. Also, the macros |\captionsamerican| and
%    |\extrasamerican| are |\let| to their English counterparts when
%    these parts are defined.
% \changes{english-3.0}{1990/04/23}{Now use \cs{adddialect} for
%    american}
% \changes{english-3.0b}{1991/06/06}{Removed \cs{global} definitions}
% \changes{english-v3.3d}{1995/02/01}{Only define american as a
%    dialect when no separate patterns have been loaded}
%    \begin{macrocode}
\ifx\l@american\undefined
  \adddialect\l@american\l@english
\fi
%    \end{macrocode}
%
%    The next step consists of defining commands to switch to (and
%    from) the English language.
%
% \begin{macro}{\captionsenglish}
%    The macro |\captionsenglish| defines all strings used
%    in the four standard document classes provided with \LaTeX.
% \changes{english-3.0b}{1991/06/06}{Removed \cs{global} definitions}
% \changes{english-3.0b}{1991/06/06}{\cs{pagename} should be
%    \cs{headpagename}}
% \changes{english-3.1a}{1991/11/11}{added \cs{seename} and
%    \cs{alsoname}}
% \changes{english-3.1b}{1992/01/26}{added \cs{prefacename}}
% \changes{english-3.2}{1993/07/15}{\cs{headpagename} should be
%    \cs{pagename}}
% \changes{english-3.3e}{1995/07/04}{Added \cs{proofname} for
%    AMS-\LaTeX}
%    \begin{macrocode}
\addto\captionsenglish{%
  \def\prefacename{Preface}%
  \def\refname{References}%
  \def\abstractname{Abstract}%
  \def\bibname{Bibliography}%
  \def\chaptername{Chapter}%
  \def\appendixname{Appendix}%
  \def\contentsname{Contents}%
  \def\listfigurename{List of Figures}%
  \def\listtablename{List of Tables}%
  \def\indexname{Index}%
  \def\figurename{Figure}%
  \def\tablename{Table}%
  \def\partname{Part}%
  \def\enclname{encl}%
  \def\ccname{cc}%
  \def\headtoname{To}%
  \def\pagename{Page}%
  \def\seename{see}%
  \def\alsoname{see also}%
  \def\proofname{Proof}%
  }
%    \end{macrocode}
% \end{macro}
%
% \begin{macro}{\captionsamerican}
%    The `captions' are the same for both versions of the language, so
%    we can |\let| the macro |\captionsamerican| be equal to
%    |\captionsenglish|.
%    \begin{macrocode}
\let\captionsamerican\captionsenglish
%    \end{macrocode}
% \end{macro}
%
% \begin{macro}{\dateenglish}
%    The macro |\dateenglish| redefines the command |\today| to
%    produce English dates.
% \changes{english-3.0b}{1991/06/06}{Removed \cs{global} definitions}
%    \begin{macrocode}
\def\dateenglish{%
\def\today{\ifcase\day\or
  1st\or 2nd\or 3rd\or 4th\or 5th\or
  6th\or 7th\or 8th\or 9th\or 10th\or
  11th\or 12th\or 13th\or 14th\or 15th\or
  16th\or 17th\or 18th\or 19th\or 20th\or
  21st\or 22nd\or 23rd\or 24th\or 25th\or
  26th\or 27th\or 28th\or 29th\or 30th\or
  31st\fi~\ifcase\month\or
  January\or February\or March\or April\or May\or June\or
  July\or August\or September\or October\or November\or December\fi
  \space \number\year}}
%    \end{macrocode}
% \end{macro}
%
% \begin{macro}{\dateamerican}
%    The macro |\dateamerican| redefines the command |\today| to
%    produce American dates.
% \changes{english-3.0b}{1991/06/06}{Removed \cs{global} definitions}
%    \begin{macrocode}
\def\dateamerican{%
\def\today{\ifcase\month\or
  January\or February\or March\or April\or May\or June\or
  July\or August\or September\or October\or November\or December\fi
  \space\number\day, \number\year}}
%    \end{macrocode}
% \end{macro}
%
% \begin{macro}{\extrasenglish}
% \begin{macro}{\noextrasenglish}
%    The macro |\extrasenglish| will perform all the extra definitions
%    needed for the English language. The macro |\extrasenglish| is
%    used to cancel the actions of |\extrasenglish|.  For the moment
%    these macros are empty but they are defined for compatibility
%    with the other language definition files.
%
%    \begin{macrocode}
\addto\extrasenglish{}
\addto\noextrasenglish{}
%    \end{macrocode}
% \end{macro}
% \end{macro}
%
% \begin{macro}{\extrasamerican}
% \begin{macro}{\noextrasamerican}
%    Also for the ``american'' variant no extra definitions are needed
%    at the moment.
%    \begin{macrocode}
\let\extrasamerican\extrasenglish
\let\noextrasamerican\noextrasenglish
%    \end{macrocode}
% \end{macro}
% \end{macro}
%
%    It is possible that a site might need to add some extra code to
%    the babel macros. To enable this we load a local configuration
%    file, \file{english.cfg} if it is found on \TeX' search path.
% \changes{english-3.3e}{1995/07/02}{Added loading of configuration
%    file}
%    \begin{macrocode}
\loadlocalcfg{english}
%    \end{macrocode}
%
%    Our last action is to make a note that the commands we have just
%    defined, will be executed by calling the macro |\selectlanguage|
%    at the beginning of the document.
%    \begin{macrocode}
\main@language{english}
%    \end{macrocode}
%    Finally, the category code of \texttt{@} is reset to its original
%    value. The macrospace used by |\atcatcode| is freed.
% \changes{english-3.0c}{1991/07/15}{Modified handling of catcode of
%    @-sign.}
%    \begin{macrocode}
\catcode`\@=\atcatcode \let\atcatcode\relax
%</code>
%    \end{macrocode}
%
% \Finale
%%
%% \CharacterTable
%%  {Upper-case    \A\B\C\D\E\F\G\H\I\J\K\L\M\N\O\P\Q\R\S\T\U\V\W\X\Y\Z
%%   Lower-case    \a\b\c\d\e\f\g\h\i\j\k\l\m\n\o\p\q\r\s\t\u\v\w\x\y\z
%%   Digits        \0\1\2\3\4\5\6\7\8\9
%%   Exclamation   \!     Double quote  \"     Hash (number) \#
%%   Dollar        \$     Percent       \%     Ampersand     \&
%%   Acute accent  \'     Left paren    \(     Right paren   \)
%%   Asterisk      \*     Plus          \+     Comma         \,
%%   Minus         \-     Point         \.     Solidus       \/
%%   Colon         \:     Semicolon     \;     Less than     \<
%%   Equals        \=     Greater than  \>     Question mark \?
%%   Commercial at \@     Left bracket  \[     Backslash     \\
%%   Right bracket \]     Circumflex    \^     Underscore    \_
%%   Grave accent  \`     Left brace    \{     Vertical bar  \|
%%   Right brace   \}     Tilde         \~}
%%
\endinput
%
  }
\DeclareOption{esperanto}{% \iffalse meta-comment
%
% Copyright 1989-1995 Johannes L. Braams and any individual authors
% listed elsewhere in this file.  All rights reserved.
% 
% For further copyright information any other copyright notices in this
% file.
% 
% This file is part of the Babel system release 3.5.
% --------------------------------------------------
%   This system is distributed in the hope that it will be useful,
%   but WITHOUT ANY WARRANTY; without even the implied warranty of
%   MERCHANTABILITY or FITNESS FOR A PARTICULAR PURPOSE.
% 
%   For error reports concerning UNCHANGED versions of this file no more
%   than one year old, see bugs.txt.
% 
%   Please do not request updates from me directly.  Primary
%   distribution is through the CTAN archives.
% 
% 
% IMPORTANT COPYRIGHT NOTICE:
% 
% You are NOT ALLOWED to distribute this file alone.
% 
% You are allowed to distribute this file under the condition that it is
% distributed together with all the files listed in manifest.txt.
% 
% If you receive only some of these files from someone, complain!
% 
% Permission is granted to copy this file to another file with a clearly
% different name and to customize the declarations in that copy to serve
% the needs of your installation, provided that you comply with
% the conditions in the file legal.txt from the LaTeX2e distribution.
% 
% However, NO PERMISSION is granted to produce or to distribute a
% modified version of this file under its original name.
%  
% You are NOT ALLOWED to change this file.
% 
% 
% \fi
% \CheckSum{287}
%\iffalse
%    Tell the \LaTeX\ system who we are and write an entry on the
%    transcript.
%<*dtx>
\ProvidesFile{esperant.dtx}
%</dtx>
%<code>\ProvidesFile{esperant.ldf}
        [1995/07/10 v1.4g Esperanto support from the babel system]
%
% Babel package for LaTeX version 2e
% Copyright (C) 1989 - 1995
%           by Johannes Braams, TeXniek
%
% Please report errors to: J.L. Braams
%                          JLBraams@cistron.nl
%
%    This file is part of the babel system, it provides the source
%    code for the Esperanto language definition file.  A contribution
%    was made by Ruiz-Altaba Marti (ruizaltb@cernvm.cern.ch) Code from
%    esperant.sty version 1.1 by Joerg Knappen
%    (\texttt{knappen@vkpmzd.kph.uni-mainz.de}) was included in
%    version 1.2.
%<*filedriver>
\documentclass{ltxdoc}
\newcommand*\TeXhax{\TeX hax}
\newcommand*\babel{\textsf{babel}}
\newcommand*\langvar{$\langle \it lang \rangle$}
\newcommand*\note[1]{}
\newcommand*\Lopt[1]{\textsf{#1}}
\newcommand*\file[1]{\texttt{#1}}
\begin{document}
 \DocInput{esperant.dtx}
\end{document}
%</filedriver>
%\fi
% \GetFileInfo{esperant.dtx}
%
% \changes{esperanto-1.0a}{1991/07/15}{Renamed \file{babel.sty} in
%    \file{babel.com}}
% \changes{esperanto-1.1}{1992/02/15}{Brought up-to-date with
%    babel~3.2a}
% \changes{esperanto-1.2}{1992/02/18}{Included code from
%    \texttt{esperant.sty}}
% \changes{esperanto-1.4a}{1994/02/04}{Updated for \LaTeXe}
% \changes{esperanto-1.4d}{1994/06/25}{Removed the use of
%    \cs{filedate}, moved Identification after loading of
%    \file{babel.def}}
% \changes{esperanto-1.4e}{1995/02/09}{Moved identification code to
%    the top of the file}
% \changes{esperant-1.4f}{1995/06/14}{Corrected typos (PR1652)}
%
%  \section{The Esperanto language}
%
%    The file \file{\filename}\footnote{The file described in this
%    section has version number \fileversion\ and was last revised on
%    \filedate. A contribution was made by Ruiz-Altaba Marti
%    (\texttt{ruizaltb@cernvm.cern.ch}). Code from the file
%    \texttt{esperant.sty} by J\"org Knappen
%    (\texttt{knappen@vkpmzd.kph.uni-mainz.de}) was included.} defines
%    all the language-specific macros for the Esperanto language.
%
%    For this language the character |^| is made active.
%    In table~\ref{tab:esp-act} an overview is given of its purpose.
% \begin{table}[htb]
%    \centering
%     \begin{tabular}{lp{8cm}}
%      |^u| & gives \u u, with hyphenation in the rest of the word
%                   allowed\\
%      |^U| & gives \u U, with hyphenation in the rest of the word
%                   allowed\\
%      |^h| & prevents h\llap{\^{}} from becoming too tall\\
%      |^j| & gives \^\j\\
%      \verb=^|= & inserts a |\discretionary{-}{}{}|\\
%      |^a| & gives \^a with hyphenation in the rest of the word
%                   allowed, this works for all other letters as
%                   well.\\
%      \end{tabular}
%      \caption{The funtions of the active character for Esperanto.}
%    \label{tab:esp-act}
% \end{table}
%
%  \StopEventually{}
%
%    As this file needs to be read only once, we check whether it was
%    read before. If it was, the command |\captionsesperanto| is already
%    defined, so we can stop processing.
%
% \changes{esperanto-1.0a}{1991/07/15}{Added reset of catcode of @
%    before \cs{endinput}.}
% \changes{esperanto-1.0b}{1991/10/29}{Removed use of
%    \cs{@ifundefined}}
%    \begin{macrocode}
%<*code>
\ifx\undefined\captionsesperanto
\else
  \selectlanguage{esperanto}
  \expandafter\endinput
\fi
%    \end{macrocode}
%
% \begin{macro}{\atcatcode}
%    This file, \file{esperant.ldf}, may have been read while \TeX\ is
%    in the middle of processing a document, so we have to make sure
%    the category code of \texttt{@} is `letter' while this file is
%    being read. We save the category code of the @-sign in
%    |\atcatcode| and make it `letter'. Later the category code can be
%    restored to whatever it was before.
%
% \changes{esperanto-1.0a}{1991/07/15}{Modified handling of catcode of
%    @ again.}
% \changes{esperanto-1.0b}{1991/10/29}{Removed use of `makeatletter and
%    hence the need to load \file{latexhax.com}}
%    \begin{macrocode}
\chardef\atcatcode=\catcode`\@
\catcode`\@=11\relax
%    \end{macrocode}
% \end{macro}
%
%    Now we determine whether the the common macros from the file
%    \file{babel.def} need to be read. We can be in one of two
%    situations: either another language option has been read earlier
%    on, in which case that other option has already read
%    \file{babel.def}, or \texttt{esperanto} is the first language
%    option to be processed. In that case we need to read
%    \file{babel.def} right here before we continue.
%
% \changes{esperanto-1.1}{1992/02/15}{Added \cs{relax} after the
%    argument of \cs{input}}
%    \begin{macrocode}
\ifx\undefined\babel@core@loaded\input babel.def\relax\fi
%    \end{macrocode}
%
%    Another check that has to be made, is if another language
%    definition file has been read already. In that case its
%    definitions have been activated. This might interfere with
%    definitions this file tries to make. Therefore we make sure that
%    we cancel any special definitions. This can be done by checking
%    the existence of the macro |\originalTeX|. If it exists we simply
%    execute it, otherwise it is |\let| to |\empty|.
% \changes{esperanto-1.0a}{1991/07/15}{Added
%    \cs{let}\cs{originalTeX}\cs{relax} to test for existence}
% \changes{esperanto-1.1}{1992/02/15}{\cs{originalTeX} should be
%    expandable, \cs{let} it to \cs{empty}}
%    \begin{macrocode}
\ifx\undefined\originalTeX \let\originalTeX\empty \else\originalTeX\fi
%    \end{macrocode}
%
%    When this file is read as an option, i.e. by the |\usepackage|
%    command, \texttt{esperanto} will be an `unknown' language in
%    which case we have to make it known. So we check for the
%    existence of |\l@esperanto| to see whether we have to do
%    something here.
%
% \changes{esperanto-1.0b}{1991/10/29}{Removed use of
%    \cs{makeatletter}}
% \changes{esperanto-1.1}{1992/02/15}{Added a warning when no
%    hyphenation patterns were loaded.}
% \changes{esperanto-1.4d}{1994/06/25}{Use \cs{@nopatterns} for the
%    warning}
%    \begin{macrocode}
\ifx\undefined\l@esperanto
  \@nopatterns{Esperanto}
  \adddialect\l@esperanto0\fi
%    \end{macrocode}
%
%    The next step consists of defining commands to switch to the
%    Esperanto language. The reason for this is that a user might want
%    to switch back and forth between languages.
%
% \begin{macro}{\captionsesperanto}
%    The macro |\captionsesperanto| defines all strings used
%    in the four standard documentclasses provided with \LaTeX.
% \changes{esperanto-1.1}{1992/02/15}{Added \cs{seename},
%    \cs{alsoname} and \cs{prefacename}}
% \changes{esperanto-1.3}{1993/07/10}{Repaired a number of mistakes,
%    indicated by D. Ederveen}
% \changes{esperanto-1.3}{1993/07/15}{\cs{headpagename} should be
%    \cs{pagename}}
% \changes{esperanto-1.4a}{1994/02/04}{added missing closing brace}
% \changes{esperanto-1.4g}{1995/07/04}{Added \cs{proofname} for
%    AMS-\LaTeX}
%    \begin{macrocode}
\addto\captionsesperanto{%
  \def\prefacename{Anta\u{u}parolo}%
  \def\refname{Cita\^\j{}oj}%
  \def\abstractname{Resumo}%
  \def\bibname{Bibliografio}%
  \def\chaptername{{\^C}apitro}%
  \def\appendixname{Apendico}%
  \def\contentsname{Enhavo}%
  \def\listfigurename{Listo de figuroj}%
  \def\listtablename{Listo de tabeloj}%
  \def\indexname{Indekso}%
  \def\figurename{Figuro}%
  \def\tablename{Tabelo}%
  \def\partname{Parto}%
  \def\enclname{Aldono(j)}%
  \def\ccname{Kopie al}%
  \def\headtoname{Al}%
  \def\pagename{Pa\^go}%
  \def\subjectname{Temo}%
  \def\seename{vidu}%   a^u: vd.
  \def\alsoname{vidu anka\u{u}}% a^u vd. anka\u{u}
  \def\proofname{Proof}%  <-- needs translation
  }
%    \end{macrocode}
% \end{macro}
%
% \begin{macro}{\dateesperanto}
%    The macro |\dateesperanto| redefines the command |\today| to
%    produce Esperanto dates.
% \changes{esperanto-1.}{1993/07/10}{Removed the capitals from
%    \cs{today}}
%    \begin{macrocode}
\def\dateesperanto{%
\def\today{\number\day{--a}~de~\ifcase\month\or
  januaro\or februaro\or marto\or aprilo\or majo\or junio\or
  julio\or a\u{u}gusto\or septembro\or oktobro\or novembro\or
  decembro\fi,\space \number\year}}
%    \end{macrocode}
% \end{macro}
%
% \begin{macro}{\extrasesperanto}
% \begin{macro}{\noextrasesperanto}
%    The macro |\extrasesperanto| performs all the extra definitions
%    needed for the Esperanto language. The macro |\noextrasesperanto|
%    is used to cancel the actions of |\extrasesperanto|.
%
%    For Esperanto the |^| character is made active. This is done
%    once, later on its definition may vary.
%
%    \begin{macrocode}
\initiate@active@char{^}
\addto\extrasesperanto{\languageshorthands{esperanto}}
\addto\extrasesperanto{\bbl@activate{^}}
\addto\noextrasesperanto{\bbl@deactivate{^}}
%    \end{macrocode}
% \end{macro}
% \end{macro}
%
%    And here are the uses of the active |^|:
%    \begin{macrocode}
\declare@shorthand{esperanto}{^u}{\u u\allowhyphens}
\declare@shorthand{esperanto}{^U}{\u U\allowhyphens}
\declare@shorthand{esperanto}{^h}{h\llap{\^{}}\allowhyphens}
\declare@shorthand{esperanto}{^j}{\^{\j}\allowhyphens}
\declare@shorthand{esperanto}{^|}{\discretionary{-}{}{}\allowhyphens}
%    \end{macrocode}
%
% \begin{macro}{\Esper}
% \begin{macro}{\esper}
%    In \file{esperant.sty} J\"org Knappen provides the macros
%    |\esper| and |\Esper| that can be used instead of |\alph| and
%    |\Alph|. These macros are available in this file as well.
%
%    Their definition takes place in three steps. First the toplevel.
%    \begin{macrocode}
\def\esper#1{\@esper{\@nameuse{c@#1}}}
\def\Esper#1{\@Esper{\@nameuse{c@#1}}}
%    \end{macrocode}
%    Then the first five occasions that are probably used the most.
%    \begin{macrocode}
\def\@esper#1{\ifcase#1\or a\or b\or c\or \^c\or d\else\@iesper{#1}\fi}
\def\@Esper#1{\ifcase#1\or A\or B\or C\or \^C\or D\else\@Iesper{#1}\fi}
%    \end{macrocode}
%    And the 33 other cases.
%    \begin{macrocode}
\def\@iesper#1{\ifcase#1\or \or \or \or \or \or e\or f\or g\or \^g\or
    h\or h\llap{\^{}}\or i\or j\or \^\j\or k\orl\or m\or n\or o\or
    p\or s\or \^s\or t\or u\or \u{u}\or v\or z\else\@ctrerr\fi}
%    \end{macrocode}
%    \begin{macrocode}
\def\@Iesper#1{\ifcase#1\or \or \or \or \or \or E\or F\or G\or \^G\or
    H\or \^H\or I\or J\or \^\J\or K\or L\or M\or N\or O\or
    P\or S\or \^S\or T\or U\or \u{U}\or V\or Z\else\@ctrerr\fi}
%    \end{macrocode}
% \end{macro}
% \end{macro}
%
% \begin{macro}{\hodiau}
% \begin{macro}{\hodiaun}
%    In \file{esperant.sty} J\"org Knappen provides two alternative
%    macros for |\today|, |\hodiau| and |\hodiaun|. The second macro
%    produces an accusative version of the date in Esperanto.
%    \begin{macrocode}
\addto\dateesperanto{\def\hodiau{la \today}}
\def\hodiaun{la \number\day --an~de~\ifcase\month\or
  januaro\or februaro\or marto\or aprilo\or majo\or junio\or
  julio\or a\u{u}gusto\or septembro\or oktobro\or novembro\or
  decembro\fi, \space \number\year}
%    \end{macrocode}
% \end{macro}
% \end{macro}
%
%    It is possible that a site might need to add some extra code to
%    the babel macros. To enable this we load a local configuration
%    file, \file{esperant.cfg} if it is found on \TeX' search path.
% \changes{esperanto-1.4g}{1995/07/04}{Added loading of configuration
%    file}
%    \begin{macrocode}
\loadlocalcfg{esperant}
%    \end{macrocode}
%
%    Our last action is to make a note that the commands we have just
%    defined, will be executed by calling the macro |\selectlanguage|
%    at the beginning of the document.
%    \begin{macrocode}
\main@language{esperanto}
%    \end{macrocode}
%    Finally, the category code of \texttt{@} is reset to its original
%    value. The macrospace used by |\atcatcode| is freed.
% \changes{esperanto-1.0a}{1991/07/15}{Modified handling of catcode of
%    the @-sign.}
%    \begin{macrocode}
\catcode`\@=\atcatcode \let\atcatcode\relax
%</code>
%    \end{macrocode}
%
% \Finale
%%
%% \CharacterTable
%%  {Upper-case    \A\B\C\D\E\F\G\H\I\J\K\L\M\N\O\P\Q\R\S\T\U\V\W\X\Y\Z
%%   Lower-case    \a\b\c\d\e\f\g\h\i\j\k\l\m\n\o\p\q\r\s\t\u\v\w\x\y\z
%%   Digits        \0\1\2\3\4\5\6\7\8\9
%%   Exclamation   \!     Double quote  \"     Hash (number) \#
%%   Dollar        \$     Percent       \%     Ampersand     \&
%%   Acute accent  \'     Left paren    \(     Right paren   \)
%%   Asterisk      \*     Plus          \+     Comma         \,
%%   Minus         \-     Point         \.     Solidus       \/
%%   Colon         \:     Semicolon     \;     Less than     \<
%%   Equals        \=     Greater than  \>     Question mark \?
%%   Commercial at \@     Left bracket  \[     Backslash     \\
%%   Right bracket \]     Circumflex    \^     Underscore    \_
%%   Grave accent  \`     Left brace    \{     Vertical bar  \|
%%   Right brace   \}     Tilde         \~}
%%
\endinput
}
%    \end{macrocode}
% \changes{babel~3.5b}{1995/06/06}{Added estonian option}
%    \begin{macrocode}
\DeclareOption{estonian}{% \iffalse meta-comment
%
% Copyright 1989-1995 Johannes L. Braams and any individual authors
% listed elsewhere in this file.  All rights reserved.
% 
% For further copyright information any other copyright notices in this
% file.
% 
% This file is part of the Babel system release 3.5.
% --------------------------------------------------
%   This system is distributed in the hope that it will be useful,
%   but WITHOUT ANY WARRANTY; without even the implied warranty of
%   MERCHANTABILITY or FITNESS FOR A PARTICULAR PURPOSE.
% 
%   For error reports concerning UNCHANGED versions of this file no more
%   than one year old, see bugs.txt.
% 
%   Please do not request updates from me directly.  Primary
%   distribution is through the CTAN archives.
% 
% 
% IMPORTANT COPYRIGHT NOTICE:
% 
% You are NOT ALLOWED to distribute this file alone.
% 
% You are allowed to distribute this file under the condition that it is
% distributed together with all the files listed in manifest.txt.
% 
% If you receive only some of these files from someone, complain!
% 
% Permission is granted to copy this file to another file with a clearly
% different name and to customize the declarations in that copy to serve
% the needs of your installation, provided that you comply with
% the conditions in the file legal.txt from the LaTeX2e distribution.
% 
% However, NO PERMISSION is granted to produce or to distribute a
% modified version of this file under its original name.
%  
% You are NOT ALLOWED to change this file.
% 
% 
% \fi
% \CheckSum{422}
% \iffalse
%    Tell the \LaTeX\ system who we are and write an entry on the
%    transcript.
%<*dtx>
\ProvidesFile{estonian.dtx}
%</dtx>
%<code>\ProvidesFile{estonian.ldf}
        [1995/07/04 v1.0c Estonian support from the babel system]
%
% Babel package for LaTeX version 2e
% Copyright (C) 1989 - 1995
%           by Johannes Braams, TeXniek
%
% Estonian language Definition File
% Copyright (C) 1991 - 1995
%           by Enn Saar, Tartu Astrophysical Observatory
%              Tartu Astrophysical Observatory
%              EE-2444 T\~oravere
%              Estonia
%              tel: +372 7 410 267
%              fax: +372 7 410 205
%              saar@aai.ee
%
%              Johannes Braams, TeXniek
%
% Please report errors to: Enn Saar <saar@aai.ee>
%                          (or J.L. Braams <JLBraams@cistron.nl)
%
%    This file is part of the babel system, it provides the source
%    code for the Estonian language definition file.  The original
%    version of this file was written by Enn Saar,
%    (saar@aai.ee).
%<*filedriver>
\documentclass{ltxdoc}
\newcommand*\TeXhax{\TeX hax}
\newcommand*\babel{\textsf{babel}}
\newcommand*\langvar{$\langle \it lang \rangle$}
\newcommand*\note[1]{}
\newcommand*\Lopt[1]{\textsf{#1}}
\newcommand*\file[1]{\texttt{#1}}
\begin{document}
 \DocInput{estonian.dtx}
\end{document}
%</filedriver>
%\fi
%
% \changes{estonian-1.0b}{1995/06/16}{corrected typos}
%
% \GetFileInfo{estonian.dtx}
%
% \section{The Estonian language}
%
%    The file \file{\filename}\footnote{The file described in this
%    section has version number \fileversion\ and was last revised on
%    \filedate. The original author is Enn Saar,
%    (\texttt{saar@aai.ee}).}  defines the language definition macro's
%    for the Estonian language.
%
%    This file was written as part of the TWGML project, and borrows
%    heavily from the \babel\ German and Spanish language files
%    \file{germanb.ldf} and \file{spanish.ldf}.
%
%    Estonian has the same umlauts as German (\"a, \"o, \"u), but in
%    addition to this, we have also \~o, and two recent characters
%    \v s and \v z, so we need at least two active characters.
%    We shall use |"| and |~| to type Estonian accents on ASCII
%    keyboards (in the 7-bit character world). Their use is given in
%    table~\ref{tab:estonian-quote}.
%    \begin{table}[htb]
%     \begin{center}
%     \begin{tabular}{lp{8cm}}
%      |~o| & |\~o|, (and uppercase); \\
%      |"a| & |\"a|, (and uppercase); \\
%      |"o| & |\"o|, (and uppercase); \\
%      |"u| & |\"u|, (and uppercase); \\
%      |~s| & |\v s|, (and uppercase); \\
%      |~z| & |\v z|, (and uppercase); \\
%      \verb="|= & disable ligature at this position;\\
%      |"-| & an explicit hyphen sign, allowing hyphenation
%                  in the rest of the word;\\
%      |\-| & like the old |\-|, but allowing hyphenation
%             in the rest of the word; \\
%      |"`| & for Estonian low left double quotes (same as German);\\
%      |"'| & for Estonian right double quotes;\\
%      |"<| & for French left double quotes (also rather popular)\\
%      |">| & for French right double quotes.\\
%     \end{tabular}
%     \caption{The extra definitions made
%              by \file{estonian.ldf}}\label{tab:estonian-quote}
%     \end{center}
%    \end{table}
%    These active accent characters behave according to their original
%    definitions if not followed by one of the characters indicated in
%    that table; the original quote character can be typed using the
%    macro |\dq|.
%
%    We support also the T1 output encoding (and Cork-encoded text
%    input).  You can choose the T1 encoding by the command
%    |\usepackage[T1]{fontenc}|.  This package must be loaded before
%    \babel. As the standard Estonian hyphenation file
%    \file{eehyph.tex} is in the Cork encoding, choosing this encoding
%    will give you better hyphenation.
%
%    As mentioned in the Spanish style file, it may happen that some
%    packages fail (usually in a \cs{message}). In this case you
%    should change the order of the \cs{usepackage} declarations
%    or the order of the style options in \cs{documentclass}.
%
% \StopEventually{}
%
% \subsection{Implementation}
%
%    Check whether the file has been read already.
%
%    \begin{macrocode}
%<*code>
\ifx\undefined\captionsestonian
\else
  \selectlanguage{estonian}
  \expandafter\endinput
\fi
%    \end{macrocode}
%
%    Change the category code of \texttt{@}, as usual.
%
%    \begin{macrocode}
\chardef\atcatcode=\catcode`\@
\catcode`\@=11\relax
%    \end{macrocode}
%
%    Check whether we need the common macros from the file
%    \file{babel.def}.
%
%    \begin{macrocode}
\ifx\undefined\babel@core@loaded\input babel.def\relax\fi
%    \end{macrocode}
%
%
%    We execute the macro |\originalTeX| to get rid of side effects
%    that could be caused by language options used before.
%
%    \begin{macrocode}
\ifx\undefined\originalTeX \let\originalTeX\empty \else\originalTeX\fi
%    \end{macrocode}
%
%    If Estonian is not included in the format file (does not have
%    hyphenation patterns), we shall use English hyphenation.
%
%    \begin{macrocode}
\ifx\undefined\l@estonian
  \@nopatterns{Estonian}
  \adddialect\l@estonian0
\fi
%    \end{macrocode}
%
%    Now come the commands to switch to (and from) Estonian.
%
%  \begin{macro}{\captionsestonian}
%    The macro |\captionsestonian| defines all strings used in the
%    four standard documentclasses provided with \LaTeX.
%
% \changes{estonian-1.0c}{1995/07/04}{Added \cs{proofname} for
%    AMS-\LaTeX}
%    \begin{macrocode}
\addto\captionsestonian{%
  \def\prefacename{Sissejuhatus}%
  \def\refname{Viited}%
  \def\bibname{Kirjandus}%
  \def\appendixname{Lisa}%
  \def\contentsname{Sisukord}%
  \def\listfigurename{Joonised}%
  \def\listtablename{Tabelid}%
  \def\indexname{Indeks}%
  \def\figurename{Joonis}%
  \def\tablename{Tabel}%
  \def\partname{Osa}%
  \def\enclname{Lisa(d)}%
  \def\ccname{Koopia(d)}%
  \def\headtoname{}%
  \def\pagename{Lk.}%
  \def\seename{vt.}%
  \def\alsoname{vt. ka}
  \def\proofname{Proof}%   <-- needs translation
  }
%    \end{macrocode}
%
%    These captions contain accented characters.
%
%    \begin{macrocode}
\begingroup \catcode`\"\active
\def\x{\endgroup
\addto\captionsestonian{%
  \def\abstractname{Kokkuv~ote}%
  \def\chaptername{Peat"ukk}}}
\x
%    \end{macrocode}
%  \end{macro}
%
%  \begin{macro}{\dateestonian}
%    The macro |\dateestonian| redefines the command |\today| to
%    produce Estonian dates.
%
%    \begin{macrocode}
\begingroup \catcode`\"\active
\def\x{\endgroup
   \def\month@estonian{\ifcase\month\or
     jaanuar\or veebruar\or m"arts\or aprill\or mai\or juuni\or
     juuli\or august\or september\or oktoober\or november\or
         detsember\fi}}
\x
\def\dateestonian{\def\today{\number\day.\space\month@estonian
  \space\number\year.\space a.}}
%    \end{macrocode}
%  \end{macro}
%
%  \begin{macro}{\extrasestonian}
%  \begin{macro}{\noextrasestonian}
%    The macro |\extrasestonian| will perform all the extra
%    definitions needed for Estonian. The macro |\noextrasestonian| is
%    used to cancel the actions of |\extrasestonian|. For Estonian,
%    |"| is made active and has to be treated as `special' (|~| is
%    active already).
%
%    \begin{macrocode}
\initiate@active@char{"}
\initiate@active@char{~}
\addto\extrasestonian{\languageshorthands{estonian}}
\addto\extrasestonian{\bbl@activate{"}\bbl@activate{~}}
%    \end{macrocode}
%    Store the original macros, and redefine accents.
%
%    \begin{macrocode}
\addto\extrasestonian{\babel@save\"\umlautlow\babel@save\~\tildelow}
%    \end{macrocode}
%
%    If we are using the T1 output encoding, we should allow for the
%    T1 input encoding, too (this has been chosen as the preliminary
%    archiving standard by TWGML).
%
%    \begin{macrocode}
\edef\next{T1}
\ifx\f@encoding\next
   \addto\extrasestonian{%
      \catcode245=11 \catcode228=11 \catcode246=11 \catcode252=11
      \catcode178=11 \catcode186=11 \catcode213=11 \catcode196=11
      \catcode214=11 \catcode220=11 \catcode146=11 \catcode154=11
      \lccode245=245 \lccode228=228 \lccode246=246 \lccode252=252
      \lccode178=178 \lccode186=186 \lccode213=245 \lccode196=228
      \lccode214=246 \lccode220=252 \lccode146=178 \lccode154=186
      \uccode245=213 \uccode228=196 \uccode246=214 \uccode252=220
      \uccode178=146 \uccode186=154 \uccode213=213 \uccode196=196
      \uccode214=214 \uccode220=220 \uccode146=146 \uccode154=154
      \sfcode245=1000 \sfcode228=1000 \sfcode246=1000 \sfcode252=1000
      \sfcode178=1000 \sfcode186=1000 \sfcode213=999 \sfcode196=999
      \sfcode214=999 \sfcode220=999 \sfcode146=999 \sfcode154=999}
\fi
%    \end{macrocode}
%
%    Estonian does not use extra spaces after sentences.
%
%    \begin{macrocode}
\addto\extrasestonian{\bbl@frenchspacing}
\addto\noextrasestonian{\bbl@nonfrenchspacing}
%    \end{macrocode}
%  \end{macro}
%  \end{macro}
%
%  \begin{macro}{\estonianhyphenmins}
%     For Estonian, |\lefthyphenmin| and |\righthyphenmin| are both 2.
%
%    \begin{macrocode}
\def\estonianhyphenmins{\tw@\tw@}
%    \end{macrocode}
%  \end{macro}
%
%  \begin{macro}{\tildelow}
%  \begin{macro}{\gentilde}
%  \begin{macro}{\newtilde}
%  \begin{macro}{\newcheck}
%    The standard \TeX\ accents are too high for Estonian typography,
%    we have to lower them (following the \babel\ German style).  For
%    a detailed explanation see the file \file{glyphs.dtx}.
%
%    \begin{macrocode}
\def\tildelow{\def\~{\protect\gentilde}}
\def\gentilde#1{\if#1o\newtilde{#1}\else\if#1O\newtilde{#1}%
    \else\newcheck{#1}%
    \fi\fi}
\def\newtilde#1{\leavevmode\allowhyphens
  {\U@D 1ex%
  {\setbox\z@\hbox{\char126}\dimen@ -.45ex\advance\dimen@\ht\z@
  \ifdim 1ex<\dimen@ \fontdimen5\font\dimen@ \fi}%
  \accent126\fontdimen5\font\U@D #1}\allowhyphens}
\def\newcheck#1{\leavevmode\allowhyphens
  {\U@D 1ex%
  {\setbox\z@\hbox{\char20}\dimen@ -.45ex\advance\dimen@\ht\z@
  \ifdim 1ex<\dimen@ \fontdimen5\font\dimen@ \fi}%
  \accent20\fontdimen5\font\U@D #1}\allowhyphens}
%    \end{macrocode}
%  \end{macro}
%  \end{macro}
%  \end{macro}
%  \end{macro}
%
%    We save the double quote character in |\dq|, and  tilde in |\til|,
%    and store the original definitions of |\"| and |~| as |\dieresis|
%    and |\texttilde|.
%
%    \begin{macrocode}
\begingroup \catcode`\"12
\edef\x{\endgroup
  \def\noexpand\dq{"}
  \def\noexpand\til{~}}
\x
\let\dieresis\"
\let\texttilde\~
%    \end{macrocode}
%
%    This part follows closely \file{spanish.ldf}. We check the
%    encoding and if it is T1, we have to tell \TeX\ about our
%    redefined accents.
%
%    \begin{macrocode}
\edef\next{T1}
\ifx\f@encoding\next
  \let\@umlaut\dieresis
  \let\@tilde\texttilde
  \DeclareTextComposite{\~}{T1}{s}{178}
  \DeclareTextComposite{\~}{T1}{S}{146}
  \DeclareTextComposite{\~}{T1}{z}{186}
  \DeclareTextComposite{\~}{T1}{Z}{154}
  \DeclareTextComposite{\"}{T1}{'}{17}
  \DeclareTextComposite{\"}{T1}{`}{18}
  \DeclareTextComposite{\"}{T1}{<}{19}
  \DeclareTextComposite{\"}{T1}{>}{20}
%    \end{macrocode}
%
%    If the encoding differs from T1, we expand the accents, enabling
%    hyphenation beyond the accent. In this case \TeX\ will not find
%    all possible breaks, and we have to warn people.
%
%    \begin{macrocode}
\else
  \wlog{Warning: Hyphenation would work better for the T1 encoding.}
  \let\@umlaut\newumlaut
  \let\@tilde\gentilde
\fi
%    \end{macrocode}
%
%     Now we define the shorthands.
%
%    \begin{macrocode}
\declare@shorthand{estonian}{\textormath{\"{a}}{\ddot a}}
\declare@shorthand{estonian}{\textormath{\"{A}}{\ddot A}}
\declare@shorthand{estonian}{\textormath{\"{o}}{\ddot o}}
\declare@shorthand{estonian}{\textormath{\"{O}}{\ddot O}}
\declare@shorthand{estonian}{\textormath{\"{u}}{\ddot u}}
\declare@shorthand{estonian}{\textormath{\"{U}}{\ddot U}}
%    \end{macrocode}
%    german and french quotes,
%    \begin{macrocode}
\declare@shorthand{estonian}{"`}{%
  \textormath{\quotedblbase{}}{\mbox{\quotedblbase}}}
\declare@shorthand{estonian}{"'}{%
  \textormath{\textquotedblleft{}}{\mbox{\textquotedblleft}}}
\declare@shorthand{estonian}{"<}{%
  \textormath{\guillemotleft{}}{\mbox{\guillemotleft}}}
\declare@shorthand{estonian}{">}{%
  \textormath{\guillemotright{}}{\mbox{\guillemotright}}}
%    \end{macrocode}
%    
%    \begin{macrocode}
\declare@shorthand{estonian}{~o}{\textormath{\@tilde o}{\tilde o}}
\declare@shorthand{estonian}{~O}{\textormath{\@tilde O}{\tilde O}}
\declare@shorthand{estonian}{~s}{\textormath{\@tilde s}{\check s}}
\declare@shorthand{estonian}{~S}{\textormath{\@tilde S}{\check S}}
\declare@shorthand{estonian}{~z}{\textormath{\@tilde z}{\check z}}
\declare@shorthand{estonian}{~Z}{\textormath{\@tilde Z}{\check Z}}
%    \end{macrocode}
%    and some additional commands:
%    \begin{macrocode}
\declare@shorthand{estonian}{"-}{\allowhyphens\-\allowhyphens}
\declare@shorthand{estonian}{"|}{%
  \textormath{\penalty\@M\discretionary{-}{}{\kern.03em}%
              \allowhyphens}{}}
\declare@shorthand{estonian}{""}{\dq}
\declare@shorthand{estonian}{~~}{\til}
%    \end{macrocode}
%
%    It is possible that a site might need to add some extra code to
%    the babel macros. To enable this we load a local configuration
%    file, \file{estonian.cfg} if it is found on \TeX' search path.
% \changes{estonian-1.0c}{1995/07/02}{Added loading of configuration
%    file}
%    \begin{macrocode}
\loadlocalcfg{estonian}
%    \end{macrocode}
%
%    Select, finally, Estonian, and restore the category code
%    of \texttt{@}. We are done.
%
%    \begin{macrocode}
\main@language{estonian}
\catcode`\@=\atcatcode \let\atcatcode\relax
%</code>
%    \end{macrocode}
%
% \Finale
%%
%% \CharacterTable
%%  {Upper-case    \A\B\C\D\E\F\G\H\I\J\K\L\M\N\O\P\Q\R\S\T\U\V\W\X\Y\Z
%%   Lower-case    \a\b\c\d\e\f\g\h\i\j\k\l\m\n\o\p\q\r\s\t\u\v\w\x\y\z
%%   Digits        \0\1\2\3\4\5\6\7\8\9
%%   Exclamation   \!     Double quote  \"     Hash (number) \#
%%   Dollar        \$     Percent       \%     Ampersand     \&
%%   Acute accent  \'     Left paren    \(     Right paren   \)
%%   Asterisk      \*     Plus          \+     Comma         \,
%%   Minus         \-     Point         \.     Solidus       \/
%%   Colon         \:     Semicolon     \;     Less than     \<
%%   Equals        \=     Greater than  \>     Question mark \?
%%   Commercial at \@     Left bracket  \[     Backslash     \\
%%   Right bracket \]     Circumflex    \^     Underscore    \_
%%   Grave accent  \`     Left brace    \{     Vertical bar  \|
%%   Right brace   \}     Tilde         \~}
%%
\endinput
}
\DeclareOption{finnish}{% \iffalse meta-comment
%
% Copyright 1989-1995 Johannes L. Braams and any individual authors
% listed elsewhere in this file.  All rights reserved.
% 
% For further copyright information any other copyright notices in this
% file.
% 
% This file is part of the Babel system release 3.5.
% --------------------------------------------------
%   This system is distributed in the hope that it will be useful,
%   but WITHOUT ANY WARRANTY; without even the implied warranty of
%   MERCHANTABILITY or FITNESS FOR A PARTICULAR PURPOSE.
% 
%   For error reports concerning UNCHANGED versions of this file no more
%   than one year old, see bugs.txt.
% 
%   Please do not request updates from me directly.  Primary
%   distribution is through the CTAN archives.
% 
% 
% IMPORTANT COPYRIGHT NOTICE:
% 
% You are NOT ALLOWED to distribute this file alone.
% 
% You are allowed to distribute this file under the condition that it is
% distributed together with all the files listed in manifest.txt.
% 
% If you receive only some of these files from someone, complain!
% 
% Permission is granted to copy this file to another file with a clearly
% different name and to customize the declarations in that copy to serve
% the needs of your installation, provided that you comply with
% the conditions in the file legal.txt from the LaTeX2e distribution.
% 
% However, NO PERMISSION is granted to produce or to distribute a
% modified version of this file under its original name.
%  
% You are NOT ALLOWED to change this file.
% 
% 
% \fi
% \CheckSum{205}
% \iffalse
%    Tell the \LaTeX\ system who we are and write an entry on the
%    transcript.
%<*dtx>
\ProvidesFile{finnish.dtx}
%</dtx>
%<code>\ProvidesFile{finnish.ldf}
        [1995/07/02 v1.3g Finnish support from the babel system]
%
% Babel package for LaTeX version 2e
% Copyright (C) 1989 - 1995
%           by Johannes Braams, TeXniek
%
% Please report errors to: J.L. Braams
%                          JLBraams@cistron.nl
%
%    This file is part of the babel system, it provides the source
%    code for the Finnish language definition file.  A contribution
%    was made by Mikko KANERVA (KANERVA@CERNVM) and Keranen Reino
%    (KERANEN@CERNVM).
%<*filedriver>
\documentclass{ltxdoc}
\newcommand*\TeXhax{\TeX hax}
\newcommand*\babel{\textsf{babel}}
\newcommand*\langvar{$\langle \it lang \rangle$}
\newcommand*\note[1]{}
\newcommand*\Lopt[1]{\textsf{#1}}
\newcommand*\file[1]{\texttt{#1}}
\begin{document}
 \DocInput{finnish.dtx}
\end{document}
%</filedriver>
%\fi
%
% \GetFileInfo{finnish.dtx}
%
% \changes{finnish-1.0a}{1991/07/15}{Renamed \file{babel.sty} in
%    \file{babel.com}}
% \changes{finnish-1.1}{1991/02/15}{Brought up-to-date with babel 3.2a}
% \changes{finnish-1.2}{1994/02/27}{Update for \LaTeXe}
% \changes{finnish-1.3c}{1994/06/26}{Removed the use of \cs{filedate}
%    and moved identification after the loading of \file{babel.def}}
% \changes{finnish-1.3d}{1994/06/30}{Removed a few references to
%    \file{babel.com}}
%
%  \section{The Finnish language}
%
%    The file \file{\filename}\footnote{The file described in this
%    section has version number \fileversion\ and was last revised on
%    \filedate.  A contribution was made by Mikko KANERVA
%    (\texttt{KANERVA@CERNVM}) and Keranen Reino
%    (\texttt{KERANEN@CERNVM}).}  defines all the language definition
%    macros for the Finnish language.
%
%    For this language the character |"| is made active. In
%    table~\ref{tab:finnish-quote} an overview is given of its purpose.
%
%    \begin{table}[htb]
%     \centering
%     \begin{tabular}{lp{8cm}}
%       \verb="|= & disable ligature at this position.\\
%        |"-| & an explicit hyphen sign, allowing hyphenation
%               in the rest of the word.\\
%        |"=| & an explicit hyphen sign for expressions such as
%               ``pakastekaapit ja -arkut''.\\
%        |""| & like \verb="-=, but producing no hyphen sign (for
%              words that should break at some sign such as
%              ``entrada/salida.''\\
%        |"`| & lowered double left quotes (looks like ,,)\\
%        |"'| & normal double right quotes\\
%        |"<| & for French left double quotes (similar to $<<$).\\
%        |">| & for French right double quotes (similar to $>>$).\\
%        |\-| & like the old |\-|, but allowing hyphenation
%               in the rest of the word.
%     \end{tabular}
%     \caption{The extra definitions made by \file{finnish.ldf}}
%     \label{tab:finnish-quote}
%    \end{table}
%
% \StopEventually{}
%
%    As this file needs to be read only once, we check whether it was
%    read before. If it was, the command |\captionsfinnish| is already
%    defined, so we can stop processing. If this command is undefined
%    we proceed with the various definitions and first show the
%    current version of this file.
%
% \changes{finnish-1.0a}{1991/07/15}{Added reset of catcode of @
%    before rns were loaded.}
%    \begin{macrocode}
%<*code>
\ifx\undefined\captionsfinnish
\else
  \selectlanguage{finnish}
  \expandafter\endinput
\fi
%    \end{macrocode}
%
% \changes{finnish-1.0b}{1991/10/29}{Removed code to load
%    \file{latexhax.com}}
%
% \begin{macro}{\atcatcode}
%    This file, \file{finnish.ldf}, may have been read while \TeX\ is
%    in the middle of processing a document, so we have to make sure
%    the category code of \texttt{@} is `letter' while this file is
%    being read.  We save the category code of the @-sign in
%    |\atcatcode| and make it `letter'. Later the category code can be
%    restored to whatever it was before.
%
% \changes{finnish-1.0a}{1991/07/15}{Modified handling of catcode of @
%    again.}
% \changes{finnish-1.0b}{1991/10/29}{Removed use of \cs{makeatletter}
%    and hence the need to load \file{latexhax.com}}
%    \begin{macrocode}
\chardef\atcatcode=\catcode`\@
\catcode`\@=11\relax
%    \end{macrocode}
% \end{macro}
%
%    Now we determine whether the the common macros from the file
%    \file{babel.def} need to be read. We can be in one of two
%    situations: either another language option has been read earlier
%    on, in which case that other option has already read
%    \file{babel.def}, or \texttt{finnish} is the first language
%    option to be processed. In that case we need to read
%    \file{babel.def} right here before we continue.
%
% \changes{finnish-1.1}{1992/02/15}{Added \cs{relax} after the
%    argument of \cs{input}}
%    \begin{macrocode}
\ifx\undefined\babel@core@loaded\input babel.def\relax\fi
%    \end{macrocode}
%
%    Another check that has to be made, is if another language
%    definition file has been read already. In that case its
%    definitions have been activated. This might interfere with
%    definitions this file tries to make. Therefore we make sure that
%    we cancel any special definitions. This can be done by checking
%    the existence of the macro |\originalTeX|. If it exists we simply
%    execute it, otherwise it is |\let| to |\empty|.
% \changes{finnish-1.0a}{1991/07/15}{Added
%    \cs{let}\cs{originalTeX}\cs{relax} to test for existence}
% \changes{finnish-1.1}{1992/02/15}{\cs{originalTeX} should be
%    expandable, \cs{let} it to \cs{empty}}
%    \begin{macrocode}
\ifx\undefined\originalTeX \let\originalTeX\empty \else\originalTeX\fi
%    \end{macrocode}
%
%    When this file is read as an option, i.e. by the |\usepackage|
%    command, \texttt{finnish} will be an `unknown' language in which
%    case we have to make it known.  So we check for the existence of
%    |\l@finnish| to see whether we have to do something here.
%
% \changes{finnish-1.0b}{1991/10/29}{Removed use of \cs{@ifundefined}}
% \changes{finnish-1.1}{1992/02/15}{Added a warning when no hyphenation
%    patterns were loaded.}
% \changes{finnish-1.3c}{1994/06/26}{Now use \cs{@nopatterns} to
%    produce the warning}
%    \begin{macrocode}
\ifx\undefined\l@finnish
    \@nopatterns{Finnish}
    \adddialect\l@finnish0\fi
%    \end{macrocode}
%
%    The next step consists of defining commands to switch to the
%    Finnish language. The reason for this is that a user might want
%    to switch back and forth between languages.
%
% \begin{macro}{\captionsfinnish}
%    The macro |\captionsfinnish| defines all strings used in the four
%    standard documentclasses provided with \LaTeX.
% \changes{finnish-1.1}{1992/02/15}{Added \cs{seename}, \cs{alsoname}
%    and \cs{prefacename}}
% \changes{finnish-1.1}{1993/07/15}{\cs{headpagename} should be
%    \cs{pagename}}
% \changes{finnish-1.1.2}{1993/09/16}{Added translations}
% \changes{finnish-1.3g}{1995/07/02}{Added \cs{proofname} for
%    AMS-\LaTeX}
%    \begin{macrocode}
\addto\captionsfinnish{%
  \def\prefacename{Esipuhe}%
  \def\refname{Viitteet}%
  \def\abstractname{Tiivistelm\"a}
  \def\bibname{Kirjallisuutta}%
  \def\chaptername{Luku}%
  \def\appendixname{Liite}%
  \def\contentsname{Sis\"alt\"o}%   /* Could be "Sis\"allys" as well */
  \def\listfigurename{Kuvat}%
  \def\listtablename{Taulukot}%
  \def\indexname{Hakemisto}%
  \def\figurename{Kuva}%
  \def\tablename{Taulukko}%
  \def\partname{Osa}%
  \def\enclname{Liitteet}%
  \def\ccname{Jakelu}%
  \def\headtoname{Vastaanottaja}%
  \def\pagename{Sivu}%
  \def\seename{katso}%
  \def\alsoname{katso my\"os}%
  \def\proofname{Proof}%  <-- needs translation!
  }%
%    \end{macrocode}
% \end{macro}
%
% \begin{macro}{\datefinnish}
%    The macro |\datefinnish| redefines the command |\today| to
%    produce Finnish dates.
% \changes{finnish-1.3e}{1994/07/12}{Added a`.' after the number of
%    the day}
%    \begin{macrocode}
\def\datefinnish{%
\def\today{\number\day.~\ifcase\month\or
  tammikuuta\or helmikuuta\or maaliskuuta\or huhtikuuta\or
  toukokuuta\or kes\"akuuta\or hein\"akuuta\or elokuuta\or
  syyskuuta\or lokakuuta\or marraskuuta\or joulukuuta\fi
  \space\number\year}}
%    \end{macrocode}
% \end{macro}
%
% \begin{macro}{\extrasfinnish}
% \begin{macro}{\noextrasfinnish}
%    Finnish has many long words (some of them compound, some not).
%    For this reason hyphenation is very often the only solution in
%    line breaking. For this reason the values of |\hyphenpenalty|,
%    |\exhyphenpenalty| and |\doublehyphendemerits| should be
%    decreased. (In one of the manuals of style Matti Rintala noticed
%    a paragraph with ten lines, eight of which ended in a hyphen!)
%
%    Matti Rintala noticed that with these changes \TeX\ handles
%    Finnish very well, although sometimes the values of |\tolerance|
%    and |\emergencystretch| must be increased. However, I don't think
%    changing these values in \file{finnish.ldf} is appropriate, as
%    the looseness of the font (and the line width) affect the correct
%    choice of these parameters.
% \changes{finnish-1.3f}{1995/05/13}{Added the setting of more
%    hyphenation parameters, according to PR1027}
%    \begin{macrocode}
\addto\extrasfinnish{%
  \babel@savevariable\hyphenpenalty\hyphenpenalty=30%
  \babel@savevariable\exhyphenpenalty\exhyphenpenalty=30%
  \babel@savevariable\doublehyphendemerits\doublehyphendemerits=5000%
  \babel@savevariable\finalhyphendemerits\finalhyphendemerits=5000%
  }
\addto\noextrasfinnish{}
%    \end{macrocode}
%
%    Another thing |\extrasfinnish| needs to do is to make sure that
%    |\frenchspacing| is in effect.  If this is not the case the
%    execution of |\noextrasfinnish| will switch it of again.
% \changes{finnish-1.3f}{1995/05/15}{Added the setting of
%    \cs{frenchspacing}}
%    \begin{macrocode}
\addto\extrasfinnish{\bbl@frenchspacing}
\addto\noextrasfinnish{\bbl@nonfrenchspacing}
%    \end{macrocode}
%
%    For Finnish the \texttt{"} character is made active. This is
%    done once, later on its definition may vary. Other languages in
%    the same document may also use the \texttt{"} character for
%    shorthands; we specify that the finnish group of shorthands
%    should be used.
% \changes{finnish-1.3g}{1995/07/02}{Added the active double quote}
%    \begin{macrocode}
\initiate@active@char{"}
\addto\extrasfinnish{\languageshorthands{finnish}}
\addto\extrasfinnish{\bbl@activate{"}}
%\addto\noextrasfinnish{\bbl@deactivate{"}}
%    \end{macrocode}
%
%
%    The `umlaut' character should be positioned lower on \emph{all}
%    vowels in Finnish texts.
%    \begin{macrocode}
\addto\extrasfinnish{\umlautlow\umlautelow}
\addto\noextrasfinnish{\umlauthigh}
%    \end{macrocode}
%
%    First we define access to the low opening double quote and
%    guillemets for quotations,
%    \begin{macrocode}
\declare@shorthand{finnish}{"`}{%
  \textormath{\quotedblbase{}}{\mbox{\quotedblbase}}}
\declare@shorthand{finnish}{"'}{%
  \textormath{\textquotedblright{}}{\mbox{\textquotedblright}}}
\declare@shorthand{finnish}{"<}{%
  \textormath{\guillemotleft{}}{\mbox{\guillemotleft}}}
\declare@shorthand{finnish}{">}{%
  \textormath{\guillemotright{}}{\mbox{\guillemotright}}}
%    \end{macrocode}
%    then we define two shorthands to be able to specify hyphenation
%    breakpoints that behavew a little different from |\-|.
%    \begin{macrocode}
\declare@shorthand{finnish}{"-}{\allowhyphens-\allowhyphens}
\declare@shorthand{finnish}{""}{\hskip\z@skip}
\declare@shorthand{finnish}{"=}{\hbox{-}\allowhyphens}
%    \end{macrocode}
%    And we want to have a shorthand for disabling a ligature.
%    \begin{macrocode}
\declare@shorthand{finnish}{"|}{%
  \textormath{\discretionary{-}{}{\kern.03em}}{}}
%    \end{macrocode}
% \end{macro}
% \end{macro}
%
%  \begin{macro}{\-}
%
%    All that is left now is the redefinition of |\-|. The new version
%    of |\-| should indicate an extra hyphenation position, while
%    allowing other hyphenation positions to be generated
%    automatically. The standard behaviour of \TeX\ in this respect is
%    very unfortunate for languages such as Dutch, Finnish and German,
%    where long compound words are quite normal and all one needs is a
%    means to indicate an extra hyphenation position on top of the
%    ones that \TeX\ can generate from the hyphenation patterns.
% \changes{finnish-1.3g}{1995/07/02}{Added change of \cs{-}}
%    \begin{macrocode}
\addto\extrasfinnish{\babel@save\-}
\addto\extrasfinnish{\def\-{\allowhyphens
                          \discretionary{-}{}{}\allowhyphens}}
%    \end{macrocode}
%  \end{macro}
%
%  \begin{macro}{\finishhyphenmins}
%    The finnish hyphenation patterns can be used with |\lefthyphenmin|
%    set to~2 and |\righthyphenmin| set to~2.
% \changes{finnish-1.3f}{1995/05/13}{use \cs{finnishhyphenmins} to
%    store the correct values}
%    \begin{macrocode}
\def\finnishhyphenmins{\tw@\tw@}
%    \end{macrocode}
%  \end{macro}
%
%    It is possible that a site might need to add some extra code to
%    the babel macros. To enable this we load a local configuration
%    file, \file{finnish.cfg} if it is found on \TeX' search path.
% \changes{finnish-1.3g}{1995/07/02}{Added loading of configuration
%    file}
%    \begin{macrocode}
\loadlocalcfg{finnish}
%    \end{macrocode}
%
%    Our last action is to make a note that the commands we have just
%    defined, will be executed by calling the macro |\selectlanguage|
%    at the beginning of the document.
%    \begin{macrocode}
\main@language{finnish}
%    \end{macrocode}
%    Finally, the category code of \texttt{@} is reset to its original
%    value. The macrospace used by |\atcatcode| is freed.
% \changes{finnish-1.0a}{1991/07/15}{Modified handling of catcode of
%    @-sign.}
%    \begin{macrocode}
\catcode`\@=\atcatcode \let\atcatcode\relax
%</code>
%    \end{macrocode}
%
% \Finale
%%
%% \CharacterTable
%%  {Upper-case    \A\B\C\D\E\F\G\H\I\J\K\L\M\N\O\P\Q\R\S\T\U\V\W\X\Y\Z
%%   Lower-case    \a\b\c\d\e\f\g\h\i\j\k\l\m\n\o\p\q\r\s\t\u\v\w\x\y\z
%%   Digits        \0\1\2\3\4\5\6\7\8\9
%%   Exclamation   \!     Double quote  \"     Hash (number) \#
%%   Dollar        \$     Percent       \%     Ampersand     \&
%%   Acute accent  \'     Left paren    \(     Right paren   \)
%%   Asterisk      \*     Plus          \+     Comma         \,
%%   Minus         \-     Point         \.     Solidus       \/
%%   Colon         \:     Semicolon     \;     Less than     \<
%%   Equals        \=     Greater than  \>     Question mark \?
%%   Commercial at \@     Left bracket  \[     Backslash     \\
%%   Right bracket \]     Circumflex    \^     Underscore    \_
%%   Grave accent  \`     Left brace    \{     Vertical bar  \|
%%   Right brace   \}     Tilde         \~}
%%
\endinput

}
%    \end{macrocode}
%    The \babel\ support or French is stored in \file{francais.ldf};
%    therefore the \LaTeX2.09 option used to be \Lopt{francais}.
%    The hyphenation patterns may be loaded as either `french' or as
%    `francais'.
%    \begin{macrocode}
\DeclareOption{francais}{%
  \ifx\l@francais\undefined
    \let\l@francais\l@french
  \fi
  % \iffalse meta-comment
%
% Copyright 1989-1995 Johannes L. Braams and any individual authors
% listed elsewhere in this file.  All rights reserved.
% 
% For further copyright information any other copyright notices in this
% file.
% 
% This file is part of the Babel system release 3.5.
% --------------------------------------------------
%   This system is distributed in the hope that it will be useful,
%   but WITHOUT ANY WARRANTY; without even the implied warranty of
%   MERCHANTABILITY or FITNESS FOR A PARTICULAR PURPOSE.
% 
%   For error reports concerning UNCHANGED versions of this file no more
%   than one year old, see bugs.txt.
% 
%   Please do not request updates from me directly.  Primary
%   distribution is through the CTAN archives.
% 
% 
% IMPORTANT COPYRIGHT NOTICE:
% 
% You are NOT ALLOWED to distribute this file alone.
% 
% You are allowed to distribute this file under the condition that it is
% distributed together with all the files listed in manifest.txt.
% 
% If you receive only some of these files from someone, complain!
% 
% Permission is granted to copy this file to another file with a clearly
% different name and to customize the declarations in that copy to serve
% the needs of your installation, provided that you comply with
% the conditions in the file legal.txt from the LaTeX2e distribution.
% 
% However, NO PERMISSION is granted to produce or to distribute a
% modified version of this file under its original name.
%  
% You are NOT ALLOWED to change this file.
% 
% 
% \fi
% \CheckSum{342}
%
% \iffalse
%    Tell the \LaTeX\ system who we are and write an entry on the
%    transcript.
%<*dtx>
\ProvidesFile{francais.dtx}
%</dtx>
%<code>\ProvidesFile{francais.ldf}
        [1995/07/09 v4.6c French support from the babel system]
%
% Babel package for LaTeX version 2e
% Copyright (C) 1989 - 1995
%           by Johannes Braams, TeXniek
%
% Francais language Definition File
% Copyright (C) 1989 - 1995
%           by Johannes Braams, TeXniek
%              Bernard Gaulle, GUTenberg
%
% Please report errors to: J.L. Braams
%                          JLBraams@cistron.nl
%
%    This file is part of the babel system, it provides the source
%    code for the French language definition file.
%<*filedriver>
\documentclass{ltxdoc}
\newcommand*\TeXhax{\TeX hax}
\newcommand*\babel{\textsf{babel}}
\newcommand*\langvar{$\langle \mathit lang \rangle$}
\newcommand*\note[1]{}
\newcommand*\Lopt[1]{\textsf{#1}}
\newcommand*\file[1]{\texttt{#1}}
\begin{document}
 \DocInput{francais.dtx}
\end{document}
%</filedriver>
%\fi
% \GetFileInfo{francais.dtx}
%
% \changes{french-2.0a}{1990/04/02}{Added checking of format}
% \changes{french-2.1}{1990/04/24}{Reflect changes in babel 2.1}
% \changes{french-2.1a}{1990/05/14}{Incorporated Nico's comments}
% \changes{french-2.1b}{1990/07/16}{Fixed some typos}
% \changes{french-2.2c}{1990/08/27}{Modified the documentation
%    somewhat}
% \changes{french-3.0}{1991/04/23}{Modified for babel 3.0}
% \changes{french-3.0a}{1991/05/23}{removed use of \cs{setlanguage}}
% \changes{french-3.0b}{1991/05/28}{renamed from \texttt{french},
%    including all control sequences}
% \changes{french-4.0}{1991/05/29}{included code from GUTenberg
%    \file{french.sty}}
% \changes{french-4.1}{1991/05/29}{Removed bug found by van der Meer}
% \changes{french-4.2a}{1991/07/15}{Renamed babel.sty in
%    \file{babel.com}}
% \changes{french-4.2f}{1992/02/15}{Brought up-to-date with babel
%    3.2a}
% \changes{french-4.2f}{1992/02/15}{Removed crossreferencing
%    macros as they are dealt with in \file{babel.com}}
% \changes{french-4.5}{1994/02/27}{Update for LaTeX2e}
% \changes{french-4.5c}{1994/06/26}{Removed the use of \cs{filedate}
%    and moved the identification after the loading of
%    \file{babel.def}}
%
%  \section{The French language}
%
%    The file \file{\filename}\footnote{The file described in this
%    section has version number \fileversion\ and was last revised on
%    \filedate. This file was initially derived from the original
%    version of \file{german.sty}, which has some definitions for
%    French. Later the definitions from \file{french.sty} version 2
%    were added.} defines all the language definition macros for the
%    French language.
%
%    French typographic rules specify that a little white space should
%    be present before `double puctuation' characters. These
%    characters are \texttt{:}, \texttt{;}, \texttt{!} and
%    \texttt{?}. In order to get this whitespace automatigically the
%    category code of these characters is made |\active|. The user
%    should input these four characters preceeded with a space; the
%    space will then be replaced by a |\thinspace|.
%
% \StopEventually{}
%
%    As this file needs to be read only once, we check whether it was
%    read before. If it was, the |\captionsfrancais| is already
%    defined, so we can stop processing. If this command is undefined
%    we proceed with the various definitions and first show the
%    current version of this file.
%
% \changes{french-4.2a}{1991/07/15}{Added reset of catcode of @ before
%    \cs{endinput}.}
% \changes{french-4.2e}{1991/10/29}{Removed use of \cs{@ifundefined}}
%    \begin{macrocode}
%<*code>
\ifx\undefined\captionsfrancais
\else
  \selectlanguage{francais}
  \expandafter\endinput
\fi
%    \end{macrocode}
%
% \changes{french-4.2e}{1991/10/29}{Removed code to load
%    \file{latexhax.com}}
%
% \begin{macro}{\atcatcode}
%    This file, \file{francais.ldf}, may have been read while \TeX\ is
%    in the middle of processing a document, so we have to make sure
%    the category code of \texttt{@} is `letter' while this file is
%    being read. We save the category code of the @-sign in
%    |\atcatcode| and make it `letter'. Later the category code can be
%    restored to whatever it was before.
% \changes{french-4.1a}{1991/06/06}{Made test of catcode of @ more
%    robust}
% \changes{french-4.2a}{1991/07/15}{Modified handling of catcode of @
%    again.}
% \changes{french-4.2e}{1991/10/29}{Removed use of \cs{makeatletter}
%    and hence the need to load \file{latexhax.com}}
%    \begin{macrocode}
\chardef\atcatcode=\catcode`\@
\catcode`\@=11\relax
%    \end{macrocode}
% \end{macro}
%
%    Now we determine whether the the common macros from the file
%    \file{babel.def} need to be read. We can be in one of two
%    situations: either another language option has been read earlier
%    on, in which case that other option has already read
%    \file{babel.def}, or \texttt{francais} is the first language
%    option to be processed. In that case we need to read
%    \file{babel.def} right here before we continue.
% \changes{french-3.0}{1991/04/23}{New check before loading
%    \file{babel.com}}
% \changes{french-4.2f}{1992/02/16}{Added \cs{relax} after the
%    argument of \cs{input}}
%    \begin{macrocode}
\ifx\undefined\babel@core@loaded\input babel.def\relax\fi
%    \end{macrocode}
%
% \changes{french-4.1}{1991/05/29}{Add a check for existence
%    \cs{originalTeX}} Another check that has to be made, is if
%    another language definition file has been read already. In that
%    case its definitions have been activated. This might interfere
%    with definitions this file tries to make. Therefore we make sure
%    that we cancel any special definitions. This can be done by
%    checking the existence of the macro |\originalTeX|. If it exists
%    we simply execute it, otherwise it is |\let| to |\empty|.
% \changes{french-4.2a}{1991/07/15}{Added
%    \cs{let}\cs{originalTeX}\cs{relax} to test for existence}
% \changes{french-4.2f}{1992/02/16}{\cs{originalTeX} should be
%    expandable, \cs{let} it to \cs{empty}}
%    \begin{macrocode}
\ifx\undefined\originalTeX \let\originalTeX\empty \fi
\originalTeX
%    \end{macrocode}
%
%    When this file is read as an option, i.e. by the |\usepackage|
%    command, \texttt{francais} will be an `unknown' language in which
%    case we have to make it known.  So we check for the existence of
%    |\l@francais| to see whether we have to do something here.
%
% \changes{french-3.0}{1991/04/23}{Now use \cs{adddialect} if language
%    undefined}
% \changes{french-4.2e}{1991/10/29}{Removed use of \cs{@ifundefined}}
% \changes{french-4.2f}{1992/02/16}{Added a warning when no hyphenation
%    patterns were loaded.}
% \changes{french-4.5c}{1994/06/26}{Now use \cs{@nopatterns} to
%    produce the warning}
% \changes{french-4.6c}{1995/07/02}{Also allow the hyphenation
%    patterns to be loaded for `french'}
%    \begin{macrocode}
\ifx\l@francais\undefined
  \ifx\l@french\undefined
    \@nopatterns{Francais}
    \adddialect\l@francais0
    \let\l@french\l@francais
  \else
    \let\l@francais\l@french
  \fi
\fi
%    \end{macrocode}
%    The next step consists of defining commands to switch to the
%    English language. The reason for this is that a user might want
%    to switch back and forth between languages.
%
% \begin{macro}{\captionsfrancais}
%    The macro |\captionsfrancais| defines all strings used in the
%    four standard document classes provided with \LaTeX.
% \changes{french-4.1a}{1991/06/06}{Removed \cs{bsl global}
%    definitions}
% \changes{french-4.2f}{1992/02/16}{Added \cs{seename}, \cs{alsoname}
%    and \cs{prefacename}}
% \changes{french-4.4}{1993/07/11}{\cs{headpagename} should be
%    \cs{pagename}}
% \changes{french-4.6c}{1995/07/02}{Added \cs{proofname} for
%    AMS-\LaTeX}
%    \begin{macrocode}
\addto\captionsfrancais{%
  \def\prefacename{Pr\'eface}%
  \def\refname{R\'ef\'erences}%
  \def\abstractname{R\'esum\'e}%
  \def\bibname{Bibliographie}%
  \def\chaptername{Chapitre}%
  \def\appendixname{Annexe}%
  \def\contentsname{Table des mati\`eres}%
  \def\listfigurename{Liste des figures}%
  \def\listtablename{Liste des tableaux}%
  \def\indexname{Index}%
  \def\figurename{Figure}%
  \def\tablename{Tableau}%
  \def\partname{Partie}%
  \def\enclname{P.~J.}%
  \def\ccname{Copie \`a}%
  \def\headtoname{A}
  \def\pagename{Page}%
  \def\seename{voir}%
  \def\alsoname{voir aussi}%
  \def\proofname{Proof}%  <-- needs translation!
  }
%    \end{macrocode}
% \end{macro}
%
% \begin{macro}{\datefrancais}
%    The macro |\datefrancais| redefines the command |\today| to
%    produce French dates.
% \changes{french-4.1a}{1991/06/06}{Removed \cs{global} definitions}
%    \begin{macrocode}
\def\datefrancais{%
\def\today{\ifnum\day=1\relax 1\/$^{\rm er}$\else
  \number\day\fi \space\ifcase\month\or
  janvier\or f\'evrier\or mars\or avril\or mai\or juin\or
  juillet\or ao\^ut\or septembre\or octobre\or novembre\or
  d\'ecembre\fi
  \space\number\year}}
%    \end{macrocode}
% \end{macro}
%
% \begin{macro}{\extrasfrancais}
% \changes{french-4.3}{1992/02/20}{Completely rewrote macro}
% \begin{macro}{\noextrasfrancais}
%    The macro |\extrasfrancais| will perform all the extra
%    definitions needed for the French language. The macro
%    |\noextrasfrancais| is used to cancel the actions of
%    |\extrasfrancais|.
%
%    The category code of the characters \texttt{:}, \texttt{;},
%    \texttt{!} and \texttt{?} is made |\active| to insert a little
%    white space.
% \changes{french-4.6a}{1995/03/07}{Use the new mechanism for dealing
%    with active chars}
%    \begin{macrocode}
\initiate@active@char{:}
\initiate@active@char{;}
\initiate@active@char{!}
\initiate@active@char{?}
%    \end{macrocode}
%    We specify that the french group of shorthands should be used.
%    \begin{macrocode}
\addto\extrasfrancais{\languageshorthands{french}}
%    \end{macrocode}
%    These characters are `turned on' once, later their definition may
%    vary. 
%    \begin{macrocode}
\addto\extrasfrancais{%
  \bbl@activate{:}\bbl@activate{;}%
  \bbl@activate{!}\bbl@activate{?}}
%\addto\noextrasfrancais{%
%  \bbl@deactivate{:}\bbl@deactivate{;}%
%  \bbl@deactivate{!}\bbl@deactivate{?}}
%    \end{macrocode}
%
%    The last thing |\extrasfrancais| needs to do is to make sure that
%    |\frenchspacing| is in effect.  If this is not the case the
%    execution of |\noextrasfrancais| will switch it off again.
% \changes{french-4.3a}{1992/07/02}{Removed spurious \cs{endgroup} and
%    \texttt{\}}}
% \changes{french-4.6a}{1995/03/14}{now use \cs{bbl@frenchspacing} and
%    \cs{bbl@nonfrenchspacing}}
%    \begin{macrocode}
\addto\extrasfrancais{\bbl@frenchspacing}
\addto\noextrasfrancais{\bbl@nonfrenchspacing}
%    \end{macrocode}
% \end{macro}
% \end{macro}
%
% \begin{macro}{\french@sh@;@}
%    We have to reduce the amount of white space before \texttt{;},
%    \texttt{:} and \texttt{!} when the user types a space in front of
%    these characters. This should only happen outside mathmode, hence
%    the test with |\ifmmode|.
%
% \changes{french-4.3b}{1993/04/04}{Replaced \cs{,} with \cs{thinspace}
%    to make it work with plain TeX.}
% \changes{french-4.6a}{1995/02/19}{Use new \cs{DefineActiveNoArg}}
% \changes{french-4.6a}{1995/03/05}{Use the more general mechanism of
%    \cs{declare@shorthand}}
%    \begin{macrocode}
\declare@shorthand{french}{;}{%
  \ifmmode
    \string;\space
  \else\relax
%    \end{macrocode}
%    In horizontal mode we check for the presence of a `space' and
%    replace it by a |\thinspace|.
%    \begin{macrocode}
    \ifhmode
      \ifdim\lastskip>\z@
        \unskip\penalty\@M\thinspace
      \fi
    \fi
%    \end{macrocode}
%    Now we can insert a |;| character.
%    \begin{macrocode}
    \string;\space
  \fi}
%    \end{macrocode}
% \end{macro}
%
% \begin{macro}{\french@sh@:@}
% \changes{french-4.3b}{1993/04/04}{Replaced \cs{,} with \cs{thinspace}
%    to make it work with plain TeX.}
% \changes{french-4.6a}{1995/02/19}{Use new \cs{DefineActiveNoArg}}
% \begin{macro}{\french@sh@!@}
% \changes{french-4.3b}{1993/04/04}{Replaced \cs{,} with \cs{thinspace}
%    to make it work with plain TeX.}
% \changes{french-4.6a}{1995/02/19}{Use new \cs{DefineActiveNoArg}}
% \changes{french-4.6a}{1995/03/05}{Use the more general mechanism of
%    \cs{declare@shorthand}}
%
%    Because these definitions are very similar only one is displayed
%    in a way that the definition can be easily checked.
%
%    \begin{macrocode}
\declare@shorthand{french}{:}{%
  \ifmmode\string:\space
  \else\relax
    \ifhmode
      \ifdim\lastskip>\z@\unskip\penalty\@M\thinspace\fi
    \fi
    \string:\space
  \fi}
\declare@shorthand{french}{!}{%
  \ifmmode\string!\space
  \else\relax
    \ifhmode
      \ifdim\lastskip>\z@\unskip\penalty\@M\thinspace\fi
    \fi
    \string!\space
  \fi}
%    \end{macrocode}
% \end{macro}
% \end{macro}
%
% \begin{macro}{\french@sh@?@}
%    For the question mark something different has to be done. In this
%    case the amount of white space that replaces the space character
%    depends on the dimensions of the font.
%
% \changes{french-4.3b}{1993/04/04}{Replaced \cs{,} with \cs{thinspace}
%    to make it work with plain TeX.}
% \changes{french-4.6a}{1995/02/19}{Use new \cs{DefineActiveNoArg}}
% \changes{french-4.6a}{1995/03/05}{Use the more general mechanism of
%    \cs{declare@shorthand}}
%    \begin{macrocode}
\declare@shorthand{french}{?}{%
  \ifmmode\string?\space
  \else\relax
    \ifhmode
      \ifdim\lastskip>\z@
        \unskip
        \kern\fontdimen2\font
        \kern-1.4\fontdimen3\font
      \fi
    \fi
    \string?\space
  \fi}
%    \end{macrocode}
% \end{macro}
%
%  \begin{macro}{\system@sh@:@}
%  \begin{macro}{\system@sh@!@}
%  \begin{macro}{\system@sh@?@}
%  \begin{macro}{\system@sh@;@}
% \changes{french-4.6b}{1995/06/03}{Added system level shorthands}
%    When the active characters appear in an environment where their
%    french behaviour is not wanted they should give an `expected'
%    result, ie not gobble up the space that follows them. Therefore
%    we define shorthands at system level as well.
%    \begin{macrocode}
\declare@shorthand{system}{:}{\string:\space}
\declare@shorthand{system}{!}{\string!\space}
\declare@shorthand{system}{?}{\string?\space}
\declare@shorthand{system}{;}{\string;\space}
%    \end{macrocode}
%  \end{macro}
%  \end{macro}
%  \end{macro}
%  \end{macro}
%
%    All that is left to do now is provide the french user with some
%    extra utilities.
%
%    Some definitions for special characters.
%    \begin{macrocode}
\DeclareTextSymbol{\at}{OT1}{64}
\DeclareTextSymbol{\at}{T1}{64}
\DeclareTextSymbolDefault{\at}{OT1}
\DeclareTextSymbol{\boi}{OT1}{92}
\DeclareTextSymbol{\boi}{T1}{16}
\DeclareTextSymbolDefault{\boi}{OT1}
\DeclareTextSymbol{\circonflexe}{OT1}{94}
\DeclareTextSymbol{\circonflexe}{T1}{2}
\DeclareTextSymbolDefault{\circonflexe}{OT1}
\DeclareTextSymbol{\tild}{OT1}{126}
\DeclareTextSymbol{\tild}{T1}{3}
\DeclareTextSymbolDefault{\tild}{OT1}
\DeclareTextSymbol{\degre}{OT1}{23}
\DeclareTextSymbol{\degre}{T1}{6}
\DeclareTextSymbolDefault{\degre}{OT1}
%    \end{macrocode}
%
%    The following macros are used in the redefinition of |\^| and
%    |\"| to handle the letter i.
% \changes{francais-4.6c}{1995/07/07}{Postpone the declaration of the
%    TextCompositeCommands untill \cs{AtBeginDocument}}
%
%    \begin{macrocode}
\AtBeginDocument{%
  \DeclareTextCompositeCommand{\^}{OT1}{i}{\^\i}
  \DeclareTextCompositeCommand{\"}{OT1}{i}{\"\i}}
%    \end{macrocode}
%
%    A macro for typesetting things like 1\raise1ex\hbox{\small er} as
%    proposed by Raymon Seroul.
%    \begin{macrocode}
\def\up#1{\raise 1ex\hbox{\small#1}}
%    \end{macrocode}
%
%    Definitions as provided by Nicolas Brouard for typing |\No3| to
%    get 3\kern-.25em\lower.2ex\hbox{\char'27} and for typing
%    |4\ieme| to get 4$^{\rm e }$\kern+.17em.
%    \begin{macrocode}
\def\No{\kern-.25em\lower.2ex\hbox{\degre}}
\def\ieme{$^{\rm e }$\kern+.17em}
%    \end{macrocode}
%
%    And some more macros for numbering.
%    First two support macros.
%    \begin{macrocode}
\def\FrenchEnumerate#1{$#1^{\rm o}$\kern+.29em}
\def\FrenchPopularEnumerate#1{#1\No\kern-.25em)\kern+.3em}
%    \end{macrocode}
%
%    Typing |\primo| should result in `$1^{\rm o}$\kern+.29em',
%    \begin{macrocode}
\def\primo{\FrenchEnumerate1}
\def\secundo{\FrenchEnumerate2}
\def\tertio{\FrenchEnumerate3}
\def\quatro{\FrenchEnumerate4}
%    \end{macrocode}
%    while typing |\fprimo)| gives
%    `1\kern-.25em\lower.2ex\hbox{\char'27}\kern-.25em)\kern+.3em'.
%    \begin{macrocode}
\def\fprimo){\FrenchPopularEnumerate1}
\def\fsecundo){\FrenchPopularEnumerate2}
\def\ftertio){\FrenchPopularEnumerate3}
\def\fquatro){\FrenchPopularEnumerate4}
%    \end{macrocode}
%
%    It is possible that a site might need to add some extra code to
%    the babel macros. To enable this we load a local configuration
%    file, \file{francais.cfg} if it is found on \TeX' search path.
% \changes{french-4.6c}{1995/07/02}{Added loading of configuration
%    file}
%    \begin{macrocode}
\loadlocalcfg{francais}
%    \end{macrocode}
%
%    Our last action is to make a note that the commands we have just
%    defined, will be executed by calling the macro |\selectlanguage|
%    at the beginning of the document.
%    \begin{macrocode}
\main@language{francais}
%    \end{macrocode}
%    Finally, the category code of \texttt{@} is reset to its original
%    value. The macrospace used by |\atcatcode| is freed.
% \changes{french-4.2a}{1991/07/15}{Modified handling of catcode of
%    @-sign.}
%    \begin{macrocode}
\catcode`\@=\atcatcode \let\atcatcode\relax
%</code>
%    \end{macrocode}
%
% \Finale
%%
%% \CharacterTable
%%  {Upper-case    \A\B\C\D\E\F\G\H\I\J\K\L\M\N\O\P\Q\R\S\T\U\V\W\X\Y\Z
%%   Lower-case    \a\b\c\d\e\f\g\h\i\j\k\l\m\n\o\p\q\r\s\t\u\v\w\x\y\z
%%   Digits        \0\1\2\3\4\5\6\7\8\9
%%   Exclamation   \!     Double quote  \"     Hash (number) \#
%%   Dollar        \$     Percent       \%     Ampersand     \&
%%   Acute accent  \'     Left paren    \(     Right paren   \)
%%   Asterisk      \*     Plus          \+     Comma         \,
%%   Minus         \-     Point         \.     Solidus       \/
%%   Colon         \:     Semicolon     \;     Less than     \<
%%   Equals        \=     Greater than  \>     Question mark \?
%%   Commercial at \@     Left bracket  \[     Backslash     \\
%%   Right bracket \]     Circumflex    \^     Underscore    \_
%%   Grave accent  \`     Left brace    \{     Vertical bar  \|
%%   Right brace   \}     Tilde         \~}
%%
\endinput
%
  }
%    \end{macrocode}
%    With \LaTeXe\ we can now also use the option \Lopt{french} and
%    still call the file \file{francais.ldf}.
% \changes{babel~3.5d}{1995/07/02}{Load \file{french.ldf} when it is
%    found instead of \file{francais.ldf}}
%    \begin{macrocode}
\DeclareOption{french}{%
  \ifx\l@french\undefined
    \let\l@french\l@francais
  \fi
  \IfFileExists{french.ldf}{%
    \input{french.ldf}}{%
    % \iffalse meta-comment
%
% Copyright 1989-1995 Johannes L. Braams and any individual authors
% listed elsewhere in this file.  All rights reserved.
% 
% For further copyright information any other copyright notices in this
% file.
% 
% This file is part of the Babel system release 3.5.
% --------------------------------------------------
%   This system is distributed in the hope that it will be useful,
%   but WITHOUT ANY WARRANTY; without even the implied warranty of
%   MERCHANTABILITY or FITNESS FOR A PARTICULAR PURPOSE.
% 
%   For error reports concerning UNCHANGED versions of this file no more
%   than one year old, see bugs.txt.
% 
%   Please do not request updates from me directly.  Primary
%   distribution is through the CTAN archives.
% 
% 
% IMPORTANT COPYRIGHT NOTICE:
% 
% You are NOT ALLOWED to distribute this file alone.
% 
% You are allowed to distribute this file under the condition that it is
% distributed together with all the files listed in manifest.txt.
% 
% If you receive only some of these files from someone, complain!
% 
% Permission is granted to copy this file to another file with a clearly
% different name and to customize the declarations in that copy to serve
% the needs of your installation, provided that you comply with
% the conditions in the file legal.txt from the LaTeX2e distribution.
% 
% However, NO PERMISSION is granted to produce or to distribute a
% modified version of this file under its original name.
%  
% You are NOT ALLOWED to change this file.
% 
% 
% \fi
% \CheckSum{342}
%
% \iffalse
%    Tell the \LaTeX\ system who we are and write an entry on the
%    transcript.
%<*dtx>
\ProvidesFile{francais.dtx}
%</dtx>
%<code>\ProvidesFile{francais.ldf}
        [1995/07/09 v4.6c French support from the babel system]
%
% Babel package for LaTeX version 2e
% Copyright (C) 1989 - 1995
%           by Johannes Braams, TeXniek
%
% Francais language Definition File
% Copyright (C) 1989 - 1995
%           by Johannes Braams, TeXniek
%              Bernard Gaulle, GUTenberg
%
% Please report errors to: J.L. Braams
%                          JLBraams@cistron.nl
%
%    This file is part of the babel system, it provides the source
%    code for the French language definition file.
%<*filedriver>
\documentclass{ltxdoc}
\newcommand*\TeXhax{\TeX hax}
\newcommand*\babel{\textsf{babel}}
\newcommand*\langvar{$\langle \mathit lang \rangle$}
\newcommand*\note[1]{}
\newcommand*\Lopt[1]{\textsf{#1}}
\newcommand*\file[1]{\texttt{#1}}
\begin{document}
 \DocInput{francais.dtx}
\end{document}
%</filedriver>
%\fi
% \GetFileInfo{francais.dtx}
%
% \changes{french-2.0a}{1990/04/02}{Added checking of format}
% \changes{french-2.1}{1990/04/24}{Reflect changes in babel 2.1}
% \changes{french-2.1a}{1990/05/14}{Incorporated Nico's comments}
% \changes{french-2.1b}{1990/07/16}{Fixed some typos}
% \changes{french-2.2c}{1990/08/27}{Modified the documentation
%    somewhat}
% \changes{french-3.0}{1991/04/23}{Modified for babel 3.0}
% \changes{french-3.0a}{1991/05/23}{removed use of \cs{setlanguage}}
% \changes{french-3.0b}{1991/05/28}{renamed from \texttt{french},
%    including all control sequences}
% \changes{french-4.0}{1991/05/29}{included code from GUTenberg
%    \file{french.sty}}
% \changes{french-4.1}{1991/05/29}{Removed bug found by van der Meer}
% \changes{french-4.2a}{1991/07/15}{Renamed babel.sty in
%    \file{babel.com}}
% \changes{french-4.2f}{1992/02/15}{Brought up-to-date with babel
%    3.2a}
% \changes{french-4.2f}{1992/02/15}{Removed crossreferencing
%    macros as they are dealt with in \file{babel.com}}
% \changes{french-4.5}{1994/02/27}{Update for LaTeX2e}
% \changes{french-4.5c}{1994/06/26}{Removed the use of \cs{filedate}
%    and moved the identification after the loading of
%    \file{babel.def}}
%
%  \section{The French language}
%
%    The file \file{\filename}\footnote{The file described in this
%    section has version number \fileversion\ and was last revised on
%    \filedate. This file was initially derived from the original
%    version of \file{german.sty}, which has some definitions for
%    French. Later the definitions from \file{french.sty} version 2
%    were added.} defines all the language definition macros for the
%    French language.
%
%    French typographic rules specify that a little white space should
%    be present before `double puctuation' characters. These
%    characters are \texttt{:}, \texttt{;}, \texttt{!} and
%    \texttt{?}. In order to get this whitespace automatigically the
%    category code of these characters is made |\active|. The user
%    should input these four characters preceeded with a space; the
%    space will then be replaced by a |\thinspace|.
%
% \StopEventually{}
%
%    As this file needs to be read only once, we check whether it was
%    read before. If it was, the |\captionsfrancais| is already
%    defined, so we can stop processing. If this command is undefined
%    we proceed with the various definitions and first show the
%    current version of this file.
%
% \changes{french-4.2a}{1991/07/15}{Added reset of catcode of @ before
%    \cs{endinput}.}
% \changes{french-4.2e}{1991/10/29}{Removed use of \cs{@ifundefined}}
%    \begin{macrocode}
%<*code>
\ifx\undefined\captionsfrancais
\else
  \selectlanguage{francais}
  \expandafter\endinput
\fi
%    \end{macrocode}
%
% \changes{french-4.2e}{1991/10/29}{Removed code to load
%    \file{latexhax.com}}
%
% \begin{macro}{\atcatcode}
%    This file, \file{francais.ldf}, may have been read while \TeX\ is
%    in the middle of processing a document, so we have to make sure
%    the category code of \texttt{@} is `letter' while this file is
%    being read. We save the category code of the @-sign in
%    |\atcatcode| and make it `letter'. Later the category code can be
%    restored to whatever it was before.
% \changes{french-4.1a}{1991/06/06}{Made test of catcode of @ more
%    robust}
% \changes{french-4.2a}{1991/07/15}{Modified handling of catcode of @
%    again.}
% \changes{french-4.2e}{1991/10/29}{Removed use of \cs{makeatletter}
%    and hence the need to load \file{latexhax.com}}
%    \begin{macrocode}
\chardef\atcatcode=\catcode`\@
\catcode`\@=11\relax
%    \end{macrocode}
% \end{macro}
%
%    Now we determine whether the the common macros from the file
%    \file{babel.def} need to be read. We can be in one of two
%    situations: either another language option has been read earlier
%    on, in which case that other option has already read
%    \file{babel.def}, or \texttt{francais} is the first language
%    option to be processed. In that case we need to read
%    \file{babel.def} right here before we continue.
% \changes{french-3.0}{1991/04/23}{New check before loading
%    \file{babel.com}}
% \changes{french-4.2f}{1992/02/16}{Added \cs{relax} after the
%    argument of \cs{input}}
%    \begin{macrocode}
\ifx\undefined\babel@core@loaded\input babel.def\relax\fi
%    \end{macrocode}
%
% \changes{french-4.1}{1991/05/29}{Add a check for existence
%    \cs{originalTeX}} Another check that has to be made, is if
%    another language definition file has been read already. In that
%    case its definitions have been activated. This might interfere
%    with definitions this file tries to make. Therefore we make sure
%    that we cancel any special definitions. This can be done by
%    checking the existence of the macro |\originalTeX|. If it exists
%    we simply execute it, otherwise it is |\let| to |\empty|.
% \changes{french-4.2a}{1991/07/15}{Added
%    \cs{let}\cs{originalTeX}\cs{relax} to test for existence}
% \changes{french-4.2f}{1992/02/16}{\cs{originalTeX} should be
%    expandable, \cs{let} it to \cs{empty}}
%    \begin{macrocode}
\ifx\undefined\originalTeX \let\originalTeX\empty \fi
\originalTeX
%    \end{macrocode}
%
%    When this file is read as an option, i.e. by the |\usepackage|
%    command, \texttt{francais} will be an `unknown' language in which
%    case we have to make it known.  So we check for the existence of
%    |\l@francais| to see whether we have to do something here.
%
% \changes{french-3.0}{1991/04/23}{Now use \cs{adddialect} if language
%    undefined}
% \changes{french-4.2e}{1991/10/29}{Removed use of \cs{@ifundefined}}
% \changes{french-4.2f}{1992/02/16}{Added a warning when no hyphenation
%    patterns were loaded.}
% \changes{french-4.5c}{1994/06/26}{Now use \cs{@nopatterns} to
%    produce the warning}
% \changes{french-4.6c}{1995/07/02}{Also allow the hyphenation
%    patterns to be loaded for `french'}
%    \begin{macrocode}
\ifx\l@francais\undefined
  \ifx\l@french\undefined
    \@nopatterns{Francais}
    \adddialect\l@francais0
    \let\l@french\l@francais
  \else
    \let\l@francais\l@french
  \fi
\fi
%    \end{macrocode}
%    The next step consists of defining commands to switch to the
%    English language. The reason for this is that a user might want
%    to switch back and forth between languages.
%
% \begin{macro}{\captionsfrancais}
%    The macro |\captionsfrancais| defines all strings used in the
%    four standard document classes provided with \LaTeX.
% \changes{french-4.1a}{1991/06/06}{Removed \cs{bsl global}
%    definitions}
% \changes{french-4.2f}{1992/02/16}{Added \cs{seename}, \cs{alsoname}
%    and \cs{prefacename}}
% \changes{french-4.4}{1993/07/11}{\cs{headpagename} should be
%    \cs{pagename}}
% \changes{french-4.6c}{1995/07/02}{Added \cs{proofname} for
%    AMS-\LaTeX}
%    \begin{macrocode}
\addto\captionsfrancais{%
  \def\prefacename{Pr\'eface}%
  \def\refname{R\'ef\'erences}%
  \def\abstractname{R\'esum\'e}%
  \def\bibname{Bibliographie}%
  \def\chaptername{Chapitre}%
  \def\appendixname{Annexe}%
  \def\contentsname{Table des mati\`eres}%
  \def\listfigurename{Liste des figures}%
  \def\listtablename{Liste des tableaux}%
  \def\indexname{Index}%
  \def\figurename{Figure}%
  \def\tablename{Tableau}%
  \def\partname{Partie}%
  \def\enclname{P.~J.}%
  \def\ccname{Copie \`a}%
  \def\headtoname{A}
  \def\pagename{Page}%
  \def\seename{voir}%
  \def\alsoname{voir aussi}%
  \def\proofname{Proof}%  <-- needs translation!
  }
%    \end{macrocode}
% \end{macro}
%
% \begin{macro}{\datefrancais}
%    The macro |\datefrancais| redefines the command |\today| to
%    produce French dates.
% \changes{french-4.1a}{1991/06/06}{Removed \cs{global} definitions}
%    \begin{macrocode}
\def\datefrancais{%
\def\today{\ifnum\day=1\relax 1\/$^{\rm er}$\else
  \number\day\fi \space\ifcase\month\or
  janvier\or f\'evrier\or mars\or avril\or mai\or juin\or
  juillet\or ao\^ut\or septembre\or octobre\or novembre\or
  d\'ecembre\fi
  \space\number\year}}
%    \end{macrocode}
% \end{macro}
%
% \begin{macro}{\extrasfrancais}
% \changes{french-4.3}{1992/02/20}{Completely rewrote macro}
% \begin{macro}{\noextrasfrancais}
%    The macro |\extrasfrancais| will perform all the extra
%    definitions needed for the French language. The macro
%    |\noextrasfrancais| is used to cancel the actions of
%    |\extrasfrancais|.
%
%    The category code of the characters \texttt{:}, \texttt{;},
%    \texttt{!} and \texttt{?} is made |\active| to insert a little
%    white space.
% \changes{french-4.6a}{1995/03/07}{Use the new mechanism for dealing
%    with active chars}
%    \begin{macrocode}
\initiate@active@char{:}
\initiate@active@char{;}
\initiate@active@char{!}
\initiate@active@char{?}
%    \end{macrocode}
%    We specify that the french group of shorthands should be used.
%    \begin{macrocode}
\addto\extrasfrancais{\languageshorthands{french}}
%    \end{macrocode}
%    These characters are `turned on' once, later their definition may
%    vary. 
%    \begin{macrocode}
\addto\extrasfrancais{%
  \bbl@activate{:}\bbl@activate{;}%
  \bbl@activate{!}\bbl@activate{?}}
%\addto\noextrasfrancais{%
%  \bbl@deactivate{:}\bbl@deactivate{;}%
%  \bbl@deactivate{!}\bbl@deactivate{?}}
%    \end{macrocode}
%
%    The last thing |\extrasfrancais| needs to do is to make sure that
%    |\frenchspacing| is in effect.  If this is not the case the
%    execution of |\noextrasfrancais| will switch it off again.
% \changes{french-4.3a}{1992/07/02}{Removed spurious \cs{endgroup} and
%    \texttt{\}}}
% \changes{french-4.6a}{1995/03/14}{now use \cs{bbl@frenchspacing} and
%    \cs{bbl@nonfrenchspacing}}
%    \begin{macrocode}
\addto\extrasfrancais{\bbl@frenchspacing}
\addto\noextrasfrancais{\bbl@nonfrenchspacing}
%    \end{macrocode}
% \end{macro}
% \end{macro}
%
% \begin{macro}{\french@sh@;@}
%    We have to reduce the amount of white space before \texttt{;},
%    \texttt{:} and \texttt{!} when the user types a space in front of
%    these characters. This should only happen outside mathmode, hence
%    the test with |\ifmmode|.
%
% \changes{french-4.3b}{1993/04/04}{Replaced \cs{,} with \cs{thinspace}
%    to make it work with plain TeX.}
% \changes{french-4.6a}{1995/02/19}{Use new \cs{DefineActiveNoArg}}
% \changes{french-4.6a}{1995/03/05}{Use the more general mechanism of
%    \cs{declare@shorthand}}
%    \begin{macrocode}
\declare@shorthand{french}{;}{%
  \ifmmode
    \string;\space
  \else\relax
%    \end{macrocode}
%    In horizontal mode we check for the presence of a `space' and
%    replace it by a |\thinspace|.
%    \begin{macrocode}
    \ifhmode
      \ifdim\lastskip>\z@
        \unskip\penalty\@M\thinspace
      \fi
    \fi
%    \end{macrocode}
%    Now we can insert a |;| character.
%    \begin{macrocode}
    \string;\space
  \fi}
%    \end{macrocode}
% \end{macro}
%
% \begin{macro}{\french@sh@:@}
% \changes{french-4.3b}{1993/04/04}{Replaced \cs{,} with \cs{thinspace}
%    to make it work with plain TeX.}
% \changes{french-4.6a}{1995/02/19}{Use new \cs{DefineActiveNoArg}}
% \begin{macro}{\french@sh@!@}
% \changes{french-4.3b}{1993/04/04}{Replaced \cs{,} with \cs{thinspace}
%    to make it work with plain TeX.}
% \changes{french-4.6a}{1995/02/19}{Use new \cs{DefineActiveNoArg}}
% \changes{french-4.6a}{1995/03/05}{Use the more general mechanism of
%    \cs{declare@shorthand}}
%
%    Because these definitions are very similar only one is displayed
%    in a way that the definition can be easily checked.
%
%    \begin{macrocode}
\declare@shorthand{french}{:}{%
  \ifmmode\string:\space
  \else\relax
    \ifhmode
      \ifdim\lastskip>\z@\unskip\penalty\@M\thinspace\fi
    \fi
    \string:\space
  \fi}
\declare@shorthand{french}{!}{%
  \ifmmode\string!\space
  \else\relax
    \ifhmode
      \ifdim\lastskip>\z@\unskip\penalty\@M\thinspace\fi
    \fi
    \string!\space
  \fi}
%    \end{macrocode}
% \end{macro}
% \end{macro}
%
% \begin{macro}{\french@sh@?@}
%    For the question mark something different has to be done. In this
%    case the amount of white space that replaces the space character
%    depends on the dimensions of the font.
%
% \changes{french-4.3b}{1993/04/04}{Replaced \cs{,} with \cs{thinspace}
%    to make it work with plain TeX.}
% \changes{french-4.6a}{1995/02/19}{Use new \cs{DefineActiveNoArg}}
% \changes{french-4.6a}{1995/03/05}{Use the more general mechanism of
%    \cs{declare@shorthand}}
%    \begin{macrocode}
\declare@shorthand{french}{?}{%
  \ifmmode\string?\space
  \else\relax
    \ifhmode
      \ifdim\lastskip>\z@
        \unskip
        \kern\fontdimen2\font
        \kern-1.4\fontdimen3\font
      \fi
    \fi
    \string?\space
  \fi}
%    \end{macrocode}
% \end{macro}
%
%  \begin{macro}{\system@sh@:@}
%  \begin{macro}{\system@sh@!@}
%  \begin{macro}{\system@sh@?@}
%  \begin{macro}{\system@sh@;@}
% \changes{french-4.6b}{1995/06/03}{Added system level shorthands}
%    When the active characters appear in an environment where their
%    french behaviour is not wanted they should give an `expected'
%    result, ie not gobble up the space that follows them. Therefore
%    we define shorthands at system level as well.
%    \begin{macrocode}
\declare@shorthand{system}{:}{\string:\space}
\declare@shorthand{system}{!}{\string!\space}
\declare@shorthand{system}{?}{\string?\space}
\declare@shorthand{system}{;}{\string;\space}
%    \end{macrocode}
%  \end{macro}
%  \end{macro}
%  \end{macro}
%  \end{macro}
%
%    All that is left to do now is provide the french user with some
%    extra utilities.
%
%    Some definitions for special characters.
%    \begin{macrocode}
\DeclareTextSymbol{\at}{OT1}{64}
\DeclareTextSymbol{\at}{T1}{64}
\DeclareTextSymbolDefault{\at}{OT1}
\DeclareTextSymbol{\boi}{OT1}{92}
\DeclareTextSymbol{\boi}{T1}{16}
\DeclareTextSymbolDefault{\boi}{OT1}
\DeclareTextSymbol{\circonflexe}{OT1}{94}
\DeclareTextSymbol{\circonflexe}{T1}{2}
\DeclareTextSymbolDefault{\circonflexe}{OT1}
\DeclareTextSymbol{\tild}{OT1}{126}
\DeclareTextSymbol{\tild}{T1}{3}
\DeclareTextSymbolDefault{\tild}{OT1}
\DeclareTextSymbol{\degre}{OT1}{23}
\DeclareTextSymbol{\degre}{T1}{6}
\DeclareTextSymbolDefault{\degre}{OT1}
%    \end{macrocode}
%
%    The following macros are used in the redefinition of |\^| and
%    |\"| to handle the letter i.
% \changes{francais-4.6c}{1995/07/07}{Postpone the declaration of the
%    TextCompositeCommands untill \cs{AtBeginDocument}}
%
%    \begin{macrocode}
\AtBeginDocument{%
  \DeclareTextCompositeCommand{\^}{OT1}{i}{\^\i}
  \DeclareTextCompositeCommand{\"}{OT1}{i}{\"\i}}
%    \end{macrocode}
%
%    A macro for typesetting things like 1\raise1ex\hbox{\small er} as
%    proposed by Raymon Seroul.
%    \begin{macrocode}
\def\up#1{\raise 1ex\hbox{\small#1}}
%    \end{macrocode}
%
%    Definitions as provided by Nicolas Brouard for typing |\No3| to
%    get 3\kern-.25em\lower.2ex\hbox{\char'27} and for typing
%    |4\ieme| to get 4$^{\rm e }$\kern+.17em.
%    \begin{macrocode}
\def\No{\kern-.25em\lower.2ex\hbox{\degre}}
\def\ieme{$^{\rm e }$\kern+.17em}
%    \end{macrocode}
%
%    And some more macros for numbering.
%    First two support macros.
%    \begin{macrocode}
\def\FrenchEnumerate#1{$#1^{\rm o}$\kern+.29em}
\def\FrenchPopularEnumerate#1{#1\No\kern-.25em)\kern+.3em}
%    \end{macrocode}
%
%    Typing |\primo| should result in `$1^{\rm o}$\kern+.29em',
%    \begin{macrocode}
\def\primo{\FrenchEnumerate1}
\def\secundo{\FrenchEnumerate2}
\def\tertio{\FrenchEnumerate3}
\def\quatro{\FrenchEnumerate4}
%    \end{macrocode}
%    while typing |\fprimo)| gives
%    `1\kern-.25em\lower.2ex\hbox{\char'27}\kern-.25em)\kern+.3em'.
%    \begin{macrocode}
\def\fprimo){\FrenchPopularEnumerate1}
\def\fsecundo){\FrenchPopularEnumerate2}
\def\ftertio){\FrenchPopularEnumerate3}
\def\fquatro){\FrenchPopularEnumerate4}
%    \end{macrocode}
%
%    It is possible that a site might need to add some extra code to
%    the babel macros. To enable this we load a local configuration
%    file, \file{francais.cfg} if it is found on \TeX' search path.
% \changes{french-4.6c}{1995/07/02}{Added loading of configuration
%    file}
%    \begin{macrocode}
\loadlocalcfg{francais}
%    \end{macrocode}
%
%    Our last action is to make a note that the commands we have just
%    defined, will be executed by calling the macro |\selectlanguage|
%    at the beginning of the document.
%    \begin{macrocode}
\main@language{francais}
%    \end{macrocode}
%    Finally, the category code of \texttt{@} is reset to its original
%    value. The macrospace used by |\atcatcode| is freed.
% \changes{french-4.2a}{1991/07/15}{Modified handling of catcode of
%    @-sign.}
%    \begin{macrocode}
\catcode`\@=\atcatcode \let\atcatcode\relax
%</code>
%    \end{macrocode}
%
% \Finale
%%
%% \CharacterTable
%%  {Upper-case    \A\B\C\D\E\F\G\H\I\J\K\L\M\N\O\P\Q\R\S\T\U\V\W\X\Y\Z
%%   Lower-case    \a\b\c\d\e\f\g\h\i\j\k\l\m\n\o\p\q\r\s\t\u\v\w\x\y\z
%%   Digits        \0\1\2\3\4\5\6\7\8\9
%%   Exclamation   \!     Double quote  \"     Hash (number) \#
%%   Dollar        \$     Percent       \%     Ampersand     \&
%%   Acute accent  \'     Left paren    \(     Right paren   \)
%%   Asterisk      \*     Plus          \+     Comma         \,
%%   Minus         \-     Point         \.     Solidus       \/
%%   Colon         \:     Semicolon     \;     Less than     \<
%%   Equals        \=     Greater than  \>     Question mark \?
%%   Commercial at \@     Left bracket  \[     Backslash     \\
%%   Right bracket \]     Circumflex    \^     Underscore    \_
%%   Grave accent  \`     Left brace    \{     Vertical bar  \|
%%   Right brace   \}     Tilde         \~}
%%
\endinput
}%
  \let\captionsfrench\captionsfrancais
  \let\datefrench\datefrancais
  \let\extrasfrench\extrasfrancais
  \let\noextrasfrench\noextrasfrancais
  \let\frenchhyphenmins\francaishyphenmins
  }
\DeclareOption{galician}{% \iffalse meta-comment
%
% Copyright 1989-1995 Johannes L. Braams and any individual authors
% listed elsewhere in this file.  All rights reserved.
% 
% For further copyright information any other copyright notices in this
% file.
% 
% This file is part of the Babel system release 3.5.
% --------------------------------------------------
%   This system is distributed in the hope that it will be useful,
%   but WITHOUT ANY WARRANTY; without even the implied warranty of
%   MERCHANTABILITY or FITNESS FOR A PARTICULAR PURPOSE.
% 
%   For error reports concerning UNCHANGED versions of this file no more
%   than one year old, see bugs.txt.
% 
%   Please do not request updates from me directly.  Primary
%   distribution is through the CTAN archives.
% 
% 
% IMPORTANT COPYRIGHT NOTICE:
% 
% You are NOT ALLOWED to distribute this file alone.
% 
% You are allowed to distribute this file under the condition that it is
% distributed together with all the files listed in manifest.txt.
% 
% If you receive only some of these files from someone, complain!
% 
% Permission is granted to copy this file to another file with a clearly
% different name and to customize the declarations in that copy to serve
% the needs of your installation, provided that you comply with
% the conditions in the file legal.txt from the LaTeX2e distribution.
% 
% However, NO PERMISSION is granted to produce or to distribute a
% modified version of this file under its original name.
%  
% You are NOT ALLOWED to change this file.
% 
% 
% \fi
% \CheckSum{282}
% \iffalse
%    Tell the \LaTeX\ system who we are and write an entry on the
%    transcript.
%<*dtx>
\ProvidesFile{galician.dtx}
%</dtx>
%<code>\ProvidesFile{galician.ldf}
        [1995/07/08 v1.2c Galician support from the babel system]
%
% Babel package for LaTeX version 2e
% Copyright (C) 1989 - 1995
%           by Johannes Braams, TeXniek
%
% Galician Language Definition File
% Copyright (C) 1989 - 1995
%           by Manuel Carriba <mcarriba@eunetcom.net>
%              Johannes Braams, TeXniek
%
% Please report errors to: J.L. Braams
%                          JLBraams@cistron.nl
%
%    This file is part of the babel system, it provides the source
%    code for the Galician language definition file.
%
%    The 'galician' style was originally adopted from the 'spanish'
%    style.
%
%    All the macrocodes have been translated from the spanish language
%    into the galician language, using the spanish-galician
%    dictionary:
%
%           X.L. Franco Grande
%           Diccionario Galego-Castelan e Vocabulario Castelan-Galego
%           Editorial Galaixa, Vigo 1968
%
%    The hyphenation patterns for the galician language should be the
%    same as the spanish language. I've inspected some galician essays
%    and nothing strange seems to point out that both languages might
%    use different hyphenation patterns. This still has to been
%    proved. I will check it anyway, and hope to report more as soon
%    as possible.
%
%    A small note to the months in the galician language:
%
%    'outono' will be used sometimes instead of 'outubro', but this
%    word will be used more to assign the season.
%
%    'nadal' will be used sometimes instead of 'decembro', but this
%    word will be used more for the event on Christmas.
%
%    Manuel Carriba (mcarriba@eunetcom.net)
%
%    The file spanish.sty was written by Julio Sanchez,
%    (jsanchez@gmv.es)
%<*filedriver>
\documentclass{ltxdoc}
\newcommand*\TeXhax{\TeX hax}
\newcommand*\babel{\textsf{babel}}
\newcommand*\langvar{$\langle \it lang \rangle$}
\newcommand*\note[1]{}
\newcommand*\Lopt[1]{\textsf{#1}}
\newcommand*\file[1]{\texttt{#1}}
\begin{document}
 \DocInput{galician.dtx}
\end{document}
%</filedriver>
%\fi
% \GetFileInfo{galician.dtx}
%
% \changes{galician-1.1}{1994/02/27}{Update for \LaTeXe}
% \changes{galician-1.1c}{1994/06/26}{Removed the use of \cs{filedate}
%    and moved identification after the loading of \file{babel.def}}
%
%  \section{The Galician language}
%
%    The file \file{\filename}\footnote{The file described in this
%    section has version number \fileversion\ and was last revised on
%    \filedate.}  defines all the language definition macros for the
%    Galician language.
%
%    For this language the characters |'| |~| and |"| are made
%    active. In table~\ref{tab:galician-quote} an overview is given of
%    their purpose.
%    \begin{table}[htb]
%     \centering
%     \begin{tabular}{lp{8cm}}
%      \verb="|= & disable ligature at this position.\\
%      |"-| & an explicit hyphen sign, allowing hyphenation
%             in the rest of the word.\\
%      |\-| & like the old |\-|, but allowing hyphenation
%             in the rest of the word. \\
%      |'a| & an accent that allows hyphenation. Valid for all
%             vowels uppercase and lowercase.\\
%      |'n| & a n with a tilde. This is included to
%             improve compatibility with FTC. Works for uppercase too.\\
%      |"u| & a u with dieresis allowing hyphenation.\\
%      |"a| & feminine ordinal as in
%             1{\raise1ex\hbox{\underbar{\scriptsize a}}}.\\
%      |"o| & masculine ordinal as in
%             1{\raise1ex\hbox{\underbar{\scriptsize o}}}.\\
%      |~n| & a n with tilde. Works for uppercase too.
%     \end{tabular}
%     \caption{The extra definitions made by \texttt{galician.ldf}}
%     \label{tab:galician-quote}
%    \end{table}
%    These active accents character behave according to their original
%    definitions if not followed by one of the characters indicated in
%    that table.
%
% \StopEventually{}
%
%    As this file needs to be read only once, we check whether it was
%    read before. If it was, the command |\captionsgalician| is already
%    defined, so we can stop processing. If this command is undefined
%    we proceed with the various definitions and first show the
%    current version of this file.
%
%    \begin{macrocode}
%<*code>
\ifx\undefined\captionsgalician
\else
  \selectlanguage{galician}
  \expandafter\endinput
\fi
%    \end{macrocode}
%
% \begin{macro}{\atcatcode}
%    This file, \file{galician.ldf}, may have been read while \TeX\ is
%    in the middle of processing a document, so we have to make sure
%    the category code of \texttt{@} is `letter' while this file is
%    being read.  We save the category code of the @-sign in
%    |\atcatcode| and make it `letter'. Later the category code can be
%    restored to whatever it was before.
%    \begin{macrocode}
\chardef\atcatcode=\catcode`\@
\catcode`\@=11\relax
%    \end{macrocode}
% \end{macro}
%
%    Now we determine whether the the common macros from the file
%    \file{babel.def} need to be read. We can be in one of two
%    situations: either another language option has been read earlier
%    on, in which case that other option has already read
%    \file{babel.def}, or \texttt{galician} is the first language option
%    to be processed. In that case we need to read \file{babel.def}
%    right here before we continue.
%
%    \begin{macrocode}
\ifx\undefined\babel@core@loaded\input babel.def\relax\fi
%    \end{macrocode}
%
%    Another check that has to be made, is if another language
%    definition file has been read already. In that case its definitions
%    have been activated. This might interfere with definitions this
%    file tries to make. Therefore we make sure that we cancel any
%    special definitions. This can be done by checking the existence
%    of the macro |\originalTeX|. If it exists we simply execute it.
%    \begin{macrocode}
\ifx\undefined\originalTeX
  \let\originalTeX\empty
\fi
\originalTeX
%    \end{macrocode}
%
%    When this file is read as an option, i.e. by the |\usepackage|
%    command, \texttt{galician} could be an `unknown' language in which
%    case we have to make it known.  So we check for the existence of
%    |\l@galician| to see whether we have to do something here.
%
% \changes{galician-1.1c}{1994/06/26}{Now use \cs{@nopatterns} to
%    produce the warning}
%    \begin{macrocode}
\ifx\undefined\l@galician
  \@nopatterns{Galician}
  \adddialect\l@galician0\fi
%    \end{macrocode}
%
%    The next step consists of defining commands to switch to (and
%    from) the Galician language.
%
% \begin{macro}{\captionsgalician}
%    The macro |\captionsgalician| defines all strings used
%    in the four standard documentclasses provided with \LaTeX.
% \changes{galician-1.1d}{1994/11/09}{Added a few missing
%    translations}
% \changes{galician-1.2b}{1995/07/02}{Added \cs{proofname} for
%    AMS-\LaTeX}
%    \begin{macrocode}
\addto\captionsgalician{%
  \def\prefacename{Prefacio}%
  \def\refname{Referencias}%
  \def\abstractname{Resumo}%
  \def\bibname{Bibliograf\'{\i}a}%
  \def\chaptername{Cap\'{\i}tulo}%
  \def\appendixname{Ap\'endice}%
  \def\contentsname{\'Indice Xeral}%
  \def\listfigurename{\'Indice de Figuras}%
  \def\listtablename{\'Indice de T\'aboas}%
  \def\indexname{\'Indice de Materias}%
  \def\figurename{Figura}%
  \def\tablename{T\'aboa}%
  \def\partname{Parte}%
  \def\enclname{Adxunto}%
  \def\ccname{Copia a}%
  \def\headtoname{A}%
  \def\pagename{P\'axina}%
  \def\seename{v\'exase}%
  \def\alsoname{v\'exase tam\'en}%
  \def\proofname{Proof}%  <-- Needs Translation!
}
%    \end{macrocode}
% \end{macro}
%
% \begin{macro}{\dategalician}
%    The macro |\dategalician| redefines the command |\today| to
%    produce Galician dates.
% \changes{galician1.1d}{1994/11/09}{Corrected the name of the month
%    marzo from marzal}
%    \begin{macrocode}
\def\dategalician{%
  \def\today{\number\day~de\space\ifcase\month\or
    xaneiro\or febreiro\or marzo\or abril\or maio\or xu\~no\or
    xullo\or agosto\or setembro\or outubro\or novembro\or decembro\fi
    \space de~\number\year}}
%    \end{macrocode}
% \end{macro}
%
% \begin{macro}{\extrasgalician}
% \changes{galician-1.2a}{1995/03/14}{Handling of active characters
%    completely rewritten}
% \begin{macro}{\noextrasgalician}
%
%    The macro |\extrasgalician| will perform all the extra
%    definitions needed for the Galician language. The macro
%    |\noextrasgalician| is used to cancel the actions of
%    |\extrasgalician|. 
%
%    For Galician, some characters are made active or are
%    redefined. In particular, the \texttt{"} character and the |~|
%    character receive new meanings this can also happen for the
%    \texttt{'} character when the option \Lopt{activeacute} is
%    specified.
%
% \changes{galician-1.2c}{1995/07/08}{Make active accent optional}
%    \begin{macrocode}
\addto\extrasgalician{\languageshorthands{galician}}
\initiate@active@char{"}
\initiate@active@char{~}
\addto\extrasgalician{%
  \bbl@activate{"}\bbl@activate{~}}
\@ifpackagewith{babel}{activeacute}{%
  \initiate@active@char{'}
  \addto\extrasgalician{\bbl@activate{'}}}{}
%\addto\noextrasgalician{%
%  \bbl@deactivate{"}\bbl@deactivate{~}\bbl@deactivate{'}}
%    \end{macrocode}
%
% \changes{galician-1.2a}{1995/03/14}{All the code for handling active
%    characters is now moved to \file{babel.def}}
%
%    Apart from the active characters some other macros get a new
%    definition. Therefore we store the current one to be able to
%    restore them later.
%
%    \begin{macrocode}
\addto\extrasgalician{%
  \babel@save\"\babel@save\~
  \def\"{\protect\@umlaut}%
  \def\~{\protect\@tilde}}
\@ifpackagewith{babel}{activeacute}{%
  \babel@save\'
  \addto\extrasgalician{\def\'{\protect\@acute}}
  }{}
%    \end{macrocode}
% \end{macro}
% \end{macro}
%
%    All the code above is necessary because we need a few extra
%    active characters. These characters are then used as indicated in
%    table~\ref{tab:galician-quote}.
%
%    This option includes some support for working with extended,
%    8-bit fonts, if available. This assumes that the user has some
%    macros predefined. For instance, if the user has a |\@ac@a| macro
%    defined, the sequence |\'a| or |'a| will both expand to whatever
%    |\@ac@a| is defined to expand, presumably \texttt{\'a}.  The
%    names of these macros are the same as those in Ferguson's
%    ML-\TeX{} compatibility package on purpose. Using this method,
%    and provided that adequate hyphenation patterns exist, it is
%    possible to get better hyphenation for Galician than before. If
%    the user has a terminal able to produce these codes directly, it
%    is possible to do so.  If the need arises to send the document to
%    someone who does not have such support, it is possible to
%    mechanically translate the document so that the receiver can make
%    use of it.
%
%    To be able to define the function of the new accents, we first
%    define a couple of `support' macros.
%
%  \begin{macro}{\dieresis}
%  \begin{macro}{\textacute}
% \changes{galician-1.1c}{1994/06/26}{Renamed from \cs{acute} as that
%    is a \cs{mathaccent}}
%  \begin{macro}{\texttilde}
% \changes{galician-1.1c}{1994/06/26}{Renamed from \cs{tilde} as that
%    is a \cs{mathaccent}}
%
%    The original definition of |\"| is stored as |\dieresis|, because
%    the definition of |\"| might not be the default plain \TeX\
%    one. If the user uses \textsc{PostScript} fonts with the Adobe
%    font encoding the \texttt{"} character is not in the same
%    position as in Knuth's font encoding. In this case |\"| will not
%    be defined as |\accent"7F #1|, but as |\accent'310 #1|. Something
%    similar happens when using fonts that follow the Cork
%    encoding. For this reason we save the definition of |\"| and use
%    that in the definition of other macros. We do likewise for |\'|
%    and |\~|.
%    \begin{macrocode}
\let\dieresis\"
\let\texttilde\~
\@ifpackagewith{babel}{activeacute}{\let\textacute\'}{}
%    \end{macrocode}
%  \end{macro}
%  \end{macro}
%  \end{macro}
%
%  \begin{macro}{\@umlaut}
%  \begin{macro}{\@acute}
%  \begin{macro}{\@tilde}
%    If the user setup has extended fonts, the Ferguson macros are
%    required to be defined. We check for their existance and, if
%    defined, expand to whatever they are defined to. For instance,
%    |\'a| would check for the existance of a |\@ac@a| macro. It is
%    assumed to expand to the code of the accented letter.  If it is
%    not defined, we assume that no extended codes are available and
%    expand to the original definition but enabling hyphenation beyond
%    the accent. This is as best as we can do. It is better if you
%    have extended fonts or ML-\TeX{} because the hyphenation
%    algorithm can work on the whole word. The following macros are
%    directly derived from ML-\TeX{}.\footnote{A problem is perceived
%    here with these macros when used in a multilingual environment
%    where extended hyphenation patterns are available for some but
%    not all languages. Assume that no extended patterns exist at some
%    site for French and that \file{french.sty} would adopt this
%    scheme too. In that case, \texttt{'e} in French would
%    produce the combined accented letter, but hyphenation around it
%    would be suppressed. Both language options would need an
%    independent method to know whether they have extended patterns
%    available. The precise impact of this problem and the possible
%    solutions are under study.}
%
%    \begin{macrocode}
\def\@umlaut#1{\allowhyphens\dieresis{#1}\allowhyphens}
\def\@tilde#1{\allowhyphens\texttilde{#1}\allowhyphens}
\@ifpackagewith{babel}{activeacute}{%
  \def\@acute#1{\allowhyphens\textacute{#1}\allowhyphens}}{}
%    \end{macrocode}
%  \end{macro}
%  \end{macro}
%  \end{macro}
%
%    Now we can define our shorthands: the umlauts,
%    \begin{macrocode}
\declare@shorthand{galician}{"-}{\allowhyphens-\allowhyphens}
\declare@shorthand{galician}{"|}{\discretionary{-}{}{\kern.03em}}
\declare@shorthand{galician}{"u}{\@umlaut{u}}
\declare@shorthand{galician}{"U}{\@umlaut{U}}
%    \end{macrocode}
%     ordinals\footnote{The code for the ordinals was taken from the
%    answer provided by Raymond Chen
%    {\texttt(raymond@math.berkeley.edu}) to a question by Joseph Gil
%    (\texttt{yogi@cs.ubc.ca}) in \texttt{comp.text.tex}.},
%    \begin{macrocode}
\declare@shorthand{galician}{"o}{%
  \raise1ex\hbox{\underbar{\scriptsize o}}}
\declare@shorthand{galician}{"a}{%
  \raise1ex\hbox{\underbar{\scriptsize a}}}
%    \end{macrocode}
%     acute accents,
% \changes{galician-1.2b}{1995/07/03}{Changed mathmode definition of
%    acute shorthands to expand to a single prime followed by the next
%    character in the input}
%    \begin{macrocode}
\@ifpackagewith{babel}{activeacute}{%
  \declare@shorthand{galician}{'a}{\textormath{\@acute a}{^{\prime} a}}
  \declare@shorthand{galician}{'e}{\textormath{\@acute e}{^{\prime} e}}
  \declare@shorthand{galician}{'i}{\textormath{\@acute \i{}}{^{\prime} i}}
  \declare@shorthand{galician}{'o}{\textormath{\@acute o}{^{\prime} o}}
  \declare@shorthand{galician}{'u}{\textormath{\@acute u}{^{\prime} u}}
  \declare@shorthand{galician}{'A}{\textormath{\@acute A}{^{\prime} A}}
  \declare@shorthand{galician}{'E}{\textormath{\@acute E}{^{\prime} E}}
  \declare@shorthand{galician}{'I}{\textormath{\@acute I}{^{\prime} I}}
  \declare@shorthand{galician}{'O}{\textormath{\@acute O}{^{\prime} O}}
  \declare@shorthand{galician}{'U}{\textormath{\@acute U}{^{\prime} U}}
%    \end{macrocode}
%         tildes,
%    \begin{macrocode}
  \declare@shorthand{galician}{'n}{\textormath{\~n}{^{\prime} n}}
  \declare@shorthand{galician}{'N}{\textormath{\~N}{^{\prime} N}}
  }{}
\declare@shorthand{galician}{~n}{\textormath{\~n}{\@tilde n}}
\declare@shorthand{galician}{~N}{\textormath{\~N}{\@tilde N}}
%    \end{macrocode}
%
%  \begin{macro}{\-}
%
%    All that is left now is the redefinition of |\-|. The new version
%    of |\-| should indicate an extra hyphenation position, while
%    allowing other hyphenation positions to be generated
%    automatically. The standard behaviour of \TeX\ in this respect is
%    unfortunate for Galician but not as much as for Dutch or German,
%    where long compound words are quite normal and all one needs is a
%    means to indicate an extra hyphenation position on top of the
%    ones that \TeX\ can generate from the hyphenation
%    patterns. However, the average length of words in Galician makes
%    this desirable and so it is kept here.
%
%    \begin{macrocode}
\addto\extrasgalician{%
  \babel@save{\-}%
  \def\-{\allowhyphens\discretionary{-}{}{}\allowhyphens}}
%    \end{macrocode}
%  \end{macro}
%
%    It is possible that a site might need to add some extra code to
%    the babel macros. To enable this we load a local configuration
%    file, \file{galician.cfg} if it is found on \TeX' search path.
% \changes{galician-1.2b}{1995/07/02}{Added loading of configuration
%    file}
%    \begin{macrocode}
\loadlocalcfg{galician}
%    \end{macrocode}
%
%    Our last action is to make a note that the commands we have just
%    defined, will be executed by calling the macro |\selectlanguage|
%    at the beginning of the document.
%    \begin{macrocode}
\main@language{galician}
%    \end{macrocode}
%
%    Finally, the category code of \texttt{@} is reset to its original
%    value. The macrospace used by |\atcatcode| is freed.
%
%    \begin{macrocode}
\catcode`\@\atcatcode \let\atcatcode\relax
%</code>
%    \end{macrocode}
%
% \Finale
%
%% \CharacterTable
%%  {Upper-case    \A\B\C\D\E\F\G\H\I\J\K\L\M\N\O\P\Q\R\S\T\U\V\W\X\Y\Z
%%   Lower-case    \a\b\c\d\e\f\g\h\i\j\k\l\m\n\o\p\q\r\s\t\u\v\w\x\y\z
%%   Digits        \0\1\2\3\4\5\6\7\8\9
%%   Exclamation   \!     Double quote  \"     Hash (number) \#
%%   Dollar        \$     Percent       \%     Ampersand     \&
%%   Acute accent  \'     Left paren    \(     Right paren   \)
%%   Asterisk      \*     Plus          \+     Comma         \,
%%   Minus         \-     Point         \.     Solidus       \/
%%   Colon         \:     Semicolon     \;     Less than     \<
%%   Equals        \=     Greater than  \>     Question mark \?
%%   Commercial at \@     Left bracket  \[     Backslash     \\
%%   Right bracket \]     Circumflex    \^     Underscore    \_
%%   Grave accent  \`     Left brace    \{     Vertical bar  \|
%%   Right brace   \}     Tilde         \~}
%%
\endinput
}
\DeclareOption{german}{% \iffalse meta-comment
%
% Copyright 1989-1995 Johannes L. Braams and any individual authors
% listed elsewhere in this file.  All rights reserved.
% 
% For further copyright information any other copyright notices in this
% file.
% 
% This file is part of the Babel system release 3.5.
% --------------------------------------------------
%   This system is distributed in the hope that it will be useful,
%   but WITHOUT ANY WARRANTY; without even the implied warranty of
%   MERCHANTABILITY or FITNESS FOR A PARTICULAR PURPOSE.
% 
%   For error reports concerning UNCHANGED versions of this file no more
%   than one year old, see bugs.txt.
% 
%   Please do not request updates from me directly.  Primary
%   distribution is through the CTAN archives.
% 
% 
% IMPORTANT COPYRIGHT NOTICE:
% 
% You are NOT ALLOWED to distribute this file alone.
% 
% You are allowed to distribute this file under the condition that it is
% distributed together with all the files listed in manifest.txt.
% 
% If you receive only some of these files from someone, complain!
% 
% Permission is granted to copy this file to another file with a clearly
% different name and to customize the declarations in that copy to serve
% the needs of your installation, provided that you comply with
% the conditions in the file legal.txt from the LaTeX2e distribution.
% 
% However, NO PERMISSION is granted to produce or to distribute a
% modified version of this file under its original name.
%  
% You are NOT ALLOWED to change this file.
% 
% 
% \fi
% \CheckSum{341}
%
% \iffalse
%    Tell the \LaTeX\ system who we are and write an entry on the
%    transcript.
%<*dtx>
\ProvidesFile{germanb.dtx}
%</dtx>
%<code>\ProvidesFile{germanb.ldf}
        [1995/07/04 v2.6b German support from the babel system]
%
% Babel package for LaTeX version 2e
% Copyright (C) 1989 - 1995
%           by Johannes Braams, TeXniek
%
% Germanb Language Definition File
% Copyright (C) 1989 - 1995
%           by Bernd Raichle <raichle@azu.Informatik.Uni-Stuttgart.de>
%              Johannes Braams, TeXniek
% This file is based on german.tex version 2.5b,
%                       by Bernd Raichle, Hubert Partl et.al.
%
% Please report errors to: J.L. Braams
%                          JSLBraams@cistron.nl
%
%<*filedriver>
\documentclass{ltxdoc}
\font\manual=logo10 % font used for the METAFONT logo, etc.
\newcommand*\MF{{\manual META}\-{\manual FONT}}
\newcommand*\TeXhax{\TeX hax}
\newcommand*\babel{\textsf{babel}}
\newcommand*\langvar{$\langle \it lang \rangle$}
\newcommand*\note[1]{}
\newcommand*\Lopt[1]{\textsf{#1}}
\newcommand*\file[1]{\texttt{#1}}
\begin{document}
 \DocInput{germanb.dtx}
\end{document}
%</filedriver>
%\fi
% \GetFileInfo{germanb.dtx}
%
% \changes{germanb-1.0a}{1990/05/14}{Incorporated Nico's comments}
% \changes{germanb-1.0b}{1990/05/22}{fixed typo in definition for
%    austrian language found by Werenfried Spit
%    \texttt{nspit@fys.ruu.nl}}
% \changes{germanb-1.0c}{1990/07/16}{Fixed some typos}
% \changes{germanb-1.1}{1990/07/30}{When using PostScript fonts with
%    the Adobe fontencoding, the dieresis-accent is located elsewhere,
%    modified code}
% \changes{germanb-1.1a}{1990/08/27}{Modified the documentation
%    somewhat}
% \changes{germanb-2.0}{1991/04/23}{Modified for babel 3.0}
% \changes{germanb-2.0a}{1991/05/25}{Removed some problems in change
%    log}
% \changes{germanb-2.1}{1991/05/29}{Removed bug found by van der Meer}
% \changes{germanb-2.2}{1991/06/11}{Removed global assignments,
%    brought uptodate with \file{german.tex} v2.3d}
% \changes{germanb-2.2a}{1991/07/15}{Renamed \file{babel.sty} in
%    \file{babel.com}}
% \changes{germanb-2.3}{1991/11/05}{Rewritten parts of the code to use
%    the new features of babel version 3.1}
% \changes{germanb-2.3e}{1991/11/10}{Brought up-to-date with
%    \file{german.tex} v2.3e (plus some bug fixes) [br]}
% \changes{germanb-2.5}{1994/02/08}{Update or \LaTeXe}
% \changes{germanb-2.5c}{1994/06/26}{Removed the use of \cs{filedate}
%    and moved the identification after the loading of
%    \file{babel.def}}
% \changes{germanb-2.6a}{1995/02/15}{Moved the identification to the
%    top of the file}
% \changes{germanb-2.6a}{1995/02/15}{Rewrote the code that handles the
%    active double quote character}
%
%  \section{The German language}
%
%    The file \file{\filename}\footnote{The file described in this
%    section has version number \fileversion\ and was last revised on
%    \filedate.}  defines all the language definition macros for the
%    German language as well as for the Austrian dialect of this
%    language\footnote{This file is a re-implementation of Hubert
%    Partl's \file{german.sty} version 2.5b, see~\cite{HP}.}.
%
%    For this language the character |"| is made active. In
%    table~\ref{tab:german-quote} an overview is given of its
%    purpose. One of the reasons for this is that in the German
%    language some character combinations change when a word is broken
%    between the combination. Also the vertical placement of the
%    umlaut can be controlled this way.
%    \begin{table}[htb]
%     \begin{center}
%     \begin{tabular}{lp{8cm}}
%      |"a| & |\"a|, also implemented for the other
%                  lowercase and uppercase vowels.                 \\
%      |"s| & to produce the German \ss{} (like |\ss{}|).          \\
%      |"z| & to produce the German \ss{} (like |\ss{}|).          \\
%      |"ck|& for |ck| to be hyphenated as |k-k|.                  \\
%      |"ff|& for |ff| to be hyphenated as |ff-f|,
%                  this is also implemented for l, m, n, p, r and t\\
%      |"S| & for |SS| to be |\uppercase{"s}|.                     \\
%      |"Z| & for |SZ| to be |\uppercase{"z}|.                     \\
%      \verb="|= & disable ligature at this position.              \\
%      |"-| & an explicit hyphen sign, allowing hyphenation
%             in the rest of the word.                             \\
%      |""| & like |"-|, but producing no hyphen sign
%             (for compund words with hyphen, e.g.\ |x-""y|).      \\
%      |"~| & for a compound word mark without a breakpoint.       \\
%      |"=| & for a compound word mark with a breakpoint, allowing
%             hyphenation in the composing words.                  \\
%      |"`| & for German left double quotes (looks like ,,).       \\
%      |"'| & for German right double quotes.                      \\
%      |"<| & for French left double quotes (similar to $<<$).     \\
%      |">| & for French right double quotes (similar to $>>$).    \\
%     \end{tabular}
%     \caption{The extra definitions made
%              by \file{german.ldf}}\label{tab:german-quote}
%     \end{center}
%    \end{table}
%    The quotes in table~\ref{tab:german-quote} can also be typeset by
%    using the commands in table~\ref{tab:more-quote}.
%    \begin{table}[htb]
%     \begin{center}
%     \begin{tabular}{lp{8cm}}
%      |\glqq| & for German left double quotes (looks like ,,).   \\
%      |\grqq| & for German right double quotes (looks like ``).  \\
%      |\glq|  & for German left single quotes (looks like ,).    \\
%      |\grq|  & for German right single quotes (looks like `).   \\
%      |\flqq| & for French left double quotes (similar to $<<$). \\
%      |\frqq| & for French right double quotes (similar to $>>$).\\
%      |\flq|  & for (French) left single quotes (similar to $<$).  \\
%      |\frq|  & for (French) right single quotes (similar to $>$). \\
%      |\dq|   & the original quotes character (|"|).        \\
%     \end{tabular}
%     \caption{More commands which produce quotes, defined
%              by \file{german.ldf}}\label{tab:more-quote}
%     \end{center}
%    \end{table}
%
% \StopEventually{}
%
% \changes{germanb-2.2d}{1991/10/27}{Removed code to load
%    \file{latexhax.com}}
%
%    As this file, \file{germanb.ldf}, needs to be read only once, we
%    check whether it was read before.  If it was, the command
%    |\captionsgerman| is already defined, so we can stop
%    processing. If this command is undefined we proceed with the
%    various definitions and first show the current version of this
%    file.
%
% \changes{germanb-2.2a}{1991/07/15}{Added reset of catcode of @ before
%                                  \cs{endinput}.}
% \changes{germanb-2.2d}{1991/10/27}{Removed use of \cs{@ifundefined}}
% \changes{germanb-2.3e}{1991/11/10}{Moved code to the beginning of
%    the file and added \cs{selectlanguage} call}
%    \begin{macrocode}
%<*code>
\ifx\undefined\captionsgerman
\else
  \selectlanguage{german}
  \expandafter\endinput
\fi
%    \end{macrocode}
%
%  \begin{macro}{\atcatcode}
%    This file, \file{germanb.ldf}, may have been read while \TeX\ is
%    in the middle of processing a document, so we have to make sure
%    the category code of \texttt{@} is `letter' while this file is
%    being read. We save the category code of the @-sign in
%    |\atcatcode| and make it `letter'. Later the category code can be
%    restored to whatever it was before.
%
% \changes{germanb-2.2}{1991/06/11}{Made test of catcode of @ more
%    robust}
% \changes{germanb-2.2a}{1991/07/15}{Modified handling of catcode of @
%    again.}
% \changes{germanb-2.2d}{1991/10/27}{Removed use of \cs{makeatletter}
%    and hence the need to load \file{latexhax.com}}
%    \begin{macrocode}
\chardef\atcatcode=\catcode`\@
\catcode`\@=11\relax
%    \end{macrocode}
%  \end{macro}
%
%
%    Now we determine whether the common macros from the file
%    \file{babel.def} need to be read. We can be in one of two
%    situations: either another language option has been read earlier
%    on, in which case that other option has already read
%    \file{babel.def}, or \file{germanb} is the first language option
%    to be processed. In that case we need to read \file{babel.def}
%    right here before we continue.
%
% \changes{germanb-2.0}{1991/04/23}{New check before loading
%    \file{babel.com}}
% \changes{germanb-2.3g}{1992/02/15}{Added \cs{relax} after the
%    argument of \cs{input}}
%    \begin{macrocode}
\ifx\undefined\babel@core@loaded\input babel.def\relax\fi
%    \end{macrocode}
%
% \changes{germanb-2.1}{1991/05/29}{Add a check for existence of
%    \cs{originalTeX}}
%
%    Another check that has to be made, is if another language
%    definition file has been read already. In that case its
%    definitions have been activated. This might interfere with
%    definitions this file tries to make. Therefore we make sure that
%    we cancel any special definitions. This can be done by checking
%    the existence of the macro |\originalTeX|. If it exists we simply
%    execute it, otherwise it is |\let| to |\empty|.
% \changes{germanb-2.2a}{1991/07/15}{Added
%    \cs{let}\cs{originalTeX}\cs{relax} to test for existence}
% \changes{germanb-2.3f}{1992/01/25}{Set \cs{originalTeX} to \cs{bsl
%    empty}, because it should be expandable.}
%    \begin{macrocode}
\ifx\undefined\originalTeX \let\originalTeX\empty\fi
\originalTeX
%    \end{macrocode}
%
%    When this file is read as an option, i.e., by the |\usaepackage|
%    command, \texttt{german} will be an `unknown' language, so we
%    have to make it known.  So we check for the existence of
%    |\l@german| to see whether we have to do something here.
%
% \changes{germanb-2.0}{1991/04/23}{Now use \cs{adddialect} if
%    language undefined}
% \changes{germanb-2.2d}{1991/10/27}{Removed use of \cs{@ifundefined}}
% \changes{germanb-2.3e}{1991/11/10}{Added warning, if no german
%    patterns loaded}
% \changes{germanb-2.5c}{1994/06/26}{Now use \cs{@nopatterns} to
%    produce the warning}
%    \begin{macrocode}
\ifx\undefined\l@german
  \@nopatterns{German}
  \adddialect\l@german0
\fi
%    \end{macrocode}
%
%    For the Austrian version of these definitions we just add another
%    language. Also, the macros |\captionsaustrian| and
%    |\extrasaustrian| are |\let| to their German counterparts if
%    these parts are defined.
% \changes{germanb-2.0}{1991/04/23}{Now use \cs{adddialect} for
%    austrian}
%    \begin{macrocode}
\adddialect\l@austrian\l@german
%    \end{macrocode}
%
%
%    The next step consists of defining commands to switch to (and
%    from) the German language.
%
%  \begin{macro}{\captionsgerman}
%    The macro |\captionsgerman| defines all strings used in the four
%    standard document classes provided with \LaTeX.
%
% \changes{germanb-2.2}{1991/06/06}{Removed \cs{global} definitions}
% \changes{germanb-2.2}{1991/06/06}{\cs{pagename} should be
%    \cs{headpagename}}
% \changes{germanb-2.3e}{1991/11/10}{Added \cs{prefacename},
%    \cs{seename} and \cs{alsoname}}
% \changes{germanb-2.4}{1993/07/15}{\cs{headpagename} should be
%    \cs{pagename}}
% \changes{german-2.6b}{1995/07/04}{Added \cs{proofname} for
%    AMS-\LaTeX}
%    \begin{macrocode}
\addto\captionsgerman{%
  \def\prefacename{Vorwort}%
  \def\refname{Literatur}%
  \def\abstractname{Zusammenfassung}%
  \def\bibname{Literaturverzeichnis}%
  \def\chaptername{Kapitel}%
  \def\appendixname{Anhang}%
  \def\contentsname{Inhaltsverzeichnis}%    % oder nur: Inhalt
  \def\listfigurename{Abbildungsverzeichnis}%
  \def\listtablename{Tabellenverzeichnis}%
  \def\indexname{Index}%
  \def\figurename{Abbildung}%
  \def\tablename{Tabelle}%                  % oder: Tafel
  \def\partname{Teil}%
  \def\enclname{Anlage(n)}%                 % oder: Beilage(n)
  \def\ccname{Verteiler}%                   % oder: Kopien an
  \def\headtoname{An}%
  \def\pagename{Seite}%
  \def\seename{siehe}%
  \def\alsoname{siehe auch}%
  \def\proofname{Beweis}%
  }
%    \end{macrocode}
%  \end{macro}
%
% \begin{macro}{\captionsgerman}
%    The `captions' are the same for both version of the language, so
%    we can |\let| the macro |\captionsaustrian| be equal to
%    |\captionsgerman|.
%    \begin{macrocode}
\let\captionsaustrian\captionsgerman
%    \end{macrocode}
%  \end{macro}
%
%  \begin{macro}{\dategerman}
%    The macro |\dategerman| redefines the command
%    |\today| to produce German dates.
% \changes{germanb-2.3e}{1991/11/10}{Added \cs{month@german}}
%    \begin{macrocode}
\def\month@german{\ifcase\month\or
  Januar\or Februar\or M\"arz\or April\or Mai\or Juni\or
  Juli\or August\or September\or Oktober\or November\or Dezember\fi}
\def\dategerman{\def\today{\number\day.~\month@german
  \space\number\year}}
%    \end{macrocode}
%  \end{macro}
%
%  \begin{macro}{\dateaustrian}
%    The macro |\dateaustrian| redefines the command
%    |\today| to produce Austrian version of the German dates.
%    \begin{macrocode}
\def\dateaustrian{\def\today{\number\day.~\ifnum1=\month
  J\"anner\else \month@german\fi \space\number\year}}
%    \end{macrocode}
%  \end{macro}
%
%
%  \begin{macro}{\extrasgerman}
% \changes{germanb-2.0b}{1991/05/29}{added some comment chars to
%    prevent white space}
% \changes{germanb-2.2}{1991/06/11}{Save all redefined macros}
%  \begin{macro}{\noextrasgerman}
% \changes{germanb-1.1}{1990/07/30}{Added \cs{dieresis}}
% \changes{germanb-2.0b}{1991/05/29}{added some comment chars to
%    prevent white space}
% \changes{germanb-2.2}{1991/06/11}{Try to restore everything to its
%    former state}
%
%    The macro |\extrasgerman| will perform all the extra definitions
%    needed for the German language. The macro |\noextrasgerman|
%    is used to cancel the actions of |\extrasgerman|.
%
%    For German (as well as for Dutch) the \texttt{"} character is
%    made active. This is done once, later on its definition may vary.
%    \begin{macrocode}
\initiate@active@char{"}
\addto\extrasgerman{\languageshorthands{german}}
\addto\extrasgerman{\bbl@activate{"}}
%\addto\noextrasgerman{\bbl@deactivate{"}}
%    \end{macrocode}
%
% \changes{germanb-2.6a}{1995/02/15}{All the code to handle the active
%    double quote has been moved to \file{babel.def}}
%
%    In order for \TeX\ to be able to hyphenate German words which
%    contain `\ss' (in the \texttt{OT1} position |^^Y|) we have to
%    give the character a nonzero |\lccode| (see Appendix H, the \TeX
%    book).
%    \begin{macrocode}
\addto\extrasgerman{%
  \babel@savevariable{\lccode`\^^Y}%
  \lccode`\^^Y`\^^Y}
%    \end{macrocode}
% \changes{germanb-2.6a}{1995/02/15}{Removeed \cs{3} as it is no
%    longer in \file{german.ldf}}
%
%    The umlaut accent macro |\"| is changed to lower the umlaut dots.
%    The redefinition is done with the help of |\umlautlow|.
%    \begin{macrocode}
\addto\extrasgerman{\babel@save\"\umlautlow}
\addto\noextrasgerman{\umlauthigh}
%    \end{macrocode}
%    The german hyphenation patterns can be used with |\lefthyphenmin|
%    and |\righthyphenmin| set to~2.
% \changes{germanb-2.6a}{1995/05/13}{use \cs{germanhyphenmins} to store
%    the correct values}
%    \begin{macrocode}
\def\germanhyphenmins{\tw@\tw@}
%    \end{macrocode}
%  \end{macro}
%  \end{macro}
%
%  \begin{macro}{\extrasaustrian}
%  \begin{macro}{\noextrasaustrian}
%    For both versions of the language the same special macros are
%    used, so we can |\let| the austrian macros be equal to their
%    german counterparts.
%    \begin{macrocode}
\let\extrasaustrian\extrasgerman
\let\noextrasaustrian\noextrasgerman
%    \end{macrocode}
%  \end{macro}
%  \end{macro}
%
% \changes{germanb-2.6a}{1995/02/15}{\cs{umlautlow} and
%    \cs{umlauthigh} moved to \file{glyphs.dtx}, as well as
%    \cs{newumlaut} (now \cs{lower@umlaut}}
%
%    The code above is necessary because we need an extra active
%    character. This character is then used as indicated in
%    table~\ref{tab:german-quote}.
%
%    To be able to define the function of |"|, we first define a
%    couple of `support' macros.
%
% \changes{germanb-2.3e}{1991/11/10}{Added \cs{save@sf@q} macro and
%    rewrote all quote macros to use it}
% \changes{germanb-2.3h}{1991/02/16}{moved definition of
%    \cs{allowhyphens}, \cs{set@low@box} and \cs{save@sf@q} to
%    \file{babel.com}}
% \changes{german-2.6a}{1995/02/15}{Moved all quotation characters to
%    \file{glyphs.dtx}}
%
%  \begin{macro}{\dq}
%    We save the original double quote character in |\dq| to keep
%    it available, the math accent |\"| can now be typed as |"|.
%    \begin{macrocode}
\begingroup \catcode`\"12
\def\x{\endgroup
  \def\@SS{\mathchar"7019 }
  \def\dq{"}}
\x
%    \end{macrocode}
%  \end{macro}
%
%  \begin{macro}{\german@dq@disc}
%    For the discretionary macros we use this macro:
%    \begin{macrocode}
\def\german@dq@disc#1#2{%
  \penalty\@M\discretionary{#2-}{}{#1}\allowhyphens}
%    \end{macrocode}
%  \end{macro}
%
% \changes{german-2.6a}{1995/02/15}{Use \cs{ddot} instead of
%    \cs{@MATHUMLAUT}}
%
%    Now we can define the doublequote macros: the umlauts,
%    \begin{macrocode}
\declare@shorthand{german}{"a}{\textormath{\"{a}}{\ddot a}}
\declare@shorthand{german}{"o}{\textormath{\"{o}}{\ddot o}}
\declare@shorthand{german}{"u}{\textormath{\"{u}}{\ddot u}}
\declare@shorthand{german}{"A}{\textormath{\"{A}}{\ddot A}}
\declare@shorthand{german}{"O}{\textormath{\"{O}}{\ddot O}}
\declare@shorthand{german}{"U}{\textormath{\"{U}}{\ddot U}}
%    \end{macrocode}
%    tremas,
%    \begin{macrocode}
\declare@shorthand{german}{"e}{\textormath{\"{e}}{\ddot e}}
\declare@shorthand{german}{"E}{\textormath{\"{E}}{\ddot E}}
\declare@shorthand{german}{"i}{\textormath{\"{\i}}{\ddot\imath}}
\declare@shorthand{german}{"I}{\textormath{\"{I}}{\ddot I}}
%    \end{macrocode}
%    german es-zet (sharp s),
%    \begin{macrocode}
\declare@shorthand{german}{"s}{\textormath{\ss{}}{\@SS{}}}
\declare@shorthand{german}{"S}{SS}
\declare@shorthand{german}{"z}{\textormath{\ss{}}{\@SS{}}}
\declare@shorthand{german}{"Z}{SZ}
%    \end{macrocode}
%    german and french quotes,
%    \begin{macrocode}
\declare@shorthand{german}{"`}{%
  \textormath{\quotedblbase{}}{\mbox{\quotedblbase}}}
\declare@shorthand{german}{"'}{%
  \textormath{\textquotedblleft{}}{\mbox{\textquotedblleft}}}
\declare@shorthand{german}{"<}{%
  \textormath{\guillemotleft{}}{\mbox{\guillemotleft}}}
\declare@shorthand{german}{">}{%
  \textormath{\guillemotright{}}{\mbox{\guillemotright}}}
%    \end{macrocode}
%    discretionary commands
%    \begin{macrocode}
\declare@shorthand{german}{"c}{\textormath{\german@dq@disc ck}{c}}
\declare@shorthand{german}{"C}{\textormath{\german@dq@disc CK}{C}}
\declare@shorthand{german}{"f}{\textormath{\german@dq@disc f{ff}}{f}}
\declare@shorthand{german}{"F}{\textormath{\german@dq@disc F{FF}}{F}}
\declare@shorthand{german}{"l}{\textormath{\german@dq@disc l{ll}}{l}}
\declare@shorthand{german}{"L}{\textormath{\german@dq@disc L{LL}}{L}}
\declare@shorthand{german}{"m}{\textormath{\german@dq@disc m{mm}}{m}}
\declare@shorthand{german}{"M}{\textormath{\german@dq@disc M{MM}}{M}}
\declare@shorthand{german}{"n}{\textormath{\german@dq@disc n{nn}}{n}}
\declare@shorthand{german}{"N}{\textormath{\german@dq@disc N{NN}}{N}}
\declare@shorthand{german}{"p}{\textormath{\german@dq@disc p{pp}}{p}}
\declare@shorthand{german}{"P}{\textormath{\german@dq@disc P{PP}}{P}}
\declare@shorthand{german}{"r}{\textormath{\german@dq@disc r{rr}}{r}}
\declare@shorthand{german}{"R}{\textormath{\german@dq@disc R{RR}}{R}}
\declare@shorthand{german}{"t}{\textormath{\german@dq@disc t{tt}}{t}}
\declare@shorthand{german}{"T}{\textormath{\german@dq@disc T{TT}}{T}}
%    \end{macrocode}
%    and some additional commands:
%    \begin{macrocode}
\declare@shorthand{german}{"-}{\penalty\@M\-\allowhyphens}
\declare@shorthand{german}{"|}{%
  \textormath{\penalty\@M\discretionary{-}{}{\kern.03em}%
              \allowhyphens}{}}
\declare@shorthand{german}{""}{\hskip\z@skip}
\declare@shorthand{german}{"~}{\textormath{\leavevmode\hbox{-}}{-}}
\declare@shorthand{german}{"=}{\penalty\@M-\hskip\z@skip}
%    \end{macrocode}
%
%  \begin{macro}{\mdqon}
%  \begin{macro}{\mdqoff}
%  \begin{macro}{\ck}
%    All that's left to do now is to  define a couple of commands
%    for reasons of compatibility with \file{german.sty}.
%    \begin{macrocode}
\def\mdqon{\bbl@activate{"}}
\def\mdqoff{\bbl@deactivate{"}}
\def\ck{\allowhyphens\discretionary{k-}{k}{ck}\allowhyphens}
%    \end{macrocode}
%  \end{macro}
%  \end{macro}
%  \end{macro}
%
%    It is possible that a site might need to add some extra code to
%    the babel macros. To enable this we load a local configuration
%    file, \file{germanb.cfg} if it is found on \TeX' search path.
% \changes{german-2.6b}{1995/07/02}{Added loading of configuration
%    file}
%    \begin{macrocode}
\loadlocalcfg{germanb}
%    \end{macrocode}
%
%    Our last action is to make a note that the commands we have just
%    defined, will be executed by calling the macro |\selectlanguage|
%    at the beginning of the document.
%    \begin{macrocode}
\main@language{german}
%    \end{macrocode}
%    Finally, the category code of \texttt{@} is reset to its original
%    value.
% \changes{germanb-2.2a}{1991/07/15}{Modified handling of catcode of
%    @-sign.}
%    \begin{macrocode}
\catcode`\@=\atcatcode
%</code>
%    \end{macrocode}
%
% \Finale
%%
%% \CharacterTable
%%  {Upper-case    \A\B\C\D\E\F\G\H\I\J\K\L\M\N\O\P\Q\R\S\T\U\V\W\X\Y\Z
%%   Lower-case    \a\b\c\d\e\f\g\h\i\j\k\l\m\n\o\p\q\r\s\t\u\v\w\x\y\z
%%   Digits        \0\1\2\3\4\5\6\7\8\9
%%   Exclamation   \!     Double quote  \"     Hash (number) \#
%%   Dollar        \$     Percent       \%     Ampersand     \&
%%   Acute accent  \'     Left paren    \(     Right paren   \)
%%   Asterisk      \*     Plus          \+     Comma         \,
%%   Minus         \-     Point         \.     Solidus       \/
%%   Colon         \:     Semicolon     \;     Less than     \<
%%   Equals        \=     Greater than  \>     Question mark \?
%%   Commercial at \@     Left bracket  \[     Backslash     \\
%%   Right bracket \]     Circumflex    \^     Underscore    \_
%%   Grave accent  \`     Left brace    \{     Vertical bar  \|
%%   Right brace   \}     Tilde         \~}
%%
\endinput
}
\DeclareOption{germanb}{% \iffalse meta-comment
%
% Copyright 1989-1995 Johannes L. Braams and any individual authors
% listed elsewhere in this file.  All rights reserved.
% 
% For further copyright information any other copyright notices in this
% file.
% 
% This file is part of the Babel system release 3.5.
% --------------------------------------------------
%   This system is distributed in the hope that it will be useful,
%   but WITHOUT ANY WARRANTY; without even the implied warranty of
%   MERCHANTABILITY or FITNESS FOR A PARTICULAR PURPOSE.
% 
%   For error reports concerning UNCHANGED versions of this file no more
%   than one year old, see bugs.txt.
% 
%   Please do not request updates from me directly.  Primary
%   distribution is through the CTAN archives.
% 
% 
% IMPORTANT COPYRIGHT NOTICE:
% 
% You are NOT ALLOWED to distribute this file alone.
% 
% You are allowed to distribute this file under the condition that it is
% distributed together with all the files listed in manifest.txt.
% 
% If you receive only some of these files from someone, complain!
% 
% Permission is granted to copy this file to another file with a clearly
% different name and to customize the declarations in that copy to serve
% the needs of your installation, provided that you comply with
% the conditions in the file legal.txt from the LaTeX2e distribution.
% 
% However, NO PERMISSION is granted to produce or to distribute a
% modified version of this file under its original name.
%  
% You are NOT ALLOWED to change this file.
% 
% 
% \fi
% \CheckSum{341}
%
% \iffalse
%    Tell the \LaTeX\ system who we are and write an entry on the
%    transcript.
%<*dtx>
\ProvidesFile{germanb.dtx}
%</dtx>
%<code>\ProvidesFile{germanb.ldf}
        [1995/07/04 v2.6b German support from the babel system]
%
% Babel package for LaTeX version 2e
% Copyright (C) 1989 - 1995
%           by Johannes Braams, TeXniek
%
% Germanb Language Definition File
% Copyright (C) 1989 - 1995
%           by Bernd Raichle <raichle@azu.Informatik.Uni-Stuttgart.de>
%              Johannes Braams, TeXniek
% This file is based on german.tex version 2.5b,
%                       by Bernd Raichle, Hubert Partl et.al.
%
% Please report errors to: J.L. Braams
%                          JSLBraams@cistron.nl
%
%<*filedriver>
\documentclass{ltxdoc}
\font\manual=logo10 % font used for the METAFONT logo, etc.
\newcommand*\MF{{\manual META}\-{\manual FONT}}
\newcommand*\TeXhax{\TeX hax}
\newcommand*\babel{\textsf{babel}}
\newcommand*\langvar{$\langle \it lang \rangle$}
\newcommand*\note[1]{}
\newcommand*\Lopt[1]{\textsf{#1}}
\newcommand*\file[1]{\texttt{#1}}
\begin{document}
 \DocInput{germanb.dtx}
\end{document}
%</filedriver>
%\fi
% \GetFileInfo{germanb.dtx}
%
% \changes{germanb-1.0a}{1990/05/14}{Incorporated Nico's comments}
% \changes{germanb-1.0b}{1990/05/22}{fixed typo in definition for
%    austrian language found by Werenfried Spit
%    \texttt{nspit@fys.ruu.nl}}
% \changes{germanb-1.0c}{1990/07/16}{Fixed some typos}
% \changes{germanb-1.1}{1990/07/30}{When using PostScript fonts with
%    the Adobe fontencoding, the dieresis-accent is located elsewhere,
%    modified code}
% \changes{germanb-1.1a}{1990/08/27}{Modified the documentation
%    somewhat}
% \changes{germanb-2.0}{1991/04/23}{Modified for babel 3.0}
% \changes{germanb-2.0a}{1991/05/25}{Removed some problems in change
%    log}
% \changes{germanb-2.1}{1991/05/29}{Removed bug found by van der Meer}
% \changes{germanb-2.2}{1991/06/11}{Removed global assignments,
%    brought uptodate with \file{german.tex} v2.3d}
% \changes{germanb-2.2a}{1991/07/15}{Renamed \file{babel.sty} in
%    \file{babel.com}}
% \changes{germanb-2.3}{1991/11/05}{Rewritten parts of the code to use
%    the new features of babel version 3.1}
% \changes{germanb-2.3e}{1991/11/10}{Brought up-to-date with
%    \file{german.tex} v2.3e (plus some bug fixes) [br]}
% \changes{germanb-2.5}{1994/02/08}{Update or \LaTeXe}
% \changes{germanb-2.5c}{1994/06/26}{Removed the use of \cs{filedate}
%    and moved the identification after the loading of
%    \file{babel.def}}
% \changes{germanb-2.6a}{1995/02/15}{Moved the identification to the
%    top of the file}
% \changes{germanb-2.6a}{1995/02/15}{Rewrote the code that handles the
%    active double quote character}
%
%  \section{The German language}
%
%    The file \file{\filename}\footnote{The file described in this
%    section has version number \fileversion\ and was last revised on
%    \filedate.}  defines all the language definition macros for the
%    German language as well as for the Austrian dialect of this
%    language\footnote{This file is a re-implementation of Hubert
%    Partl's \file{german.sty} version 2.5b, see~\cite{HP}.}.
%
%    For this language the character |"| is made active. In
%    table~\ref{tab:german-quote} an overview is given of its
%    purpose. One of the reasons for this is that in the German
%    language some character combinations change when a word is broken
%    between the combination. Also the vertical placement of the
%    umlaut can be controlled this way.
%    \begin{table}[htb]
%     \begin{center}
%     \begin{tabular}{lp{8cm}}
%      |"a| & |\"a|, also implemented for the other
%                  lowercase and uppercase vowels.                 \\
%      |"s| & to produce the German \ss{} (like |\ss{}|).          \\
%      |"z| & to produce the German \ss{} (like |\ss{}|).          \\
%      |"ck|& for |ck| to be hyphenated as |k-k|.                  \\
%      |"ff|& for |ff| to be hyphenated as |ff-f|,
%                  this is also implemented for l, m, n, p, r and t\\
%      |"S| & for |SS| to be |\uppercase{"s}|.                     \\
%      |"Z| & for |SZ| to be |\uppercase{"z}|.                     \\
%      \verb="|= & disable ligature at this position.              \\
%      |"-| & an explicit hyphen sign, allowing hyphenation
%             in the rest of the word.                             \\
%      |""| & like |"-|, but producing no hyphen sign
%             (for compund words with hyphen, e.g.\ |x-""y|).      \\
%      |"~| & for a compound word mark without a breakpoint.       \\
%      |"=| & for a compound word mark with a breakpoint, allowing
%             hyphenation in the composing words.                  \\
%      |"`| & for German left double quotes (looks like ,,).       \\
%      |"'| & for German right double quotes.                      \\
%      |"<| & for French left double quotes (similar to $<<$).     \\
%      |">| & for French right double quotes (similar to $>>$).    \\
%     \end{tabular}
%     \caption{The extra definitions made
%              by \file{german.ldf}}\label{tab:german-quote}
%     \end{center}
%    \end{table}
%    The quotes in table~\ref{tab:german-quote} can also be typeset by
%    using the commands in table~\ref{tab:more-quote}.
%    \begin{table}[htb]
%     \begin{center}
%     \begin{tabular}{lp{8cm}}
%      |\glqq| & for German left double quotes (looks like ,,).   \\
%      |\grqq| & for German right double quotes (looks like ``).  \\
%      |\glq|  & for German left single quotes (looks like ,).    \\
%      |\grq|  & for German right single quotes (looks like `).   \\
%      |\flqq| & for French left double quotes (similar to $<<$). \\
%      |\frqq| & for French right double quotes (similar to $>>$).\\
%      |\flq|  & for (French) left single quotes (similar to $<$).  \\
%      |\frq|  & for (French) right single quotes (similar to $>$). \\
%      |\dq|   & the original quotes character (|"|).        \\
%     \end{tabular}
%     \caption{More commands which produce quotes, defined
%              by \file{german.ldf}}\label{tab:more-quote}
%     \end{center}
%    \end{table}
%
% \StopEventually{}
%
% \changes{germanb-2.2d}{1991/10/27}{Removed code to load
%    \file{latexhax.com}}
%
%    As this file, \file{germanb.ldf}, needs to be read only once, we
%    check whether it was read before.  If it was, the command
%    |\captionsgerman| is already defined, so we can stop
%    processing. If this command is undefined we proceed with the
%    various definitions and first show the current version of this
%    file.
%
% \changes{germanb-2.2a}{1991/07/15}{Added reset of catcode of @ before
%                                  \cs{endinput}.}
% \changes{germanb-2.2d}{1991/10/27}{Removed use of \cs{@ifundefined}}
% \changes{germanb-2.3e}{1991/11/10}{Moved code to the beginning of
%    the file and added \cs{selectlanguage} call}
%    \begin{macrocode}
%<*code>
\ifx\undefined\captionsgerman
\else
  \selectlanguage{german}
  \expandafter\endinput
\fi
%    \end{macrocode}
%
%  \begin{macro}{\atcatcode}
%    This file, \file{germanb.ldf}, may have been read while \TeX\ is
%    in the middle of processing a document, so we have to make sure
%    the category code of \texttt{@} is `letter' while this file is
%    being read. We save the category code of the @-sign in
%    |\atcatcode| and make it `letter'. Later the category code can be
%    restored to whatever it was before.
%
% \changes{germanb-2.2}{1991/06/11}{Made test of catcode of @ more
%    robust}
% \changes{germanb-2.2a}{1991/07/15}{Modified handling of catcode of @
%    again.}
% \changes{germanb-2.2d}{1991/10/27}{Removed use of \cs{makeatletter}
%    and hence the need to load \file{latexhax.com}}
%    \begin{macrocode}
\chardef\atcatcode=\catcode`\@
\catcode`\@=11\relax
%    \end{macrocode}
%  \end{macro}
%
%
%    Now we determine whether the common macros from the file
%    \file{babel.def} need to be read. We can be in one of two
%    situations: either another language option has been read earlier
%    on, in which case that other option has already read
%    \file{babel.def}, or \file{germanb} is the first language option
%    to be processed. In that case we need to read \file{babel.def}
%    right here before we continue.
%
% \changes{germanb-2.0}{1991/04/23}{New check before loading
%    \file{babel.com}}
% \changes{germanb-2.3g}{1992/02/15}{Added \cs{relax} after the
%    argument of \cs{input}}
%    \begin{macrocode}
\ifx\undefined\babel@core@loaded\input babel.def\relax\fi
%    \end{macrocode}
%
% \changes{germanb-2.1}{1991/05/29}{Add a check for existence of
%    \cs{originalTeX}}
%
%    Another check that has to be made, is if another language
%    definition file has been read already. In that case its
%    definitions have been activated. This might interfere with
%    definitions this file tries to make. Therefore we make sure that
%    we cancel any special definitions. This can be done by checking
%    the existence of the macro |\originalTeX|. If it exists we simply
%    execute it, otherwise it is |\let| to |\empty|.
% \changes{germanb-2.2a}{1991/07/15}{Added
%    \cs{let}\cs{originalTeX}\cs{relax} to test for existence}
% \changes{germanb-2.3f}{1992/01/25}{Set \cs{originalTeX} to \cs{bsl
%    empty}, because it should be expandable.}
%    \begin{macrocode}
\ifx\undefined\originalTeX \let\originalTeX\empty\fi
\originalTeX
%    \end{macrocode}
%
%    When this file is read as an option, i.e., by the |\usaepackage|
%    command, \texttt{german} will be an `unknown' language, so we
%    have to make it known.  So we check for the existence of
%    |\l@german| to see whether we have to do something here.
%
% \changes{germanb-2.0}{1991/04/23}{Now use \cs{adddialect} if
%    language undefined}
% \changes{germanb-2.2d}{1991/10/27}{Removed use of \cs{@ifundefined}}
% \changes{germanb-2.3e}{1991/11/10}{Added warning, if no german
%    patterns loaded}
% \changes{germanb-2.5c}{1994/06/26}{Now use \cs{@nopatterns} to
%    produce the warning}
%    \begin{macrocode}
\ifx\undefined\l@german
  \@nopatterns{German}
  \adddialect\l@german0
\fi
%    \end{macrocode}
%
%    For the Austrian version of these definitions we just add another
%    language. Also, the macros |\captionsaustrian| and
%    |\extrasaustrian| are |\let| to their German counterparts if
%    these parts are defined.
% \changes{germanb-2.0}{1991/04/23}{Now use \cs{adddialect} for
%    austrian}
%    \begin{macrocode}
\adddialect\l@austrian\l@german
%    \end{macrocode}
%
%
%    The next step consists of defining commands to switch to (and
%    from) the German language.
%
%  \begin{macro}{\captionsgerman}
%    The macro |\captionsgerman| defines all strings used in the four
%    standard document classes provided with \LaTeX.
%
% \changes{germanb-2.2}{1991/06/06}{Removed \cs{global} definitions}
% \changes{germanb-2.2}{1991/06/06}{\cs{pagename} should be
%    \cs{headpagename}}
% \changes{germanb-2.3e}{1991/11/10}{Added \cs{prefacename},
%    \cs{seename} and \cs{alsoname}}
% \changes{germanb-2.4}{1993/07/15}{\cs{headpagename} should be
%    \cs{pagename}}
% \changes{german-2.6b}{1995/07/04}{Added \cs{proofname} for
%    AMS-\LaTeX}
%    \begin{macrocode}
\addto\captionsgerman{%
  \def\prefacename{Vorwort}%
  \def\refname{Literatur}%
  \def\abstractname{Zusammenfassung}%
  \def\bibname{Literaturverzeichnis}%
  \def\chaptername{Kapitel}%
  \def\appendixname{Anhang}%
  \def\contentsname{Inhaltsverzeichnis}%    % oder nur: Inhalt
  \def\listfigurename{Abbildungsverzeichnis}%
  \def\listtablename{Tabellenverzeichnis}%
  \def\indexname{Index}%
  \def\figurename{Abbildung}%
  \def\tablename{Tabelle}%                  % oder: Tafel
  \def\partname{Teil}%
  \def\enclname{Anlage(n)}%                 % oder: Beilage(n)
  \def\ccname{Verteiler}%                   % oder: Kopien an
  \def\headtoname{An}%
  \def\pagename{Seite}%
  \def\seename{siehe}%
  \def\alsoname{siehe auch}%
  \def\proofname{Beweis}%
  }
%    \end{macrocode}
%  \end{macro}
%
% \begin{macro}{\captionsgerman}
%    The `captions' are the same for both version of the language, so
%    we can |\let| the macro |\captionsaustrian| be equal to
%    |\captionsgerman|.
%    \begin{macrocode}
\let\captionsaustrian\captionsgerman
%    \end{macrocode}
%  \end{macro}
%
%  \begin{macro}{\dategerman}
%    The macro |\dategerman| redefines the command
%    |\today| to produce German dates.
% \changes{germanb-2.3e}{1991/11/10}{Added \cs{month@german}}
%    \begin{macrocode}
\def\month@german{\ifcase\month\or
  Januar\or Februar\or M\"arz\or April\or Mai\or Juni\or
  Juli\or August\or September\or Oktober\or November\or Dezember\fi}
\def\dategerman{\def\today{\number\day.~\month@german
  \space\number\year}}
%    \end{macrocode}
%  \end{macro}
%
%  \begin{macro}{\dateaustrian}
%    The macro |\dateaustrian| redefines the command
%    |\today| to produce Austrian version of the German dates.
%    \begin{macrocode}
\def\dateaustrian{\def\today{\number\day.~\ifnum1=\month
  J\"anner\else \month@german\fi \space\number\year}}
%    \end{macrocode}
%  \end{macro}
%
%
%  \begin{macro}{\extrasgerman}
% \changes{germanb-2.0b}{1991/05/29}{added some comment chars to
%    prevent white space}
% \changes{germanb-2.2}{1991/06/11}{Save all redefined macros}
%  \begin{macro}{\noextrasgerman}
% \changes{germanb-1.1}{1990/07/30}{Added \cs{dieresis}}
% \changes{germanb-2.0b}{1991/05/29}{added some comment chars to
%    prevent white space}
% \changes{germanb-2.2}{1991/06/11}{Try to restore everything to its
%    former state}
%
%    The macro |\extrasgerman| will perform all the extra definitions
%    needed for the German language. The macro |\noextrasgerman|
%    is used to cancel the actions of |\extrasgerman|.
%
%    For German (as well as for Dutch) the \texttt{"} character is
%    made active. This is done once, later on its definition may vary.
%    \begin{macrocode}
\initiate@active@char{"}
\addto\extrasgerman{\languageshorthands{german}}
\addto\extrasgerman{\bbl@activate{"}}
%\addto\noextrasgerman{\bbl@deactivate{"}}
%    \end{macrocode}
%
% \changes{germanb-2.6a}{1995/02/15}{All the code to handle the active
%    double quote has been moved to \file{babel.def}}
%
%    In order for \TeX\ to be able to hyphenate German words which
%    contain `\ss' (in the \texttt{OT1} position |^^Y|) we have to
%    give the character a nonzero |\lccode| (see Appendix H, the \TeX
%    book).
%    \begin{macrocode}
\addto\extrasgerman{%
  \babel@savevariable{\lccode`\^^Y}%
  \lccode`\^^Y`\^^Y}
%    \end{macrocode}
% \changes{germanb-2.6a}{1995/02/15}{Removeed \cs{3} as it is no
%    longer in \file{german.ldf}}
%
%    The umlaut accent macro |\"| is changed to lower the umlaut dots.
%    The redefinition is done with the help of |\umlautlow|.
%    \begin{macrocode}
\addto\extrasgerman{\babel@save\"\umlautlow}
\addto\noextrasgerman{\umlauthigh}
%    \end{macrocode}
%    The german hyphenation patterns can be used with |\lefthyphenmin|
%    and |\righthyphenmin| set to~2.
% \changes{germanb-2.6a}{1995/05/13}{use \cs{germanhyphenmins} to store
%    the correct values}
%    \begin{macrocode}
\def\germanhyphenmins{\tw@\tw@}
%    \end{macrocode}
%  \end{macro}
%  \end{macro}
%
%  \begin{macro}{\extrasaustrian}
%  \begin{macro}{\noextrasaustrian}
%    For both versions of the language the same special macros are
%    used, so we can |\let| the austrian macros be equal to their
%    german counterparts.
%    \begin{macrocode}
\let\extrasaustrian\extrasgerman
\let\noextrasaustrian\noextrasgerman
%    \end{macrocode}
%  \end{macro}
%  \end{macro}
%
% \changes{germanb-2.6a}{1995/02/15}{\cs{umlautlow} and
%    \cs{umlauthigh} moved to \file{glyphs.dtx}, as well as
%    \cs{newumlaut} (now \cs{lower@umlaut}}
%
%    The code above is necessary because we need an extra active
%    character. This character is then used as indicated in
%    table~\ref{tab:german-quote}.
%
%    To be able to define the function of |"|, we first define a
%    couple of `support' macros.
%
% \changes{germanb-2.3e}{1991/11/10}{Added \cs{save@sf@q} macro and
%    rewrote all quote macros to use it}
% \changes{germanb-2.3h}{1991/02/16}{moved definition of
%    \cs{allowhyphens}, \cs{set@low@box} and \cs{save@sf@q} to
%    \file{babel.com}}
% \changes{german-2.6a}{1995/02/15}{Moved all quotation characters to
%    \file{glyphs.dtx}}
%
%  \begin{macro}{\dq}
%    We save the original double quote character in |\dq| to keep
%    it available, the math accent |\"| can now be typed as |"|.
%    \begin{macrocode}
\begingroup \catcode`\"12
\def\x{\endgroup
  \def\@SS{\mathchar"7019 }
  \def\dq{"}}
\x
%    \end{macrocode}
%  \end{macro}
%
%  \begin{macro}{\german@dq@disc}
%    For the discretionary macros we use this macro:
%    \begin{macrocode}
\def\german@dq@disc#1#2{%
  \penalty\@M\discretionary{#2-}{}{#1}\allowhyphens}
%    \end{macrocode}
%  \end{macro}
%
% \changes{german-2.6a}{1995/02/15}{Use \cs{ddot} instead of
%    \cs{@MATHUMLAUT}}
%
%    Now we can define the doublequote macros: the umlauts,
%    \begin{macrocode}
\declare@shorthand{german}{"a}{\textormath{\"{a}}{\ddot a}}
\declare@shorthand{german}{"o}{\textormath{\"{o}}{\ddot o}}
\declare@shorthand{german}{"u}{\textormath{\"{u}}{\ddot u}}
\declare@shorthand{german}{"A}{\textormath{\"{A}}{\ddot A}}
\declare@shorthand{german}{"O}{\textormath{\"{O}}{\ddot O}}
\declare@shorthand{german}{"U}{\textormath{\"{U}}{\ddot U}}
%    \end{macrocode}
%    tremas,
%    \begin{macrocode}
\declare@shorthand{german}{"e}{\textormath{\"{e}}{\ddot e}}
\declare@shorthand{german}{"E}{\textormath{\"{E}}{\ddot E}}
\declare@shorthand{german}{"i}{\textormath{\"{\i}}{\ddot\imath}}
\declare@shorthand{german}{"I}{\textormath{\"{I}}{\ddot I}}
%    \end{macrocode}
%    german es-zet (sharp s),
%    \begin{macrocode}
\declare@shorthand{german}{"s}{\textormath{\ss{}}{\@SS{}}}
\declare@shorthand{german}{"S}{SS}
\declare@shorthand{german}{"z}{\textormath{\ss{}}{\@SS{}}}
\declare@shorthand{german}{"Z}{SZ}
%    \end{macrocode}
%    german and french quotes,
%    \begin{macrocode}
\declare@shorthand{german}{"`}{%
  \textormath{\quotedblbase{}}{\mbox{\quotedblbase}}}
\declare@shorthand{german}{"'}{%
  \textormath{\textquotedblleft{}}{\mbox{\textquotedblleft}}}
\declare@shorthand{german}{"<}{%
  \textormath{\guillemotleft{}}{\mbox{\guillemotleft}}}
\declare@shorthand{german}{">}{%
  \textormath{\guillemotright{}}{\mbox{\guillemotright}}}
%    \end{macrocode}
%    discretionary commands
%    \begin{macrocode}
\declare@shorthand{german}{"c}{\textormath{\german@dq@disc ck}{c}}
\declare@shorthand{german}{"C}{\textormath{\german@dq@disc CK}{C}}
\declare@shorthand{german}{"f}{\textormath{\german@dq@disc f{ff}}{f}}
\declare@shorthand{german}{"F}{\textormath{\german@dq@disc F{FF}}{F}}
\declare@shorthand{german}{"l}{\textormath{\german@dq@disc l{ll}}{l}}
\declare@shorthand{german}{"L}{\textormath{\german@dq@disc L{LL}}{L}}
\declare@shorthand{german}{"m}{\textormath{\german@dq@disc m{mm}}{m}}
\declare@shorthand{german}{"M}{\textormath{\german@dq@disc M{MM}}{M}}
\declare@shorthand{german}{"n}{\textormath{\german@dq@disc n{nn}}{n}}
\declare@shorthand{german}{"N}{\textormath{\german@dq@disc N{NN}}{N}}
\declare@shorthand{german}{"p}{\textormath{\german@dq@disc p{pp}}{p}}
\declare@shorthand{german}{"P}{\textormath{\german@dq@disc P{PP}}{P}}
\declare@shorthand{german}{"r}{\textormath{\german@dq@disc r{rr}}{r}}
\declare@shorthand{german}{"R}{\textormath{\german@dq@disc R{RR}}{R}}
\declare@shorthand{german}{"t}{\textormath{\german@dq@disc t{tt}}{t}}
\declare@shorthand{german}{"T}{\textormath{\german@dq@disc T{TT}}{T}}
%    \end{macrocode}
%    and some additional commands:
%    \begin{macrocode}
\declare@shorthand{german}{"-}{\penalty\@M\-\allowhyphens}
\declare@shorthand{german}{"|}{%
  \textormath{\penalty\@M\discretionary{-}{}{\kern.03em}%
              \allowhyphens}{}}
\declare@shorthand{german}{""}{\hskip\z@skip}
\declare@shorthand{german}{"~}{\textormath{\leavevmode\hbox{-}}{-}}
\declare@shorthand{german}{"=}{\penalty\@M-\hskip\z@skip}
%    \end{macrocode}
%
%  \begin{macro}{\mdqon}
%  \begin{macro}{\mdqoff}
%  \begin{macro}{\ck}
%    All that's left to do now is to  define a couple of commands
%    for reasons of compatibility with \file{german.sty}.
%    \begin{macrocode}
\def\mdqon{\bbl@activate{"}}
\def\mdqoff{\bbl@deactivate{"}}
\def\ck{\allowhyphens\discretionary{k-}{k}{ck}\allowhyphens}
%    \end{macrocode}
%  \end{macro}
%  \end{macro}
%  \end{macro}
%
%    It is possible that a site might need to add some extra code to
%    the babel macros. To enable this we load a local configuration
%    file, \file{germanb.cfg} if it is found on \TeX' search path.
% \changes{german-2.6b}{1995/07/02}{Added loading of configuration
%    file}
%    \begin{macrocode}
\loadlocalcfg{germanb}
%    \end{macrocode}
%
%    Our last action is to make a note that the commands we have just
%    defined, will be executed by calling the macro |\selectlanguage|
%    at the beginning of the document.
%    \begin{macrocode}
\main@language{german}
%    \end{macrocode}
%    Finally, the category code of \texttt{@} is reset to its original
%    value.
% \changes{germanb-2.2a}{1991/07/15}{Modified handling of catcode of
%    @-sign.}
%    \begin{macrocode}
\catcode`\@=\atcatcode
%</code>
%    \end{macrocode}
%
% \Finale
%%
%% \CharacterTable
%%  {Upper-case    \A\B\C\D\E\F\G\H\I\J\K\L\M\N\O\P\Q\R\S\T\U\V\W\X\Y\Z
%%   Lower-case    \a\b\c\d\e\f\g\h\i\j\k\l\m\n\o\p\q\r\s\t\u\v\w\x\y\z
%%   Digits        \0\1\2\3\4\5\6\7\8\9
%%   Exclamation   \!     Double quote  \"     Hash (number) \#
%%   Dollar        \$     Percent       \%     Ampersand     \&
%%   Acute accent  \'     Left paren    \(     Right paren   \)
%%   Asterisk      \*     Plus          \+     Comma         \,
%%   Minus         \-     Point         \.     Solidus       \/
%%   Colon         \:     Semicolon     \;     Less than     \<
%%   Equals        \=     Greater than  \>     Question mark \?
%%   Commercial at \@     Left bracket  \[     Backslash     \\
%%   Right bracket \]     Circumflex    \^     Underscore    \_
%%   Grave accent  \`     Left brace    \{     Vertical bar  \|
%%   Right brace   \}     Tilde         \~}
%%
\endinput
}
%    \end{macrocode}
%    \Lopt{hungarian} is just a synonym for \Lopt{magyar}
%    \begin{macrocode}
\DeclareOption{hungarian}{% \iffalse meta-comment
%
% Copyright 1989-1995 Johannes L. Braams and any individual authors
% listed elsewhere in this file.  All rights reserved.
% 
% For further copyright information any other copyright notices in this
% file.
% 
% This file is part of the Babel system release 3.5.
% --------------------------------------------------
%   This system is distributed in the hope that it will be useful,
%   but WITHOUT ANY WARRANTY; without even the implied warranty of
%   MERCHANTABILITY or FITNESS FOR A PARTICULAR PURPOSE.
% 
%   For error reports concerning UNCHANGED versions of this file no more
%   than one year old, see bugs.txt.
% 
%   Please do not request updates from me directly.  Primary
%   distribution is through the CTAN archives.
% 
% 
% IMPORTANT COPYRIGHT NOTICE:
% 
% You are NOT ALLOWED to distribute this file alone.
% 
% You are allowed to distribute this file under the condition that it is
% distributed together with all the files listed in manifest.txt.
% 
% If you receive only some of these files from someone, complain!
% 
% Permission is granted to copy this file to another file with a clearly
% different name and to customize the declarations in that copy to serve
% the needs of your installation, provided that you comply with
% the conditions in the file legal.txt from the LaTeX2e distribution.
% 
% However, NO PERMISSION is granted to produce or to distribute a
% modified version of this file under its original name.
%  
% You are NOT ALLOWED to change this file.
% 
% 
% \fi
% \CheckSum{282}
% \iffalse
%    Tell the \LaTeX\ system who we are and write an entry on the
%    transcript.
%<*dtx>
\ProvidesFile{magyar.dtx}
%</dtx>
%<code>\ProvidesFile{magyar.ldf}
        [1995/07/04 v1.3e Magyar support from the babel system]
%
% Babel package for LaTeX version 2e
% Copyright (C) 1989 - 1995
%           by Johannes Braams, TeXniek
%
% Magyar Language Definition File
% Copyright (C) 1989 - 1995
%           by Johannes Braams, TeXniek
%              \'Arp\'ad B\'IR\'O
%
% Please report errors to: J.L. Braams <JLBraams@cistron.nl>
%
%    This file is part of the babel system, it provides the source
%    code for the Hungarian language definition file.  A contribution
%    was made by Attila Koppanyi (attila@cernvm.cern.ch).
%<*filedriver>
\documentclass{ltxdoc}
\newcommand*\TeXhax{\TeX hax}
\newcommand*\babel{\textsf{babel}}
\newcommand*\langvar{$\langle \it lang \rangle$}
\newcommand*\note[1]{}
\newcommand*\Lopt[1]{\textsf{#1}}
\newcommand*\file[1]{\texttt{#1}}
\begin{document}
 \DocInput{magyar.dtx}
\end{document}
%</filedriver>
%\fi
% \GetFileInfo{magyar.dtx}
%
% \changes{magyar-1.0a}{1991/07/15}{Renamed \file{babel.sty} in
%    \file{babel.com}}
% \changes{magyar-1.1}{1992/02/16}{Brought up-to-date with babel 3.2a}
% \changes{magyar-1.1.4}{1994/02/08}{Further spelling corrections}
% \changes{magyar-1.1.5}{1994/02/09}{Still more spelling corrections}
% \changes{magyar-1.2}{1994/02/27}{Update for \LaTeXe}
% \changes{magyar-1.3c}{1994/06/26}{Removed the use of \cs{filedate}
%    and moved identification after the loading of \file{babel.def}}
%
%  \section{The Hungarian language}
%
%    The file option \file{\filename}\footnote{The file described in
%    this section has version number \fileversion\ and was last
%    revised on \filedate.  A contribution was made by Attila Koppanyi
%    (\texttt{attila@cernvm.cern.ch}). Later updates and suggestions
%    by \'Arp\'ad B\'ir\'o (\texttt{JZP1104@HUSZEG11.bitnet}), Istvan
%    Hamecz (\texttt{hami@ursus.bke.hu)} and Horvath Dezso
%    (\texttt{horvath@pisa.infn.it}).}  defines all the
%    language definition macros for the Hungarian language.
%
% \DescribeMacro\ontoday
%    For this language currently the only special definition that is
%    added is the |\ontoday| command which works like |\today| but
%    produces a slightly different date format used in expressions suh
%    as `on february 10th'.
%
% \StopEventually{}
%
%    As this file needs to be read only once, we check whether it was
%    read before. If it was, the command |\captionsmagyar| is already
%    defined, so we can stop processing. If this command is undefined
%    we proceed with the various definitions and first show the
%    current version of this file.
%
% \changes{magyar-1.0a}{1991/07/15}{Added reset of catcode of @ before
%    \cs{endinput}.}
% \changes{magyar-1.0b}{1991/10/29}{Removed use of \cs{@ifundefined}}
%    \begin{macrocode}
%<*code>
\ifx\undefined\captionsmagyar
\else
  \selectlanguage{magyar}
  \expandafter\endinput
\fi
%    \end{macrocode}
%
% \changes{magyar-1.0b}{1991/10/29}{Removed code to load
%    \file{latexhax.com}}
%
% \begin{macro}{\atcatcode}
%    This file, \file{magyar.ldf}, may have been read while \TeX\ is
%    in the middle of processing a document, so we have to make sure
%    the category code of \texttt{@} is `letter' while this file is
%    being read.  We save the category code of the @-sign in
%    |\atcatcode| and make it `letter'. Later the category code can be
%    restored to whatever it was before.
%
% \changes{magyar-1.0a}{1991/07/15}{Modified handling of catcode of @
%    again.}
% \changes{magyar-1.0b}{1991/10/29}{Removed use of \cs{makeatletter}
%    and hence the need to load \file{latexhax.com}}
%    \begin{macrocode}
\chardef\atcatcode=\catcode`\@
\catcode`\@=11\relax
%    \end{macrocode}
% \end{macro}
%
%    Now we determine whether the the common macros from the file
%    \file{babel.def} need to be read. We can be in one of two
%    situations: either another language option has been read earlier
%    on, in which case that other option has already read
%    \file{babel.def}, or \texttt{magyar} is the first language option
%    to be processed. In that case we need to read \file{babel.def}
%    right here before we continue.
%
% \changes{magyar-1.1}{1992/02/16}{Added \cs{relax} after the argument
%    of \cs{input}}
%    \begin{macrocode}
\ifx\undefined\babel@core@loaded\input babel.def\relax\fi
%    \end{macrocode}
%
%    Another check that has to be made, is if another language
%    definition file has been read already. In that case its definitions
%    have been activated. This might interfere with definitions this
%    file tries to make. Therefore we make sure that we cancel any
%    special definitions. This can be done by checking the existence
%    of the macro |\originalTeX|. If it exists we simply execute it,
%    otherwise it is |\let| to |\empty|.
% \changes{magyar-1.0a}{1991/07/15}{Added
%    \cs{let}\cs{originalTeX}\cs{relax} to test for existence}
% \changes{magyar-1.1}{1992/02/16}{\cs{originalTeX} should be
%    expandable, \cs{let} it to \cs{empty}}
%    \begin{macrocode}
\ifx\undefined\originalTeX \let\originalTeX\empty \else\originalTeX\fi
%    \end{macrocode}
%
%    When this file is read as an option, i.e. by the |\usepackage|
%    command, \texttt{magyar} will be an `unknown' language in which
%    case we have to make it known.  So we check for the existence of
%    |\l@magyar| to see whether we have to do something here.
%
% \changes{magyar-1.0b}{1991/10/29}{Removed use of \cs{@ifundefined}}
% \changes{magyar-1.1}{1992/02/16}{Added a warning when no hyphenation
%    patterns were loaded.}
% \changes{magyar-1.3c}{1994/06/26}{Now use \cs{@nopatterns} to
%    produce the warning}
%    \begin{macrocode}
\ifx\undefined\l@magyar
    \@nopatterns{Magyar}
    \adddialect\l@magyar0\fi
%    \end{macrocode}
%
%    An additional note about formatting Hungarian texts: One should
%    invert the order of the number and text in things like chapter
%    headings, page references etc. So one should write `I. r\'esz'
%    instead of `Part I', or `3. oldal' for `page 3'.
%
%    For chapter headings this could be accomplished by a redefinition
%    of the macros |\@makechapterhead| and |\@makeschapterhead|, for
%    other instances this a lot harder to accomplish. Therefore I
%    think complete document classes should be written to accomadate
%    the needed formatting.
%
%    The next step consists of defining commands to switch to (and
%    from) the Hungarian language.
%
% \begin{macro}{\captionsmagyar}
%    The macro |\captionsmagyar| defines all strings used in the four
%    standard documentclasses provided with \LaTeX.
% \changes{magyar-1.1}{1992/02/16}{Added \cs{seename}, \cs{alsoname}
%    and \cs{prefacename}}
% \changes{magyar-1.1}{1993/07/15}{\cs{headpagename} should be
%    \cs{pagename}}
% \changes{magyar-1.1.3}{1994/01/05}{Added translations, fixed typos}
% \changes{magyar-1.3e}{1995/07/04}{Added \cs{proofname} for
%    AMS-\LaTeX}
%    \begin{macrocode}
\addto\captionsmagyar{%
  \def\prefacename{El\H osz\'o}%
  \def\refname{Referenci\'ak}%
  \def\abstractname{Kivonat}%
  \def\bibname{Bibliogr\'afia}%
  \def\chaptername{fejezet}%
  \def\appendixname{f\"uggel\'ek}%
  \def\contentsname{Tartalom}%
  \def\listfigurename{\'Abr\'ak jegyz\'eke}%
  \def\listtablename{T\'abl\'azatok jegyz\'eke}%
  \def\indexname{T\'argymutat\'o}%
  \def\figurename{\'abra}%
  \def\tablename{t\'abl\'azat}%
  \def\partname{r\'esz}%
  \def\enclname{Mell\'eklet}%
  \def\ccname{K\"orlev\'el--c\'\i mzettek}%
  \def\headtoname{C\'\i mzett}%
  \def\pagename{oldal}%
  \def\seename{L\'asd}%
  \def\alsoname{L\'asd m\'eg}%
  \def\proofname{Proof}%   <-- needs translation
  }%
%    \end{macrocode}
% \end{macro}
%
% \begin{macro}{\datemagyar}
%    The macro |\datemagyar| redefines the command |\today| to produce
%    Hungarian dates.
% \changes{magyar-1.1.4}{1994/02/08}{Rewritten to produce the correct
%    date format}
%    \begin{macrocode}
\def\datemagyar{%
  \def\today{\number\year.~\ifcase\month\or
  janu\'ar\or febru\'ar\or m\'arcius\or
  \'aprilis\or m\'ajus\or j\'unius\or
  j\'ulius\or augusztus\or szeptember\or
  okt\'ober\or november\or december\fi
    \space\ifcase\day\or
    1.\or  2.\or  3.\or  4.\or  5.\or
    6.\or  7.\or  8.\or  9.\or 10.\or
   11.\or 12.\or 13.\or 14.\or 15.\or
   16.\or 17.\or 18.\or 19.\or 20.\or
   21.\or 22.\or 23.\or 24.\or 25.\or
   26.\or 27.\or 28.\or 29.\or 30.\or
   31.\fi}}
%    \end{macrocode}
% \end{macro}
%
% \begin{macro}{\ondatemagyar}
%    The macro |\ondatemagyar| produces Hungarian dates which have the
%    meaning `\emph{on this day}'.  It does not redefine the command
%    |\today|.
% \changes{magyar-1.1.3}{1994/01/05}{The date number should not be
%    followed by a dot.}
% \changes{magyar-1.1.4}{1994/02/08}{Renamed from \cs{datemagyar};
%    nolonger redefines \cs{today}.}
%    \begin{macrocode}
\def\ondatemagyar{%
  \number\year.~\ifcase\month\or
  janu\'ar\or febru\'ar\or m\'arcius\or
  \'aprilis\or m\'ajus\or j\'unius\or
  j\'ulius\or augusztus\or szeptember\or
  okt\'ober\or november\or december\fi
    \space\ifcase\day\or
    1-j\'en\or  2-\'an\or  3-\'an\or  4-\'en\or  5-\'en\or
    6-\'an\or  7-\'en\or  8-\'an\or  9-\'en\or 10-\'en\or
   11-\'en\or 12-\'en\or 13-\'an\or 14-\'en\or 15-\'en\or
   16-\'an\or 17-\'en\or 18-\'an\or 19-\'en\or 20-\'an\or
   21-\'en\or 22-\'en\or 23-\'an\or 24-\'en\or 25-\'en\or
   26-\'an\or 27-\'en\or 28-\'an\or 29-\'en\or 30-\'an\or
   31-\'en\fi}
%    \end{macrocode}
% \end{macro}
%
% \begin{macro}{\extrasmagyar}
% \begin{macro}{\noextrasmagyar}
%    The macro |\extrasmagyar| will perform all the extra definitions
%    needed for the Hungarian language. The macro |\noextrasmagyar| is
%    used to cancel the actions of |\extrasmagyar|.  For the moment
%    these macros are nearly empty; only the user command |\ontoday|
%    to access |\ondatemagyar| is defined.
%
%    \begin{macrocode}
\addto\extrasmagyar{\let\ontoday\ondatemagyar}
\addto\noextrasmagyar{\let\ontoday\undefined}
%    \end{macrocode}
% \end{macro}
% \end{macro}
%
%    It is possible that a site might need to add some extra code to
%    the babel macros. To enable this we load a local configuration
%    file, \file{magyar.cfg} if it is found on \TeX' search path.
% \changes{magyar-1.3e}{1995/07/02}{Added loading of configuration
%    file}
%    \begin{macrocode}
\loadlocalcfg{magyar}
%    \end{macrocode}
%
%    Our last action is to make a note that the commands we have just
%    defined, will be executed by calling the macro |\selectlanguage|
%    at the beginning of the document.
%    \begin{macrocode}
\main@language{magyar}
%    \end{macrocode}
%    Finally, the category code of \texttt{@} is reset to its original
%    value. The macrospace used by |\atcatcode| is freed.
% \changes{magyar-1.0a}{1991/07/15}{Modified handling of catcode of
%    @-sign.}
%    \begin{macrocode}
\catcode`\@=\atcatcode \let\atcatcode\relax
%</code>
%    \end{macrocode}
%
% \Finale
%%
%% \CharacterTable
%%  {Upper-case    \A\B\C\D\E\F\G\H\I\J\K\L\M\N\O\P\Q\R\S\T\U\V\W\X\Y\Z
%%   Lower-case    \a\b\c\d\e\f\g\h\i\j\k\l\m\n\o\p\q\r\s\t\u\v\w\x\y\z
%%   Digits        \0\1\2\3\4\5\6\7\8\9
%%   Exclamation   \!     Double quote  \"     Hash (number) \#
%%   Dollar        \$     Percent       \%     Ampersand     \&
%%   Acute accent  \'     Left paren    \(     Right paren   \)
%%   Asterisk      \*     Plus          \+     Comma         \,
%%   Minus         \-     Point         \.     Solidus       \/
%%   Colon         \:     Semicolon     \;     Less than     \<
%%   Equals        \=     Greater than  \>     Question mark \?
%%   Commercial at \@     Left bracket  \[     Backslash     \\
%%   Right bracket \]     Circumflex    \^     Underscore    \_
%%   Grave accent  \`     Left brace    \{     Vertical bar  \|
%%   Right brace   \}     Tilde         \~}
%%
\endinput
%
  \let\captionshungarian\captionsmagyar
  \let\datehungarian\datemagyar
  \let\extrashungarian\extrasmagyar
  \let\noextrashungarian\noextrasmagyar
  \let\hugarianhyphenmins\magyarhyphenmins
  }
\DeclareOption{irish}{% \iffalse meta-comment
%
% Copyright 1989-1995 Johannes L. Braams and any individual authors
% listed elsewhere in this file.  All rights reserved.
% 
% For further copyright information any other copyright notices in this
% file.
% 
% This file is part of the Babel system release 3.5.
% --------------------------------------------------
%   This system is distributed in the hope that it will be useful,
%   but WITHOUT ANY WARRANTY; without even the implied warranty of
%   MERCHANTABILITY or FITNESS FOR A PARTICULAR PURPOSE.
% 
%   For error reports concerning UNCHANGED versions of this file no more
%   than one year old, see bugs.txt.
% 
%   Please do not request updates from me directly.  Primary
%   distribution is through the CTAN archives.
% 
% 
% IMPORTANT COPYRIGHT NOTICE:
% 
% You are NOT ALLOWED to distribute this file alone.
% 
% You are allowed to distribute this file under the condition that it is
% distributed together with all the files listed in manifest.txt.
% 
% If you receive only some of these files from someone, complain!
% 
% Permission is granted to copy this file to another file with a clearly
% different name and to customize the declarations in that copy to serve
% the needs of your installation, provided that you comply with
% the conditions in the file legal.txt from the LaTeX2e distribution.
% 
% However, NO PERMISSION is granted to produce or to distribute a
% modified version of this file under its original name.
%  
% You are NOT ALLOWED to change this file.
% 
% 
% \fi
% \CheckSum{143}
% \iffalse
%    Tell the \LaTeX\ system who we are and write an entry on the
%    transcript.
%<*dtx>
\ProvidesFile{irish.dtx}
%</dtx>
%<code>\ProvidesFile{irish.ldf}
        [1995/07/04 v1.0c Irish support from the babel system]
%
% Babel package for LaTeX version 2e
% Copyright (C) 1989 -- 1995
%           by Johannes Braams, TeXniek
%
% Please report errors to: J.L. Braams
%                          JLBraams@cistron.nl
%
%    This file is part of the babel system, it provides the source
%    code for the Irish language definition file.
%
%    The Gaeilge or Irish Gaelic terms were tranlated from those
%    provided by Fraser Grant \texttt{FRASER@CERNVM} by Marion Gunn.
%<*filedriver>
\documentclass{ltxdoc}
\newcommand*{\TeXhax}{\TeX hax}
\newcommand*{\babel}{\textsf{babel}}
\newcommand*{\langvar}{$\langle \mathit lang \rangle$}
\newcommand*{\note}[1]{}
\newcommand*{\Lopt}[1]{\textsf{#1}}
\newcommand*{\file}[1]{\texttt{#1}}
\begin{document}
 \DocInput{irish.dtx}
\end{document}
%</filedriver>
%\fi
% \GetFileInfo{irish.dtx}
%
% \changes{itish-1.0b}{1995/06/14}{Corrected typo (PR1652)}
%
%  \section{The Irish language}
%
%    The file \file{\filename}\footnote{The file described in this
%    section has version number \fileversion\ and was last revised on
%    \filedate. A contribution was made by Marion Gunn.}  defines all
%    the language definition macros for the Irish language.
%
% \StopEventually{}
%
%    As this file needs to be read only once, we check whether it was
%    read before. If it was, the command |\captionsirish| is already
%    defined, so we can stop processing. If this command is undefined
%    we proceed with the various definitions and first show the
%    current version of this file.
%
%    \begin{macrocode}
%<*code>
\ifx\undefined\captionsirish
\else
  \selectlanguage{irish}
  \expandafter\endinput
\fi
%    \end{macrocode}
%
% \begin{macro}{\atcatcode}
%    This file, \file{irish.ldf}, may have been read while \TeX\ is in
%    the middle of processing a document, so we have to make sure the
%    category code of \texttt{@} is `letter' while this file is being
%    read.  We save the category code of the @-sign in |\atcatcode|
%    and make it `letter'. Later the category code can be restored to
%    whatever it was before.
%    \begin{macrocode}
\chardef\atcatcode=\catcode`\@
\catcode`\@=11\relax
%    \end{macrocode}
% \end{macro}
%
%    Now we determine whether the the common macros from the file
%    \file{babel.def} need to be read. We can be in one of two
%    situations: either another language option has been read earlier
%    on, in which case that other option has already read
%    \file{babel.def}, or \texttt{irish} is the first language option
%    to be processed. In that case we need to read \file{babel.def}
%    right here before we continue.
%
%    \begin{macrocode}
\ifx\undefined\babel@core@loaded\input babel.def\relax\fi
%    \end{macrocode}
%
%    Another check that has to be made, is if another language
%    definition file has been read already. In that case its definitions
%    have been activated. This might interfere with definitions this
%    file tries to make. Therefore we make sure that we cancel any
%    special definitions. This can be done by checking the existence
%    of the macro |\originalTeX|. If it exists we simply execute it.
%    \begin{macrocode}
\ifx\undefined\originalTeX
  \let\originalTeX\empty
\fi
\originalTeX
%    \end{macrocode}
%
%    When this file is read as an option, i.e. by the |\usepackage|
%    command, \texttt{irish} could be an `unknown' language in which
%    case we have to make it known.  So we check for the existence of
%    |\l@irish| to see whether we have to do something here.
%
%    \begin{macrocode}
\ifx\undefined\l@irish
  \@nopatterns{irish}
  \adddialect\l@irish0\fi
%    \end{macrocode}
%
%    The next step consists of defining commands to switch to (and
%    from) the Irish language.
%
% \begin{macro}{\captionsirish}
%    The macro |\captionsirish| defines all strings used in the
%    four standard documentclasses provided with \LaTeX.
% \changes{irish-1.0c}{1995/07/04}{Added \cs{proofname} for
%    AMS-\LaTeX}
%    \begin{macrocode}
\addto\captionsirish{%
  \def\prefacename{Preface}%    <-- needs translation
  \def\refname{Tagairt\'{\i}}%
  \def\abstractname{Achoimre}%
  \def\bibname{Leabharliosta}%
  \def\chaptername{Caibidil}%
  \def\appendixname{Aguis\'{\i}n}%
  \def\contentsname{Cl\'ar \'Abhair}%
  \def\listfigurename{L\'ear\'aid\'{\i}}%
  \def\listtablename{T\'abla\'{\i}}%
  \def\indexname{Inn\'eacs}%
  \def\figurename{L\'ear\'aid}%
  \def\tablename{T\'abla}%
  \def\partname{Cuid}%
  \def\enclname{faoi iamh}%
  \def\ccname{cc}%                 abrv. `c\'oip chuig'
  \def\headtoname{Go}%
  \def\pagename{Leathanach}%
  \def\seename{see}%    <-- needs translation
  \def\alsoname{see also}%    <-- needs translation
  \def\proofname{Proof}%    <-- needs translation
  }
%    \end{macrocode}
% \end{macro}
%
% \begin{macro}{\dateirish}
%    The macro |\dateirish| redefines the command |\today| to produce
%    Irish dates.
%    \begin{macrocode}
\def\dateirish{%
  \number\day\space \ifcase\month\or
  Ean\'air\or Feabhra\or M\'arta\or Aibre\'an\or
  Bealtaine\or Meitheamh\or I\'uil\or L\'unasa\or
  Me\'an F\'omhair\or Deireadh F\'omhair\or
  M\'{\i} na Samhna\or M\'{\i} na Nollag\fi
  \space \number\year}
%    \end{macrocode}
% \end{macro}
%
% \begin{macro}{\extrasirish}
% \begin{macro}{\noextrasirish}
%    The macro |\extrasirish| will perform all the extra definitions
%    needed for the Irish language. The macro |\noextrasirish| is used
%    to cancel the actions of |\extrasirish|.  For the moment these
%    macros are empty but they are defined for compatibility with the
%    other language definition files.
%
%    \begin{macrocode}
\addto\extrasirish{}
\addto\noextrasirish{}
%    \end{macrocode}
% \end{macro}
% \end{macro}
%
%    It is possible that a site might need to add some extra code to
%    the babel macros. To enable this we load a local configuration
%    file, \file{danish.cfg} if it is found on \TeX' search path.
% \changes{danish-1.3h}{1995/07/02}{Added loading of configuration
%    file}
%    \begin{macrocode}
\loadlocalcfg{danish}
%    \end{macrocode}
%
%    Our last action is to make a note that the commands we have just
%    defined, will be executed by calling the macro |\selectlanguage|
%    at the beginning of the document.
%    \begin{macrocode}
\main@language{irish}
%    \end{macrocode}
%    Finally, the category code of \texttt{@} is reset to its original
%    value. The macrospace used by |\atcatcode| is freed.
%    \begin{macrocode}
\catcode`\@=\atcatcode \let\atcatcode\relax
%</code>
%    \end{macrocode}
%
% \Finale
%\endinput
%% \CharacterTable
%%  {Upper-case    \A\B\C\D\E\F\G\H\I\J\K\L\M\N\O\P\Q\R\S\T\U\V\W\X\Y\Z
%%   Lower-case    \a\b\c\d\e\f\g\h\i\j\k\l\m\n\o\p\q\r\s\t\u\v\w\x\y\z
%%   Digits        \0\1\2\3\4\5\6\7\8\9
%%   Exclamation   \!     Double quote  \"     Hash (number) \#
%%   Dollar        \$     Percent       \%     Ampersand     \&
%%   Acute accent  \'     Left paren    \(     Right paren   \)
%%   Asterisk      \*     Plus          \+     Comma         \,
%%   Minus         \-     Point         \.     Solidus       \/
%%   Colon         \:     Semicolon     \;     Less than     \<
%%   Equals        \=     Greater than  \>     Question mark \?
%%   Commercial at \@     Left bracket  \[     Backslash     \\
%%   Right bracket \]     Circumflex    \^     Underscore    \_
%%   Grave accent  \`     Left brace    \{     Vertical bar  \|
%%   Right brace   \}     Tilde         \~}
%%
}
\DeclareOption{italian}{% \iffalse meta-comment
%
% Copyright 1989-1995 Johannes L. Braams and any individual authors
% listed elsewhere in this file.  All rights reserved.
% 
% For further copyright information any other copyright notices in this
% file.
% 
% This file is part of the Babel system release 3.5.
% --------------------------------------------------
%   This system is distributed in the hope that it will be useful,
%   but WITHOUT ANY WARRANTY; without even the implied warranty of
%   MERCHANTABILITY or FITNESS FOR A PARTICULAR PURPOSE.
% 
%   For error reports concerning UNCHANGED versions of this file no more
%   than one year old, see bugs.txt.
% 
%   Please do not request updates from me directly.  Primary
%   distribution is through the CTAN archives.
% 
% 
% IMPORTANT COPYRIGHT NOTICE:
% 
% You are NOT ALLOWED to distribute this file alone.
% 
% You are allowed to distribute this file under the condition that it is
% distributed together with all the files listed in manifest.txt.
% 
% If you receive only some of these files from someone, complain!
% 
% Permission is granted to copy this file to another file with a clearly
% different name and to customize the declarations in that copy to serve
% the needs of your installation, provided that you comply with
% the conditions in the file legal.txt from the LaTeX2e distribution.
% 
% However, NO PERMISSION is granted to produce or to distribute a
% modified version of this file under its original name.
%  
% You are NOT ALLOWED to change this file.
% 
% 
% \fi
% \CheckSum{127}
% \iffalse
%    Tell the \LaTeX\ system who we are and write an entry on the
%    transcript.
%<*dtx>
\ProvidesFile{italian.dtx}
%</dtx>
%<code>\ProvidesFile{italian.ldf}
        [1995/07/10 v1.2g Italian support from the babel system]
%    \end{macrocode}
%
% Babel package for LaTeX version 2e
% Copyright (C) 1989 - 1995
%           by Johannes Braams, TeXniek
%
% Please report errors to: J.L. Braams
%                          JLBraams@cistron.nl
%
%    This file is part of the babel system, it provides the source
%    code for the Italian language definition file.
%    The original version of this file was written by Maurizio
%    Codogno, (urcm@ur785.cselt.stet.it).
%<*filedriver>
\documentclass{ltxdoc}
\newcommand*\TeXhax{\TeX hax}
\newcommand*\babel{\textsf{babel}}
\newcommand*\langvar{$\langle \it lang \rangle$}
\newcommand*\note[1]{}
\newcommand*\Lopt[1]{\textsf{#1}}
\newcommand*\file[1]{\texttt{#1}}
\begin{document}
 \DocInput{italian.dtx}
\end{document}
%</filedriver>
%\fi
% \GetFileInfo{italian.dtx}
%
% \changes{italian-0.99}{1990/07/11}{First version, from english.doc}
% \changes{italian-1.0}{1991/04/23}{Modified for babel 3.0}
% \changes{italian-1.0a}{1991/05/23}{removed typo}
% \changes{italian-1.0b}{1991/05/29}{Removed bug found by van der Meer}
% \changes{italian-1.0e}{1991/07/15}{Renamed \file{babel.sty} in
%    \file{babel.com}}
% \changes{italian-1.1}{1992/02/16}{Brought up-to-date with babel 3.2a}
% \changes{italian-1.2}{1994/02/09}{Update for\ LaTeXe}
% \changes{italian-1.2e}{1994/06/26}{Removed the use of \cs{filedate}
%    and moved identification after the loading of \file{babel.def}}
% \changes{italian-1.2f}{1995/05/28}{Updated for babel 3.5}
%
%  \section{The Italian language}
%
%    The file \file{\filename}\footnote{The file described in this
%    section has version number \fileversion\ and was last revised on
%    \filedate. The original author is Maurizio Codogno,
%    (\texttt{urcm@ur785.cselt.stet.it}).}  It defines all the
%    language-specific macros for the Italian language.
%
%    For this language the |\clubpenalty|, |\widowpenalty| and
%    |\finalhyphendemerits| are set to rather high values.
%
% \StopEventually{}
%
%    As this file needs to be read only once, we check whether it was
%    read before. If it was, the command |\captionsitalian| is already
%    defined, so we can stop processing. If this command is undefined
%    we proceed with the various definitions and first show the
%    current version of this file.
%
% \changes{italian-1.0e}{1991/07/15}{Added reset of catcode of @
%    before \cs{endinput}.}
% \changes{italian-1.0h}{1991/10/08}{Removed use of \cs{@ifundefined}}
%    \begin{macrocode}
%<*code>
\ifx\undefined\captionsitalian
\else
  \selectlanguage{italian}
  \expandafter\endinput
\fi
%    \end{macrocode}
%
% \changes{italian-1.0h}{1991/10/07}{Removed code to load
%    \file{latexhax.com}}
%
% \begin{macro}{\atcatcode}
%    This file, \file{italian.sty}, may have been read while \TeX\ is
%    in the middle of processing a document, so we have to make sure
%    the category code of \texttt{@} is `letter' while this file is
%    being read. We save the category code of the @-sign in
%    |\atcatcode| and make it `letter'. Later the category code can be
%    restored to whatever it was before.
%
% \changes{italian-1.0c}{1991/06/06}{Made test of catcode of @ more
%    robust}
% \changes{italian-1.0e}{1991/07/15}{Modified handling of catcode of @
%    again.}
% \changes{italian-1.0f}{1991/08/29}{fixed typo, missing right brace}
% \changes{italian-1.0h}{1991/10/07}{Removed use of \cs{ makeatletter}
%    and hence the need to load \file{latexhax.com}}
%    \begin{macrocode}
\chardef\atcatcode=\catcode`\@
\catcode`\@=11\relax
%    \end{macrocode}
% \end{macro}
%
%    Now we determine whether the the common macros from the file
%    \file{babel.def} need to be read. We can be in one of two
%    situations: either another language option has been read earlier
%    on, in which case that other option has already read
%    \file{babel.def}, or \texttt{italian} is the first language
%    option to be processed. In that case we need to read
%    \file{babel.def} right here before we continue.
%
% \changes{italian-1.0}{1991/04/23}{New check before loading
%    \file{babel.com}}
% \changes{italian-1.1}{1992/02/16}{Added \cs{relax} after the
%    argument of \cs{input}}
%    \begin{macrocode}
\ifx\undefined\babel@core@loaded\input babel.def\relax\fi
%    \end{macrocode}
%
% \changes{italian-1.0b}{1991/05/29}{Add a check for existence
%    \cs{originalTeX}}
%
%    Another check that has to be made, is if another language
%    definition file has been read already. In that case its
%    definitions have been activated. This might interfere with
%    definitions this file tries to make. Therefore we make sure that
%    we cancel any special definitions. This can be done by checking
%    the existence of the macro |\originalTeX|. If it exists we simply
%    execute it, otherwise it is |\let| to |\empty|.
% \changes{italian-1.0e}{1991/07/15}{Added
%    \cs{let}\cs{originalTeX}\cs{relax} to test for existence}
% \changes{italian-1.1}{1992/02/16}{\cs{originalTeX} should be
%    expandable, \cs{let} it to \cs{empty}}
%    \begin{macrocode}
\ifx\undefined\originalTeX \let\originalTeX\empty \fi
\originalTeX
%    \end{macrocode}
%
%    When this file is read as an option, i.e. by the |\usepackage|
%    command, \texttt{italian} will be an `unknown' language in which
%    case we have to make it known.  So we check for the existence of
%    |\l@italian| to see whether we have to do something here.
%
% \changes{italian-1.0}{1991/04/23}{Now use \cs{adddialect} if
%    language undefined}
% \changes{italian-1.0h}{1991/10/08}{Removed use of \cs{@ifundefined}}
% \changes{italian-1.1}{1992/02/16}{Added a warning when no
%    hyphenation patterns were loaded.}
% \changes{italian-1.2e}{1994/06/26}{Now use \cs{@nopatterns} to
%    produce the warning}
%    \begin{macrocode}
\ifx\undefined\l@italian
    \@nopatterns{Italian}
    \adddialect\l@italian0\fi
%    \end{macrocode}
%
%    The next step consists of defining commands to switch to (and
%    from) the Italian language.
%
% \begin{macro}{\captionsitalian}
%    The macro |\captionsitalian| defines all strings used
%    in the four standard documentclasses provided with \LaTeX.
% \changes{italian-1.0c}{1991/06/06}{Removed \cs{global} definitions}
% \changes{italian-1.0c}{1991/06/06}{\cs{pagename} should be
%    \cs{headpagename}}
% \changes{italian-1.0d}{1991/07/01}{`contine' substitued by `Allegati'
%    as suggested by Marco Bozzo (\texttt{BOZZO@CERNVM}).}
% \changes{italian-1.1}{1992/02/16}{Added \cs{seename}, \cs{alsoname}
%    and \cs{prefacename}}
% \changes{italian-1.1}{1993/07/15}{\cs{headpagename} should be
%    \cs{pagename}}
% \changes{italian-1.2b}{1994/05/19}{Changed some of the words
%    following suggestions from Claudio Beccari}
% \changes{italian-1.2g}{1995/07/04}{Added \cs{proofname} for
%    AMS-\LaTeX}
%    \begin{macrocode}
\addto\captionsitalian{%
  \def\prefacename{Prefazione}%
  \def\refname{Riferimenti bibliografici}%
  \def\abstractname{Sommario}%
  \def\bibname{Bibliografia}%
  \def\chaptername{Capitolo}%
  \def\appendixname{Appendice}%
  \def\contentsname{Indice}%
  \def\listfigurename{Elenco delle figure}%
  \def\listtablename{Elenco delle tabelle}%
  \def\indexname{Indice analitico}%
  \def\figurename{Figura}%
  \def\tablename{Tabella}%
  \def\partname{Parte}%
  \def\enclname{Allegati}%
  \def\ccname{e~p.~c.}%
  \def\headtoname{Per}%
  \def\pagename{Pag.}%    % in Italian abbreviation is preferred
  \def\seename{vedi}%
  \def\alsoname{vedi anche}%
  \def\proofname{Proof}%  <-- needs translation
  }
%    \end{macrocode}
% \end{macro}
%
% \begin{macro}{\dateitalian}
%    The macro |\dateitalian| redefines the command
%    |\today| to produce Italian dates.
% \changes{italian-1.0c}{1991/06/06}{Removed \cs{global} definitions}
%    \begin{macrocode}
\def\dateitalian{%
\def\today{\number\day~\ifcase\month\or
  gennaio\or febbraio\or marzo\or aprile\or maggio\or giugno\or
  luglio\or agosto\or settembre\or ottobre\or novembre\or dicembre\fi
  \space \number\year}}
%    \end{macrocode}
% \end{macro}
%
% \begin{macro}{\italianhyphenmins}
% \changes{italian-1.2b}{1994/05/19}{Added setting of left and
%    righthyphenmin according to Claudio Beccari's suggestion}
%
%    The italian hyphenation patterns can be used with both
%    |\lefthyphenmin| and |\righthyphenmin| set to~2.
%    \begin{macrocode}
\def\italianhyphenmins{\tw@\tw@}
%    \end{macrocode}
% \end{macro}
%
% \begin{macro}{\extrasitalian}
% \begin{macro}{\noextrasitalian}
%
% \changes{italian-1.2b}{1994/05/19}{Added setting of club- and
%    widowpenalty}
%    Lower the chance that clubs or widows occur.
%    \begin{macrocode}
\addto\extrasitalian{%
  \babel@savevariable\clubpenalty
  \babel@savevariable\widowpenalty
  \clubpenalty3000\widowpenalty3000}
%    \end{macrocode}
%
% \changes{italian-1.2b}{1994/05/19}{Added setting of
%    finalhyphendemerits}
%
%    Never ever break a word between the last two lines of a paragraph
%    in italian texts.
%    \begin{macrocode}
\addto\extrasitalian{%
  \babel@savevariable\finalhyphendemerits
  \finalhyphendemerits50000000}
%    \end{macrocode}
% \end{macro}
% \end{macro}
%
%    It is possible that a site might need to add some extra code to
%    the babel macros. To enable this we load a local configuration
%    file, \file{italian.cfg} if it is found on \TeX' search path.
% \changes{italian-1.2g}{1995/07/02}{Added loading of configuration
%    file}
%    \begin{macrocode}
\loadlocalcfg{italian}
%    \end{macrocode}
%
%    Our last action is to make a note that the commands we have just
%    defined, will be executed by calling the macro |\selectlanguage|
%    at the beginning of the document.
%    \begin{macrocode}
\main@language{italian}
%    \end{macrocode}
%    Finally, the category code of \texttt{@} is reset to its original
%    value. The macrospace used by |\atcatcode| is freed.
% \changes{italian-1.0e}{1991/07/15}{Modified handling of catcode of
%    @-sign.}
%    \begin{macrocode}
\catcode`\@=\atcatcode \let\atcatcode\relax
%</code>
%    \end{macrocode}
%
% \Finale
%%
%% \CharacterTable
%%  {Upper-case    \A\B\C\D\E\F\G\H\I\J\K\L\M\N\O\P\Q\R\S\T\U\V\W\X\Y\Z
%%   Lower-case    \a\b\c\d\e\f\g\h\i\j\k\l\m\n\o\p\q\r\s\t\u\v\w\x\y\z
%%   Digits        \0\1\2\3\4\5\6\7\8\9
%%   Exclamation   \!     Double quote  \"     Hash (number) \#
%%   Dollar        \$     Percent       \%     Ampersand     \&
%%   Acute accent  \'     Left paren    \(     Right paren   \)
%%   Asterisk      \*     Plus          \+     Comma         \,
%%   Minus         \-     Point         \.     Solidus       \/
%%   Colon         \:     Semicolon     \;     Less than     \<
%%   Equals        \=     Greater than  \>     Question mark \?
%%   Commercial at \@     Left bracket  \[     Backslash     \\
%%   Right bracket \]     Circumflex    \^     Underscore    \_
%%   Grave accent  \`     Left brace    \{     Vertical bar  \|
%%   Right brace   \}     Tilde         \~}
%%
\endinput
}
\DeclareOption{lowersorbian}{% \iffalse meta-comment
%
% Copyright 1989-1995 Johannes L. Braams and any individual authors
% listed elsewhere in this file.  All rights reserved.
% 
% For further copyright information any other copyright notices in this
% file.
% 
% This file is part of the Babel system release 3.5.
% --------------------------------------------------
%   This system is distributed in the hope that it will be useful,
%   but WITHOUT ANY WARRANTY; without even the implied warranty of
%   MERCHANTABILITY or FITNESS FOR A PARTICULAR PURPOSE.
% 
%   For error reports concerning UNCHANGED versions of this file no more
%   than one year old, see bugs.txt.
% 
%   Please do not request updates from me directly.  Primary
%   distribution is through the CTAN archives.
% 
% 
% IMPORTANT COPYRIGHT NOTICE:
% 
% You are NOT ALLOWED to distribute this file alone.
% 
% You are allowed to distribute this file under the condition that it is
% distributed together with all the files listed in manifest.txt.
% 
% If you receive only some of these files from someone, complain!
% 
% Permission is granted to copy this file to another file with a clearly
% different name and to customize the declarations in that copy to serve
% the needs of your installation, provided that you comply with
% the conditions in the file legal.txt from the LaTeX2e distribution.
% 
% However, NO PERMISSION is granted to produce or to distribute a
% modified version of this file under its original name.
%  
% You are NOT ALLOWED to change this file.
% 
% 
% \fi
% \CheckSum{162}
% \iffalse
%
%    Tell the \LaTeX\ system who we are and write an entry on the
%    transcript.
%<*dtx>
\ProvidesFile{lsorbian.dtx}
%</dtx>
%<code>\ProvidesFile{lsorbian.ldf}
        [1995/07/04 v1.0b Lower Sorbian support from the babel system]
%
% Babel package for LaTeX version 2e
% Copyright (C) 1989 - 1995
%           by Johannes Braams, TeXniek
%
% Lower Sorbian Language Definition File
% Copyright (C) 1994 - 1995
%           by Eduard Werner
%           Werner, Eduard",
%           Serbski institut z. t.,
%           Dw\'orni\v{s}\'cowa 6
%           02625 Budy\v{s}in/Bautzen
%           Germany",
%           (??)3591 497223",
%           edi@kaihh.hanse.de",
%
% Please report errors to: Eduard Werner <edi@kaihh.hanse.de>
%
%    This file is part of the babel system, it provides the source
%    code for the Lower Sorbian definition file.
%<*filedriver>
\documentclass{ltxdoc}
\newcommand*\TeXhax{\TeX hax}
\newcommand*\babel{\textsf{babel}}
\newcommand*\langvar{$\langle \it lang \rangle$}
\newcommand*\note[1]{}
\newcommand*\Lopt[1]{\textsf{#1}}
\newcommand*\file[1]{\texttt{#1}}
\begin{document}
 \DocInput{lsorbian.dtx}
\end{document}
%</filedriver>
%\fi
%
% \GetFileInfo{lsorbian.dtx}
%
% \changes{Lsorbian-0.1}{1994/10/10}{First version}
%
%  \section{The Lower Sorbian language}
%
%    The file \file{\filename}\footnote{The file described in this
%    section has version number \fileversion\ and was last revised on
%    \filedate.  It was written by Eduard Werner
%    (\texttt{edi@kaihh.hanse.de}).}  It defines all the
%    language-specific macros for Lower Sorbian.
%
% \StopEventually{}
%
%    As this file needs to be read only once, we check whether it was
%    read before. If it was, the command |\captionlsorbian| is already
%    defined, so we can stop processing. If this command is undefined
%    we proceed with the various definitions and first show the
%    current version of this file.
%
%    \begin{macrocode}
%<*code>
\ifx\undefined\captionslsorbian
\else
  \selectlanguage{lsorbian}
  \expandafter\endinput
\fi
%    \end{macrocode}
%
%  \begin{macro}{\atcatcode}
%    This file, \file{lsorbian.ldf}, may have been read while \TeX\ is
%    in the middle of processing a document, so we have to make sure
%    the category code of \texttt{@} is `letter' while this file is
%    being read.  We save the category code of the @-sign in
%    |\atcatcode| and make it `letter'. Later the category code can be
%    restored to whatever it was before.
%
%    \begin{macrocode}
\chardef\atcatcode=\catcode`\@
\catcode`\@=11\relax
%    \end{macrocode}
%  \end{macro}
%
%    Now we determine whether the the common macros from the file
%    \file{babel.def} need to be read. We can be in one of two
%    situations: either another language option has been read earlier
%    on, in which case that other option has already read
%    \file{babel.def}, or \texttt{lsorbian} is the first language
%    option to be processed. In that case we need to read
%    \file{babel.def} right here before we continue.
%
%    \begin{macrocode}
\ifx\undefined\babel@core@loaded\input babel.def\relax\fi
%    \end{macrocode}
%
%    Another check that has to be made, is if another language
%    definition file has been read already. In that case its
%    definitions have been activated. This might interfere with
%    definitions this file tries to make. Therefore we make sure that
%    we cancel any special definitions. This can be done by checking
%    the existence of the macro |\originalTeX|. If it exists we simply
%    execute it, otherwise it is |\let| to |\empty|.
%
%    \begin{macrocode}
\ifx\undefined\originalTeX \let\originalTeX\empty \else\originalTeX\fi
%    \end{macrocode}
%
%    When this file is read as an option, i.e. by the |\usepackage|
%    command, \texttt{lsorbian} will be an `unknown' languagein which
%    case we have to make it known. So we check for the existence of
%    |\l@lsorbian| to see whether we have to do something here.
%
%    \begin{macrocode}
\ifx\undefined\l@lsorbian
    \@nopatterns{Lsorbian}
    \adddialect\l@lsorbian\l@usorbian\fi
%    \end{macrocode}
%
%    The next step consists of defining commands to switch to (and
%    from) the Lower Sorbian language.
%
%  \begin{macro}{\captionslsorbian}
%    The macro |\captionslsorbian| defines all strings used in the four
%    standard documentclasses provided with \LaTeX.
% \changes{lsorbian-1.0b}{1995/07/04}{Added \cs{proofname} for
%    AMS-\LaTeX}
%    \begin{macrocode}
\addto\captionslsorbian{%
  \def\prefacename{Zawod}%
  \def\refname{Referency}%
  \def\abstractname{Abstrakt}%
  \def\bibname{Literatura}%
  \def\chaptername{Kapitl}%
  \def\appendixname{Dodawki}%
  \def\contentsname{Wop\'simje\'se}%
  \def\listfigurename{Zapis wobrazow}%
  \def\listtablename{Zapis tabulkow}%
  \def\indexname{Indeks}%
  \def\figurename{Wobraz}%
  \def\tablename{Tabulka}%
  \def\partname{\'Z\v el}%
  \def\enclname{P\'si\l oga}%
  \def\ccname{CC}%
  \def\headtoname{Komu}%
  \def\pagename{Strona}%
  \def\seename{gl.}%
  \def\alsoname{gl.~teke}%
  \def\proofname{Proof}%  <-- needs translation
  }%
%    \end{macrocode}
%  \end{macro}
%
%  \begin{macro}{\newdatelsorbian}
%    The macro |\newdatelsorbian| redefines the command |\today| to
%    produce Lower Sorbian dates.
%    \begin{macrocode}
\def\newdatelsorbian{%
\def\today{\number\day.~\ifcase\month\or
januara\or februara\or m\v erca\or apryla\or maja\or junija\or
  julija\or awgusta\or septembra\or oktobra\or
  nowembra\or decembra\fi
    \space \number\year}}
%    \end{macrocode}
%  \end{macro}
%
%  \begin{macro}{\olddatelsorbian}
%    The macro |\olddatelsorbian| redefines the command |\today| to
%    produce old-style Lower Sorbian dates.
%    \begin{macrocode}
\def\olddatelsorbian{%
  \def\today{\number\day.~\ifcase\month\or
    wjelikego ro\v zka\or
    ma\l ego ro\v zka\or
    nal\v etnika\or
    jat\v sownika\or
    ro\v zownika\or
    sma\v znika\or
    pra\v znika\or
    \v znje\'nca\or
    po\v znje\'nca\or
    winowca\or
    nazymnika\or 
    godownika\fi \space \number\year}}
%    \end{macrocode}
%  \end{macro}
%
%    The default will be the new-style dates.
%    \begin{macrocode}
\let\datelsorbian\newdatelsorbian
%    \end{macrocode}
%
% \begin{macro}{\extraslsorbian}
% \begin{macro}{\noextraslsorbian}
%    The macro |\extraslsorbian| will perform all the extra
%    definitions needed for the lsorbian language. The macro
%    |\noextraslsorbian| is used to cancel the actions of
%    |\extraslsorbian|.  For the moment these macros are empty but
%    they are defined for compatibility with the other language
%    definition files.
%
%    \begin{macrocode}
\addto\extraslsorbian{}
\addto\noextraslsorbian{}
%    \end{macrocode}
% \end{macro}
% \end{macro}
%
%    It is possible that a site might need to add some extra code to
%    the babel macros. To enable this we load a local configuration
%    file, \file{lsorbian.cfg} if it is found on \TeX' search path.
% \changes{lsorbian-1.0b}{1995/07/02}{Added loading of configuration
%    file}
%    \begin{macrocode}
\loadlocalcfg{lsorbian}
%    \end{macrocode}
%
%    Our last action is to make a note that the commands we have just
%    defined, will be executed by calling the macro |\selectlanguage|
%    at the beginning of the document.
%    \begin{macrocode}
\main@language{lsorbian}
%    \end{macrocode}
%
%    Finally, the category code of \texttt{@} is reset to its original
%    value. The macrospace used by |\atcatcode| is freed.
%    \begin{macrocode}
\catcode`\@=\atcatcode \let\atcatcode\relax
%</code>
%    \end{macrocode}
%
% \Finale
%%
%% \CharacterTable
%%  {Upper-case    \A\B\C\D\E\F\G\H\I\J\K\L\M\N\O\P\Q\R\S\T\U\V\W\X\Y\Z
%%   Lower-case    \a\b\c\d\e\f\g\h\i\j\k\l\m\n\o\p\q\r\s\t\u\v\w\x\y\z
%%   Digits        \0\1\2\3\4\5\6\7\8\9
%%   Exclamation   \!     Double quote  \"     Hash (number) \#
%%   Dollar        \$     Percent       \%     Ampersand     \&
%%   Acute accent  \'     Left paren    \(     Right paren   \)
%%   Asterisk      \*     Plus          \+     Comma         \,
%%   Minus         \-     Point         \.     Solidus       \/
%%   Colon         \:     Semicolon     \;     Less than     \<
%%   Equals        \=     Greater than  \>     Question mark \?
%%   Commercial at \@     Left bracket  \[     Backslash     \\
%%   Right bracket \]     Circumflex    \^     Underscore    \_
%%   Grave accent  \`     Left brace    \{     Vertical bar  \|
%%   Right brace   \}     Tilde         \~}
%%
\endinput
}
\DeclareOption{magyar}{% \iffalse meta-comment
%
% Copyright 1989-1995 Johannes L. Braams and any individual authors
% listed elsewhere in this file.  All rights reserved.
% 
% For further copyright information any other copyright notices in this
% file.
% 
% This file is part of the Babel system release 3.5.
% --------------------------------------------------
%   This system is distributed in the hope that it will be useful,
%   but WITHOUT ANY WARRANTY; without even the implied warranty of
%   MERCHANTABILITY or FITNESS FOR A PARTICULAR PURPOSE.
% 
%   For error reports concerning UNCHANGED versions of this file no more
%   than one year old, see bugs.txt.
% 
%   Please do not request updates from me directly.  Primary
%   distribution is through the CTAN archives.
% 
% 
% IMPORTANT COPYRIGHT NOTICE:
% 
% You are NOT ALLOWED to distribute this file alone.
% 
% You are allowed to distribute this file under the condition that it is
% distributed together with all the files listed in manifest.txt.
% 
% If you receive only some of these files from someone, complain!
% 
% Permission is granted to copy this file to another file with a clearly
% different name and to customize the declarations in that copy to serve
% the needs of your installation, provided that you comply with
% the conditions in the file legal.txt from the LaTeX2e distribution.
% 
% However, NO PERMISSION is granted to produce or to distribute a
% modified version of this file under its original name.
%  
% You are NOT ALLOWED to change this file.
% 
% 
% \fi
% \CheckSum{282}
% \iffalse
%    Tell the \LaTeX\ system who we are and write an entry on the
%    transcript.
%<*dtx>
\ProvidesFile{magyar.dtx}
%</dtx>
%<code>\ProvidesFile{magyar.ldf}
        [1995/07/04 v1.3e Magyar support from the babel system]
%
% Babel package for LaTeX version 2e
% Copyright (C) 1989 - 1995
%           by Johannes Braams, TeXniek
%
% Magyar Language Definition File
% Copyright (C) 1989 - 1995
%           by Johannes Braams, TeXniek
%              \'Arp\'ad B\'IR\'O
%
% Please report errors to: J.L. Braams <JLBraams@cistron.nl>
%
%    This file is part of the babel system, it provides the source
%    code for the Hungarian language definition file.  A contribution
%    was made by Attila Koppanyi (attila@cernvm.cern.ch).
%<*filedriver>
\documentclass{ltxdoc}
\newcommand*\TeXhax{\TeX hax}
\newcommand*\babel{\textsf{babel}}
\newcommand*\langvar{$\langle \it lang \rangle$}
\newcommand*\note[1]{}
\newcommand*\Lopt[1]{\textsf{#1}}
\newcommand*\file[1]{\texttt{#1}}
\begin{document}
 \DocInput{magyar.dtx}
\end{document}
%</filedriver>
%\fi
% \GetFileInfo{magyar.dtx}
%
% \changes{magyar-1.0a}{1991/07/15}{Renamed \file{babel.sty} in
%    \file{babel.com}}
% \changes{magyar-1.1}{1992/02/16}{Brought up-to-date with babel 3.2a}
% \changes{magyar-1.1.4}{1994/02/08}{Further spelling corrections}
% \changes{magyar-1.1.5}{1994/02/09}{Still more spelling corrections}
% \changes{magyar-1.2}{1994/02/27}{Update for \LaTeXe}
% \changes{magyar-1.3c}{1994/06/26}{Removed the use of \cs{filedate}
%    and moved identification after the loading of \file{babel.def}}
%
%  \section{The Hungarian language}
%
%    The file option \file{\filename}\footnote{The file described in
%    this section has version number \fileversion\ and was last
%    revised on \filedate.  A contribution was made by Attila Koppanyi
%    (\texttt{attila@cernvm.cern.ch}). Later updates and suggestions
%    by \'Arp\'ad B\'ir\'o (\texttt{JZP1104@HUSZEG11.bitnet}), Istvan
%    Hamecz (\texttt{hami@ursus.bke.hu)} and Horvath Dezso
%    (\texttt{horvath@pisa.infn.it}).}  defines all the
%    language definition macros for the Hungarian language.
%
% \DescribeMacro\ontoday
%    For this language currently the only special definition that is
%    added is the |\ontoday| command which works like |\today| but
%    produces a slightly different date format used in expressions suh
%    as `on february 10th'.
%
% \StopEventually{}
%
%    As this file needs to be read only once, we check whether it was
%    read before. If it was, the command |\captionsmagyar| is already
%    defined, so we can stop processing. If this command is undefined
%    we proceed with the various definitions and first show the
%    current version of this file.
%
% \changes{magyar-1.0a}{1991/07/15}{Added reset of catcode of @ before
%    \cs{endinput}.}
% \changes{magyar-1.0b}{1991/10/29}{Removed use of \cs{@ifundefined}}
%    \begin{macrocode}
%<*code>
\ifx\undefined\captionsmagyar
\else
  \selectlanguage{magyar}
  \expandafter\endinput
\fi
%    \end{macrocode}
%
% \changes{magyar-1.0b}{1991/10/29}{Removed code to load
%    \file{latexhax.com}}
%
% \begin{macro}{\atcatcode}
%    This file, \file{magyar.ldf}, may have been read while \TeX\ is
%    in the middle of processing a document, so we have to make sure
%    the category code of \texttt{@} is `letter' while this file is
%    being read.  We save the category code of the @-sign in
%    |\atcatcode| and make it `letter'. Later the category code can be
%    restored to whatever it was before.
%
% \changes{magyar-1.0a}{1991/07/15}{Modified handling of catcode of @
%    again.}
% \changes{magyar-1.0b}{1991/10/29}{Removed use of \cs{makeatletter}
%    and hence the need to load \file{latexhax.com}}
%    \begin{macrocode}
\chardef\atcatcode=\catcode`\@
\catcode`\@=11\relax
%    \end{macrocode}
% \end{macro}
%
%    Now we determine whether the the common macros from the file
%    \file{babel.def} need to be read. We can be in one of two
%    situations: either another language option has been read earlier
%    on, in which case that other option has already read
%    \file{babel.def}, or \texttt{magyar} is the first language option
%    to be processed. In that case we need to read \file{babel.def}
%    right here before we continue.
%
% \changes{magyar-1.1}{1992/02/16}{Added \cs{relax} after the argument
%    of \cs{input}}
%    \begin{macrocode}
\ifx\undefined\babel@core@loaded\input babel.def\relax\fi
%    \end{macrocode}
%
%    Another check that has to be made, is if another language
%    definition file has been read already. In that case its definitions
%    have been activated. This might interfere with definitions this
%    file tries to make. Therefore we make sure that we cancel any
%    special definitions. This can be done by checking the existence
%    of the macro |\originalTeX|. If it exists we simply execute it,
%    otherwise it is |\let| to |\empty|.
% \changes{magyar-1.0a}{1991/07/15}{Added
%    \cs{let}\cs{originalTeX}\cs{relax} to test for existence}
% \changes{magyar-1.1}{1992/02/16}{\cs{originalTeX} should be
%    expandable, \cs{let} it to \cs{empty}}
%    \begin{macrocode}
\ifx\undefined\originalTeX \let\originalTeX\empty \else\originalTeX\fi
%    \end{macrocode}
%
%    When this file is read as an option, i.e. by the |\usepackage|
%    command, \texttt{magyar} will be an `unknown' language in which
%    case we have to make it known.  So we check for the existence of
%    |\l@magyar| to see whether we have to do something here.
%
% \changes{magyar-1.0b}{1991/10/29}{Removed use of \cs{@ifundefined}}
% \changes{magyar-1.1}{1992/02/16}{Added a warning when no hyphenation
%    patterns were loaded.}
% \changes{magyar-1.3c}{1994/06/26}{Now use \cs{@nopatterns} to
%    produce the warning}
%    \begin{macrocode}
\ifx\undefined\l@magyar
    \@nopatterns{Magyar}
    \adddialect\l@magyar0\fi
%    \end{macrocode}
%
%    An additional note about formatting Hungarian texts: One should
%    invert the order of the number and text in things like chapter
%    headings, page references etc. So one should write `I. r\'esz'
%    instead of `Part I', or `3. oldal' for `page 3'.
%
%    For chapter headings this could be accomplished by a redefinition
%    of the macros |\@makechapterhead| and |\@makeschapterhead|, for
%    other instances this a lot harder to accomplish. Therefore I
%    think complete document classes should be written to accomadate
%    the needed formatting.
%
%    The next step consists of defining commands to switch to (and
%    from) the Hungarian language.
%
% \begin{macro}{\captionsmagyar}
%    The macro |\captionsmagyar| defines all strings used in the four
%    standard documentclasses provided with \LaTeX.
% \changes{magyar-1.1}{1992/02/16}{Added \cs{seename}, \cs{alsoname}
%    and \cs{prefacename}}
% \changes{magyar-1.1}{1993/07/15}{\cs{headpagename} should be
%    \cs{pagename}}
% \changes{magyar-1.1.3}{1994/01/05}{Added translations, fixed typos}
% \changes{magyar-1.3e}{1995/07/04}{Added \cs{proofname} for
%    AMS-\LaTeX}
%    \begin{macrocode}
\addto\captionsmagyar{%
  \def\prefacename{El\H osz\'o}%
  \def\refname{Referenci\'ak}%
  \def\abstractname{Kivonat}%
  \def\bibname{Bibliogr\'afia}%
  \def\chaptername{fejezet}%
  \def\appendixname{f\"uggel\'ek}%
  \def\contentsname{Tartalom}%
  \def\listfigurename{\'Abr\'ak jegyz\'eke}%
  \def\listtablename{T\'abl\'azatok jegyz\'eke}%
  \def\indexname{T\'argymutat\'o}%
  \def\figurename{\'abra}%
  \def\tablename{t\'abl\'azat}%
  \def\partname{r\'esz}%
  \def\enclname{Mell\'eklet}%
  \def\ccname{K\"orlev\'el--c\'\i mzettek}%
  \def\headtoname{C\'\i mzett}%
  \def\pagename{oldal}%
  \def\seename{L\'asd}%
  \def\alsoname{L\'asd m\'eg}%
  \def\proofname{Proof}%   <-- needs translation
  }%
%    \end{macrocode}
% \end{macro}
%
% \begin{macro}{\datemagyar}
%    The macro |\datemagyar| redefines the command |\today| to produce
%    Hungarian dates.
% \changes{magyar-1.1.4}{1994/02/08}{Rewritten to produce the correct
%    date format}
%    \begin{macrocode}
\def\datemagyar{%
  \def\today{\number\year.~\ifcase\month\or
  janu\'ar\or febru\'ar\or m\'arcius\or
  \'aprilis\or m\'ajus\or j\'unius\or
  j\'ulius\or augusztus\or szeptember\or
  okt\'ober\or november\or december\fi
    \space\ifcase\day\or
    1.\or  2.\or  3.\or  4.\or  5.\or
    6.\or  7.\or  8.\or  9.\or 10.\or
   11.\or 12.\or 13.\or 14.\or 15.\or
   16.\or 17.\or 18.\or 19.\or 20.\or
   21.\or 22.\or 23.\or 24.\or 25.\or
   26.\or 27.\or 28.\or 29.\or 30.\or
   31.\fi}}
%    \end{macrocode}
% \end{macro}
%
% \begin{macro}{\ondatemagyar}
%    The macro |\ondatemagyar| produces Hungarian dates which have the
%    meaning `\emph{on this day}'.  It does not redefine the command
%    |\today|.
% \changes{magyar-1.1.3}{1994/01/05}{The date number should not be
%    followed by a dot.}
% \changes{magyar-1.1.4}{1994/02/08}{Renamed from \cs{datemagyar};
%    nolonger redefines \cs{today}.}
%    \begin{macrocode}
\def\ondatemagyar{%
  \number\year.~\ifcase\month\or
  janu\'ar\or febru\'ar\or m\'arcius\or
  \'aprilis\or m\'ajus\or j\'unius\or
  j\'ulius\or augusztus\or szeptember\or
  okt\'ober\or november\or december\fi
    \space\ifcase\day\or
    1-j\'en\or  2-\'an\or  3-\'an\or  4-\'en\or  5-\'en\or
    6-\'an\or  7-\'en\or  8-\'an\or  9-\'en\or 10-\'en\or
   11-\'en\or 12-\'en\or 13-\'an\or 14-\'en\or 15-\'en\or
   16-\'an\or 17-\'en\or 18-\'an\or 19-\'en\or 20-\'an\or
   21-\'en\or 22-\'en\or 23-\'an\or 24-\'en\or 25-\'en\or
   26-\'an\or 27-\'en\or 28-\'an\or 29-\'en\or 30-\'an\or
   31-\'en\fi}
%    \end{macrocode}
% \end{macro}
%
% \begin{macro}{\extrasmagyar}
% \begin{macro}{\noextrasmagyar}
%    The macro |\extrasmagyar| will perform all the extra definitions
%    needed for the Hungarian language. The macro |\noextrasmagyar| is
%    used to cancel the actions of |\extrasmagyar|.  For the moment
%    these macros are nearly empty; only the user command |\ontoday|
%    to access |\ondatemagyar| is defined.
%
%    \begin{macrocode}
\addto\extrasmagyar{\let\ontoday\ondatemagyar}
\addto\noextrasmagyar{\let\ontoday\undefined}
%    \end{macrocode}
% \end{macro}
% \end{macro}
%
%    It is possible that a site might need to add some extra code to
%    the babel macros. To enable this we load a local configuration
%    file, \file{magyar.cfg} if it is found on \TeX' search path.
% \changes{magyar-1.3e}{1995/07/02}{Added loading of configuration
%    file}
%    \begin{macrocode}
\loadlocalcfg{magyar}
%    \end{macrocode}
%
%    Our last action is to make a note that the commands we have just
%    defined, will be executed by calling the macro |\selectlanguage|
%    at the beginning of the document.
%    \begin{macrocode}
\main@language{magyar}
%    \end{macrocode}
%    Finally, the category code of \texttt{@} is reset to its original
%    value. The macrospace used by |\atcatcode| is freed.
% \changes{magyar-1.0a}{1991/07/15}{Modified handling of catcode of
%    @-sign.}
%    \begin{macrocode}
\catcode`\@=\atcatcode \let\atcatcode\relax
%</code>
%    \end{macrocode}
%
% \Finale
%%
%% \CharacterTable
%%  {Upper-case    \A\B\C\D\E\F\G\H\I\J\K\L\M\N\O\P\Q\R\S\T\U\V\W\X\Y\Z
%%   Lower-case    \a\b\c\d\e\f\g\h\i\j\k\l\m\n\o\p\q\r\s\t\u\v\w\x\y\z
%%   Digits        \0\1\2\3\4\5\6\7\8\9
%%   Exclamation   \!     Double quote  \"     Hash (number) \#
%%   Dollar        \$     Percent       \%     Ampersand     \&
%%   Acute accent  \'     Left paren    \(     Right paren   \)
%%   Asterisk      \*     Plus          \+     Comma         \,
%%   Minus         \-     Point         \.     Solidus       \/
%%   Colon         \:     Semicolon     \;     Less than     \<
%%   Equals        \=     Greater than  \>     Question mark \?
%%   Commercial at \@     Left bracket  \[     Backslash     \\
%%   Right bracket \]     Circumflex    \^     Underscore    \_
%%   Grave accent  \`     Left brace    \{     Vertical bar  \|
%%   Right brace   \}     Tilde         \~}
%%
\endinput
}
\DeclareOption{norsk}{% \iffalse meta-comment
%
% Copyright 1989-1995 Johannes L. Braams and any individual authors
% listed elsewhere in this file.  All rights reserved.
% 
% For further copyright information any other copyright notices in this
% file.
% 
% This file is part of the Babel system release 3.5.
% --------------------------------------------------
%   This system is distributed in the hope that it will be useful,
%   but WITHOUT ANY WARRANTY; without even the implied warranty of
%   MERCHANTABILITY or FITNESS FOR A PARTICULAR PURPOSE.
% 
%   For error reports concerning UNCHANGED versions of this file no more
%   than one year old, see bugs.txt.
% 
%   Please do not request updates from me directly.  Primary
%   distribution is through the CTAN archives.
% 
% 
% IMPORTANT COPYRIGHT NOTICE:
% 
% You are NOT ALLOWED to distribute this file alone.
% 
% You are allowed to distribute this file under the condition that it is
% distributed together with all the files listed in manifest.txt.
% 
% If you receive only some of these files from someone, complain!
% 
% Permission is granted to copy this file to another file with a clearly
% different name and to customize the declarations in that copy to serve
% the needs of your installation, provided that you comply with
% the conditions in the file legal.txt from the LaTeX2e distribution.
% 
% However, NO PERMISSION is granted to produce or to distribute a
% modified version of this file under its original name.
%  
% You are NOT ALLOWED to change this file.
% 
% 
% \fi
%\CheckSum{182}
% \iffalse
%    Tell the \LaTeX\ system who we are and write an entry on the
%    transcript.
%<*dtx>
\ProvidesFile{norsk.dtx}
%</dtx>
%<code>\ProvidesFile{norsk.ldf}
        [1995/07/02 v1.2f Norsk support from the babel system]
%
% Babel package for LaTeX version 2e
% Copyright (C) 1989 - 1995
%           by Johannes Braams, TeXniek
%
% Please report errors to: J.L. Braams
%                          JLBraams@cistron.nl
%
%    This file is part of the babel system, it provides the source
%    code for the Norwegian language definition file.  Contributions
%    were made by Haavard Helstrup (HAAVARD@CERNVM) and Alv Kjetil
%    Holme (HOLMEA@CERNVM); the `nynorsk' variant has been supplied by
%    Per Steinar Iversen (iversen@vxcern.cern.ch) and Terje Engeset
%    Petterst (TERJEEP@VSFYS1.FI.UIB.NO)
%<*filedriver>
\documentclass{ltxdoc}
\newcommand*\TeXhax{\TeX hax}
\newcommand*\babel{\textsf{babel}}
\newcommand*\langvar{$\langle \it lang \rangle$}
\newcommand*\note[1]{}
\newcommand*\Lopt[1]{\textsf{#1}}
\newcommand*\file[1]{\texttt{#1}}
\begin{document}
 \DocInput{norsk.dtx}
\end{document}
%</filedriver>
%\fi
% \GetFileInfo{norsk.dtx}
%
% \changes{norsk-1.0a}{1991/07/15}{Renamed \file{babel.sty} in
%    \file{babel.com}}
% \changes{norsk-1.1}{1992/02/16}{Brought up-to-date with babel 3.2a}
% \changes{norsk-1.1.3}{1993/11/11}{Added a couple of translations
%    (from Per Norman Oma, TeX@itk.unit.no)}
% \changes{norsk-1.2}{1994/02/27}{Update for \LaTeXe}
% \changes{norsk-1.2d}{1994/06/26}{Removed the use of \cs{filedate}
%    and moved identification after the loading of \file{babel.def}}
%
%  \section{The Norwegian language}
%
%    The file \file{\filename}\footnote{The file described in this
%    section has version number \fileversion\ and was last revised on
%    \filedate.  Contributions were made by Haavard Helstrup
%    (\texttt{HAAVARD@CERNVM}) and Alv Kjetil Holme
%    (\texttt{HOLMEA@CERNVM}); the `nynorsk' variant has been supplied
%    by Per Steinar Iversen \texttt{iversen@vxcern.cern.ch}) and Terje
%    Engeset Petterst (\texttt{TERJEEP@VSFYS1.FI.UIB.NO)}.}  defines
%    all the language definition macros for the Norwegian language as
%    well as for a new spelling variant `nynorsk' for this language.
%
%    For this language currently no special definitions are needed or
%    available.
%
% \StopEventually{}
%
%    As this file needs to be read only once, we check whether it was
%    read before. If it was, the command |\captionsnorsk| is already
%    defined, so we can stop processing. If this command is undefined
%    we proceed with the various definitions and first show the
%    current version of this file.
%
% \changes{norsk-1.0a}{1991/07/15}{Added reset of catcode of @ before
%    \cs{endinput}.}
% \changes{norsk-1.0c}{1991/10/29}{Removed use of \cs{@ifundefined}}
%    \begin{macrocode}
%<*code>
\ifx\undefined\captionsnorsk
\else
  \selectlanguage{norsk}
  \expandafter\endinput
\fi
%    \end{macrocode}
%
% \begin{macro}{\atcatcode}
%    This file, \file{norsk.sty}, may have been read while \TeX\ is in
%    the middle of processing a document, so we have to make sure the
%    category code of \texttt{@} is `letter' while this file is being
%    read.  We save the category code of the @-sign in |\atcatcode|
%    and make it `letter'. Later the category code can be restored to
%    whatever it was before.
%
% \changes{norsk-1.0a}{1991/07/15}{Modified handling of catcode of @
%    again.}
% \changes{norsk-1.0c}{1991/10/29}{Removed use of \cs{makeatletter} and
%    hence the need to load \file{latexhax.com}}
%    \begin{macrocode}
\chardef\atcatcode=\catcode`\@
\catcode`\@=11\relax
%    \end{macrocode}
% \end{macro}
%
%    Now we determine whether the the common macros from the file
%    \file{babel.def} need to be read. We can be in one of two
%    situations: either another language option has been read earlier
%    on, in which case that other option has already read
%    \file{babel.def}, or \texttt{norsk} is the first language option
%    to be processed. In that case we need to read \file{babel.def}
%    right here before we continue.
%
% \changes{norsk-1.1}{1992/02/16}{Added \cs{relax} after the
%    argument of \cs{input}}
%    \begin{macrocode}
\ifx\undefined\babel@core@loaded\input babel.def\relax\fi
%    \end{macrocode}
%
% \changes{norsk-1.0c}{1991/10/29}{Removed code to load
%    \file{latexhax.com}}
%
%    Another check that has to be made, is if another language
%    definition file has been read already. In that case its
%    definitions have been activated. This might interfere with
%    definitions this file tries to make. Therefore we make sure that
%    we cancel any special definitions. This can be done by checking
%    the existence of the macro |\originalTeX|. If it exists we simply
%    execute it, otherwise it is |\let| to |\empty|.
% \changes{norsk-1.0a}{1991/07/15}{Added
%    \cs{let}\cs{originalTeX}\cs{relax} to test for existence}
% \changes{norsk-1.1}{1992/02/16}{\cs{originalTeX} should be
%    expandable, \cs{let} it to \cs{empty}}
%    \begin{macrocode}
\ifx\undefined\originalTeX \let\originalTeX\empty \else\originalTeX\fi
%    \end{macrocode}
%
%    When this file is read as an option, i.e. by the |\usepackage|
%    command, \texttt{norsk} will be an `unknown' language in which
%    case we have to make it known.  So we check for the existence of
%    |\l@norsk| to see whether we have to do something here.
%
% \changes{norsk-1.0c}{1991/10/29}{Removed use of \cs{@ifundefined}}
% \changes{norsk-1.1}{1992/02/16}{Added a warning when no hyphenation
%    patterns were loaded.}
% \changes{norsk-1.2d}{1994/06/26}{Now use \cs{@nopatterns} to produce
%    the warning}
%    \begin{macrocode}
\ifx\undefined\l@norsk
    \@nopatterns{Norsk}
    \adddialect\l@norsk0\fi
%    \end{macrocode}
%
%    For the `nynorsk' version of these definitions we just add a
%    ``dialect''. Also, the macros |\datenynorsk| and |\extrasnynorsk|
%    are |\let| to their `norsk' counterparts when these parts are
%    defined.
%    \begin{macrocode}
\adddialect\l@nynorsk\l@norsk
%    \end{macrocode}
%
%  \begin{macro}{\norskhyphenmins}
%  \begin{macro}{\nynorskhyphenmins}
%    The Norwegian hyphenation patterns can be used with
%    |\lefthyphenmin| set to~1 and |\righthyphenmin| set to~2. This is
%    true for both `versions' of the language.
% \changes{norsk-1.2f}{1995/07/02}{Added setting of hyphenmin
%    parameters}
%    \begin{macrocode}
\def\norskhyphenmins{\@ne\tw@}
\let\nynorskhyphenmins\norskhyphenmins
%    \end{macrocode}
%  \end{macro}
%  \end{macro}
%
%    The next step consists of defining commands to switch to (and
%    from) the Norwegian language.
%
% \begin{macro}{\captionsnorsk}
%    The macro |\captionsnorsk| defines all strings used
%    in the four standard documentclasses provided with \LaTeX.
% \changes{norsk-1.1}{1992/02/16}{Added \cs{seename}, \cs{alsoname} and
%    \cs{prefacename}}
% \changes{norsk-1.1}{1993/07/15}{\cs{headpagename} should be
%    \cs{pagename}}
% \changes{norsk-1.2f}{1995/07/02}{Added \cs{proofname} for
%    AMS-\LaTeX}
%    \begin{macrocode}
\addto\captionsnorsk{%
  \def\prefacename{Forord}%
  \def\refname{Referanser}%
  \def\abstractname{Sammendrag}%
  \def\bibname{Bibliografi}%      or Litteraturoversikt
%               or Litteratur or Referanser
  \def\chaptername{Kapittel}%
  \def\appendixname{Tillegg}%    or Appendiks
  \def\contentsname{Innhold}%
  \def\listfigurename{Figurer}%  or Figurliste
  \def\listtablename{Tabeller}%  or Tabelliste
  \def\indexname{Register}%
  \def\figurename{Figur}%
  \def\tablename{Tabell}%
  \def\partname{Del}%
  \def\enclname{Vedlegg}%
  \def\ccname{Kopi sendt}%
  \def\headtoname{Til}% in letter
  \def\pagename{Side}%
  \def\seename{Se}%
  \def\alsoname{Se ogs\aa{}}%
  \def\proofname{Proof}%
  }
%    \end{macrocode}
% \end{macro}
%
% \begin{macro}{\captionsnynorsk}
%    The macro |\captionsnynorsk| defines all strings used in the four
%    standard documentclasses provided with \LaTeX, but using a
%    different spelling than in the command |\captionsnorsk|.
% \changes{norsk-1.1}{1992/02/16}{Added \cs{seename}, \cs{alsoname} and
%    \cs{prefacename}}
% \changes{norsk-1.1}{1993/07/15}{\cs{headpagename} should be
%    \cs{pagename}}
%    \begin{macrocode}
\addto\captionsnynorsk{%
  \def\prefacename{Forord}%
  \def\refname{Referansar}%
  \def\abstractname{Samandrag}%
  \def\bibname{Litteratur}%     or Litteraturoversyn
                         %    or Referansar
  \def\chaptername{Kapittel}%
  \def\appendixname{Tillegg}%   or Appendiks
  \def\contentsname{Innhald}%
  \def\listfigurename{Figurar}% or Figurliste
  \def\listtablename{Tabellar}% or Tabelliste
  \def\indexname{Register}%
  \def\figurename{Figur}%
  \def\tablename{Tabell}%
  \def\partname{Del}%
  \def\enclname{Vedlegg}%
  \def\ccname{Kopi sendt}%
  \def\headtoname{Til}% in letter
  \def\pagename{Side}%
  \def\seename{Sj\aa{}}%
  \def\alsoname{Sj\aa{} ogs\aa{}}%
  \def\proofname{Proof}%
  }
%    \end{macrocode}
% \end{macro}
%
% \begin{macro}{\datenorsk}
%    The macro |\datenorsk| redefines the command |\today| to produce
%    Norwegian dates.
%    \begin{macrocode}
\def\datenorsk{%
\def\today{\number\day.~\ifcase\month\or
  januar\or februar\or mars\or april\or mai\or juni\or
  juli\or august\or september\or oktober\or november\or desember\fi
  \space\number\year}}
%    \end{macrocode}
% \end{macro}
%
% \begin{macro}{\datenynorsk}
%    The spelling of the names of the months is the same for both
%    versions of the ``Norsk'' language, so we simply |\let| the macro
%    |\datenynorsk| be equal to |\datenorsk|
%    \begin{macrocode}
\let\datenynorsk\datenorsk
%    \end{macrocode}
% \end{macro}
%
% \begin{macro}{\extrasnorsk}
% \begin{macro}{\extrasnynorsk}
%    The macro |\extrasnorsk| will perform all the extra definitions
%    needed for the Norwegian language. The macro |\noextrasnorsk| is
%    used to cancel the actions of |\extrasnorsk|.  
%
%    Norwegian typesetting requires |\frencspacing| to be in effect.
%    \begin{macrocode}
\addto\extrasnorsk{\bbl@frenchspacing}
\addto\noextrasnorsk{\bbl@nonfrenchspacing}
%    \end{macrocode}
% \end{macro}
% \end{macro}
%
% \begin{macro}{\extrasnynorsk}
% \begin{macro}{\noextrasnynorsk}
%    Also for the ``nynorsk'' variant no extra definitions are needed
%    at the moment.
%    \begin{macrocode}
\let\extrasnynorsk\extrasnorsk
\let\noextrasnynorsk\noextrasnorsk
%    \end{macrocode}
% \end{macro}
% \end{macro}
%
%    It is possible that a site might need to add some extra code to
%    the babel macros. To enable this we load a local configuration
%    file, \file{norsk.cfg} if it is found on \TeX' search path.
% \changes{norsk-1.2f}{1995/07/02}{Added loading of configuration
%    file}
%    \begin{macrocode}
\loadlocalcfg{norsk}
%    \end{macrocode}
%
%    Our last action is to make a note that the commands we have just
%    defined, will be executed by calling the macro |\selectlanguage|
%    at the beginning of the document.
%    \begin{macrocode}
\main@language{norsk}
%    \end{macrocode}
%    Finally, the category code of \texttt{@} is reset to its original
%    value. The macrospace used by |\atcatcode| is freed.
% \changes{norsk-1.0a}{1991/07/15}{Modified handling of catcode of
%    @-sign.}
%    \begin{macrocode}
\catcode`\@=\atcatcode \let\atcatcode\relax
%</code>
%    \end{macrocode}
%
% \Finale
%%
%% \CharacterTable
%%  {Upper-case    \A\B\C\D\E\F\G\H\I\J\K\L\M\N\O\P\Q\R\S\T\U\V\W\X\Y\Z
%%   Lower-case    \a\b\c\d\e\f\g\h\i\j\k\l\m\n\o\p\q\r\s\t\u\v\w\x\y\z
%%   Digits        \0\1\2\3\4\5\6\7\8\9
%%   Exclamation   \!     Double quote  \"     Hash (number) \#
%%   Dollar        \$     Percent       \%     Ampersand     \&
%%   Acute accent  \'     Left paren    \(     Right paren   \)
%%   Asterisk      \*     Plus          \+     Comma         \,
%%   Minus         \-     Point         \.     Solidus       \/
%%   Colon         \:     Semicolon     \;     Less than     \<
%%   Equals        \=     Greater than  \>     Question mark \?
%%   Commercial at \@     Left bracket  \[     Backslash     \\
%%   Right bracket \]     Circumflex    \^     Underscore    \_
%%   Grave accent  \`     Left brace    \{     Vertical bar  \|
%%   Right brace   \}     Tilde         \~}
%%
\endinput
}
%    \end{macrocode}
%    For Norwegian two spelling variants are provided.
%    \begin{macrocode}
\DeclareOption{nynorsk}{%
  % \iffalse meta-comment
%
% Copyright 1989-1995 Johannes L. Braams and any individual authors
% listed elsewhere in this file.  All rights reserved.
% 
% For further copyright information any other copyright notices in this
% file.
% 
% This file is part of the Babel system release 3.5.
% --------------------------------------------------
%   This system is distributed in the hope that it will be useful,
%   but WITHOUT ANY WARRANTY; without even the implied warranty of
%   MERCHANTABILITY or FITNESS FOR A PARTICULAR PURPOSE.
% 
%   For error reports concerning UNCHANGED versions of this file no more
%   than one year old, see bugs.txt.
% 
%   Please do not request updates from me directly.  Primary
%   distribution is through the CTAN archives.
% 
% 
% IMPORTANT COPYRIGHT NOTICE:
% 
% You are NOT ALLOWED to distribute this file alone.
% 
% You are allowed to distribute this file under the condition that it is
% distributed together with all the files listed in manifest.txt.
% 
% If you receive only some of these files from someone, complain!
% 
% Permission is granted to copy this file to another file with a clearly
% different name and to customize the declarations in that copy to serve
% the needs of your installation, provided that you comply with
% the conditions in the file legal.txt from the LaTeX2e distribution.
% 
% However, NO PERMISSION is granted to produce or to distribute a
% modified version of this file under its original name.
%  
% You are NOT ALLOWED to change this file.
% 
% 
% \fi
%\CheckSum{182}
% \iffalse
%    Tell the \LaTeX\ system who we are and write an entry on the
%    transcript.
%<*dtx>
\ProvidesFile{norsk.dtx}
%</dtx>
%<code>\ProvidesFile{norsk.ldf}
        [1995/07/02 v1.2f Norsk support from the babel system]
%
% Babel package for LaTeX version 2e
% Copyright (C) 1989 - 1995
%           by Johannes Braams, TeXniek
%
% Please report errors to: J.L. Braams
%                          JLBraams@cistron.nl
%
%    This file is part of the babel system, it provides the source
%    code for the Norwegian language definition file.  Contributions
%    were made by Haavard Helstrup (HAAVARD@CERNVM) and Alv Kjetil
%    Holme (HOLMEA@CERNVM); the `nynorsk' variant has been supplied by
%    Per Steinar Iversen (iversen@vxcern.cern.ch) and Terje Engeset
%    Petterst (TERJEEP@VSFYS1.FI.UIB.NO)
%<*filedriver>
\documentclass{ltxdoc}
\newcommand*\TeXhax{\TeX hax}
\newcommand*\babel{\textsf{babel}}
\newcommand*\langvar{$\langle \it lang \rangle$}
\newcommand*\note[1]{}
\newcommand*\Lopt[1]{\textsf{#1}}
\newcommand*\file[1]{\texttt{#1}}
\begin{document}
 \DocInput{norsk.dtx}
\end{document}
%</filedriver>
%\fi
% \GetFileInfo{norsk.dtx}
%
% \changes{norsk-1.0a}{1991/07/15}{Renamed \file{babel.sty} in
%    \file{babel.com}}
% \changes{norsk-1.1}{1992/02/16}{Brought up-to-date with babel 3.2a}
% \changes{norsk-1.1.3}{1993/11/11}{Added a couple of translations
%    (from Per Norman Oma, TeX@itk.unit.no)}
% \changes{norsk-1.2}{1994/02/27}{Update for \LaTeXe}
% \changes{norsk-1.2d}{1994/06/26}{Removed the use of \cs{filedate}
%    and moved identification after the loading of \file{babel.def}}
%
%  \section{The Norwegian language}
%
%    The file \file{\filename}\footnote{The file described in this
%    section has version number \fileversion\ and was last revised on
%    \filedate.  Contributions were made by Haavard Helstrup
%    (\texttt{HAAVARD@CERNVM}) and Alv Kjetil Holme
%    (\texttt{HOLMEA@CERNVM}); the `nynorsk' variant has been supplied
%    by Per Steinar Iversen \texttt{iversen@vxcern.cern.ch}) and Terje
%    Engeset Petterst (\texttt{TERJEEP@VSFYS1.FI.UIB.NO)}.}  defines
%    all the language definition macros for the Norwegian language as
%    well as for a new spelling variant `nynorsk' for this language.
%
%    For this language currently no special definitions are needed or
%    available.
%
% \StopEventually{}
%
%    As this file needs to be read only once, we check whether it was
%    read before. If it was, the command |\captionsnorsk| is already
%    defined, so we can stop processing. If this command is undefined
%    we proceed with the various definitions and first show the
%    current version of this file.
%
% \changes{norsk-1.0a}{1991/07/15}{Added reset of catcode of @ before
%    \cs{endinput}.}
% \changes{norsk-1.0c}{1991/10/29}{Removed use of \cs{@ifundefined}}
%    \begin{macrocode}
%<*code>
\ifx\undefined\captionsnorsk
\else
  \selectlanguage{norsk}
  \expandafter\endinput
\fi
%    \end{macrocode}
%
% \begin{macro}{\atcatcode}
%    This file, \file{norsk.sty}, may have been read while \TeX\ is in
%    the middle of processing a document, so we have to make sure the
%    category code of \texttt{@} is `letter' while this file is being
%    read.  We save the category code of the @-sign in |\atcatcode|
%    and make it `letter'. Later the category code can be restored to
%    whatever it was before.
%
% \changes{norsk-1.0a}{1991/07/15}{Modified handling of catcode of @
%    again.}
% \changes{norsk-1.0c}{1991/10/29}{Removed use of \cs{makeatletter} and
%    hence the need to load \file{latexhax.com}}
%    \begin{macrocode}
\chardef\atcatcode=\catcode`\@
\catcode`\@=11\relax
%    \end{macrocode}
% \end{macro}
%
%    Now we determine whether the the common macros from the file
%    \file{babel.def} need to be read. We can be in one of two
%    situations: either another language option has been read earlier
%    on, in which case that other option has already read
%    \file{babel.def}, or \texttt{norsk} is the first language option
%    to be processed. In that case we need to read \file{babel.def}
%    right here before we continue.
%
% \changes{norsk-1.1}{1992/02/16}{Added \cs{relax} after the
%    argument of \cs{input}}
%    \begin{macrocode}
\ifx\undefined\babel@core@loaded\input babel.def\relax\fi
%    \end{macrocode}
%
% \changes{norsk-1.0c}{1991/10/29}{Removed code to load
%    \file{latexhax.com}}
%
%    Another check that has to be made, is if another language
%    definition file has been read already. In that case its
%    definitions have been activated. This might interfere with
%    definitions this file tries to make. Therefore we make sure that
%    we cancel any special definitions. This can be done by checking
%    the existence of the macro |\originalTeX|. If it exists we simply
%    execute it, otherwise it is |\let| to |\empty|.
% \changes{norsk-1.0a}{1991/07/15}{Added
%    \cs{let}\cs{originalTeX}\cs{relax} to test for existence}
% \changes{norsk-1.1}{1992/02/16}{\cs{originalTeX} should be
%    expandable, \cs{let} it to \cs{empty}}
%    \begin{macrocode}
\ifx\undefined\originalTeX \let\originalTeX\empty \else\originalTeX\fi
%    \end{macrocode}
%
%    When this file is read as an option, i.e. by the |\usepackage|
%    command, \texttt{norsk} will be an `unknown' language in which
%    case we have to make it known.  So we check for the existence of
%    |\l@norsk| to see whether we have to do something here.
%
% \changes{norsk-1.0c}{1991/10/29}{Removed use of \cs{@ifundefined}}
% \changes{norsk-1.1}{1992/02/16}{Added a warning when no hyphenation
%    patterns were loaded.}
% \changes{norsk-1.2d}{1994/06/26}{Now use \cs{@nopatterns} to produce
%    the warning}
%    \begin{macrocode}
\ifx\undefined\l@norsk
    \@nopatterns{Norsk}
    \adddialect\l@norsk0\fi
%    \end{macrocode}
%
%    For the `nynorsk' version of these definitions we just add a
%    ``dialect''. Also, the macros |\datenynorsk| and |\extrasnynorsk|
%    are |\let| to their `norsk' counterparts when these parts are
%    defined.
%    \begin{macrocode}
\adddialect\l@nynorsk\l@norsk
%    \end{macrocode}
%
%  \begin{macro}{\norskhyphenmins}
%  \begin{macro}{\nynorskhyphenmins}
%    The Norwegian hyphenation patterns can be used with
%    |\lefthyphenmin| set to~1 and |\righthyphenmin| set to~2. This is
%    true for both `versions' of the language.
% \changes{norsk-1.2f}{1995/07/02}{Added setting of hyphenmin
%    parameters}
%    \begin{macrocode}
\def\norskhyphenmins{\@ne\tw@}
\let\nynorskhyphenmins\norskhyphenmins
%    \end{macrocode}
%  \end{macro}
%  \end{macro}
%
%    The next step consists of defining commands to switch to (and
%    from) the Norwegian language.
%
% \begin{macro}{\captionsnorsk}
%    The macro |\captionsnorsk| defines all strings used
%    in the four standard documentclasses provided with \LaTeX.
% \changes{norsk-1.1}{1992/02/16}{Added \cs{seename}, \cs{alsoname} and
%    \cs{prefacename}}
% \changes{norsk-1.1}{1993/07/15}{\cs{headpagename} should be
%    \cs{pagename}}
% \changes{norsk-1.2f}{1995/07/02}{Added \cs{proofname} for
%    AMS-\LaTeX}
%    \begin{macrocode}
\addto\captionsnorsk{%
  \def\prefacename{Forord}%
  \def\refname{Referanser}%
  \def\abstractname{Sammendrag}%
  \def\bibname{Bibliografi}%      or Litteraturoversikt
%               or Litteratur or Referanser
  \def\chaptername{Kapittel}%
  \def\appendixname{Tillegg}%    or Appendiks
  \def\contentsname{Innhold}%
  \def\listfigurename{Figurer}%  or Figurliste
  \def\listtablename{Tabeller}%  or Tabelliste
  \def\indexname{Register}%
  \def\figurename{Figur}%
  \def\tablename{Tabell}%
  \def\partname{Del}%
  \def\enclname{Vedlegg}%
  \def\ccname{Kopi sendt}%
  \def\headtoname{Til}% in letter
  \def\pagename{Side}%
  \def\seename{Se}%
  \def\alsoname{Se ogs\aa{}}%
  \def\proofname{Proof}%
  }
%    \end{macrocode}
% \end{macro}
%
% \begin{macro}{\captionsnynorsk}
%    The macro |\captionsnynorsk| defines all strings used in the four
%    standard documentclasses provided with \LaTeX, but using a
%    different spelling than in the command |\captionsnorsk|.
% \changes{norsk-1.1}{1992/02/16}{Added \cs{seename}, \cs{alsoname} and
%    \cs{prefacename}}
% \changes{norsk-1.1}{1993/07/15}{\cs{headpagename} should be
%    \cs{pagename}}
%    \begin{macrocode}
\addto\captionsnynorsk{%
  \def\prefacename{Forord}%
  \def\refname{Referansar}%
  \def\abstractname{Samandrag}%
  \def\bibname{Litteratur}%     or Litteraturoversyn
                         %    or Referansar
  \def\chaptername{Kapittel}%
  \def\appendixname{Tillegg}%   or Appendiks
  \def\contentsname{Innhald}%
  \def\listfigurename{Figurar}% or Figurliste
  \def\listtablename{Tabellar}% or Tabelliste
  \def\indexname{Register}%
  \def\figurename{Figur}%
  \def\tablename{Tabell}%
  \def\partname{Del}%
  \def\enclname{Vedlegg}%
  \def\ccname{Kopi sendt}%
  \def\headtoname{Til}% in letter
  \def\pagename{Side}%
  \def\seename{Sj\aa{}}%
  \def\alsoname{Sj\aa{} ogs\aa{}}%
  \def\proofname{Proof}%
  }
%    \end{macrocode}
% \end{macro}
%
% \begin{macro}{\datenorsk}
%    The macro |\datenorsk| redefines the command |\today| to produce
%    Norwegian dates.
%    \begin{macrocode}
\def\datenorsk{%
\def\today{\number\day.~\ifcase\month\or
  januar\or februar\or mars\or april\or mai\or juni\or
  juli\or august\or september\or oktober\or november\or desember\fi
  \space\number\year}}
%    \end{macrocode}
% \end{macro}
%
% \begin{macro}{\datenynorsk}
%    The spelling of the names of the months is the same for both
%    versions of the ``Norsk'' language, so we simply |\let| the macro
%    |\datenynorsk| be equal to |\datenorsk|
%    \begin{macrocode}
\let\datenynorsk\datenorsk
%    \end{macrocode}
% \end{macro}
%
% \begin{macro}{\extrasnorsk}
% \begin{macro}{\extrasnynorsk}
%    The macro |\extrasnorsk| will perform all the extra definitions
%    needed for the Norwegian language. The macro |\noextrasnorsk| is
%    used to cancel the actions of |\extrasnorsk|.  
%
%    Norwegian typesetting requires |\frencspacing| to be in effect.
%    \begin{macrocode}
\addto\extrasnorsk{\bbl@frenchspacing}
\addto\noextrasnorsk{\bbl@nonfrenchspacing}
%    \end{macrocode}
% \end{macro}
% \end{macro}
%
% \begin{macro}{\extrasnynorsk}
% \begin{macro}{\noextrasnynorsk}
%    Also for the ``nynorsk'' variant no extra definitions are needed
%    at the moment.
%    \begin{macrocode}
\let\extrasnynorsk\extrasnorsk
\let\noextrasnynorsk\noextrasnorsk
%    \end{macrocode}
% \end{macro}
% \end{macro}
%
%    It is possible that a site might need to add some extra code to
%    the babel macros. To enable this we load a local configuration
%    file, \file{norsk.cfg} if it is found on \TeX' search path.
% \changes{norsk-1.2f}{1995/07/02}{Added loading of configuration
%    file}
%    \begin{macrocode}
\loadlocalcfg{norsk}
%    \end{macrocode}
%
%    Our last action is to make a note that the commands we have just
%    defined, will be executed by calling the macro |\selectlanguage|
%    at the beginning of the document.
%    \begin{macrocode}
\main@language{norsk}
%    \end{macrocode}
%    Finally, the category code of \texttt{@} is reset to its original
%    value. The macrospace used by |\atcatcode| is freed.
% \changes{norsk-1.0a}{1991/07/15}{Modified handling of catcode of
%    @-sign.}
%    \begin{macrocode}
\catcode`\@=\atcatcode \let\atcatcode\relax
%</code>
%    \end{macrocode}
%
% \Finale
%%
%% \CharacterTable
%%  {Upper-case    \A\B\C\D\E\F\G\H\I\J\K\L\M\N\O\P\Q\R\S\T\U\V\W\X\Y\Z
%%   Lower-case    \a\b\c\d\e\f\g\h\i\j\k\l\m\n\o\p\q\r\s\t\u\v\w\x\y\z
%%   Digits        \0\1\2\3\4\5\6\7\8\9
%%   Exclamation   \!     Double quote  \"     Hash (number) \#
%%   Dollar        \$     Percent       \%     Ampersand     \&
%%   Acute accent  \'     Left paren    \(     Right paren   \)
%%   Asterisk      \*     Plus          \+     Comma         \,
%%   Minus         \-     Point         \.     Solidus       \/
%%   Colon         \:     Semicolon     \;     Less than     \<
%%   Equals        \=     Greater than  \>     Question mark \?
%%   Commercial at \@     Left bracket  \[     Backslash     \\
%%   Right bracket \]     Circumflex    \^     Underscore    \_
%%   Grave accent  \`     Left brace    \{     Vertical bar  \|
%%   Right brace   \}     Tilde         \~}
%%
\endinput
%
  \main@language{nynorsk}}
\DeclareOption{polish}{% \iffalse meta-comment
%
% Copyright 1989-1995 Johannes L. Braams and any individual authors
% listed elsewhere in this file.  All rights reserved.
% 
% For further copyright information any other copyright notices in this
% file.
% 
% This file is part of the Babel system release 3.5.
% --------------------------------------------------
%   This system is distributed in the hope that it will be useful,
%   but WITHOUT ANY WARRANTY; without even the implied warranty of
%   MERCHANTABILITY or FITNESS FOR A PARTICULAR PURPOSE.
% 
%   For error reports concerning UNCHANGED versions of this file no more
%   than one year old, see bugs.txt.
% 
%   Please do not request updates from me directly.  Primary
%   distribution is through the CTAN archives.
% 
% 
% IMPORTANT COPYRIGHT NOTICE:
% 
% You are NOT ALLOWED to distribute this file alone.
% 
% You are allowed to distribute this file under the condition that it is
% distributed together with all the files listed in manifest.txt.
% 
% If you receive only some of these files from someone, complain!
% 
% Permission is granted to copy this file to another file with a clearly
% different name and to customize the declarations in that copy to serve
% the needs of your installation, provided that you comply with
% the conditions in the file legal.txt from the LaTeX2e distribution.
% 
% However, NO PERMISSION is granted to produce or to distribute a
% modified version of this file under its original name.
%  
% You are NOT ALLOWED to change this file.
% 
% 
% \fi
% \CheckSum{513}
%
% \iffalse
%    Tell the \LaTeX\ system who we are and write an entry on the
%    transcript.
%<*dtx>
\ProvidesFile{polish.dtx}
%</dtx>
%<code>\ProvidesFile{polish.ldf}
        [1995/07/04 v1.2b Polish support from the babel system]
%
% Babel package for LaTeX version 2e
% Copyright (C) 1989 -- 1995
%           by Johannes Braams, TeXniek
%
% Polish Language Definition File
% Copyright (C) 1989 - 1995
%           by Elmar Schalueck, Michael Janich
%              Universitaet-Gesamthochschule Paderborn
%              Warburger Strasse 100
%              4790 Paderborn
%              Germany
%              elmar@uni-paderborn.de
%              massa@uni-paderborn.de
%
% Please report errors to: J.L. Braams
%                          JLBraams@cistron.nl
%
%    This file is part of the babel system, it provides the source
%    code for the Polish language definition file. It was developped
%    out of Polish.tex, which was written by Elmar Schalueck and
%    Michael Janich. Polish.tex was based on code by Leszek
%    Holenderski, Jerzy Ryll and J. S. Bie\'n from Faculty of
%    Mathematics,Informatics and Mechanics of Warsaw University, exept
%    of Jerzy Ryll (Instytut Informatyki Uniwersytetu Warszawskiego).
%<*filedriver>
\documentclass{ltxdoc}
\newcommand*\TeXhax{\TeX hax}
\newcommand*\babel{\textsf{babel}}
\newcommand*\langvar{$\langle \it lang \rangle$}
\newcommand*\note[1]{}
\newcommand*\Lopt[1]{\textsf{#1}}
\newcommand*\file[1]{\texttt{#1}}
\begin{document}
 \DocInput{polish.dtx}
\end{document}
%</filedriver>
%\fi
% \GetFileInfo{polish.dtx}
%
% \changes{polish-1.1c}{1994/06/26}{Removed the use of \cs{filedate}
%    and moved identification after the loading of babel.def}
%
%  \section{The Polish language}
%
%    The file \file{\filename}\footnote{The file described in this
%    section has version number \fileversion\ and was last revised on
%    \filedate.}  defines all the language-specific macros for the
%    Polish language.
%
%    For this language the character |"| is made active. In
%    table~\ref{tab:polish-quote} an overview is given of its purpose.
%    \begin{table}[htb]
%     \begin{center}
%     \begin{tabular}{lp{8cm}}
%      |"a| & or |\aob|, for tailed-a (like \c{a})\\
%      |"A| & or |\Aob|, for tailed-A (like \c{A})\\
%      |"e| & or |\eob|, for tailed-e (like \c{e})\\
%      |"E| & or |\Eob|, for tailed-E (like \c{E})\\
%      |"c| & or |\'c|,  for accented c (like \'c),
%                      same with uppercase letters and n,o,s\\
%      |"l| & or |\lpb{}|, for l with stroke (like \l)\\
%      |"L| & or |\Lpb{}|, for L with stroke (like \L)\\
%      |"r| & or |\zkb{}|, for pointed z (like \.z), cf.
%      pronounciation\\
%      |"R| & or |\Zkb{}|, for pointed Z (like \.Z)\\
%      |"z| & or |\'z|,  for accented z\\
%      |"Z| & or |\'Z|,  for accented Z\\
%      \verb="|= & disable ligature at this position.\\
%      |"-| & an explicit hyphen sign, allowing hyphenation
%                  in the rest of the word.\\
%      |""| & like |"-|, but producing no hyphen sign
%                  (for compund words with hyphen, e.g.\ |x-""y|). \\
%      |"`| & for German left double quotes (looks like ,,).   \\
%      |"'| & for German right double quotes.                  \\
%      |"<| & for French left double quotes (similar to $<<$). \\
%      |">| & for French right double quotes (similar to $>>$).\\
%     \end{tabular}
%     \caption{The extra definitions made by \file{polish.sty}}
%     \label{tab:polish-quote}
%     \end{center}
%    \end{table}
%
% \StopEventually{}
%
%    As this file needs to be read only once, we check whether it was
%    read before. If it was, the command |\captionspolish| is already
%    defined, so we can stop processing. If this command is undefined
%    we proceed with the various definitions and first show the
%    current version of this file.
%
%    \begin{macrocode}
\ifx\undefined\captionspolish
\else
  \selectlanguage{polish}
  \expandafter\endinput
\fi
%    \end{macrocode}
%
% \begin{macro}{\atcatcode}
%    This file, \file{polish.sty}, may have been read while \TeX\ is
%    in the middle of processing a document, so we have to make sure
%    the category code of \texttt{@} is `letter' while this file is
%    being read.  We save the category code of the @-sign in
%    |\atcatcode| and make it `letter'. Later the category code can be
%    restored to whatever it was before.
%    \begin{macrocode}
%<*code>
\chardef\atcatcode=\catcode`\@
\catcode`\@=11\relax
%    \end{macrocode}
% \end{macro}
%
%    Now we determine whether the the common macros from the file
%    \file{babel.def} need to be read. We can be in one of two
%    situations: either another language option has been read earlier
%    on, in which case that other option has already read
%    \file{babel.def}, or \texttt{polish} is the first language option
%    to be processed. In that case we need to read \file{babel.def}
%    right here before we continue.
%
%    \begin{macrocode}
\ifx\undefined\babel@core@loaded\input babel.def\relax\fi
%    \end{macrocode}
%
%    Another check that has to be made, is if another language
%    definition file has been read already. In that case its
%    definitions have been activated. This might interfere with
%    definitions this file tries to make. Therefore we make sure that
%    we cancel any special definitions. This can be done by checking
%    the existence of the macro |\originalTeX|. If it exists we simply
%    execute it.
%    \begin{macrocode}
\ifx\undefined\originalTeX
  \let\originalTeX\empty
\fi
\originalTeX
%    \end{macrocode}
%
%    When this file is read as an option, i.e. by the |\usepackage|
%    command, \texttt{polish} could be an `unknown' language in which
%    case we have to make it known. So we check for the existence of
%    |\l@polish| to see whether we have to do something here.
%
% \changes{polish-1.1c}{1994/06/26}{Now use \cs{@nopatterns} to
%    produce the warning}
%    \begin{macrocode}
\ifx\undefined\l@polish
  \@nopatterns{Polish}
  \adddialect\l@polish0\fi
%    \end{macrocode}
%
%    The next step consists of defining commands to switch to (and
%    from) the Polish language.
%
% \begin{macro}{\captionspolish}
%    The macro |\captionspolish| defines all strings used in the four
%    standard documentclasses provided with \LaTeX.
% \changes{polish-1.2b}{1995/07/04}{Added \cs{proofname} for
%    AMS-\LaTeX}
%    \begin{macrocode}
\addto\captionspolish{%
  \def\prefacename{Przedmowa}%
  \def\refname{Bibliografia}%
  \def\abstractname{Streszczenie}%
  \def\bibname{Literatura}%
  \def\chaptername{Rozdzia\l}%
  \def\appendixname{Dodatek}%
  \def\contentsname{Spis rzeczy}%
  \def\listfigurename{Spis rysunk\'ow}%
  \def\listtablename{Spis tablic}%
  \def\indexname{Indeks}%
  \def\figurename{Rysunek}%
  \def\tablename{Tablica}%
  \def\partname{Cz\eob{}\'s\'c}%
  \def\enclname{Za\l\aob{}cznik}%
  \def\ccname{Kopie:}%
  \def\headtoname{Do}%
  \def\pagename{Strona}%
  \def\seename{Por\'ownaj}%
  \def\alsoname{Por\'ownaj tak\.ze}%
  \def\proofname{Proof}%   <-- needs translation
}
%    \end{macrocode}
% \end{macro}
%
% \begin{macro}{\datepolish}
%    The macro |\datepolish| redefines the command |\today| to produce
%    Polish dates.
%    \begin{macrocode}
\def\datepolish{%
  \def\today{\number\day~\ifcase\month\or
  stycznia\or lutego\or marca\or kwietnia\or maja\or czerwca\or lipca\or
  sierpnia\or wrze\'snia\or pa\'zdziernika\or listopada\or grudnia\fi
  \space\number\year}
}
%    \end{macrocode}
% \end{macro}
%
% \begin{macro}{\extraspolish}
% \begin{macro}{\noextraspolish}
%    The macro |\extraspolish| will perform all the extra definitions
%    needed for the Polish language. The macro |\noextraspolish| is
%    used to cancel the actions of |\extraspolish|.
%
%    For Polish the \texttt{"} character is made active. This is
%    done once, later on its definition may vary. Other languages in
%    the same document may also use the \texttt{"} character for
%    shorthands; we specify that the polish group of shorthands
%    should be used.
%
%    \begin{macrocode}
\initiate@active@char{"}
\addto\extraspolish{\languageshorthands{polish}}
\addto\extraspolish{\bbl@activate{"}}
%\addto\noextraspolish{\bbl@deactivate{"}}
%    \end{macrocode}
% \end{macro}
% \end{macro}
%
%    The code above is necessary because we need an extra
%    active character. This character is then used as indicated in
%    table~\ref{tab:polish-quote}.
%
%    If you have problems at the end of a word with a linebreak, use
%    the other version without hyphenation tricks. Some TeX wizard may
%    produce a better solution with forcasting another token to decide
%    whether the character after the double quote is the last in a
%    word. Do it and let us know.
%
%    In Polish texts some letters get special diacritical marks.
%    Leszek Holenderski designed the following code to position the
%    diacritics correctly for every font in every size. These macros
%    need a few extra dimension variables.
%
%    \begin{macrocode}
\newdimen\pl@left
\newdimen\pl@down
\newdimen\pl@right
\newdimen\pl@temp
%    \end{macrocode}
%
%  \begin{macro}{\sob}
%    The macro |\sob| is used to put the `ogonek' in the right
%    place.
%
%    \begin{macrocode}
\def\sob#1#2#3#4#5{%parameters: letter and fractions hl,ho,vl,vo
  \setbox0\hbox{#1}\setbox1\hbox{$_\mathchar'454$}\setbox2\hbox{p}%
  \pl@right=#2\wd0 \advance\pl@right by-#3\wd1
  \pl@down=#5\ht1 \advance\pl@down by-#4\ht0
  \pl@left=\pl@right \advance\pl@left by\wd1
  \pl@temp=-\pl@down \advance\pl@temp by\dp2 \dp1=\pl@temp
  \kern\pl@right\lower\pl@down\box1\kern-\pl@left #1}
%    \end{macrocode}
%  \end{macro}
%
%  \begin{macro}{\aob}
%  \begin{macro}{\Aob}
%  \begin{macro}{\eob}
%  \begin{macro}{\Eob}
%    The ogonek is placed with the letters `a', `A', `e', and `E'.
%    \begin{macrocode}
\def\aob{\sob a{.66}{.20}{0}{.90}}
\def\Aob{\sob A{.80}{.50}{0}{.90}}
\def\eob{\sob e{.50}{.35}{0}{.93}}
\def\Eob{\sob E{.60}{.35}{0}{.90}}
%    \end{macrocode}
%  \end{macro}
%  \end{macro}
%  \end{macro}
%  \end{macro}
%
%  \begin{macro}{\spb}
%    The macro |\spb| is used to put the `poprzeczka' in the
%    right place.
%
%    \begin{macrocode}
\def\spb#1#2#3#4#5{%
  \setbox0\hbox{#1}\setbox1\hbox{\char'023}%
  \pl@right=#2\wd0 \advance\pl@right by-#3\wd1
  \pl@down=#5\ht1 \advance\pl@down by-#4\ht0
  \pl@left=\pl@right \advance\pl@left by\wd1
  \ht1=\pl@down \dp1=-\pl@down
  \kern\pl@right\lower\pl@down\box1\kern-\pl@left #1}
%    \end{macrocode}
%  \end{macro}
%
%  \begin{macro}{\skb}
%    The macro |\skb| is used to put the `kropka' in the
%    right place.
%
%    \begin{macrocode}
\def\skb#1#2#3#4#5{%
  \setbox0\hbox{#1}\setbox1\hbox{\char'056}%
  \pl@right=#2\wd0 \advance\pl@right by-#3\wd1
  \pl@down=#5\ht1 \advance\pl@down by-#4\ht0
  \pl@left=\pl@right \advance\pl@left by\wd1
  \kern\pl@right\lower\pl@down\box1\kern-\pl@left #1}
%    \end{macrocode}
%  \end{macro}
%
%  \begin{macro}{\textpl}
%    For the `poprzeczka' and the `kropka' in text fonts we don't need
%    any special coding, but we can (almost) use what is already
%    available.
%
%    \begin{macrocode}
\def\textpl{%
  \def\lpb{\plll}%
  \def\Lpb{\pLLL}%
  \def\zkb{\.z}%
  \def\Zkb{\.Z}}
%    \end{macrocode}
%    Initially we assume that typesetting is done with text fonts.
% \changes{polish-1.0a}{1993/11/05}{Initially execute `textpl}
%    \begin{macrocode}
\textpl
%    \end{macrocode}
%
%    \begin{macrocode}
\let\lll=\l \let\LLL=\L
\def\plll{\lll}
\def\pLLL{\LLL}
%    \end{macrocode}
%  \end{macro}
%
%  \begin{macro}{\telepl}
%    But for the `teletype' font in `OT1' encoding we have to take some
%    special actions, involving the macros defined above.
%
%    \begin{macrocode}
\def\telepl{%
  \def\lpb{\spb l{.45}{.5}{.4}{.8}}%
  \def\Lpb{\spb L{.23}{.5}{.4}{.8}}%
  \def\zkb{\skb z{.5}{.5}{1.2}{0}}%
  \def\Zkb{\skb Z{.5}{.5}{1.1}{0}}}
%    \end{macrocode}
%  \end{macro}
%
%    To activate these codes the font changing commands as they are
%    defined in \LaTeX\ are modified. The same is done for plain
%    \TeX's font changing commands.
%
%    When |\selectfont| is undefined the current format is spposed to be
%    either plain (based) or \LaTeX$\:$2.09.
% \changes{polish-1.2a}{1995/06/06}{Don't modify \cs{rm} and friends for
%    \LaTeXe, take \cs{selectfont } instead}
%    \begin{macrocode}
\ifx\selectfont\undefined
  \ifx\prm\undefined \addto\rm{\textpl}\else \addto\prm{\textpl}\fi
  \ifx\pit\undefined \addto\it{\textpl}\else \addto\pit{\textpl}\fi
  \ifx\pbf\undefined \addto\bf{\textpl}\else \addto\pbf{\textpl}\fi
  \ifx\psl\undefined \addto\sl{\textpl}\else \addto\psl{\textpl}\fi
  \ifx\psf\undefined                   \else \addto\psf{\textpl}\fi
  \ifx\psc\undefined                   \else \addto\psc{\textpl}\fi
  \ifx\ptt\undefined \addto\tt{\telepl}\else \addto\ptt{\telepl}\fi
\else
%    \end{macrocode}
%    When |\selectfont| exists we assume \LaTeXe.
%    \begin{macrocode}
  \expandafter\addto\csname selectfont \endcsname{%
    \csname\f@encoding @pl\endcsname}
\fi
%    \end{macrocode}
%    Currently we support the OT1 and T1 encodings. For T1 we don't
%    have to make a difference between typewriter fonts and other
%    fonts, they all have the same glyphs.
%    \begin{macrocode}
\expandafter\let\csname T1@pl\endcsname\textpl
%    \end{macrocode}
%    For OT1 we need to check the current font family, stored in
%    |\f@family|. Unfortunately we need a hack as |\ttdefault| is
%    defined as a |\long| macro, while |\f@family| is not.
%    \begin{macrocode}
\expandafter\def\csname OT1@pl\endcsname{%
  \long\edef\curr@family{\f@family}%
  \ifx\curr@family\ttdefault
    \telepl
  \else
    \textpl
  \fi}
%    \end{macrocode}
%
%  \begin{macro}{\dq}
%    We save the original double quote character in |\dq| to keep
%    it available, the math accent |\"| can now be typed as |"|.
%    \begin{macrocode}
\begingroup \catcode`\"12
\def\x{\endgroup
  \def\dq{"}}
\x
%    \end{macrocode}
%  \end{macro}
%
%    Now we can define the doublequote macros for diacritics,
% \changes{polish-1.1d}{1995/01/31}{The dqmacro for C used \cs{'c}}
%    \begin{macrocode}
\declare@shorthand{polish}{"a}{\textormath{\aob}{\ddot a}}
\declare@shorthand{polish}{"A}{\textormath{\Aob}{\ddot A}}
\declare@shorthand{polish}{"c}{\textormath{\'c}{\acute c}}
\declare@shorthand{polish}{"C}{\textormath{\'C}{\acute C}}
\declare@shorthand{polish}{"e}{\textormath{\eob}{\ddot e}}
\declare@shorthand{polish}{"E}{\textormath{\Eob}{\ddot E}}
\declare@shorthand{polish}{"l}{\textormath{\lpb}{\ddot l}}
\declare@shorthand{polish}{"L}{\textormath{\Lpb}{\ddot L}}
\declare@shorthand{polish}{"n}{\textormath{\'n}{\acute n}}
\declare@shorthand{polish}{"N}{\textormath{\'N}{\acute N}}
\declare@shorthand{polish}{"o}{\textormath{\'o}{\acute o}}
\declare@shorthand{polish}{"O}{\textormath{\'O}{\acute O}}
\declare@shorthand{polish}{"r}{\textormath{\zkb}{\ddot r}}
\declare@shorthand{polish}{"R}{\textormath{\Zkb}{\ddot R}}
\declare@shorthand{polish}{"s}{\textormath{\'s}{\acute s}}
\declare@shorthand{polish}{"S}{\textormath{\'S}{\acute S}}
\declare@shorthand{polish}{"z}{\textormath{\'z}{\acute z}}
\declare@shorthand{polish}{"Z}{\textormath{\'Z}{\acute Z}}
%    \end{macrocode}
%
%    Then we define access to two forms of quotation marks, similar
%    to the german and french quotation marks.
%    \begin{macrocode}
\declare@shorthand{polish}{"`}{%
  \textormath{\quotedblbase{}}{\mbox{\quotedblbase}}}
\declare@shorthand{polish}{"'}{%
  \textormath{\textquotedblleft{}}{\mbox{\textquotedblleft}}}
\declare@shorthand{polish}{"<}{%
  \textormath{\guillemotleft{}}{\mbox{\guillemotleft}}}
\declare@shorthand{polish}{">}{%
  \textormath{\guillemotright{}}{\mbox{\guillemotright}}}
%    \end{macrocode}
%    then we define two shorthands to be able to specify hyphenation
%    breakpoints that behavew a little different from |\-|.
%    \begin{macrocode}
\declare@shorthand{polish}{"-}{\allowhyphens-\allowhyphens}
\declare@shorthand{polish}{""}{\hskip\z@skip}
%    \end{macrocode}
%    And we want to have a shorthand for disabling a ligature.
%    \begin{macrocode}
\declare@shorthand{polish}{"|}{%
  \textormath{\discretionary{-}{}{\kern.03em}}{}}
%    \end{macrocode}
%
%
%  \begin{macro}{\mdqon}
%  \begin{macro}{\mdqoff}
%    All that's left to do now is to  define a couple of commands
%    for reasons of compatibility with \file{polish.tex}.
%    \begin{macrocode}
\def\mdqon{\bbl@activate{"}}
\def\mdqoff{\bbl@deactivate{"}}
%    \end{macrocode}
%  \end{macro}
%  \end{macro}
%
%    It is possible that a site might need to add some extra code to
%    the babel macros. To enable this we load a local configuration
%    file, \file{polish.cfg} if it is found on \TeX' search path.
% \changes{polish-1.2b}{1995/07/02}{Added loading of configuration
%    file}
%    \begin{macrocode}
\loadlocalcfg{polish}
%    \end{macrocode}
%
%    Our last action is to make a note that activate the commands we
%    have just defined, will be executed by calling the macro
%    |\selectlanguage| at the beginning of the document.
%
%    \begin{macrocode}
\main@language{polish}
%    \end{macrocode}
%
%    Finally, the category code of \texttt{@} is reset to its original
%    value. The macrospace used by |\atcatcode| is freed.
%
%    \begin{macrocode}
\catcode`\@=\atcatcode \let\atcatcode\relax
%</code>
%    \end{macrocode}
%
% \Finale
%
%% \CharacterTable
%%  {Upper-case    \A\B\C\D\E\F\G\H\I\J\K\L\M\N\O\P\Q\R\S\T\U\V\W\X\Y\Z
%%   Lower-case    \a\b\c\d\e\f\g\h\i\j\k\l\m\n\o\p\q\r\s\t\u\v\w\x\y\z
%%   Digits        \0\1\2\3\4\5\6\7\8\9
%%   Exclamation   \!     Double quote  \"     Hash (number) \#
%%   Dollar        \$     Percent       \%     Ampersand     \&
%%   Acute accent  \'     Left paren    \(     Right paren   \)
%%   Asterisk      \*     Plus          \+     Comma         \,
%%   Minus         \-     Point         \.     Solidus       \/
%%   Colon         \:     Semicolon     \;     Less than     \<
%%   Equals        \=     Greater than  \>     Question mark \?
%%   Commercial at \@     Left bracket  \[     Backslash     \\
%%   Right bracket \]     Circumflex    \^     Underscore    \_
%%   Grave accent  \`     Left brace    \{     Vertical bar  \|
%%   Right brace   \}     Tilde         \~}
%%
\endinput
}
\DeclareOption{portuges}{% \iffalse meta-comment
%
% Copyright 1989-1995 Johannes L. Braams and any individual authors
% listed elsewhere in this file.  All rights reserved.
% 
% For further copyright information any other copyright notices in this
% file.
% 
% This file is part of the Babel system release 3.5.
% --------------------------------------------------
%   This system is distributed in the hope that it will be useful,
%   but WITHOUT ANY WARRANTY; without even the implied warranty of
%   MERCHANTABILITY or FITNESS FOR A PARTICULAR PURPOSE.
% 
%   For error reports concerning UNCHANGED versions of this file no more
%   than one year old, see bugs.txt.
% 
%   Please do not request updates from me directly.  Primary
%   distribution is through the CTAN archives.
% 
% 
% IMPORTANT COPYRIGHT NOTICE:
% 
% You are NOT ALLOWED to distribute this file alone.
% 
% You are allowed to distribute this file under the condition that it is
% distributed together with all the files listed in manifest.txt.
% 
% If you receive only some of these files from someone, complain!
% 
% Permission is granted to copy this file to another file with a clearly
% different name and to customize the declarations in that copy to serve
% the needs of your installation, provided that you comply with
% the conditions in the file legal.txt from the LaTeX2e distribution.
% 
% However, NO PERMISSION is granted to produce or to distribute a
% modified version of this file under its original name.
%  
% You are NOT ALLOWED to change this file.
% 
% 
% \fi
% \CheckSum{275}
% \iffalse
%    Tell the \LaTeX\ system who we are and write an entry on the
%    transcript.
%<*dtx>
\ProvidesFile{portuges.dtx}
%</dtx>
%<code>\ProvidesFile{portuges.ldf}
        [1995/07/04 v1.2h Portuguese support from the babel system]
%
% Babel package for LaTeX version 2e
% Copyright (C) 1989 - 1995
%           by Johannes Braams, TeXniek
%
% Portuguese Language Definition File
% Copyright (C) 1989 - 1995
%           by Johannes Braams, TeXniek
%
% Please report errors to: J.L. Braams
%                          JLBraams@cistron.nl
%
%    This file is part of the babel system, it provides the source
%    code for the Portuguese language definition file.  The Portuguese
%    words were contributed by Jose Pedro Ramalhete, (JRAMALHE@CERNVM
%    or Jose-Pedro_Ramalhete@MACMAIL).
%
%    Arnaldo Viegas de Lima <arnaldo@VNET.IBM.COM> contributed
%    brazilian translations and suggestions for enhancements.
%<*filedriver>
\documentclass{ltxdoc}
\newcommand*\TeXhax{\TeX hax}
\newcommand*\babel{\textsf{babel}}
\newcommand*\langvar{$\langle \it lang \rangle$}
\newcommand*\note[1]{}
\newcommand*\Lopt[1]{\textsf{#1}}
\newcommand*\file[1]{\texttt{#1}}
\begin{document}
 \DocInput{portuges.dtx}
\end{document}
%</filedriver>
%\fi
%
% \GetFileInfo{portuges.dtx}
%
% \changes{portuges-1.0a}{1991/07/15}{Renamed \file{babel.sty} in
%    \file{babel.com}}
% \changes{portuges-1.1}{1992/02/16}{Brought up-to-date with babel 3.2a}
% \changes{portuges-1.2}{1994/02/26}{Update for \LaTeXe}
% \changes{portuges-1.2d}{1994/06/26}{Removed the use of \cs{filedate}
%    and moved identification after the loading of \file{babel.def}}
% \changes{portuges-1.2g}{1995/06/04}{Enhanced support for brasilian}
%
%  \section{The Portuguese language}
%
%    The file \file{\filename}\footnote{The file described in this
%    section has version number \fileversion\ and was last revised on
%    \filedate.  Contributions were made by Jose Pedro Ramalhete
%    (\texttt{JRAMALHE@CERNVM} or
%    \texttt{Jose-Pedro\_Ramalhete@MACMAIL}) and Arnaldo Viegas de
%    Lima \texttt{arnaldo@VNET.IBM.COM}.}  defines all the
%    language-specific macros for the Portuguese language as well as
%    for the Brasilian version of this language.
%
%    For this language the character |"| is made active. In
%    table~\ref{tab:port-quote} an overview is given of its purpose.
%
%    \begin{table}[htb]
%     \centering
%     \begin{tabular}{lp{8cm}}
%       \verb="|= & disable ligature at this position.\\
%        |"-| & an explicit hyphen sign, allowing hyphenation
%               in the rest of the word.\\
%        |""| & like \verb="-=, but producing no hyphen sign (for
%              words that should break at some sign such as
%              ``entrada/salida.''\\
%        |"<| & for French left double quotes (similar to $<<$).\\
%        |">| & for French right double quotes (similar to $>>$).\\
%        |\-| & like the old |\-|, but allowing hyphenation
%               in the rest of the word. \\
%     \end{tabular}
%     \caption{The extra definitions made by \file{portuges.ldf}}
%     \label{tab:port-quote}
%    \end{table}
%
% \StopEventually{}

%    As this file needs to be read only once, we check whether it was
%    read before. If it was, the command |\captionsportuges| is
%    already defined, so we can stop processing. If this command is
%    undefined we proceed with the various definitions and first show
%    the current version of this file.
%
% \changes{portuges-1.0a}{1991/07/15}{Added reset of catcode of @
%    before \cs{endinput}.}
% \changes{portuges-1.0b}{1991/10/29}{Removed use of
%    \cs{@ifundefined}}
%    \begin{macrocode}
%<*code>
\ifx\undefined\captionsportuges
\else
  \selectlanguage{portuges}
  \expandafter\endinput
\fi
%    \end{macrocode}
%
% \changes{portuges-1.0b}{1991/10/29}{Removed code to load
%    \file{latexhax.com}}
%
% \begin{macro}{\atcatcode}
%    This file, \file{portuges.ldf}, may have been read while \TeX\ is
%    in the middle of processing a document, so we have to make sure
%    the category code of \texttt{@} is `letter' while this file is
%    being read.  We save the category code of the @-sign in
%    |\atcatcode| and make it `letter'. Later the category code can be
%    restored to whatever it was before.
%
% \changes{portuges-1.0a}{1991/07/15}{Modified handling of catcode of
%    @ again.}
% \changes{portuges-1.0b}{1991/10/ 29}{Removed use of
%    \cs{makeatletter} and hence the need to load \file{latexhax.com}}
%    \begin{macrocode}
\chardef\atcatcode=\catcode`\@
\catcode`\@=11\relax
%    \end{macrocode}
% \end{macro}
%
%    Now we determine whether the the common macros from the file
%    \file{babel.def} need to be read. We can be in one of two
%    situations: either another language option has been read earlier
%    on, in which case that other option has already read
%    \file{babel.def}, or \texttt{portuges} is the first language
%    option to be processed. In that case we need to read
%    \file{babel.def} right here before we continue.
%
% \changes{portuges-1.1}{1992/02/16}{Added \cs{relax} after the
%    argument of \cs{input}}
%    \begin{macrocode}
\ifx\undefined\babel@core@loaded\input babel.def\relax\fi
%    \end{macrocode}
%
%    Another check that has to be made, is if another language
%    definition file has been read already. In that case its
%    definitions have been activated. This might interfere with
%    definitions this file tries to make. Therefore we make sure that
%    we cancel any special definitions. This can be done by checking
%    the existence of the macro |\originalTeX|. If it exists we simply
%    execute it, otherwise it is |\let| to |\empty|.
% \changes{portuges-1.0a}{1991/07/ 15}{Added
%    \cs{let}\cs{originalTeX}\cs{relax} to test for existence}
% \changes{portuges-1.1}{1992/02/16}{\cs{originalTeX} should be
%    expandable, \cs{let} it to \cs{empty}}
%    \begin{macrocode}
\ifx\undefined\originalTeX \let\originalTeX\empty \else\originalTeX\fi
%    \end{macrocode}
%
%    When this file is read as an option, i.e. by the |\usepackage|
%    command, \texttt{portuges} will be an `unknown' language in which
%    case we have to make it known. So we check for the existence of
%    |\l@portuges| to see whether we have to do something here.
%
% \changes{portuges-1.0b}{1991/10/29}{Removed use of cs{@ifundefined}}
% \changes{portuges-1.1}{1992/02/16}{Added a warning when no
%    hyphenation patterns were loaded.}
% \changes{portuges-1.2d}{1994/06/26}{Now use \cs{@nopatterns} to
%    produce the warning}
%    \begin{macrocode}
\ifx\undefined\l@portuges
    \@nopatterns{Portuges}
    \adddialect\l@portuges0\fi
%    \end{macrocode}
%
%    For the Brasilian version of these definitions we just add a
%    ``dialect''. Also, the macros |\captionsbrazil| and
%    |\extrasbrazil| are |\let| to their Portuguese counterparts when
%    these parts are defined.
%    \begin{macrocode}
\adddialect\l@brazil\l@portuges
%    \end{macrocode}
%
%    The next step consists of defining commands to switch to (and from)
%    the Portuguese language.
%
% \begin{macro}{\captionsportuges}
%    The macro |\captionsportuges| defines all strings used
%    in the four standard documentclasses provided with \LaTeX.
% \changes{portuges-1.1}{1992/02/16}{Added \cs{seename}, \cs{alsoname}
%    and \cs{prefacename}}
% \changes{portuges-1.1}{1993/07/15}{\cs{headpagename} should be
%    \cs{pagename}}
% \changes{portuges-1.2e}{1994/11/09}{Added a few missing
%    translations}
% \changes{portuges-1.2h}{1995/07/04}{Added \cs{proofname} for
%    AMS-\LaTeX}
%    \begin{macrocode}
\addto\captionsportuges{%
  \def\prefacename{Pref\'acio}%
  \def\refname{Refer\^encias}%
  \def\abstractname{Resumo}%
  \def\bibname{Bibliografia}%
  \def\chaptername{Cap\'{\i}tulo}%
  \def\appendixname{Ap\^endice}%
  \def\contentsname{\'Indice}%
  \def\listfigurename{Lista de Figuras}%
  \def\listtablename{Lista de Tabelas}%
  \def\indexname{\'Indice Remissivo}%
  \def\figurename{Figura}%
  \def\tablename{Tabela}%
  \def\partname{Parte}%
  \def\enclname{Anexos}%
  \def\ccname{C\'opia a}%
  \def\headtoname{Para}%
  \def\pagename{P\'agina}%
  \def\seename{ver}%
  \def\alsoname{ver tamb\'em}%
  \def\proofname{Proof}%  <-- needs translation
  }%
%    \end{macrocode}
% \end{macro}
%
% \begin{macro}{\captionsbrazil}
% \changes{portuges-1.2g}{1995/06/04}{The coptions for brazilian and
%    portuguese are different now}
%
%    The ``captions'' are different for both versions of the language,
%    so we define the macro |\captionsbrazil| here.
%    \begin{macrocode}
\addto\captionsbrazil{%
  \def\prefacename{Pref\'acio}%
  \def\refname{Refer\^encias}%
  \def\abstractname{Resumo}%
  \def\bibname{Refer\^encias Bibliogr\'aficas}%
  \def\chaptername{Cap\'{\i}tulo}%
  \def\appendixname{Ap\^endice}%
  \def\contentsname{Sum\'ario}%
  \def\listfigurename{Lista de Figuras}%
  \def\listtablename{Lista de Tabelas}%
  \def\indexname{\'Indice}%
  \def\figurename{Figura}%
  \def\tablename{Tabela}%
  \def\partname{Parte}%
  \def\enclname{Anexo}%
  \def\ccname{C\'opia para}%
  \def\headtoname{Para}%
  \def\pagename{P\'agina}%
  \def\seename{veja}%
  \def\alsoname{veja tamb\'em}%
  }
%    \end{macrocode}
% \end{macro}
%
% \begin{macro}{\dateportuges}
%    The macro |\dateportuges| redefines the command |\today| to
%    produce Portuguese dates.
%    \begin{macrocode}
\def\dateportuges{%
\def\today{\number\day\space de\space\ifcase\month\or
  Janeiro\or Fevereiro\or Mar\c{c}o\or Abril\or Maio\or Junho\or
  Julho\or Agosto\or Setembro\or Outubro\or Novembro\or Dezembro\fi
  \space de\space\number\year}}
%    \end{macrocode}
% \end{macro}
%
% \begin{macro}{\datebrazil}
%    The macro |\datebrazil| redefines the command
%    |\today| to produce Brasilian dates, for which the names
%    of the months are not capitalized.
%    \begin{macrocode}
\def\datebrazil{%
\def\today{\number\day\space de\space\ifcase\month\or
  janeiro\or fevereiro\or mar\c{c}o\or abril\or maio\or junho\or
  julho\or agosto\or setembro\or outubro\or novembro\or dezembro\fi
  \space de\space\number\year}}
%    \end{macrocode}
% \end{macro}
%
%  \begin{macro}{\portugeshyphenmins}
%  \begin{macro}{\brasilhyphenmins}
% \changes{portuges-1.2g}{1995/06/04}{Added setting of hyphenmin
%    values}
%    Set correct values for |\lefthyphenmin| and |\righthyphenmin|.
%    \begin{macrocode}
\def\portugeshyphenmins{\tw@\tw@}
\def\brazilhyphenmins{\tw@\tw@}
%    \end{macrocode}
%  \end{macro}
%  \end{macro}
%
% \begin{macro}{\extrasportuges}
% \changes{portuges-1.2g}{1995/06/04}{Added using some \texttt{"}
%    shorthands}
% \begin{macro}{\noextrasportuges}
%    The macro |\extrasportuges| will perform all the extra
%    definitions needed for the Portuguese language. The macro
%    |\noextrasportuges| is used to cancel the actions of
%    |\extrasportuges|.
%
%    For Portuguese the \texttt{"} character is made active. This is
%    done once, later on its definition may vary. Other languages in
%    the same document may also use the \texttt{"} character for
%    shorthands; we specify that the portuguese group of shorthands
%    should be used.
%
%    \begin{macrocode}
\initiate@active@char{"}
\addto\extrasportuges{\languageshorthands{portuges}}
\addto\extrasportuges{\bbl@activate{"}}
%\addto\noextrasportuges{\bbl@deactivate{"}}
%    \end{macrocode}
%    First we define access to the guillemets for quotations,
%    \begin{macrocode}
\declare@shorthand{portuges}{"<}{%
  \textormath{\guillemotleft{}}{\mbox{\guillemotleft}}}
\declare@shorthand{portuges}{">}{%
  \textormath{\guillemotright{}}{\mbox{\guillemotright}}}
%    \end{macrocode}
%    then we define two shorthands to be able to specify hyphenation
%    breakpoints that behavew a little different from |\-|.
%    \begin{macrocode}
\declare@shorthand{portuges}{"-}{\allowhyphens-\allowhyphens}
\declare@shorthand{portuges}{""}{\hskip\z@skip}
%    \end{macrocode}
%    And we want to have a shorthand for disabling a ligature.
%    \begin{macrocode}
\declare@shorthand{portuges}{"|}{%
  \textormath{\discretionary{-}{}{\kern.03em}}{}}
%    \end{macrocode}
% \end{macro}
% \end{macro}
%
%  \begin{macro}{\-}
%
%    All that is left now is the redefinition of |\-|. The new version
%    of |\-| should indicate an extra hyphenation position, while
%    allowing other hyphenation positions to be generated
%    automatically. The standard behaviour of \TeX\ in this respect is
%    very unfortunate for languages such as Dutch and German, where
%    long compound words are quite normal and all one needs is a means
%    to indicate an extra hyphenation position on top of the ones that
%    \TeX\ can generate from the hyphenation patterns.
%    \begin{macrocode}
\addto\extrasportuges{\babel@save\-}
\addto\extrasportuges{\def\-{\allowhyphens
                          \discretionary{-}{}{}\allowhyphens}}
%    \end{macrocode}
%  \end{macro}
%
%  \begin{macro}{\ord}
% \changes{portuges-1.2g}{1995/06/04}{Added macro}
%  \begin{macro}{\ro}
% \changes{portuges-1.2g}{1995/06/04}{Added macro}
%  \begin{macro}{\orda}
% \changes{portuges-1.2g}{1995/06/04}{Added macro}
%  \begin{macro}{\ra}
% \changes{portuges-1.2g}{1995/06/04}{Added macro}
%    We also provide an easy way to typeset ordinals, both in the male
%    (|\ord| or |\ro|) and the female (|orda| or |\ra|) form.
%    \begin{macrocode}
\def\ord{$^{\rm o}$}
\def\orda{$^{\rm a}$}
\let\ro\ord\let\ra\orda
%    \end{macrocode}
%  \end{macro}
%  \end{macro}
%  \end{macro}
%  \end{macro}
%
% \begin{macro}{\extrasbrazil}
% \begin{macro}{\noextrasbrazil}
%    Also for the ``brazil'' variant no extra definitions are needed
%    at the moment.
%    \begin{macrocode}
\let\extrasbrazil\extrasportuges
\let\noextrasbrazil\noextrasportuges
%    \end{macrocode}
% \end{macro}
% \end{macro}
%
%    It is possible that a site might need to add some extra code to
%    the babel macros. To enable this we load a local configuration
%    file, \file{portuges.cfg} if it is found on \TeX' search path.
% \changes{portuges-1.2h}{1995/07/02}{Added loading of configuration
%    file}
%    \begin{macrocode}
\loadlocalcfg{portuges}
%    \end{macrocode}
%
%    Our last action is to make a note that the commands we have just
%    defined, will be executed by calling the macro |\selectlanguage|
%    at the beginning of the document.
% \changes{portuges-1.2f}{1995/03/14}{Use \cs{main@language} instead
%    of \cs{selectlanguage}}
%    \begin{macrocode}
\main@language{portuges}
%    \end{macrocode}
%    Finally, the category code of \texttt{@} is reset to its original
%    value. The macrospace used by |\atcatcode| is freed.
% \changes{portuges-1.0a}{1991/07/15}{Modified handling of catcode of
%    @-sign.}
%    \begin{macrocode}
\catcode`\@=\atcatcode \let\atcatcode\relax
%</code>
%    \end{macrocode}
%
% \Finale
%%
%% \CharacterTable
%%  {Upper-case    \A\B\C\D\E\F\G\H\I\J\K\L\M\N\O\P\Q\R\S\T\U\V\W\X\Y\Z
%%   Lower-case    \a\b\c\d\e\f\g\h\i\j\k\l\m\n\o\p\q\r\s\t\u\v\w\x\y\z
%%   Digits        \0\1\2\3\4\5\6\7\8\9
%%   Exclamation   \!     Double quote  \"     Hash (number) \#
%%   Dollar        \$     Percent       \%     Ampersand     \&
%%   Acute accent  \'     Left paren    \(     Right paren   \)
%%   Asterisk      \*     Plus          \+     Comma         \,
%%   Minus         \-     Point         \.     Solidus       \/
%%   Colon         \:     Semicolon     \;     Less than     \<
%%   Equals        \=     Greater than  \>     Question mark \?
%%   Commercial at \@     Left bracket  \[     Backslash     \\
%%   Right bracket \]     Circumflex    \^     Underscore    \_
%%   Grave accent  \`     Left brace    \{     Vertical bar  \|
%%   Right brace   \}     Tilde         \~}
%%
\endinput
}
\DeclareOption{portuguese}{% \iffalse meta-comment
%
% Copyright 1989-1995 Johannes L. Braams and any individual authors
% listed elsewhere in this file.  All rights reserved.
% 
% For further copyright information any other copyright notices in this
% file.
% 
% This file is part of the Babel system release 3.5.
% --------------------------------------------------
%   This system is distributed in the hope that it will be useful,
%   but WITHOUT ANY WARRANTY; without even the implied warranty of
%   MERCHANTABILITY or FITNESS FOR A PARTICULAR PURPOSE.
% 
%   For error reports concerning UNCHANGED versions of this file no more
%   than one year old, see bugs.txt.
% 
%   Please do not request updates from me directly.  Primary
%   distribution is through the CTAN archives.
% 
% 
% IMPORTANT COPYRIGHT NOTICE:
% 
% You are NOT ALLOWED to distribute this file alone.
% 
% You are allowed to distribute this file under the condition that it is
% distributed together with all the files listed in manifest.txt.
% 
% If you receive only some of these files from someone, complain!
% 
% Permission is granted to copy this file to another file with a clearly
% different name and to customize the declarations in that copy to serve
% the needs of your installation, provided that you comply with
% the conditions in the file legal.txt from the LaTeX2e distribution.
% 
% However, NO PERMISSION is granted to produce or to distribute a
% modified version of this file under its original name.
%  
% You are NOT ALLOWED to change this file.
% 
% 
% \fi
% \CheckSum{275}
% \iffalse
%    Tell the \LaTeX\ system who we are and write an entry on the
%    transcript.
%<*dtx>
\ProvidesFile{portuges.dtx}
%</dtx>
%<code>\ProvidesFile{portuges.ldf}
        [1995/07/04 v1.2h Portuguese support from the babel system]
%
% Babel package for LaTeX version 2e
% Copyright (C) 1989 - 1995
%           by Johannes Braams, TeXniek
%
% Portuguese Language Definition File
% Copyright (C) 1989 - 1995
%           by Johannes Braams, TeXniek
%
% Please report errors to: J.L. Braams
%                          JLBraams@cistron.nl
%
%    This file is part of the babel system, it provides the source
%    code for the Portuguese language definition file.  The Portuguese
%    words were contributed by Jose Pedro Ramalhete, (JRAMALHE@CERNVM
%    or Jose-Pedro_Ramalhete@MACMAIL).
%
%    Arnaldo Viegas de Lima <arnaldo@VNET.IBM.COM> contributed
%    brazilian translations and suggestions for enhancements.
%<*filedriver>
\documentclass{ltxdoc}
\newcommand*\TeXhax{\TeX hax}
\newcommand*\babel{\textsf{babel}}
\newcommand*\langvar{$\langle \it lang \rangle$}
\newcommand*\note[1]{}
\newcommand*\Lopt[1]{\textsf{#1}}
\newcommand*\file[1]{\texttt{#1}}
\begin{document}
 \DocInput{portuges.dtx}
\end{document}
%</filedriver>
%\fi
%
% \GetFileInfo{portuges.dtx}
%
% \changes{portuges-1.0a}{1991/07/15}{Renamed \file{babel.sty} in
%    \file{babel.com}}
% \changes{portuges-1.1}{1992/02/16}{Brought up-to-date with babel 3.2a}
% \changes{portuges-1.2}{1994/02/26}{Update for \LaTeXe}
% \changes{portuges-1.2d}{1994/06/26}{Removed the use of \cs{filedate}
%    and moved identification after the loading of \file{babel.def}}
% \changes{portuges-1.2g}{1995/06/04}{Enhanced support for brasilian}
%
%  \section{The Portuguese language}
%
%    The file \file{\filename}\footnote{The file described in this
%    section has version number \fileversion\ and was last revised on
%    \filedate.  Contributions were made by Jose Pedro Ramalhete
%    (\texttt{JRAMALHE@CERNVM} or
%    \texttt{Jose-Pedro\_Ramalhete@MACMAIL}) and Arnaldo Viegas de
%    Lima \texttt{arnaldo@VNET.IBM.COM}.}  defines all the
%    language-specific macros for the Portuguese language as well as
%    for the Brasilian version of this language.
%
%    For this language the character |"| is made active. In
%    table~\ref{tab:port-quote} an overview is given of its purpose.
%
%    \begin{table}[htb]
%     \centering
%     \begin{tabular}{lp{8cm}}
%       \verb="|= & disable ligature at this position.\\
%        |"-| & an explicit hyphen sign, allowing hyphenation
%               in the rest of the word.\\
%        |""| & like \verb="-=, but producing no hyphen sign (for
%              words that should break at some sign such as
%              ``entrada/salida.''\\
%        |"<| & for French left double quotes (similar to $<<$).\\
%        |">| & for French right double quotes (similar to $>>$).\\
%        |\-| & like the old |\-|, but allowing hyphenation
%               in the rest of the word. \\
%     \end{tabular}
%     \caption{The extra definitions made by \file{portuges.ldf}}
%     \label{tab:port-quote}
%    \end{table}
%
% \StopEventually{}

%    As this file needs to be read only once, we check whether it was
%    read before. If it was, the command |\captionsportuges| is
%    already defined, so we can stop processing. If this command is
%    undefined we proceed with the various definitions and first show
%    the current version of this file.
%
% \changes{portuges-1.0a}{1991/07/15}{Added reset of catcode of @
%    before \cs{endinput}.}
% \changes{portuges-1.0b}{1991/10/29}{Removed use of
%    \cs{@ifundefined}}
%    \begin{macrocode}
%<*code>
\ifx\undefined\captionsportuges
\else
  \selectlanguage{portuges}
  \expandafter\endinput
\fi
%    \end{macrocode}
%
% \changes{portuges-1.0b}{1991/10/29}{Removed code to load
%    \file{latexhax.com}}
%
% \begin{macro}{\atcatcode}
%    This file, \file{portuges.ldf}, may have been read while \TeX\ is
%    in the middle of processing a document, so we have to make sure
%    the category code of \texttt{@} is `letter' while this file is
%    being read.  We save the category code of the @-sign in
%    |\atcatcode| and make it `letter'. Later the category code can be
%    restored to whatever it was before.
%
% \changes{portuges-1.0a}{1991/07/15}{Modified handling of catcode of
%    @ again.}
% \changes{portuges-1.0b}{1991/10/ 29}{Removed use of
%    \cs{makeatletter} and hence the need to load \file{latexhax.com}}
%    \begin{macrocode}
\chardef\atcatcode=\catcode`\@
\catcode`\@=11\relax
%    \end{macrocode}
% \end{macro}
%
%    Now we determine whether the the common macros from the file
%    \file{babel.def} need to be read. We can be in one of two
%    situations: either another language option has been read earlier
%    on, in which case that other option has already read
%    \file{babel.def}, or \texttt{portuges} is the first language
%    option to be processed. In that case we need to read
%    \file{babel.def} right here before we continue.
%
% \changes{portuges-1.1}{1992/02/16}{Added \cs{relax} after the
%    argument of \cs{input}}
%    \begin{macrocode}
\ifx\undefined\babel@core@loaded\input babel.def\relax\fi
%    \end{macrocode}
%
%    Another check that has to be made, is if another language
%    definition file has been read already. In that case its
%    definitions have been activated. This might interfere with
%    definitions this file tries to make. Therefore we make sure that
%    we cancel any special definitions. This can be done by checking
%    the existence of the macro |\originalTeX|. If it exists we simply
%    execute it, otherwise it is |\let| to |\empty|.
% \changes{portuges-1.0a}{1991/07/ 15}{Added
%    \cs{let}\cs{originalTeX}\cs{relax} to test for existence}
% \changes{portuges-1.1}{1992/02/16}{\cs{originalTeX} should be
%    expandable, \cs{let} it to \cs{empty}}
%    \begin{macrocode}
\ifx\undefined\originalTeX \let\originalTeX\empty \else\originalTeX\fi
%    \end{macrocode}
%
%    When this file is read as an option, i.e. by the |\usepackage|
%    command, \texttt{portuges} will be an `unknown' language in which
%    case we have to make it known. So we check for the existence of
%    |\l@portuges| to see whether we have to do something here.
%
% \changes{portuges-1.0b}{1991/10/29}{Removed use of cs{@ifundefined}}
% \changes{portuges-1.1}{1992/02/16}{Added a warning when no
%    hyphenation patterns were loaded.}
% \changes{portuges-1.2d}{1994/06/26}{Now use \cs{@nopatterns} to
%    produce the warning}
%    \begin{macrocode}
\ifx\undefined\l@portuges
    \@nopatterns{Portuges}
    \adddialect\l@portuges0\fi
%    \end{macrocode}
%
%    For the Brasilian version of these definitions we just add a
%    ``dialect''. Also, the macros |\captionsbrazil| and
%    |\extrasbrazil| are |\let| to their Portuguese counterparts when
%    these parts are defined.
%    \begin{macrocode}
\adddialect\l@brazil\l@portuges
%    \end{macrocode}
%
%    The next step consists of defining commands to switch to (and from)
%    the Portuguese language.
%
% \begin{macro}{\captionsportuges}
%    The macro |\captionsportuges| defines all strings used
%    in the four standard documentclasses provided with \LaTeX.
% \changes{portuges-1.1}{1992/02/16}{Added \cs{seename}, \cs{alsoname}
%    and \cs{prefacename}}
% \changes{portuges-1.1}{1993/07/15}{\cs{headpagename} should be
%    \cs{pagename}}
% \changes{portuges-1.2e}{1994/11/09}{Added a few missing
%    translations}
% \changes{portuges-1.2h}{1995/07/04}{Added \cs{proofname} for
%    AMS-\LaTeX}
%    \begin{macrocode}
\addto\captionsportuges{%
  \def\prefacename{Pref\'acio}%
  \def\refname{Refer\^encias}%
  \def\abstractname{Resumo}%
  \def\bibname{Bibliografia}%
  \def\chaptername{Cap\'{\i}tulo}%
  \def\appendixname{Ap\^endice}%
  \def\contentsname{\'Indice}%
  \def\listfigurename{Lista de Figuras}%
  \def\listtablename{Lista de Tabelas}%
  \def\indexname{\'Indice Remissivo}%
  \def\figurename{Figura}%
  \def\tablename{Tabela}%
  \def\partname{Parte}%
  \def\enclname{Anexos}%
  \def\ccname{C\'opia a}%
  \def\headtoname{Para}%
  \def\pagename{P\'agina}%
  \def\seename{ver}%
  \def\alsoname{ver tamb\'em}%
  \def\proofname{Proof}%  <-- needs translation
  }%
%    \end{macrocode}
% \end{macro}
%
% \begin{macro}{\captionsbrazil}
% \changes{portuges-1.2g}{1995/06/04}{The coptions for brazilian and
%    portuguese are different now}
%
%    The ``captions'' are different for both versions of the language,
%    so we define the macro |\captionsbrazil| here.
%    \begin{macrocode}
\addto\captionsbrazil{%
  \def\prefacename{Pref\'acio}%
  \def\refname{Refer\^encias}%
  \def\abstractname{Resumo}%
  \def\bibname{Refer\^encias Bibliogr\'aficas}%
  \def\chaptername{Cap\'{\i}tulo}%
  \def\appendixname{Ap\^endice}%
  \def\contentsname{Sum\'ario}%
  \def\listfigurename{Lista de Figuras}%
  \def\listtablename{Lista de Tabelas}%
  \def\indexname{\'Indice}%
  \def\figurename{Figura}%
  \def\tablename{Tabela}%
  \def\partname{Parte}%
  \def\enclname{Anexo}%
  \def\ccname{C\'opia para}%
  \def\headtoname{Para}%
  \def\pagename{P\'agina}%
  \def\seename{veja}%
  \def\alsoname{veja tamb\'em}%
  }
%    \end{macrocode}
% \end{macro}
%
% \begin{macro}{\dateportuges}
%    The macro |\dateportuges| redefines the command |\today| to
%    produce Portuguese dates.
%    \begin{macrocode}
\def\dateportuges{%
\def\today{\number\day\space de\space\ifcase\month\or
  Janeiro\or Fevereiro\or Mar\c{c}o\or Abril\or Maio\or Junho\or
  Julho\or Agosto\or Setembro\or Outubro\or Novembro\or Dezembro\fi
  \space de\space\number\year}}
%    \end{macrocode}
% \end{macro}
%
% \begin{macro}{\datebrazil}
%    The macro |\datebrazil| redefines the command
%    |\today| to produce Brasilian dates, for which the names
%    of the months are not capitalized.
%    \begin{macrocode}
\def\datebrazil{%
\def\today{\number\day\space de\space\ifcase\month\or
  janeiro\or fevereiro\or mar\c{c}o\or abril\or maio\or junho\or
  julho\or agosto\or setembro\or outubro\or novembro\or dezembro\fi
  \space de\space\number\year}}
%    \end{macrocode}
% \end{macro}
%
%  \begin{macro}{\portugeshyphenmins}
%  \begin{macro}{\brasilhyphenmins}
% \changes{portuges-1.2g}{1995/06/04}{Added setting of hyphenmin
%    values}
%    Set correct values for |\lefthyphenmin| and |\righthyphenmin|.
%    \begin{macrocode}
\def\portugeshyphenmins{\tw@\tw@}
\def\brazilhyphenmins{\tw@\tw@}
%    \end{macrocode}
%  \end{macro}
%  \end{macro}
%
% \begin{macro}{\extrasportuges}
% \changes{portuges-1.2g}{1995/06/04}{Added using some \texttt{"}
%    shorthands}
% \begin{macro}{\noextrasportuges}
%    The macro |\extrasportuges| will perform all the extra
%    definitions needed for the Portuguese language. The macro
%    |\noextrasportuges| is used to cancel the actions of
%    |\extrasportuges|.
%
%    For Portuguese the \texttt{"} character is made active. This is
%    done once, later on its definition may vary. Other languages in
%    the same document may also use the \texttt{"} character for
%    shorthands; we specify that the portuguese group of shorthands
%    should be used.
%
%    \begin{macrocode}
\initiate@active@char{"}
\addto\extrasportuges{\languageshorthands{portuges}}
\addto\extrasportuges{\bbl@activate{"}}
%\addto\noextrasportuges{\bbl@deactivate{"}}
%    \end{macrocode}
%    First we define access to the guillemets for quotations,
%    \begin{macrocode}
\declare@shorthand{portuges}{"<}{%
  \textormath{\guillemotleft{}}{\mbox{\guillemotleft}}}
\declare@shorthand{portuges}{">}{%
  \textormath{\guillemotright{}}{\mbox{\guillemotright}}}
%    \end{macrocode}
%    then we define two shorthands to be able to specify hyphenation
%    breakpoints that behavew a little different from |\-|.
%    \begin{macrocode}
\declare@shorthand{portuges}{"-}{\allowhyphens-\allowhyphens}
\declare@shorthand{portuges}{""}{\hskip\z@skip}
%    \end{macrocode}
%    And we want to have a shorthand for disabling a ligature.
%    \begin{macrocode}
\declare@shorthand{portuges}{"|}{%
  \textormath{\discretionary{-}{}{\kern.03em}}{}}
%    \end{macrocode}
% \end{macro}
% \end{macro}
%
%  \begin{macro}{\-}
%
%    All that is left now is the redefinition of |\-|. The new version
%    of |\-| should indicate an extra hyphenation position, while
%    allowing other hyphenation positions to be generated
%    automatically. The standard behaviour of \TeX\ in this respect is
%    very unfortunate for languages such as Dutch and German, where
%    long compound words are quite normal and all one needs is a means
%    to indicate an extra hyphenation position on top of the ones that
%    \TeX\ can generate from the hyphenation patterns.
%    \begin{macrocode}
\addto\extrasportuges{\babel@save\-}
\addto\extrasportuges{\def\-{\allowhyphens
                          \discretionary{-}{}{}\allowhyphens}}
%    \end{macrocode}
%  \end{macro}
%
%  \begin{macro}{\ord}
% \changes{portuges-1.2g}{1995/06/04}{Added macro}
%  \begin{macro}{\ro}
% \changes{portuges-1.2g}{1995/06/04}{Added macro}
%  \begin{macro}{\orda}
% \changes{portuges-1.2g}{1995/06/04}{Added macro}
%  \begin{macro}{\ra}
% \changes{portuges-1.2g}{1995/06/04}{Added macro}
%    We also provide an easy way to typeset ordinals, both in the male
%    (|\ord| or |\ro|) and the female (|orda| or |\ra|) form.
%    \begin{macrocode}
\def\ord{$^{\rm o}$}
\def\orda{$^{\rm a}$}
\let\ro\ord\let\ra\orda
%    \end{macrocode}
%  \end{macro}
%  \end{macro}
%  \end{macro}
%  \end{macro}
%
% \begin{macro}{\extrasbrazil}
% \begin{macro}{\noextrasbrazil}
%    Also for the ``brazil'' variant no extra definitions are needed
%    at the moment.
%    \begin{macrocode}
\let\extrasbrazil\extrasportuges
\let\noextrasbrazil\noextrasportuges
%    \end{macrocode}
% \end{macro}
% \end{macro}
%
%    It is possible that a site might need to add some extra code to
%    the babel macros. To enable this we load a local configuration
%    file, \file{portuges.cfg} if it is found on \TeX' search path.
% \changes{portuges-1.2h}{1995/07/02}{Added loading of configuration
%    file}
%    \begin{macrocode}
\loadlocalcfg{portuges}
%    \end{macrocode}
%
%    Our last action is to make a note that the commands we have just
%    defined, will be executed by calling the macro |\selectlanguage|
%    at the beginning of the document.
% \changes{portuges-1.2f}{1995/03/14}{Use \cs{main@language} instead
%    of \cs{selectlanguage}}
%    \begin{macrocode}
\main@language{portuges}
%    \end{macrocode}
%    Finally, the category code of \texttt{@} is reset to its original
%    value. The macrospace used by |\atcatcode| is freed.
% \changes{portuges-1.0a}{1991/07/15}{Modified handling of catcode of
%    @-sign.}
%    \begin{macrocode}
\catcode`\@=\atcatcode \let\atcatcode\relax
%</code>
%    \end{macrocode}
%
% \Finale
%%
%% \CharacterTable
%%  {Upper-case    \A\B\C\D\E\F\G\H\I\J\K\L\M\N\O\P\Q\R\S\T\U\V\W\X\Y\Z
%%   Lower-case    \a\b\c\d\e\f\g\h\i\j\k\l\m\n\o\p\q\r\s\t\u\v\w\x\y\z
%%   Digits        \0\1\2\3\4\5\6\7\8\9
%%   Exclamation   \!     Double quote  \"     Hash (number) \#
%%   Dollar        \$     Percent       \%     Ampersand     \&
%%   Acute accent  \'     Left paren    \(     Right paren   \)
%%   Asterisk      \*     Plus          \+     Comma         \,
%%   Minus         \-     Point         \.     Solidus       \/
%%   Colon         \:     Semicolon     \;     Less than     \<
%%   Equals        \=     Greater than  \>     Question mark \?
%%   Commercial at \@     Left bracket  \[     Backslash     \\
%%   Right bracket \]     Circumflex    \^     Underscore    \_
%%   Grave accent  \`     Left brace    \{     Vertical bar  \|
%%   Right brace   \}     Tilde         \~}
%%
\endinput
%
  \let\captionsportuguese\captionsportuges
  \let\dateportuguese\dateportuges
  \let\extrasportuguese\extrasportuges
  \let\noextrasportuguese\noextrasportuges
  \let\portuguesehyphenmins\portugeshyphenmins
}
\DeclareOption{romanian}{% \iffalse meta-comment
%
% Copyright 1989-1995 Johannes L. Braams and any individual authors
% listed elsewhere in this file.  All rights reserved.
% 
% For further copyright information any other copyright notices in this
% file.
% 
% This file is part of the Babel system release 3.5.
% --------------------------------------------------
%   This system is distributed in the hope that it will be useful,
%   but WITHOUT ANY WARRANTY; without even the implied warranty of
%   MERCHANTABILITY or FITNESS FOR A PARTICULAR PURPOSE.
% 
%   For error reports concerning UNCHANGED versions of this file no more
%   than one year old, see bugs.txt.
% 
%   Please do not request updates from me directly.  Primary
%   distribution is through the CTAN archives.
% 
% 
% IMPORTANT COPYRIGHT NOTICE:
% 
% You are NOT ALLOWED to distribute this file alone.
% 
% You are allowed to distribute this file under the condition that it is
% distributed together with all the files listed in manifest.txt.
% 
% If you receive only some of these files from someone, complain!
% 
% Permission is granted to copy this file to another file with a clearly
% different name and to customize the declarations in that copy to serve
% the needs of your installation, provided that you comply with
% the conditions in the file legal.txt from the LaTeX2e distribution.
% 
% However, NO PERMISSION is granted to produce or to distribute a
% modified version of this file under its original name.
%  
% You are NOT ALLOWED to change this file.
% 
% 
% \fi
% \CheckSum{121}
% \iffalse
%    Tell the \LaTeX\ system who we are and write an entry on the
%    transcript.
%<*dtx>
\ProvidesFile{romanian.dtx}
%</dtx>
%<code>\ProvidesFile{romanian.ldf}
        [1995/07/04 v1.2f Romanian support from the babel system]
%
% Babel package for LaTeX version 2e
% Copyright (C) 1989 - 1995
%           by Johannes Braams, TeXniek
%
% Please report errors to: J.L. Braams
%                          JLBraams@cistron.nl
%
%    This file is part of the babel system, it provides the source
%    code for the Romanian language definition file. A contribution
%    was made by Umstatter Horst (hhu@cernvm.cern.ch) and Robert
%    Juhasz (robertj@uni-paderborn.de)
%<*filedriver>
\documentclass{ltxdoc}
\newcommand*\TeXhax{\TeX hax}
\newcommand*\babel{\textsf{babel}}
\newcommand*\langvar{$\langle \it lang \rangle$}
\newcommand*\note[1]{}
\newcommand*\Lopt[1]{\textsf{#1}}
\newcommand*\file[1]{\texttt{#1}}
\begin{document}
 \DocInput{romanian.dtx}
\end{document}
%</filedriver>
%\fi
%
% \GetFileInfo{romanian.dtx}
%
% \changes{romanian-1.0a}{1991/07/15}{Renamed babel.sty in babel.com}
% \changes{romanian-1.1}{1992/02/16}{Brought up-to-date with babel 3.2a}
% \changes{romanian-1.2}{1994/02/27}{Update for LaTeX2e}
% \changes{romanian-1.2d}{1994/06/26}{Removed the use of \cs{filedate}
%    and moved identification after the loading of babel.def}
% \changes{romanian-1.2e}{1995/05/25}{Updated for babel release 3.5}
%
%  \section{The Romanian language}
%
%    The file \file{\filename}\footnote{The file described in this
%    section has version number \fileversion\ and was last revised on
%    \filedate.  A contribution was made by Umstatter Horst
%    (\texttt{hhu@cernvm.cern.ch}).}  defines all the
%    language-specific macros for the Romanian language.
%
%    For this language currently no special definitions are needed or
%    available.
%
% \StopEventually{}
%
%    As this file needs to be read only once, we check whether it was
%    read before. If it was, the command |\captionsromanian| is
%    already defined, so we can stop processing. If this command is
%    undefined we proceed with the various definitions and first show
%    the current version of this file.
%
% \changes{romanian-1.0a}{1991/07/15}{Added reset of catcode of @
%    before \cs{endinput}.}
% \changes{romanian-1.0b}{1991/10/29}{Removed use of
%    \cs{@ifundefined}}
% \changes{romanian-1.1b}{1993/11/5}{Added translations}
%    \begin{macrocode}
%<*code>
\ifx\undefined\captionsromanian
\else
  \selectlanguage{romanian}
  \expandafter\endinput
\fi
%    \end{macrocode}
%
% \changes{romanian-1.0b}{1991/10/29}{Removed code to load
%    \file{latexhax.com}}
%
% \begin{macro}{\atcatcode}
%    This file, \file{romanian.sty}, may have been read while \TeX\ is
%    in the middle of processing a document, so we have to make sure
%    the category code of \texttt{@} is `letter' while this file is
%    being read.  We save the category code of the @-sign in
%    |\atcatcode| and make it `letter'. Later the category code can be
%    restored to whatever it was before.
%
% \changes{romanian-1.0a}{1991/07/15}{Modified handling of catcode of
%    @ again.}
% \changes{romanian-1.0b}{1991/10/29}{Removed use of \cs{makeatletter}
%    and hence the need to load \file{latexhax.com}}
%    \begin{macrocode}
\chardef\atcatcode=\catcode`\@
\catcode`\@=11\relax
%    \end{macrocode}
% \end{macro}
%
%    Now we determine whether the the common macros from the file
%    \file{babel.def} need to be read. We can be in one of two
%    situations: either another language option has been read earlier
%    on, in which case that other option has already read
%    \file{babel.def}, or \texttt{romanian} is the first language option
%    to be processed. In that case we need to read \file{babel.def}
%    right here before we continue.
%
% \changes{romanian-1.1}{1992/02/16}{Added \cs{relax} after the
%    argument of \cs{input}}
%    \begin{macrocode}
\ifx\undefined\babel@core@loaded\input babel.def\relax\fi
%    \end{macrocode}
%
%    Another check that has to be made, is if another language
%    definition file has been read already. In that case its
%    definitions have been activated. This might interfere with
%    definitions this file tries to make. Therefore we make sure that
%    we cancel any special definitions. This can be done by checking
%    the existence of the macro |\originalTeX|. If it exists we simply
%    execute it, otherwise it is |\let| to |\empty|.
% \changes{romanian-1.0a}{1991/07/15}{Added
%    \cs{let}\cs{originalTeX}\cs{relax} to test for existence}
% \changes{romanian-1.1}{1992/02/16}{\cs{originalTeX} should be
%    expandable, \cs{let} it to \cs{empty}}
%    \begin{macrocode}
\ifx\undefined\originalTeX \let\originalTeX\empty \else\originalTeX\fi
%    \end{macrocode}
%
%    When this file is read as an option, i.e. by the |\usepackage|
%    command, \texttt{romanian} will be an `unknown' language in which
%    case we have to make it known. So we check for the existence of
%    |\l@romanian| to see whether we have to do something here.
%
% \changes{romanian-1.0b}{1991/10/29}{Removed use of
%    \cs{@ifundefined}}
% \changes{romanian-1.1}{1992/02/16}{Added a warning when no
%    hyphenation patterns were loaded.}
% \changes{romanian-1.2d}{1994/06/26}{Now use \cs{@nopatterns} to
%    produce the warning}
%    \begin{macrocode}
\ifx\undefined\l@romanian
    \@nopatterns{Romanian}
    \adddialect\l@romanian0\fi
%    \end{macrocode}
%
%    The next step consists of defining commands to switch to (and
%    from) the Romanian language.
%
% \begin{macro}{\captionsromanian}
%    The macro |\captionsromanian| defines all strings used in the
%    four standard documentlasses provided with \LaTeX.
% \changes{romanian-1.1}{1992/02/16}{Added \cs{seename}, \cs{alsoname}
%    and \cs{prefacename}}
% \changes{romanian-1.1}{1992/02/17}{Translation errors found by
%    Robert Juhasz fixed}
% \changes{romanian-1.1}{1993/07/15}{\cs{headpagename} should be
%    \cs{pagename}}
% \changes{romanian-1.2f}{1995/07/04}{Added \cs{proofname} for
%    AMS-\LaTeX}
%    \begin{macrocode}
\addto\captionsromanian{%
  \def\prefacename{Prefa\c{t}\u{a}}%
  \def\refname{Bibliografie}%
  \def\abstractname{Rezumat}%
  \def\bibname{Bibliografie}%
  \def\chaptername{Capitolul}%
  \def\appendixname{Anexa}%
  \def\contentsname{Cuprins}%
  \def\listfigurename{List\u{a} de figuri}%
  \def\listtablename{List\u{a} de tabele}%
  \def\indexname{Glosar}%
  \def\figurename{Figura}%    % sau Plan\c{s}a
  \def\tablename{Tabela}%
  \def\partname{Partea}%
  \def\enclname{Anex\u{a}}%   % sau Anexe
  \def\ccname{Copie}%
  \def\headtoname{Pentru}%
  \def\pagename{Pagina}%
  \def\seename{Vezi}%
  \def\alsoname{Vezi de asemenea}%
  \def\proofname{Proof}%   <-- needs translation
  }%
%    \end{macrocode}
% \end{macro}
%
% \begin{macro}{\dateromanian}
%    The macro |\dateromanian| redefines the command |\today| to
%    produce Romanian dates.
% \changes{romanian-1.1}{1992/02/17}{Translation errors found by Robert
%    Juhasz fixed}
%    \begin{macrocode}
\def\dateromanian{%
\def\today{\number\day~\ifcase\month\or
  ianuarie\or februarie\or martie\or aprilie\or mai\or
  iunie\or iulie\or august\or septembrie\or octombrie\or
  noiembrie\or decembrie\fi
  \space \number\year}}
%    \end{macrocode}
% \end{macro}
%
% \begin{macro}{\extrasromanian}
% \begin{macro}{\noextrasromanian}
%    The macro |\extrasromanian| will perform all the extra
%    definitions needed for the Romanian language. The macro
%    |\noextrasromanian| is used to cancel the actions of
%    |\extrasromanian| For the moment these macros are empty but they
%    are defined for compatibility with the other language definition
%    files.
%
%    \begin{macrocode}
\addto\extrasromanian{}
\addto\noextrasromanian{}
%    \end{macrocode}
% \end{macro}
% \end{macro}
%
%    It is possible that a site might need to add some extra code to
%    the babel macros. To enable this we load a local configuration
%    file, \file{romanian.cfg} if it is found on \TeX' search path.
% \changes{romanian-1.2f}{1995/07/02}{Added loading of configuration
%    file}
%    \begin{macrocode}
\loadlocalcfg{romanian}
%    \end{macrocode}
%
%    Our last action is to make a note that the commands we have just
%    defined, will be executed by calling the macro |\selectlanguage|
%    at the beginning of the document.
%    \begin{macrocode}
\main@language{romanian}
%    \end{macrocode}
%    Finally, the category code of \texttt{@} is reset to its original
%    value. The macrospace used by |\atcatcode| is freed.
% \changes{romanian-1.0a}{1991/07/15}{Modified handling of catcode of
%    @-sign.}
%    \begin{macrocode}
\catcode`\@=\atcatcode \let\atcatcode\relax
%</code>
%    \end{macrocode}
%
% \Finale
%%
%% \CharacterTable
%%  {Upper-case    \A\B\C\D\E\F\G\H\I\J\K\L\M\N\O\P\Q\R\S\T\U\V\W\X\Y\Z
%%   Lower-case    \a\b\c\d\e\f\g\h\i\j\k\l\m\n\o\p\q\r\s\t\u\v\w\x\y\z
%%   Digits        \0\1\2\3\4\5\6\7\8\9
%%   Exclamation   \!     Double quote  \"     Hash (number) \#
%%   Dollar        \$     Percent       \%     Ampersand     \&
%%   Acute accent  \'     Left paren    \(     Right paren   \)
%%   Asterisk      \*     Plus          \+     Comma         \,
%%   Minus         \-     Point         \.     Solidus       \/
%%   Colon         \:     Semicolon     \;     Less than     \<
%%   Equals        \=     Greater than  \>     Question mark \?
%%   Commercial at \@     Left bracket  \[     Backslash     \\
%%   Right bracket \]     Circumflex    \^     Underscore    \_
%%   Grave accent  \`     Left brace    \{     Vertical bar  \|
%%   Right brace   \}     Tilde         \~}
%%
\endinput
}
%\DeclareOption{russian}{\input{russian.ldf}}
\DeclareOption{scottish}{% \iffalse meta-comment
%
% Copyright 1989-1995 Johannes L. Braams and any individual authors
% listed elsewhere in this file.  All rights reserved.
% 
% For further copyright information any other copyright notices in this
% file.
% 
% This file is part of the Babel system release 3.5.
% --------------------------------------------------
%   This system is distributed in the hope that it will be useful,
%   but WITHOUT ANY WARRANTY; without even the implied warranty of
%   MERCHANTABILITY or FITNESS FOR A PARTICULAR PURPOSE.
% 
%   For error reports concerning UNCHANGED versions of this file no more
%   than one year old, see bugs.txt.
% 
%   Please do not request updates from me directly.  Primary
%   distribution is through the CTAN archives.
% 
% 
% IMPORTANT COPYRIGHT NOTICE:
% 
% You are NOT ALLOWED to distribute this file alone.
% 
% You are allowed to distribute this file under the condition that it is
% distributed together with all the files listed in manifest.txt.
% 
% If you receive only some of these files from someone, complain!
% 
% Permission is granted to copy this file to another file with a clearly
% different name and to customize the declarations in that copy to serve
% the needs of your installation, provided that you comply with
% the conditions in the file legal.txt from the LaTeX2e distribution.
% 
% However, NO PERMISSION is granted to produce or to distribute a
% modified version of this file under its original name.
%  
% You are NOT ALLOWED to change this file.
% 
% 
% \fi
% \CheckSum{125}
%
% \iffalse
%    Tell the \LaTeX\ system who we are and write an entry on the
%    transcript.
%<*dtx>
\ProvidesFile{scottish.dtx}
%</dtx>
%<code>\ProvidesFile{scottish.ldf}
        [1995/07/10 v1.0c scottish support from the babel system]
%
% Babel package for LaTeX version 2e
% Copyright (C) 1989 -- 1995
%           by Johannes Braams, TeXniek
%
% Please report errors to: J.L. Braams
%                          JLBraams@cistron.nl
%
%    This file is part of the babel system, it provides the source
%    code for the scottish language definition file.
%
%    The Gaidhlig or Scottish Gaelic terms were provided by Fraser
%    Grant \texttt{FRASER@CERNVM}.
%<*filedriver>
\documentclass{ltxdoc}
\newcommand*{\TeXhax}{\TeX hax}
\newcommand*{\babel}{\textsf{babel}}
\newcommand*{\langvar}{$\langle \mathit lang \rangle$}
\newcommand*{\note}[1]{}
\newcommand*{\Lopt}[1]{\textsf{#1}}
\newcommand*{\file}[1]{\texttt{#1}}
\begin{document}
 \DocInput{scottish.dtx}
\end{document}
%</filedriver>
%\fi
% \GetFileInfo{scottish.dtx}
%
% \changes{scottish-1.0b}{1995/06/14}{Corrected typos (PR1652)}
%
%  \section{The scottish language}
%
%    The file \file{\filename}\footnote{The file described in this
%    section has version number \fileversion\ and was last revised on
%    \filedate. A contribution was made by Fraser Grant
%    (\texttt{FRASER@CERNVM}).}  defines all the language definition
%    macros for the scottish language.
%
% \StopEventually{}
%
%    As this file needs to be read only once, we check whether it was
%    read before. If it was, the command |\captionsscottish| is
%    already defined, so we can stop processing. If this command is
%    undefined we proceed with the various definitions and first show
%    the current version of this file.
%
%    \begin{macrocode}
%<*code>
\ifx\undefined\captionsscottish
\else
  \selectlanguage{scottish}
  \expandafter\endinput
\fi
%    \end{macrocode}
%
% \begin{macro}{\atcatcode}
%    This file, \file{scottish.ldf}, may have been read while \TeX\
%    is in the middle of processing a document, so we have to make
%    sure the category code of \texttt{@} is `letter' while this file
%    is being read.  We save the category code of the @-sign in
%    |\atcatcode| and make it `letter'. Later the category code can be
%    restored to whatever it was before.
%    \begin{macrocode}
\chardef\atcatcode=\catcode`\@
\catcode`\@=11\relax
%    \end{macrocode}
% \end{macro}
%
%    Now we determine whether the the common macros from the file
%    \file{babel.def} need to be read. We can be in one of two
%    situations: either another language option has been read earlier
%    on, in which case that other option has already read
%    \file{babel.def}, or \texttt{scottish} is the first language
%    option to be processed. In that case we need to read
%    \file{babel.def} right here before we continue.
%
%    \begin{macrocode}
\ifx\undefined\babel@core@loaded\input babel.def\relax\fi
%    \end{macrocode}
%
%    Another check that has to be made, is if another language
%    definition file has been read already. In that case its
%    definitions have been activated. This might interfere with
%    definitions this file tries to make. Therefore we make sure that
%    we cancel any special definitions. This can be done by checking
%    the existence of the macro |\originalTeX|. If it exists we simply
%    execute it.
%    \begin{macrocode}
\ifx\undefined\originalTeX
  \let\originalTeX\empty
\fi
\originalTeX
%    \end{macrocode}
%
%    When this file is read as an option, i.e. by the |\usepackage|
%    command, \texttt{scottish} could be an `unknown' language in
%    which case we have to make it known.  So we check for the
%    existence of |\l@scottish| to see whether we have to do something
%    here.
%
%    \begin{macrocode}
\ifx\undefined\l@scottish
  \@nopatterns{scottish}
  \adddialect\l@scottish0\fi
%    \end{macrocode}
%    The next step consists of defining commands to switch to (and
%    from) the scottish language.
%
% \begin{macro}{\captionsscottish}
%    The macro |\captionsscottish| defines all strings used in the
%    four standard documentclasses provided with \LaTeX.
% \changes{scottish-1.0c}{1995/07/04}{Added \cs{proofname} for
%    AMS-\LaTeX}
%    \begin{macrocode}
\addto\captionsscottish{%
  \def\prefacename{Preface}%    <-- needs translation
  \def\refname{Iomraidh}%
  \def\abstractname{Br\`{\i}gh}%
  \def\bibname{Leabhraichean}%
  \def\chaptername{Caibideil}%
  \def\appendixname{Ath-sgr`{\i}obhadh}%
  \def\contentsname{Cl\`ar-obrach}%
  \def\listfigurename{Liosta Dhealbh }%
  \def\listtablename{Liosta Chl\`ar}%
  \def\indexname{Cl\`ar-innse}%
  \def\figurename{Dealbh}%
  \def\tablename{Cl\`ar}%
  \def\partname{Cuid}%
  \def\enclname{a-staigh}%
  \def\ccname{lethbhreac gu}%
  \def\headtoname{gu}%
  \def\pagename{t.d.}%             abrv. `taobh duilleag'
  \def\seename{see}%    <-- needs translation
  \def\alsoname{see also}%    <-- needs translation
  \def\proofname{Proof}%    <-- needs translation
}
%    \end{macrocode}
% \end{macro}
%
% \begin{macro}{\datescottish}
%    The macro |\datescottish| redefines the command |\today| to
%    produce Scottish dates.
%    \begin{macrocode}
\def\datescottish{%
  \number\day\space \ifcase\month\or
  am Faoilteach\or an Gearran\or am M\`art\or an Giblean\or
  an C\`eitean\or an t-\`Og mhios\or an t-Iuchar\or
  L\`unasdal\or an Sultuine\or an D\`amhar\or
  an t-Samhainn\or an Dubhlachd\fi
  \space \number\year}}
%    \end{macrocode}
% \end{macro}
%
% \begin{macro}{\extrasscottish}
% \begin{macro}{\noextrasscottish}
%    The macro |\extrasscottish| will perform all the extra
%    definitions needed for the Scottish language. The macro
%    |\noextrasscottish| is used to cancel the actions of
%    |\extrasscottish|.  For the moment these macros are empty but
%    they are defined for compatibility with the other language
%    definition files.
%
%    \begin{macrocode}
\addto\extrasscottish{}
\addto\noextrasscottish{}
%    \end{macrocode}
% \end{macro}
% \end{macro}
%
%    It is possible that a site might need to add some extra code to
%    the babel macros. To enable this we load a local configuration
%    file, \file{scottish.cfg} if it is found on \TeX' search path.
% \changes{scottish-1.0c}{1995/07/02}{Added loading of configuration
%    file}
%    \begin{macrocode}
\loadlocalcfg{scottish}
%    \end{macrocode}
%
%    Our last action is to make a note that the commands we have just
%    defined, will be executed by calling the macro |\selectlanguage|
%    at the beginning of the document.
%    \begin{macrocode}
\main@language{scottish}
%    \end{macrocode}
%    Finally, the category code of \texttt{@} is reset to its original
%    value. The macrospace used by |\atcatcode| is freed.
%    \begin{macrocode}
\catcode`\@=\atcatcode \let\atcatcode\relax
%</code>
%    \end{macrocode}
%
% \Finale
%\endinput
%% \CharacterTable
%%  {Upper-case    \A\B\C\D\E\F\G\H\I\J\K\L\M\N\O\P\Q\R\S\T\U\V\W\X\Y\Z
%%   Lower-case    \a\b\c\d\e\f\g\h\i\j\k\l\m\n\o\p\q\r\s\t\u\v\w\x\y\z
%%   Digits        \0\1\2\3\4\5\6\7\8\9
%%   Exclamation   \!     Double quote  \"     Hash (number) \#
%%   Dollar        \$     Percent       \%     Ampersand     \&
%%   Acute accent  \'     Left paren    \(     Right paren   \)
%%   Asterisk      \*     Plus          \+     Comma         \,
%%   Minus         \-     Point         \.     Solidus       \/
%%   Colon         \:     Semicolon     \;     Less than     \<
%%   Equals        \=     Greater than  \>     Question mark \?
%%   Commercial at \@     Left bracket  \[     Backslash     \\
%%   Right bracket \]     Circumflex    \^     Underscore    \_
%%   Grave accent  \`     Left brace    \{     Vertical bar  \|
%%   Right brace   \}     Tilde         \~}
%%
}
\DeclareOption{spanish}{% \iffalse meta-comment
%
% Copyright 1989-1995 Johannes L. Braams and any individual authors
% listed elsewhere in this file.  All rights reserved.
% 
% For further copyright information any other copyright notices in this
% file.
% 
% This file is part of the Babel system release 3.5.
% --------------------------------------------------
%   This system is distributed in the hope that it will be useful,
%   but WITHOUT ANY WARRANTY; without even the implied warranty of
%   MERCHANTABILITY or FITNESS FOR A PARTICULAR PURPOSE.
% 
%   For error reports concerning UNCHANGED versions of this file no more
%   than one year old, see bugs.txt.
% 
%   Please do not request updates from me directly.  Primary
%   distribution is through the CTAN archives.
% 
% 
% IMPORTANT COPYRIGHT NOTICE:
% 
% You are NOT ALLOWED to distribute this file alone.
% 
% You are allowed to distribute this file under the condition that it is
% distributed together with all the files listed in manifest.txt.
% 
% If you receive only some of these files from someone, complain!
% 
% Permission is granted to copy this file to another file with a clearly
% different name and to customize the declarations in that copy to serve
% the needs of your installation, provided that you comply with
% the conditions in the file legal.txt from the LaTeX2e distribution.
% 
% However, NO PERMISSION is granted to produce or to distribute a
% modified version of this file under its original name.
%  
% You are NOT ALLOWED to change this file.
% 
% 
% \fi
% \CheckSum{343}
% \iffalse
%    Tell the \LaTeX\ system who we are and write an entry on the
%    transcript.
%<*dtx>
\ProvidesFile{spanish.dtx}
%</dtx>
%<code>\ProvidesFile{spanish.ldf}
        [1995/07/08 v3.4c Spanish support from the babel system]
%
% Babel package for LaTeX version 2e
% Copyright (C) 1989 - 1995
%           by Johannes Braams, TeXniek
%
% Spanish Language Definition File
% Copyright (C) 1991 - 1995
%           by Julio Sanchez
%              GMV, SA
%              c/ Isaac Newton 11
%              PTM - Tres Cantos
%              E-28760 Madrid
%              Spain
%              tel: +34 1 807 21 85
%              fax +34 1 807 21 99
%              jsanchez@gmv.es
%
%              Johannes Braams, TeXniek
%
% Please report errors to: Julio Sanchez <jsanchez@gmv.es>
%                          (or J.L. Braams <JLBraams@cistron.nl)
%
%    This file is part of the babel system, it provides the source
%    code for the Spanish language definition file.  The original
%    version of this file was written by Julio Sanchez,
%    (jsanchez@gmv.es) The code for the catalan language has been
%    removed and now is in an independent file.
%<*filedriver>
\documentclass{ltxdoc}
\newcommand*\TeXhax{\TeX hax}
\newcommand*\babel{\textsf{babel}}
\newcommand*\langvar{$\langle \it lang \rangle$}
\newcommand*\note[1]{}
\newcommand*\Lopt[1]{\textsf{#1}}
\newcommand*\file[1]{\texttt{#1}}
\begin{document}
 \DocInput{spanish.dtx}
\end{document}
%</filedriver>
%\fi
% \GetFileInfo{spanish.dtx}
%
% \changes{spanish-1.1}{1990/08/19}{Date format corrected.  Wrong
%    change history deleted}
% \changes{spanish-1.1a}{1990/08/27}{\cs{I} does not exist, modified}
% \changes{spanish-2.0}{1991/04/23}{Modified for babel 3.0}
% \changes{spanish-2.0a}{1991/05/23}{removed use of \cs{setlanguage}}
% \changes{spanish-2.0b}{1991/04/23}{New check before loading
%    \file{babel.sty}}
% \changes{spanish-2.1}{1991/07/03}{Added catalan as a `dialect'}
% \changes{spanish-2.1a}{1991/07/15}{Renamed \file{babel}.sty in
%    \file{babel.com}}
% \changes{spanish-3.0}{1991/11/25}{Major rewriting, new macros,
%    active accents, catalan removed}
% \changes{spanish-3.1}{1992/02/20}{Brought up-to-date with babel
%    3.2a}
% \changes{spanish-3.1.1}{1993/09/9}{The accents had to be made active
%    during their own definition. Changed address for goya.}
% \changes{spanish-3.1.2}{1993/09/13}{Added address, phone and fax for
%    Julio S\'anchez. The definition of the active tilde was not being
%    restored on exit.}
% \changes{spanish-3.2}{1994/03/20}{Active character definitions
%    changed as in germanb.}
% \changes{spanish-3.2}{1994/03/20}{Update for \LaTeXe}
% \changes{spanish-3.3d}{1994/06/26}{Removed the use of \cs{filedate}
%    and moved identification after the loading of \file{babel.def}}
% \changes{spanish-3.4b}{1995/06/14}{corrected typo (PR1652)}
% \changes{spanish-3.4c}{1995/07/08}{made active acute optional}
%
%  \iffalse
%       Missing things, ideas, etc.:
%          - The \spechyphcodes idea in ML-TeX should be explored
%          - Support for people with extended keyboards but no
%            8-bit chars should be added (or not?)
%  \fi
%
%  \section{The Spanish language}
%
% \changes{spanish-3.0}{1991/11/25}{Catalan deleted}
%
%    The file \file{\filename}\footnote{The file described in this
%    section has version number \fileversion\ and was last revised on
%    \filedate. The original author is Julio S\'anchez,
%    (\texttt{jsanchez@gmv.es}).}  defines all the language definition
%    macro's for the Spanish\footnote{Catalan used to be part of this
%    file but is now on its own file.} language.
%
%    This file\footnote{In writing this file, many ideas and actual
%    coding solutions have been taken from a number of sources. The
%    language definition files \file{dutch.sty} and \file{germanb.sty}
%    are the main contributors and are not explicitly mentioned in the
%    sequel. J.~L.~Braams and Bernd Raichle have given helpful
%    advice. Another source of inspiration is the experience gained in
%    the use of FTC, a software package written by Jos\'e A. Ma\~nas.
%    The members of the Spanish-\TeX\ list have helped clarify a
%    number of issues. Other sources are explicitly acknowledged when
%    used.  If you think that you contributed something and you are
%    not mentioned, please let me (\texttt{jsanchez@gmv.es}) know. I
%    humbly apologize for any omission.} incorporates the result of
%    discussions held in the
%    Spanish-\TeX\footnote{\texttt{spanish-tex@goya.eunet.es},
%    subscription requests can be sent to the address
%    \texttt{listserv@goya.eunet.es}. This list is devoted to
%    discussions on support in \TeX\ for Spanish.  Comments on this
%    language option are welcome there or directly to
%    \texttt{jsanchez@gmv.es}.}  electronic mail list.
%
%    For this language the characters |'| |~| and |"| are made
%    active. In table~\ref{tab:spanish-quote} an overview is given of
%    their purpose.
%    \begin{table}[htb]
%     \centering
%     \begin{tabular}{lp{8cm}}
%      |'a| & an accent that allows hyphenation. Valid for all
%             vowels uppercase and lowercase.\\
%      |'n| & a n with a tilde. This is included to
%             improve compatibility with FTC. Works for uppercase too.\\
%      \verb="|= & disable ligature at this position.\\
%      |"-| & an explicit hyphen sign, allowing hyphenation
%             in the rest of the word.\\
%      |""| & like \verb="-=, but producing no hyphen sign (for
%             words that should break at some sign such as
%             ``entrada/salida.''\\
%      |\-| & like the old |\-|, but allowing hyphenation
%             in the rest of the word. \\
%      |"u| & a u with dieresis allowing hyphenation.\\
%      |"a| & feminine ordinal as in
%             1{\raise1ex\hbox{\underbar{\scriptsize a}}}.\\
%      |"o| & masculine ordinal as in
%             1{\raise1ex\hbox{\underbar{\scriptsize o}}}.\\
%      |"<| & for French left double quotes (similar to $<<$).\\
%      |">| & for French right double quotes (similar to $>>$).\\
%      |~n| & a n with tilde. Works for uppercase too.
%     \end{tabular}
%     \caption{The extra definitions made by \file{spanish.ldf}}
%     \label{tab:spanish-quote}
%    \end{table}
%    These active accent characters behave according to their original
%    definitions if not followed by one of the characters indicated in
%    that table.
%
%    This option includes support for working with extended, 8-bit
%    fonts, if available. Old versions of this file based this support
%    on the existance of special macros with names as in Ferguson's
%    ML-\TeX{}. This is no longer the case. Support is now based on
%    providing an appropriate definition for the accent macros on
%    entry to the Spanish language. This is automatically done by
%    \LaTeXe\ or NFSS2. If T1 encoding is chosen, and provided that
%    adequate hyphenation patterns\footnote{One source for such
%    patterns is the archive at \texttt{ftp.eunet.es} that can be
%    accessed by anonymous FTP or electronic mail to
%    \texttt{ftpmail@goya.eunet.es}. They are in the \texttt{info}
%    directory \texttt{src/TeX/spanish}. The list of Frequently Asked
%    Questions with Answers about \TeX{} for Spanish is kept there as
%    well. That list is meant to be a summary of the discussions held
%    in the Spanish-\TeX{} mail list. Warning: It is in Spanish.}
%    exist, it is possible to get better hyphenation for Spanish than
%    before.  The easiest way to use the new encoding with \LaTeXe{}
%    to load the package \texttt{t1enc} with |\usepackage|. This must
%    be done before loading \babel.
%
%    If the combination of keyboard and \TeX{} version that the user
%    has is able to produce the accented characters in the T1
%    enconding, the user could see the accented characters in the
%    editor, greatly improving the readability of the document source.
%    As of today, this is not a recommended method for producing
%    documents for distribution, although it is possible to
%    mechanically translate the document so that the receiver can make
%    use of it. If care is taken to define the encoding needed by the
%    document, the results are pretty portable.
%
%    This option file will automatically detect if the T1 encoding is
%    being used and behave appropriately.  If any other encoding is
%    being used, the accent macros will be redefined to allow
%    hyphenation on the accented words.
%
% \StopEventually{}
%
% \changes{spanish-3.1}{1992/02/20}{Removed code to load
%    \file{latexhax.com}}
%
%    As this file needs to be read only once, we check whether it was
%    read before. If it was, the |\captionsspanish| is already
%    defined, so we can stop processing. If this command is undefined
%    we proceed with the various definitions and first show the
%    current version of this file.
%
% \changes{spanish-2.1a}{1991/07/15}{Added reset of catcode of @
%    before \cs{endinput}.}
% \changes{spanish-3.1}{1992/02/20}{removed use of \cs{@ifundefined}}
% \changes{spanish-3.1}{1992/02/20}{Moved code to the beginning of the
%    file and added \cs{selectlanguage} call}
%    \begin{macrocode}
%<*code>
\ifx\undefined\captionsspanish
\else
  \selectlanguage{spanish}
  \expandafter\endinput
\fi
%    \end{macrocode}
%
% \begin{macro}{\atcatcode}
%    This file, \file{spanish.ldf}, may have been read while \TeX\ is
%    in the middle of processing a document, so we have to make sure
%    the category code of \texttt{@} is `letter' while this file is
%    being read.  We save the category code of the @-sign in
%    |\atcatcode| and make it `letter'. Later the category code can be
%    restored to whatever it was before.
%
% \changes{spanish-2.0c}{1991/06/06}{Made test of catcode of @ more
%    robust}
% \changes{spanish-2.1a}{1991/07/15}{Modified handling of catcode of @
%    again.}
% \changes{spanish-3.1}{1992/02/20}{Removed use of \cs{makeatletter}
%    and hence the need to load \file{latexhax.com}}
%    \begin{macrocode}
\chardef\atcatcode=\catcode`\@
\catcode`\@=11\relax
%    \end{macrocode}
% \end{macro}
%
%    Now we determine whether the common macros from the file
%    \file{babel.def} need to be read. We can be in one of two
%    situations: either another language option has been read earlier
%    on, in which case that other option has already read
%    \file{babel.def}, or \texttt{spanish} is the first language
%    option to be processed. In that case we need to read
%    \file{babel.def} right here before we continue.
%
% \changes{spanish-2.0b}{1991/04/23}{New check before loading
%    babel.com}
% \changes{spanish-3.1}{1992/02/20}{Added \cs{relax} after the
%    argument of \cs{input}}
%    \begin{macrocode}
\ifx\undefined\babel@core@loaded\input babel.def\relax\fi
%    \end{macrocode}
%
% \changes{spanish-2.0a}{1991/05/29}{Add a check for existence
%    \cs{originalTeX}}
%    Another check that has to be made, is if another language
%    definition file has been read already. In that case its definitions
%    have been activated. This might interfere with definitions this
%    file tries to make. Therefore we make sure that we cancel any
%    special definitions. This can be done by checking the existence
%    of the macro |\originalTeX|. If it exists we simply execute it,
%    otherwise it is |\let| to |\empty|.
% \changes{spanish-2.1a}{1991/07/15}{Added \cs{let}\cs{originalTeX}%
%    \cs{relax} to test for existence}
% \changes{spanish-3.1}{1992/02/20}{Set \cs{originalTeX} to
%    \cs{empty}, because it should be expandable.}
%    \begin{macrocode}
\ifx\undefined\originalTeX \let\originalTeX\empty \else\originalTeX\fi
%    \end{macrocode}
%
%    When this file is read as an option, i.e. by the |\usepackage|
%    command, \texttt{spanish} could be an `unknown' language in which
%    case we have to make it known.  So we check for the existence of
%    |\l@spanish| to see whether we have to do something here.
%
% \changes{spanish-2.0}{1991/04/23}{Now use \cs{adddialect} if
%    language undefined}
% \changes{spanish-3.1}{1992/02/20}{removed use of \cs{@ifundefined}}
% \changes{spanish-3.1}{1992/02/20}{Added warning, if no spanish
%    patterns were loaded}
% \changes{spanish-3.3d}{1994/06/26}{Now use \cs{@nopatterns} to
%    produce the warning}
%    \begin{macrocode}
\ifx\undefined\l@spanish
  \@nopatterns{Spanish}
  \adddialect\l@spanish0
\fi
%    \end{macrocode}
%
%    The next step consists of defining commands to switch to (and
%    from) the Spanish language.
%
% \changes{spanish-3.0a}{1991/11/26}{Text fixed}
% \begin{macro}{\captionsspanish}
%    The macro |\captionsspanish| defines all strings\footnote{The
%    accent on the uppercase `I' is intentional, following the
%    recommendation of the \emph{Real Academia de la Lengua} in 
%    \emph{Esbozo de una Nueva Gram\'atica de la Lengua Espa\~nola,
%    Comisi\'on de Gram\'atica, Espasa-Calpe, 1973}.} used in the four
%    standard documentclasses provided with \LaTeX.
% \changes{spanish-2.0c}{1991/06/06}{Removed \cs{global} definitions}
% \changes{spanish-3.0}{1991/11/25}{Capitals are accented, some
%    strings changed}
% \changes{spanish-3.1}{1992/02/20}{added \cs{seename}, and
%    \cs{alsoname} and \cs{prefacename}}
% \changes{spanish-3.1}{1993/07/13}{\cs{headpagename} should be
%    \cs{pagename}}
% \changes{spanish-3.2}{1994/03/20}{added translated strings for
%    \cs{seename} \cs{alsoname} and \cs{prefacename}}
% \changes{spanish-3.4c}{1995/07/03}{Added \cs{proofname} for
%    AMS-\LaTeX}
%    \begin{macrocode}
\addto\captionsspanish{%
  \def\prefacename{Prefacio}%
  \def\refname{Referencias}%
  \def\abstractname{Resumen}%
  \def\bibname{Bibliograf\'{\i}a}%
  \def\chaptername{Cap\'{\i}tulo}%
  \def\appendixname{Ap\'endice}%
  \def\contentsname{\'Indice General}%
  \def\listfigurename{\'Indice de Figuras}%
  \def\listtablename{\'Indice de Tablas}%
  \def\indexname{\'Indice de Materias}%
  \def\figurename{Figura}%
  \def\tablename{Tabla}%
  \def\partname{Parte}%
  \def\enclname{Adjunto}%
  \def\ccname{Copia a}%
  \def\headtoname{A}%
  \def\pagename{P\'agina}%
  \def\seename{v\'ease}%
  \def\alsoname{v\'ease tambi\'en}%
  \def\proofname{Proof}%  <-- needs translation!
  }%
%    \end{macrocode}
% \end{macro}
%
% \begin{macro}{\datespanish}
%    The macro |\datespanish| redefines the command |\today| to
%    produce Spanish\footnote{Months are written lowercased. This has
%    been cause of some controversy. This file follows
%    \emph{Diccionario de Uso de la Lengua Espa\~nola, Mar\'{\i}a
%    Moliner, 1990,} that is in agreement with the most common
%    practice.}  dates.
% \changes{spanish-2.0c}{1991/06/06}{Removed cs{global} definitions}
% \changes{spanish-2.0d}{1991/07/01}{Capitalize months as suggested by
%    E. Torrente (\texttt{TORRENTE@CERNVM}).}
% \changes{spanish-3.0}{1991/11/25}{Uncapitalize months, since that
%    seems to be the correct, modern usage}
%    \begin{macrocode}
\def\datespanish{%
\def\today{\number\day~de\space\ifcase\month\or
  enero\or febrero\or marzo\or abril\or mayo\or junio\or
  julio\or agosto\or septiembre\or octubre\or noviembre\or diciembre\fi
  \space de~\number\year}}
%    \end{macrocode}
% \end{macro}
%
% \begin{macro}{\extrasspanish}
% \changes{spanish-3.0}{1991/11/25}{Formerly empty, all code is new.}
% \changes{spanish-3.1}{1992/02/20}{Rewrote the macro.}
% \changes{spanish-3.2}{1994/03/20}{Major rewrite. Now works like in
%    germanb and dutch.}
% \changes{spanish-3.4a}{1995/03/11}{Yet another major rewrite}
% \begin{macro}{\noextrasspanish}
%    The macro |\extrasspanish| will perform all the extra definitions
%    needed for the Spanish language. The macro |\noextrasspanish| is
%    used to cancel the actions of |\extrasspanish|. For Spanish, some
%    characters are made active or are redefined. In particular, the
%    \texttt{"} character, the \texttt{'} character and the |~|
%    character receive new meanings. Therefore these characters have
%    to be treated as `special' characters.
%
%    \begin{macrocode}
\addto\extrasspanish{\languageshorthands{spanish}}
\initiate@active@char{"}
\initiate@active@char{~}
\addto\extrasspanish{%
  \bbl@activate{"}%
  \bbl@activate{~}}
\@ifpackagewith{babel}{activeacute}{%
  \initiate@active@char{'}
  \addto\extrasspanish{\bbl@activate{'}}}{}
%\addto\noextrasspanish{
%  \bbl@deactivate{"}\bbl@deactivate{~}\bbl@deactivate{'}}
%    \end{macrocode}
%
% \changes{spanish-3.4a}{1995/03/07}{All the code for handling active
%    characters is now moved to \file{babel.def}}
%
%    Apart from the active characters some other macros get a new
%    definition. Therefore we store the current one to be able to
%    restore them later.
%    \begin{macrocode}
\addto\extrasspanish{%
  \babel@save\"
  \babel@save\~
  \def\"{\protect\@umlaut}%
  \def\~{\protect\@tilde}}
\@ifpackagewith{babel}{activeacute}{%
  \babel@save\'
  \addto\extrasspanish{\def\'{\protect\@acute}}
  }{}
%    \end{macrocode}
% \end{macro}
% \end{macro}
%
%  \begin{macro}{\spanishhyphenmins}
%    Spanish hyphenation uses |\lefthyphenmin| and |\righthyphenmin|
%    both set to~2.
%    \begin{macrocode}
\def\spanishhyphenmins{\tw@\tw@}
%    \end{macrocode}
% \end{macro}
%
% \changes{spanish-3.2}{1994/03/20}{Changed \cs{acute} to
%    \cs{textacute} and \cs{tilde} to \cs{texttilde} because the old
%    names were already used for math accents.}
%  \begin{macro}{\dieresis}
%  \begin{macro}{\textacute}
%  \begin{macro}{\texttilde}
%    The original definition of |\"| is stored as |\dieresis|, because
%    the we do not know what is its definition, since it depends on
%    the encoding we are using or on special macros that the user
%    might have loaded. The expansion of the macro might use the \TeX\
%    |\accent| primitive using some particular accent that the font
%    provides or might check if a combined accent exists in the font.
%    These two cases happen with respectively OT1 and T1 encodings.
%    For this reason we save the definition of |\"| and use that in
%    the definition of other macros. We do likewise for |\'| and
%    |\~|. The present coding of this option file is incorrect in that
%    it can break when the encoding changes. We do not use |\acute| or
%    |\tilde| as the macro names because they are already defined as
%    |\mathaccent|.
%    \begin{macrocode}
\let\dieresis\"
\let\texttilde\~
\@ifpackagewith{babel}{activeacute}{\let\textacute\'}{}
%    \end{macrocode}
%  \end{macro}
%  \end{macro}
%  \end{macro}
%
%  \begin{macro}{\@umlaut}
%  \begin{macro}{\@acute}
%  \begin{macro}{\@tilde}
%    We check the encoding and if not using T1, we make the accents
%    expand but enabling hyphenation beyond the accent. If this is the
%    case, not all break positions will be found in words that contain
%    accents, but this is a limitation in \TeX. An unsolved problem
%    here is that the encoding can change at any time. The definitions
%    below are made in such a way that a change between two 256-char
%    encodings are supported, but changes between a 128-char and a
%    256-char encoding are not properly supported. We check if T1 is
%    in use. If not, we will give a warning and proceed redefining the
%    accent macros so that \TeX{} at least finds the breaks that are
%    not too close to the accent. The warning will only be printed to
%    the log file.
% \changes{spanish-3.0a}{1991/11/26}{Added fix for \cs{dotlessi}}
% \changes{spanish-3.2}{1994/03/20}{All this code is new}
%    \begin{macrocode}
\ifx\undefined\DeclareFontShape
  \wlog{Warning: You are using an old LaTeX}
  \wlog{Some word breaks will not be found.}
  \def\@umlaut#1{\allowhyphens\dieresis{#1}\allowhyphens}
  \def\@tilde#1{\allowhyphens\texttilde{#1}\allowhyphens}
  \@ifpackagewith{babel}{activeacute}{%
    \def\@acute#1{\allowhyphens\textacute{#1}\allowhyphens}}{}
\else
  \edef\next{T1}
  \ifx\f@encoding\next
    \let\@umlaut\dieresis
    \let\@tilde\texttilde
    \@ifpackagewith{babel}{activeacute}{%
      \let\@acute\textacute}{}
  \else
    \wlog{Warning: You are using encoding \f@encoding\space
      instead of T1.}
    \wlog{Some word breaks will not be found.}
    \def\@umlaut#1{\allowhyphens\dieresis{#1}\allowhyphens}
    \def\@tilde#1{\allowhyphens\texttilde{#1}\allowhyphens}
    \@ifpackagewith{babel}{activeacute}{%
      \def\@acute#1{\allowhyphens\textacute{#1}\allowhyphens}}{}
  \fi
\fi
%    \end{macrocode}
%  \end{macro}
%  \end{macro}
%  \end{macro}
%
%     Now we can define our shorthands: the umlauts,
%    \begin{macrocode}
\declare@shorthand{spanish}{"u}{\@umlaut u}
\declare@shorthand{spanish}{"U}{\@umlaut U}
%    \end{macrocode}
%     french quotes,
%    \begin{macrocode}
\declare@shorthand{spanish}{"<}{%
  \textormath{\guillemotleft{}}{\mbox{\guillemotleft}}}
\declare@shorthand{spanish}{">}{%
  \textormath{\guillemotright{}}{\mbox{\guillemotright}}}
%    \end{macrocode}
%     ordinals\footnote{The code for the ordinals was taken from the
%    answer provided by Raymond Chen
%    (\texttt{raymond@math.berkeley.edu}) to a question by Joseph Gil
%    (\texttt{yogi@cs.ubc.ca}) in \texttt{comp.text.tex}.},
%    \begin{macrocode}
\declare@shorthand{spanish}{"o}{%
  \raise1ex\hbox{\underbar{\scriptsize o}}}
\declare@shorthand{spanish}{"a}{%
  \raise1ex\hbox{\underbar{\scriptsize a}}}
%    \end{macrocode}
%     acute accents,
% \changes{spanish-3.4c}{1995/07/03}{Changed mathmode definition of
%    acute shorthands to expand to a single prime followed by the next
%    character in the input}
%    \begin{macrocode}
\@ifpackagewith{babel}{activeacute}{%
  \declare@shorthand{spanish}{'a}{\textormath{\@acute a}{^{\prime} a}}
  \declare@shorthand{spanish}{'e}{\textormath{\@acute e}{^{\prime} e}}
  \declare@shorthand{spanish}{'i}{\textormath{\@acute \i{}}{^{\prime} i}}
  \declare@shorthand{spanish}{'o}{\textormath{\@acute o}{^{\prime} o}}
  \declare@shorthand{spanish}{'u}{\textormath{\@acute u}{^{\prime} u}}
  \declare@shorthand{spanish}{'A}{\textormath{\@acute A}{^{\prime} A}}
  \declare@shorthand{spanish}{'E}{\textormath{\@acute E}{^{\prime} E}}
  \declare@shorthand{spanish}{'I}{\textormath{\@acute I}{^{\prime} I}}
  \declare@shorthand{spanish}{'O}{\textormath{\@acute O}{^{\prime} O}}
  \declare@shorthand{spanish}{'U}{\textormath{\@acute U}{^{\prime} U}}
%    \end{macrocode}
%         the acute accent,
% \changes{spanish-3.4c}{1995/07/08}{Added '{}' as an axtra shorthand,
%    removed 'n as a shorthand}
%    \begin{macrocode}
  \declare@shorthand{spanish}{''}{%
    \textormath{\textquotedblright}{\sp\bgroup\prim@s'}}
%    \end{macrocode}
%     tildes,
%    \begin{macrocode}
  \declare@shorthand{spanish}{'n}{\textormath{\~n}{^{\prime} n}}
  \declare@shorthand{spanish}{'N}{\textormath{\~N}{^{\prime} N}}
  }{}
\declare@shorthand{spanish}{~n}{\textormath{\~n}{\@tilde n}}
\declare@shorthand{spanish}{~N}{\textormath{\~N}{\@tilde N}}
%    \end{macrocode}
%     and some additional commands:
%    \begin{macrocode}
\declare@shorthand{spanish}{"-}{\allowhyphens\-\allowhyphens}
\declare@shorthand{spanish}{"|}{%
  \textormath{\penalty\@M\discretionary{-}{}{\kern.03em}%
              \allowhyphens}{}}
\declare@shorthand{spanish}{""}{\hskip\z@skip}
%    \end{macrocode}
%
%    It is possible that a site might need to add some extra code to
%    the babel macros. To enable this we load a local configuration
%    file, \file{spanish.cfg} if it is found on \TeX' search path.
% \changes{spanish-3.4c}{1995/07/02}{Added loading of configuration
%    file}
%    \begin{macrocode}
\loadlocalcfg{spanish}
%    \end{macrocode}
%
%    Next the \babel\ macro |\main@language| is used to activate the
%    definitions for Spanish at the beginning of the document.
%
%    \begin{macrocode}
\main@language{spanish}
%    \end{macrocode}
%
%    Finally, the category code of \texttt{@} is reset to its original
%    value. The macrospace used by |\atcatcode| is freed.
% \changes{spanish-2.1a}{1991/07/15}{Modified handling of catcode of
%    @-sign.}
%    \begin{macrocode}
\catcode`\@=\atcatcode \let\atcatcode\relax
%</code>
%    \end{macrocode}
%
% \Finale
%
%%
%% \CharacterTable
%%  {Upper-case    \A\B\C\D\E\F\G\H\I\J\K\L\M\N\O\P\Q\R\S\T\U\V\W\X\Y\Z
%%   Lower-case    \a\b\c\d\e\f\g\h\i\j\k\l\m\n\o\p\q\r\s\t\u\v\w\x\y\z
%%   Digits        \0\1\2\3\4\5\6\7\8\9
%%   Exclamation   \!     Double quote  \"     Hash (number) \#
%%   Dollar        \$     Percent       \%     Ampersand     \&
%%   Acute accent  \'     Left paren    \(     Right paren   \)
%%   Asterisk      \*     Plus          \+     Comma         \,
%%   Minus         \-     Point         \.     Solidus       \/
%%   Colon         \:     Semicolon     \;     Less than     \<
%%   Equals        \=     Greater than  \>     Question mark \?
%%   Commercial at \@     Left bracket  \[     Backslash     \\
%%   Right bracket \]     Circumflex    \^     Underscore    \_
%%   Grave accent  \`     Left brace    \{     Vertical bar  \|
%%   Right brace   \}     Tilde         \~}
%%
\endinput
}
\DeclareOption{slovak}{% \iffalse meta-comment
%
% Copyright 1989-1995 Johannes L. Braams and any individual authors
% listed elsewhere in this file.  All rights reserved.
% 
% For further copyright information any other copyright notices in this
% file.
% 
% This file is part of the Babel system release 3.5.
% --------------------------------------------------
%   This system is distributed in the hope that it will be useful,
%   but WITHOUT ANY WARRANTY; without even the implied warranty of
%   MERCHANTABILITY or FITNESS FOR A PARTICULAR PURPOSE.
% 
%   For error reports concerning UNCHANGED versions of this file no more
%   than one year old, see bugs.txt.
% 
%   Please do not request updates from me directly.  Primary
%   distribution is through the CTAN archives.
% 
% 
% IMPORTANT COPYRIGHT NOTICE:
% 
% You are NOT ALLOWED to distribute this file alone.
% 
% You are allowed to distribute this file under the condition that it is
% distributed together with all the files listed in manifest.txt.
% 
% If you receive only some of these files from someone, complain!
% 
% Permission is granted to copy this file to another file with a clearly
% different name and to customize the declarations in that copy to serve
% the needs of your installation, provided that you comply with
% the conditions in the file legal.txt from the LaTeX2e distribution.
% 
% However, NO PERMISSION is granted to produce or to distribute a
% modified version of this file under its original name.
%  
% You are NOT ALLOWED to change this file.
% 
% 
% \fi
% \CheckSum{142}
% \iffalse
%    Tell the \LaTeX\ system who we are and write an entry on the
%    transcript.
%<*dtx>
\ProvidesFile{slovak.dtx}
%</dtx>
%<code>\ProvidesFile{slovak.ldf}
        [1995/07/04 v1.2g Slovak support from the babel system]
%
% Babel package for LaTeX version 2e
% Copyright (C) 1989 - 1995
%           by Johannes Braams, TeXniek
%
% Slovak Language Definition File
% Copyright (C) 1989 - 1995
%           by Jana Chlebikova
%           Department of Artificial Intelligence
%           Faculty of Mathematics and Physics
%           Mlynska dolina
%           84215 Bratislava
%           Slovakia
%           (42)(7) 720003 l. 835
%           (42)(7) 725882
%           chlebikj@mff.uniba.cs (Internet)
%           and Johannes Braams, TeXniek
%
% Please report errors to: J.L. Braams  <JLBraams@cistron.nl>
%                          Chlebikova Jana <chlebikj@mff.uniba.cs>
%
%    This file is part of the babel system, it provides the source
%    code for the Slovak language definition file.
%<*filedriver>
\documentclass{ltxdoc}
\newcommand*\TeXhax{\TeX hax}
\newcommand*\babel{\textsf{babel}}
\newcommand*\langvar{$\langle \it lang \rangle$}
\newcommand*\note[1]{}
\newcommand*\Lopt[1]{\textsf{#1}}
\newcommand*\file[1]{\texttt{#1}}
\begin{document}
 \DocInput{slovak.dtx}
\end{document}
%</filedriver>
%\fi
% \GetFileInfo{slovak.dtx}
%
% \changes{slovak-1.0}{1992/07/15}{First version}
% \changes{slovak-1.2}{1994/02/27}{Update for \LaTeXe}
% \changes{slovak-1.2d}{1994/06/26}{Removed the use of \cs{filedate}
%    and moved identification after the loading of \file{babel.def}}
%
%  \section{The Slovak language}
%
%    The file \file{\filename}\footnote{The file described in this
%    section has version number \fileversion\ and was last revised on
%    \filedate.  It was written by Jana Chlebikova
%    (\texttt{chlebik@euromath.dk}).}  defines all the
%    language-specific macros for the Slovak language.
%
%    For this language the macro |\q| is defined. It is used with the
%    letters (\texttt{t}, \texttt{d}, \texttt{l}, and \texttt{L}) and
%    adds a \texttt{'} to them to simulate a `hook' that should be
%    there.  The result looks like t\kern-2pt\char'47.
%
% \StopEventually{}
%
%    As this file needs to be read only once, we check whether it was
%    read before. If it was, the command |\captionsslovak| is already
%    defined, so we can stop processing. If this command is undefined
%    we proceed with the various definitions and first show the
%    current version of this file.
%
%    \begin{macrocode}
%<*code>
\ifx\undefined\captionsslovak
\else
  \selectlanguage{slovak}
  \expandafter\endinput
\fi
%    \end{macrocode}
%
%  \begin{macro}{\atcatcode}
%    This file, \file{slovak.sty}, may have been read while \TeX\ is
%    in the middle of processing a document, so we have to make sure
%    the category code of \texttt{@} is `letter' while this file is
%    being read.  We save the category code of the @-sign in
%    |\atcatcode| and make it `letter'. Later the category code can be
%    restored to whatever it was before.
%
%    \begin{macrocode}
\chardef\atcatcode=\catcode`\@
\catcode`\@=11\relax
%    \end{macrocode}
% \end{macro}
%
%    Now we determine whether the the common macros from the file
%    \file{babel.def} need to be read. We can be in one of two
%    situations: either another language option has been read earlier
%    on, in which case that other option has already read
%    \file{babel.def}, or \texttt{slovak} is the first language option
%    to be processed. In that case we need to read \file{babel.def}
%    right here before we continue.
%
%    \begin{macrocode}
\ifx\undefined\babel@core@loaded\input babel.def\relax\fi
%    \end{macrocode}
%
%    Another check that has to be made, is if another language
%    definition file has been read already. In that case its
%    definitions have been activated. This might interfere with
%    definitions this file tries to make. Therefore we make sure that
%    we cancel any special definitions. This can be done by checking
%    the existence of the macro |\originalTeX|. If it exists we simply
%    execute it, otherwise it is |\let| to |\empty|.
%
%    \begin{macrocode}
\ifx\undefined\originalTeX \let\originalTeX\empty \else\originalTeX\fi
%    \end{macrocode}
%
%    When this file is read as an option, i.e. by the |\usepackage|
%    command, \texttt{slovak} will be an `unknown' language in which
%    case we have to make it known. So we check for the existence of
%    |\l@slovak| to see whether we have to do something here.
%
% \changes{slovak-1.2d}{1994/06/26}{Now use \cs{@nopatterns} to
%    produce the warning}
%    \begin{macrocode}
\ifx\undefined\l@slovak
    \@nopatterns{Slovak}
    \adddialect\l@slovak0\fi
%    \end{macrocode}
%
%    The next step consists of defining commands to switch to (and
%    from) the Slovak language.
%
% \begin{macro}{\captionsslovak}
%    The macro |\captionsslovak| defines all strings used in the four
%    standard documentclasses provided with \LaTeX.
% \changes{slovak-1.2g}{1995/07/04}{Added \cs{proofname} for
%    AMS-\LaTeX}
%    \begin{macrocode}
\addto\captionsslovak{%
  \def\prefacename{\'Uvod}%
  \def\refname{Referencia}%
  \def\abstractname{Abstrakt}%
  \def\bibname{Literat\'ura}%
  \def\chaptername{Kapitola}%
  \def\appendixname{Dodatok}%
  \def\contentsname{Obsah}%
  \def\listfigurename{Zoznam obr\'azkov}%
  \def\listtablename{Zoznam tabuliek}%
  \def\indexname{Index}%
  \def\figurename{Obr\'azok}%
  \def\tablename{Tabu\q lka}%%% special letter l with hook
  \def\partname{\v{C}as\q t}%%% special letter t with hook
  \def\enclname{Pr\'{\i}loha}%
  \def\ccname{CC}%
  \def\headtoname{Komu}%
  \def\pagename{Strana}%
  \def\seename{vi\q d}%%%  Special letter d with hook
  \def\alsoname{vi\q d tie\v z}%%%  Special letter d with hook
  \def\proofname{Proof}%  <-- needs translation
  }
%    \end{macrocode}
% \end{macro}
%
% \begin{macro}{\dateslovak}
%    The macro |\dateslovak| redefines the command |\today| to produce
%    Slovak dates.
%    \begin{macrocode}
\def\dateslovak{%
\def\today{\number\day.~\ifcase\month\or
janu\'ara\or febru\'ara\or marca\or apr\'{\i}la\or m\'aja\or j\'una\or
  j\'ula\or augusat\or septembra\or okt\'obra\or
  novembra\or decembra\fi
    \space \number\year}}
%    \end{macrocode}
% \end{macro}
%
% \begin{macro}{\extrasslovak}
% \begin{macro}{\noextrasslovak}
%    The macro |\extrasslovak| will perform all the extra definitions
%    needed for the Slovak language. The macro |\noextrasslovak| is
%    used to cancel the actions of |\extrasslovak|.  This currently
%    means saving the meaning of one one-letter control sequence
%    before defining it.
%
% \changes{slovak-1.2e}{1995/05/28}{Use \LaTeX's \cs{v} accent
%    command}
%    \begin{macrocode}
\addto\extrasslovak{\babel@save\q\let\q\v}
%    \end{macrocode}
%
% \changes{slovak-1.2b}{1994/06/04}{Added setting of left- and
%    righthyphenmin}
%
%    The slovak hyphenation patterns should be used with
%    |\lefthyphenmin| set to~2 and |\righthyphenmin| set to~2.
% \changes{slovak-1.2e}{1995/05/28}{Now use \cs{slovakhyphenmins}}
%    \begin{macrocode}
\def\slovakhyphenmins{\tw@\tw@}
%    \end{macrocode}
% \end{macro}
% \end{macro}
%
%    It is possible that a site might need to add some extra code to
%    the babel macros. To enable this we load a local configuration
%    file, \file{slovak.cfg} if it is found on \TeX' search path.
% \changes{slovak-1.2g}{1995/07/02}{Added loading of configuration
%    file}
%    \begin{macrocode}
\loadlocalcfg{slovak}
%    \end{macrocode}
%
%    Our last action is to make a note that the commands we have just
%    defined, will be executed by calling the macro |\selectlanguage|
%    at the beginning of the document.
%    \begin{macrocode}
\main@language{slovak}
%    \end{macrocode}
%    Finally, the category code of \texttt{@} is reset to its original
%    value. The macrospace used by |\atcatcode| is freed.
%    \begin{macrocode}
\catcode`\@=\atcatcode \let\atcatcode\relax
%</code>
%    \end{macrocode}
%
% \Finale
%%
%% \CharacterTable
%%  {Upper-case    \A\B\C\D\E\F\G\H\I\J\K\L\M\N\O\P\Q\R\S\T\U\V\W\X\Y\Z
%%   Lower-case    \a\b\c\d\e\f\g\h\i\j\k\l\m\n\o\p\q\r\s\t\u\v\w\x\y\z
%%   Digits        \0\1\2\3\4\5\6\7\8\9
%%   Exclamation   \!     Double quote  \"     Hash (number) \#
%%   Dollar        \$     Percent       \%     Ampersand     \&
%%   Acute accent  \'     Left paren    \(     Right paren   \)
%%   Asterisk      \*     Plus          \+     Comma         \,
%%   Minus         \-     Point         \.     Solidus       \/
%%   Colon         \:     Semicolon     \;     Less than     \<
%%   Equals        \=     Greater than  \>     Question mark \?
%%   Commercial at \@     Left bracket  \[     Backslash     \\
%%   Right bracket \]     Circumflex    \^     Underscore    \_
%%   Grave accent  \`     Left brace    \{     Vertical bar  \|
%%   Right brace   \}     Tilde         \~}
%%
\endinput
}
\DeclareOption{slovene}{% \iffalse meta-comment
%
% Copyright 1989-1995 Johannes L. Braams and any individual authors
% listed elsewhere in this file.  All rights reserved.
% 
% For further copyright information any other copyright notices in this
% file.
% 
% This file is part of the Babel system release 3.5.
% --------------------------------------------------
%   This system is distributed in the hope that it will be useful,
%   but WITHOUT ANY WARRANTY; without even the implied warranty of
%   MERCHANTABILITY or FITNESS FOR A PARTICULAR PURPOSE.
% 
%   For error reports concerning UNCHANGED versions of this file no more
%   than one year old, see bugs.txt.
% 
%   Please do not request updates from me directly.  Primary
%   distribution is through the CTAN archives.
% 
% 
% IMPORTANT COPYRIGHT NOTICE:
% 
% You are NOT ALLOWED to distribute this file alone.
% 
% You are allowed to distribute this file under the condition that it is
% distributed together with all the files listed in manifest.txt.
% 
% If you receive only some of these files from someone, complain!
% 
% Permission is granted to copy this file to another file with a clearly
% different name and to customize the declarations in that copy to serve
% the needs of your installation, provided that you comply with
% the conditions in the file legal.txt from the LaTeX2e distribution.
% 
% However, NO PERMISSION is granted to produce or to distribute a
% modified version of this file under its original name.
%  
% You are NOT ALLOWED to change this file.
% 
% 
% \fi
% \CheckSum{179}
% \iffalse
%    Tell the \LaTeX\ system who we are and write an entry on the
%    transcript.
%<*dtx>
\ProvidesFile{slovene.dtx}
%</dtx>
%<code>\ProvidesFile{slovene.ldf}
        [1995/07/04 v1.2g Slovene support from the babel system]
%    \end{macrocode}
%
% Babel package for LaTeX version 2e
% Copyright (C) 1989 - 1995
%           by Johannes Braams, TeXniek
%
% Please report errors to: J.L. Braams
%                          JLBraams@cistron.nl
%
%    This file is part of the babel system, it provides the source
%    code for the Slovanian language definition file.  The Slovanian
%    words were contributed by Danilo Zavrtanik, University of
%    Ljubljana (YU).
%    The usage of the active " was introduced by
%        Leon \v{Z}lajpah
%        Jo\v{z}ef Stefan Institute,
%        Jamova 39, Ljubljana,
%        Slovenia
%        e-mail: leon.zlajpah@ijs.si
%
%<*filedriver>
\documentclass{ltxdoc}
\newcommand*\TeXhax{\TeX hax}
\newcommand*\babel{\textsf{babel}}
\newcommand*\langvar{$\langle \it lang \rangle$}
\newcommand*\note[1]{}
\newcommand*\Lopt[1]{{textsf \1}}
\newcommand*\file[1]{\texttt{#1}}
\begin{document}
 \DocInput{slovene.dtx}
\end{document}
%</filedriver>
%\fi
% \GetFileInfo{slovene.dtx}
%
% \changes{slovene-1.0a}{1991/07/15}{Renamed babel.sty in babel.com}
% \changes{slovene-1.1}{1992/02/16}{Brought up-to-date with babel 3.2a}
% \changes{slovene-1.2}{1994/02/27}{Update for \LaTeXe}
% \changes{slovene-1.2d}{1994/06/26}{Removed the use of \cs{filedate}
%    and moved identification after the loading of \file{babel.def}}
%
%  \section{The Slovanian language}
%
%    The file \file{\filename}\footnote{The file described in this
%    section has version number \fileversion\ and was last revised on
%    \filedate.  Contributions were made by Danilo Zavrtanik,
%    University of Ljubljana (YU) and Leon \v{Z}lajpah
%    (\texttt{leon.zlajpah@ijs.si}).}  defines all the
%    language-specific macros for the Slovanian language.
%
%    For this language the character |"| is made active. In
%    table~\ref{tab:slovene-quote} an overview is given of its
%    purpose. One of the reasons for this is that in the Slovene
%    language some special characters are used.
%
%    \begin{table}[htb]
%     \begin{center}
%     \begin{tabular}{lp{8cm}}
%      |"c| & |\"c|, also implemented for the 
%                  lowercase and uppercase s and z.                 \\
%      |"-| & an explicit hyphen sign, allowing hyphenation
%                  in the rest of the word.                         \\
%      |""| & like |"-|, but producing no hyphen sign
%                  (for compund words with hyphen, e.g.\ |x-""y|). \\
%      |"`| & for Slovene left double quotes (looks like ,,).   \\
%      |"'| & for Slovene right double quotes.                  \\
%      |"<| & for French left double quotes (similar to $<<$). \\
%      |">| & for French right double quotes (similar to $>>$).\\
%     \end{tabular}
%     \caption{The extra definitions made
%              by \file{slovene.ldf}}\label{tab:slovene-quote}
%     \end{center}
%    \end{table}
%
% \StopEventually{}
%
%    As this file needs to be read only once, we check whether it was
%    read before. If it was, the command |\captionsslovene| is already
%    defined, so we can stop processing. If this command is undefined
%    we proceed with the various definitions and first show the
%    current version of this file.
%
% \changes{slovene-1.0a}{1991/07/15}{Added reset of catcode of @
%    before \cs{endinput}.}
% \changes{slovene-1.0b}{1991/10/29}{Removed use of \cs{@ifundefined}}
%    \begin{macrocode}
%<*code>
\ifx\undefined\captionsslovene
\else
  \selectlanguage{slovene}
  \expandafter\endinput
\fi
%    \end{macrocode}
%
% \changes{slovene-1.0b}{1991/10/29}{Removed code to load
%    \file{latexhax.com}}
%
% \begin{macro}{\atcatcode}
%    This file, \file{slovene.ldf}, may have been read while \TeX\ is
%    in the middle of processing a document, so we have to make sure
%    the category code of \texttt{@} is `letter' while this file is
%    being read.  We save the category code of the @-sign in
%    |\atcatcode| and make it `letter'. Later the category code can be
%    restored to whatever it was before.
%
% \changes{slovene-1.0a}{1991/07/15}{Modified handling of catcode of @
%    again.}
% \changes{slovene-1.0b}{1991/10/29}{Removed use of 
% \cs{makeatletter} and hence the need to load \file{latexhax.com}}
%    \begin{macrocode}
\chardef\atcatcode=\catcode`\@
\catcode`\@=11\relax
%    \end{macrocode}
% \end{macro}
%
%    Now we determine whether the the common macros from the file
%    \file{babel.def} need to be read. We can be in one of two
%    situations: either another language option has been read earlier
%    on, in which case that other option has already read
%    \file{babel.def}, or \texttt{slovene} is the first language
%    option to be processed. In that case we need to read
%    \file{babel.def} right here before we continue.
%
% \changes{slovene-1.1}{1992/02/16}{Added \cs{relax} after the
%    argument of \cs{input}}
%    \begin{macrocode}
\ifx\undefined\babel@core@loaded\input babel.def\relax\fi
%    \end{macrocode}
%
%    Another check that has to be made, is if another language
%    definition file has been read already. In that case its
%    definitions have been activated. This might interfere with
%    definitions this file tries to make. Therefore we make sure that
%    we cancel any special definitions. This can be done by checking
%    the existence of the macro |\originalTeX|. If it exists we simply
%    execute it, otherwise it is |\let| to |\empty|.
% \changes{slovene-1.0a}{1991/07/15}{Added
%    \cs{let}]cs{originalTeX}\cs{relax} to test for existence}
% \changes{slovene-1.1}{1992/02/16}{\cs{originalTeX} should be
%    expandable, {\cs\let} it to \cs{empty}}
%    \begin{macrocode}
\ifx\undefined\originalTeX \let\originalTeX\empty \else\originalTeX\fi
%    \end{macrocode}
%
%    When this file is read as an option, i.e. by the |\usepackage|
%    command, \texttt{slovene} will be an `unknown' language in which
%    case we have to make it known. So we check for the existence of
%    |\l@slovene| to see whether we have to do something here.
%
% \changes{slovene-1.0b}{1991/10/29}{Removed use of \cs{@ifundefined}}
% \changes{slovene-1.1}{1992/02/16}{Added a warning when no
%    hyphenation patterns were loaded.}
% \changes{slovene-1.2d}{1994/06/26}{Now use \cs{@nopatterns} to
%    produce the warning}
%    \begin{macrocode}
\ifx\undefined\l@slovene
    \@nopatterns{Slovene}
    \adddialect\l@slovene0\fi
%    \end{macrocode}
%
%    The next step consists of defining commands to switch to the
%    Slovanian language. The reason for this is that a user might want
%    to switch back and forth between languages.
%
% \begin{macro}{\captionsslovene}
%    The macro |\captionsslovene| defines all strings used in the four
%    standard documentlasses provided with \LaTeX.
% \changes{slovene-1.1}{1992/02/16}{Added \cs{seename}, \cs{alsoname}
%    and \cs{prefacename}}
% \changes{slovene-1.1}{1993/07/15}{\cs{headpagename} should be
%    \cs{pagename}}
% \changes{slovene-1.2b}{1994/06/04}{Added extra tranlations from
%    Josef Leydold, \texttt{leydold@statrix2.wu-wien.ac.at}}
% \changes{slovene-1.2g}{1995/07/04}{Added \cs{proofname} for
%    AMS-\LaTeX}
%    \begin{macrocode}
\addto\captionsslovene{%
  \def\prefacename{Predgovor}%
  \def\refname{Literatura}%
  \def\abstractname{Povzetek}%
  \def\bibname{Literatura}%
  \def\chaptername{Poglavje}%
  \def\appendixname{Dodatek}%
  \def\contentsname{Kazalo}%
  \def\listfigurename{Slike}%
  \def\listtablename{Tabele}%
  \def\indexname{Indeks}%
  \def\figurename{Slika}%
  \def\tablename{Tabela}%
  \def\partname{Del}%
  \def\enclname{Priloge}%
  \def\ccname{Kopije}%
  \def\headtoname{Prejme}%
  \def\pagename{Stran}%
  \def\seename{glej}%
  \def\alsoname{glej tudi}%
  \def\proofname{Proof}%  <-- needs translation
  }%
%    \end{macrocode}
% \end{macro}
%
% \begin{macro}{\dateslovene}
%    The macro |\dateslovene| redefines the command |\today| to
%    produce Slovanian dates.
%    \begin{macrocode}
\def\dateslovene{%
\def\today{\number\day.~\ifcase\month\or
  januar\or februar\or marec\or april\or maj\or junij\or
  julij\or avgust\or september\or oktober\or november\or december\fi
  \space \number\year}}
%    \end{macrocode}
% \end{macro}
%
% \begin{macro}{\extrasslovene}
% \begin{macro}{\noextrasslovene}
%    The macro |\extrasslovene| performs all the extra definitions
%    needed for the Slovanian language. The macro |\noextrasslovene|
%    is used to cancel the actions of |\extrasslovene|. 
%
%    For Slovene the \texttt{"} character is made active. This is done
%    once, later on its definition may vary. Other languages in the
%    same document may also use the \texttt{"} character for
%    shorthands; we specify that the slovanian group of shorthands
%    should be used.
%
% \changes{slovene-1.2f}{1995/06/04}{Introduced the active \texttt{"}}
%    \begin{macrocode}
\initiate@active@char{"}
\addto\extrasslovene{\languageshorthands{slovene}}
\addto\extrasslovene{\bbl@activate{"}}
%\addto\noextrasslovene{\bbl@deactivate{"}}
%    \end{macrocode}
%    First we define shorthands to facilitate the occurence of letters
%    such as \v{c}.
%    \begin{macrocode}
\declare@shorthand{slovene}{"c}{\textormath{\v c}{\check c}}
\declare@shorthand{slovene}{"s}{\textormath{\v s}{\check s}}
\declare@shorthand{slovene}{"z}{\textormath{\v z}{\check z}}
\declare@shorthand{slovene}{"C}{\textormath{\v C}{\check C}}
\declare@shorthand{slovene}{"L}{\textormath{\v L}{\check L}}
\declare@shorthand{slovene}{"S}{\textormath{\v S}{\check S}}
\declare@shorthand{slovene}{"Z}{\textormath{\v Z}{\check Z}}
%    \end{macrocode}
%
%    Then we define access to two forms of quotation marks, similar
%    to the german and french quotation marks.
%    \begin{macrocode}
\declare@shorthand{slovene}{"`}{%
  \textormath{\quotedblbase{}}{\mbox{\quotedblbase}}}
\declare@shorthand{slovene}{"'}{%
  \textormath{\textquotedblleft{}}{\mbox{\textquotedblleft}}}
\declare@shorthand{slovene}{"<}{%
  \textormath{\guillemotleft{}}{\mbox{\guillemotleft}}}
\declare@shorthand{slovene}{">}{%
  \textormath{\guillemotright{}}{\mbox{\guillemotright}}}
%    \end{macrocode}
%    then we define two shorthands to be able to specify hyphenation
%    breakpoints that behavew a little different from |\-|.
%    \begin{macrocode}
\declare@shorthand{slovene}{"-}{\allowhyphens-\allowhyphens}
\declare@shorthand{slovene}{""}{\hskip\z@skip}
%    \end{macrocode}
%    And we want to have a shorthand for disabling a ligature.
%    \begin{macrocode}
\declare@shorthand{slovene}{"|}{%
  \textormath{\discretionary{-}{}{\kern.03em}}{}}
%    \end{macrocode}
% \end{macro}
% \end{macro}
%
%    It is possible that a site might need to add some extra code to
%    the babel macros. To enable this we load a local configuration
%    file, \file{slovene.cfg} if it is found on \TeX' search path.
% \changes{slovene-1.2g}{1995/07/02}{Added loading of configuration
%    file}
%    \begin{macrocode}
\loadlocalcfg{slovene}
%    \end{macrocode}
%
%    Our last action is to make a note that the commands we have just
%    defined, will be executed by calling the macro |\selectlanguage|
%    at the beginning of the document.
%    \begin{macrocode}
\main@language{slovene}
%    \end{macrocode}
%    Finally, the category code of \texttt{@} is reset to its original
%    value. The macrospace used by |\atcatcode| is freed.
% \changes{slovene-1.0a}{1991/07/15}{Modified handling of catcode of
%    @-sign.}
%    \begin{macrocode}
\catcode`\@=\atcatcode \let\atcatcode\relax
%</code>
%    \end{macrocode}
%
% \Finale
%%
%% \CharacterTable
%%  {Upper-case    \A\B\C\D\E\F\G\H\I\J\K\L\M\N\O\P\Q\R\S\T\U\V\W\X\Y\Z
%%   Lower-case    \a\b\c\d\e\f\g\h\i\j\k\l\m\n\o\p\q\r\s\t\u\v\w\x\y\z
%%   Digits        \0\1\2\3\4\5\6\7\8\9
%%   Exclamation   \!     Double quote  \"     Hash (number) \#
%%   Dollar        \$     Percent       \%     Ampersand     \&
%%   Acute accent  \'     Left paren    \(     Right paren   \)
%%   Asterisk      \*     Plus          \+     Comma         \,
%%   Minus         \-     Point         \.     Solidus       \/
%%   Colon         \:     Semicolon     \;     Less than     \<
%%   Equals        \=     Greater than  \>     Question mark \?
%%   Commercial at \@     Left bracket  \[     Backslash     \\
%%   Right bracket \]     Circumflex    \^     Underscore    \_
%%   Grave accent  \`     Left brace    \{     Vertical bar  \|
%%   Right brace   \}     Tilde         \~}
%%
\endinput
}
\DeclareOption{swedish}{% \iffalse meta-comment
%
% Copyright 1989-1995 Johannes L. Braams and any individual authors
% listed elsewhere in this file.  All rights reserved.
% 
% For further copyright information any other copyright notices in this
% file.
% 
% This file is part of the Babel system release 3.5.
% --------------------------------------------------
%   This system is distributed in the hope that it will be useful,
%   but WITHOUT ANY WARRANTY; without even the implied warranty of
%   MERCHANTABILITY or FITNESS FOR A PARTICULAR PURPOSE.
% 
%   For error reports concerning UNCHANGED versions of this file no more
%   than one year old, see bugs.txt.
% 
%   Please do not request updates from me directly.  Primary
%   distribution is through the CTAN archives.
% 
% 
% IMPORTANT COPYRIGHT NOTICE:
% 
% You are NOT ALLOWED to distribute this file alone.
% 
% You are allowed to distribute this file under the condition that it is
% distributed together with all the files listed in manifest.txt.
% 
% If you receive only some of these files from someone, complain!
% 
% Permission is granted to copy this file to another file with a clearly
% different name and to customize the declarations in that copy to serve
% the needs of your installation, provided that you comply with
% the conditions in the file legal.txt from the LaTeX2e distribution.
% 
% However, NO PERMISSION is granted to produce or to distribute a
% modified version of this file under its original name.
%  
% You are NOT ALLOWED to change this file.
% 
% 
% \fi
% \CheckSum{128}
% \iffalse
%    Tell the \LaTeX\ system who we are and write an entry on the
%    transcript.
%<*dtx>
\ProvidesFile{swedish.dtx}
%</dtx>
%<code>\ProvidesFile{swedish.ldf}
        [1995/07/04 v1.3f Swedish support from the babel system]
%
% Babel package for LaTeX version 2e
% Copyright (C) 1989 - 1995
%           by Johannes Braams, TeXniek
%
% Please report errors to: J.L. Braams
%                          JLBraams@cistron.nl
%
%    This file is part of the babel system, it provides the source
%    code for the Swedish language definition file.  A contribution was
%    made by Sten Hellman HELLMAN@CERNVM.CERN.CH
%<*filedriver>
\documentclass{ltxdoc}
\newcommand*\TeXhax{\TeX hax}
\newcommand*\babel{\textsf{babel}}
\newcommand*\langvar{$\langle \it lang \rangle$}
\newcommand*\note[1]{}
\newcommand*\Lopt[1]{\textsf{#1}}
\newcommand*\file[1]{\texttt{#1}}
\begin{document}
 \DocInput{swedish.dtx}
\end{document}
%</filedriver>
%\fi
% \GetFileInfo{swedish.dtx}
%
% \changes{swedish-1.0a}{1991/07/15}{Renamed \file{babel.sty} in
%    \file{babel.com}}
% \changes{swedish-1.1}{1992/02/16}{Brought up-to-date with babel 3.2a}
% \changes{swedish-1.2}{1994/02/27}{Update for LaTeX2e}
% \changes{swedish-1.3d}{1994/06/26}{Removed the use of \cs{filedate}
%    and moved identification after the loading of \file{babel.def}}
% \changes{swedish-1.3e}{1995/05/28}{Update for release 3.5}
%
%  \section{The Swedish language}
%
%    The file \file{\filename}\footnote{The file described in this
%    section has version number \fileversion\ and was last revised on
%    \filedate.  A contribution was made by Sten Hellman
%    (\texttt{HELLMAN@CERNVM.CERN.CH}).}  defines all the
%    language-specific macros for the Swedish language.
%
%    For this language currently no special definitions are needed or
%    available.
%
% \StopEventually{}
%
%    As this file needs to be read only once, we check whether it was
%    read before. If it was, the command |\captionsswedish| is already
%    defined, so we can stop processing. If this command is undefined
%    we proceed with the various definitions and first show the
%    current version of this file.
%
% \changes{swedish-1.0a}{1991/07/15}{Added reset of catcode of @
%    before \cs{endinput}.}
% \changes{swedish-1.0c}{1991/10/29}{Removed use of \cs{@ifundefined}}
%    \begin{macrocode}
%<*code>
\ifx\undefined\captionsswedish
\else
  \selectlanguage{swedish}
  \expandafter\endinput
\fi
%    \end{macrocode}
%
% \changes{swedish-1.0c}{1991/10/29}{Removed code to load
%    \file{latexhax.com}}
%
% \begin{macro}{\atcatcode}
%    This file, \file{swedish.sty}, may have been read while \TeX\ is
%    in the middle of processing a document, so we have to make sure
%    the category code of \texttt{@} is `letter' while this file is
%    being read.  We save the category code of the @-sign in
%    |\atcatcode| and make it `letter'. Later the category code can be
%    restored to whatever it was before.
%
% \changes{swedish-1.0a}{1991/07/15}{Modified handling of catcode of @
%    again.}
% \changes{swedish-1.0c}{1991/10/29}{Removed use of \cs{makeatletter}
%    and hence the need to load \file{latexhax.com}}
%    \begin{macrocode}
\chardef\atcatcode=\catcode`\@
\catcode`\@=11\relax
%    \end{macrocode}
% \end{macro}
%
%    Now we determine whether the the common macros from the file
%    \file{babel.def} need to be read. We can be in one of two
%    situations: either another language option has been read earlier
%    on, in which case that other option has already read
%    \file{babel.def}, or \texttt{swedish} is the first language
%    option to be processed. In that case we need to read
%    \file{babel.def} right here before we continue.
%
% \changes{swedish-1.1}{1992/02/16}{Added \cs{relax} after the
%    argument of \cs{input}}
%    \begin{macrocode}
\ifx\undefined\babel@core@loaded\input babel.def\relax\fi
%    \end{macrocode}
%
%    Another check that has to be made, is if another language
%    definition file has been read already. In that case its
%    definitions have been activated. This might interfere with
%    definitions this file tries to make. Therefore we make sure that
%    we cancel any special definitions. This can be done by checking
%    the existence of the macro |\originalTeX|. If it exists we simply
%    execute it, otherwise it is |\let| to |\empty|.
% \changes{swedish-1.0a}{1991/07/15}{Added
%    \cs{let}\cs{originalTeX}\cs{relax} to test for existence}
% \changes{swedish-1.1}{1992/02/16}{\cs{originalTeX} should be
%    expandable, \cs{let} it to \cs{empty}}
%    \begin{macrocode}
\ifx\undefined\originalTeX \let\originalTeX\empty \else\originalTeX\fi
%    \end{macrocode}
%
%    When this file is read as an option, i.e. by the |\usepackage|
%    command, \texttt{swedish} will be an `unknown' language in which
%    case we have to make it known. So we check for the existence of
%    |\l@swedish| to see whether we have to do something here.
%
% \changes{swedish-1.0c}{1991/10/29}{Removed use of \cs{@ifundefined}}
% \changes{swedish-1.1}{1992/02/16}{Added a warning when no hyphenation
%    patterns were loaded.}
% \changes{swedish-1.3d}{1994/06/26}{Now use \cs{@nopatterns} to
%    producew the warning}
%    \begin{macrocode}
\ifx\undefined\l@swedish
    \@nopatterns{Swedish}
    \adddialect\l@swedish0\fi
%    \end{macrocode}
%
%    The next step consists of defining commands to switch to the
%    Swedish language. The reason for this is that a user might want
%    to switch back and forth between languages.
%
% \begin{macro}{\captionsswedish}
%    The macro |\captionsswedish| defines all strings used in the four
%    standard documentclasses provided with \LaTeX.
% \changes{swedish-1.0b}{1991/08/21}{removed type in definition of
%    \cs{contentsname}}
% \changes{swedish-1.0b}{1991/08/21}{added definition for \cs{enclname}}
% \changes{swedish-1.0b}{1991/08/21}{made definition of \cs{refname}
%    pluralis}
% \changes{swedish-1.1}{1992/02/16}{Added \cs{seename}, \cs{alsoname}
%    and \cs{prefacename}}
% \changes{swedish-1.1}{1993/07/15}{\cs{headpagename} should be
%    \cs{pagename}}
% \changes{swedish-1.1b}{1993/09/16}{Added translations}
% \changes{swedish-1.3d}{1994/07/27}{Changed \cs{aa} to
%    \cs{csname}\texttt{ aa}\cs{endcsname}, to make \cs{uppercase} do
%    the right thing}
% \changes{swedish-1.3f}{1995/07/04}{Added \cs{proofname} for
%    AMS-\LaTeX}
%    \begin{macrocode}
\addto\captionsswedish{%
  \def\prefacename{F\"orord}%
  \def\refname{Referenser}%
  \def\abstractname{Sammanfattning}%
  \def\bibname{Litteraturf\"orteckning}%
  \def\chaptername{Kapitel}%
  \def\appendixname{Bilaga}%
  \def\contentsname{Inneh\csname aa\endcsname ll}%
  \def\listfigurename{Figurer}%
  \def\listtablename{Tabeller}%
  \def\indexname{Sakregister}%
  \def\figurename{Figur}%
  \def\tablename{Tabell}%
  \def\partname{Del}%
  \def\enclname{Bil}%
  \def\ccname{Kopia f\"or k\"annedom}%
  \def\headtoname{Till}% in letter
  \def\pagename{Sida}%
  \def\seename{se}%
%    \end{macrocode}
% \changes{swedish-1.3e}{1995/05/29}{Changed \cs{alsoname} from
%    `\texttt{se ocks\aa}'}
%    \begin{macrocode}
  \def\alsoname{se \"aven}%
  \def\proofname{Proof}%   <-- needs translation
  }%
%    \end{macrocode}
% \end{macro}
%
% \begin{macro}{\dateswedish}
%    The macro |\dateswedish| redefines the command |\today| to
%    produce Swedish dates.
%    \begin{macrocode}
\def\dateswedish{%
\def\today{\number\day~\ifcase\month\or
  januari\or februari\or mars\or april\or maj\or juni\or
  juli\or augusti\or september\or oktober\or november\or december\fi
  \space\number\year}}
%    \end{macrocode}
% \end{macro}
%
%  \begin{macro}{\swedishhyphenmins}
%    The swedish hyphenation patterns can be used with |\lefthyphenmin|
%    set to~2 and |\righthyphenmin| set to~2.
% \changes{swedish-1.3e}{1995/06/02}{use \cs{swedishhyphenmins} to
%    store the correct values}
%    \begin{macrocode}
\def\swedishhyphenmins{\tw@\tw@}
%    \end{macrocode}
%  \end{macro}
%
% \begin{macro}{\extrasswedish}
% \changes{swedish-1.3e}{1995/05/29}{Added \cs{bbl@frenchspacing}}
% \begin{macro}{\noextrasswedish}
% \changes{swedish-1.3e}{1995/05/29}{Added \cs{bbl@nonfrenchspacing}}
%    The macro |\extrasswedish| performs all the extra definitions
%    needed for the Swedish language. The macro |\noextrasswedish| is
%    used to cancel the actions of |\extrasswedish|.  For the moment
%    this macro is empty but it is defined for compatibility with the
%    other language definition files.
%
%    For Swedish texts |\frenchspacing| should be in effect.  We make
%    sure this is the case and reset it if necessary.
%
%    \begin{macrocode}
\addto\extrasswedish{\bbl@frenchspacing}
\addto\noextrasswedish{\bbl@nonfrenchspacing}
%    \end{macrocode}
% \end{macro}
% \end{macro}
%
%    It is possible that a site might need to add some extra code to
%    the babel macros. To enable this we load a local configuration
%    file, \file{swedish.cfg} if it is found on \TeX' search path.
% \changes{swedish-1.3f}{1995/07/02}{Added loading of configuration
%    file}
%    \begin{macrocode}
\loadlocalcfg{swedish}
%    \end{macrocode}
%
%    Our last action is to make a note that the commands we have just
%    defined, will be executed by calling the macro |\selectlanguage|
%    at the beginning of the document.
%    \begin{macrocode}
\main@language{swedish}
%    \end{macrocode}
%    Finally, the category code of \texttt{@} is reset to its original
%    value. The macrospace used by |\atcatcode| is freed.
% \changes{swedish-1.0a}{1991/07/15}{Modified handling of catcode of
%    @-sign.}
%    \begin{macrocode}
\catcode`\@=\atcatcode \let\atcatcode\relax
%</code>
%    \end{macrocode}
%
%\Finale
%%
%% \CharacterTable
%%  {Upper-case    \A\B\C\D\E\F\G\H\I\J\K\L\M\N\O\P\Q\R\S\T\U\V\W\X\Y\Z
%%   Lower-case    \a\b\c\d\e\f\g\h\i\j\k\l\m\n\o\p\q\r\s\t\u\v\w\x\y\z
%%   Digits        \0\1\2\3\4\5\6\7\8\9
%%   Exclamation   \!     Double quote  \"     Hash (number) \#
%%   Dollar        \$     Percent       \%     Ampersand     \&
%%   Acute accent  \'     Left paren    \(     Right paren   \)
%%   Asterisk      \*     Plus          \+     Comma         \,
%%   Minus         \-     Point         \.     Solidus       \/
%%   Colon         \:     Semicolon     \;     Less than     \<
%%   Equals        \=     Greater than  \>     Question mark \?
%%   Commercial at \@     Left bracket  \[     Backslash     \\
%%   Right bracket \]     Circumflex    \^     Underscore    \_
%%   Grave accent  \`     Left brace    \{     Vertical bar  \|
%%   Right brace   \}     Tilde         \~}
%%
\endinput}
\DeclareOption{turkish}{% \iffalse meta-comment
%
% Copyright 1989-1995 Johannes L. Braams and any individual authors
% listed elsewhere in this file.  All rights reserved.
% 
% For further copyright information any other copyright notices in this
% file.
% 
% This file is part of the Babel system release 3.5.
% --------------------------------------------------
%   This system is distributed in the hope that it will be useful,
%   but WITHOUT ANY WARRANTY; without even the implied warranty of
%   MERCHANTABILITY or FITNESS FOR A PARTICULAR PURPOSE.
% 
%   For error reports concerning UNCHANGED versions of this file no more
%   than one year old, see bugs.txt.
% 
%   Please do not request updates from me directly.  Primary
%   distribution is through the CTAN archives.
% 
% 
% IMPORTANT COPYRIGHT NOTICE:
% 
% You are NOT ALLOWED to distribute this file alone.
% 
% You are allowed to distribute this file under the condition that it is
% distributed together with all the files listed in manifest.txt.
% 
% If you receive only some of these files from someone, complain!
% 
% Permission is granted to copy this file to another file with a clearly
% different name and to customize the declarations in that copy to serve
% the needs of your installation, provided that you comply with
% the conditions in the file legal.txt from the LaTeX2e distribution.
% 
% However, NO PERMISSION is granted to produce or to distribute a
% modified version of this file under its original name.
%  
% You are NOT ALLOWED to change this file.
% 
% 
% \fi
% \CheckSum{203}
% \iffalse
%    Tell the \LaTeX\ system who we are and write an entry on the
%    transcript.
%<*dtx>
\ProvidesFile{turkish.dtx}
%</dtx>
%<code>\ProvidesFile{turkish.ldf}
        [1995/07/04 v1.2f Turkish support from the babel system]
%    \end{macrocode}
% Babel package for LaTeX version 2e
% Copyright (C) 1989 - 1995
%           by Johannes Braams, TeXniek
%
% Please report errors to: J.L. Braams
%                          JLBraams@cistron.nl
%
% Turkish Language Definition File
% Copyright (C) 1994 - 1995
%           by Mustafa Burc
%           rz6001@rziris01.rrz.uni-hamburg.de
%          (40) 2503476
%
%              Johannes Braams, TeXniek
%              Kooienswater 62
%              2715 AJ Zoetermeer
%              The Netherlands
%
%    This file is part of the babel system, it provides the source
%    code for the Turkish language definition file.
%<*filedriver>
\documentclass{ltxdoc}
\newcommand*\TeXhax{\TeX hax}
\newcommand*\babel{\textsf{babel}}
\newcommand*\langvar{$\langle \it lang \rangle$}
\newcommand*\note[1]{}
\newcommand*\Lopt[1]{\textsf{#1}}
\newcommand*\file[1]{\texttt{#1}}
\begin{document}
 \DocInput{turkish.dtx}
\end{document}
%</filedriver>
%\fi
% \GetFileInfo{turkish.dtx}
%
% \changes{turkish-1.2}{1994/02/27}{Update for \LaTeXe}
% \changes{turkish-1.2c}{1994/06/26}{Removed the use of \cs{filedate}
%    and moved identification after the loading of \file{babel.def}}
%
%  \section{The Turkish language}
%
%    The file \file{\filename}\footnote{The file described in this
%    section has version number \fileversion\ and was last revised on
%    \filedate.}  defines all the language definition macros for the
%    Turkish language\footnote{Mustafa Burc,
%    \texttt{z6001@rziris01.rrz.uni-hamburg.de} provided the code for
%    this file. It is based on the work by Pierre Mackay}.
%
%    Turkish typographic rules specify that a little `white space'
%    should be added before the characters `\texttt{:}', `\texttt{!}'
%    and `\texttt{=}'. In order to insert this white space
%    automatically these characters are made `active'. Also
%    |\frenhspacing| is set.
%
% \StopEventually{}
%
%    As this file needs to be read only once, we check whether it was
%    read before. If it was, the command |\captionsturkish| is already
%    defined, so we can stop processing. If this command is undefined
%    we proceed with the various definitions and first show the
%    current version of this file.
%
%    \begin{macrocode}
%<*code>
\ifx\undefined\captionsturkish
\else
  \selectlanguage{turkish}
  \expandafter\endinput
\fi
%    \end{macrocode}
%
% \begin{macro}{\atcatcode}
%    This file, \file{turkish.sty}, may have been read while \TeX\ is
%    in the middle of processing a document, so we have to make sure
%    the category code of \texttt{@} is `letter' while this file is
%    being read.  We save the category code of the @-sign in
%    |\atcatcode| and make it `letter'. Later the category code can be
%    restored to whatever it was before.
%    \begin{macrocode}
\chardef\atcatcode=\catcode`\@
\catcode`\@=11\relax
%    \end{macrocode}
% \end{macro}
%
%    Now we determine whether the the common macros from the file
%    \file{babel.def} need to be read. We can be in one of two
%    situations: either another language option has been read earlier
%    on, in which case that other option has already read
%    \file{babel.def}, or \texttt{turkish} is the first language
%    option to be processed. In that case we need to read
%    \file{babel.def} right here before we continue.
%
%    \begin{macrocode}
\ifx\undefined\babel@core@loaded\input babel.def\relax\fi
%    \end{macrocode}
%
%    Another check that has to be made, is if another language
%    definition file has been read already. In that case its definitions
%    have been activated. This might interfere with definitions this
%    file tries to make. Therefore we make sure that we cancel any
%    special definitions. This can be done by checking the existence
%    of the macro |\originalTeX|. If it exists we simply execute it.
%    \begin{macrocode}
\ifx\undefined\originalTeX \let\originalTeX\empty\fi
\originalTeX
%    \end{macrocode}
%
%    When this file is read as an option, i.e. by the |\usepackage|
%    command, \texttt{turkish} could be an `unknown' language in which
%    case we have to make it known. So we check for the existence of
%    |\l@turkish| to see whether we have to do something here.
%
% \changes{turkish-1.2c}{1994/06/26}{Now use \cs{@nopatterns} to
%    produce the warning}
%    \begin{macrocode}
\ifx\undefined\l@turkish
  \@nopatterns{Turkish}
  \adddialect\l@turkish0\fi
%    \end{macrocode}
%
%    The next step consists of defining commands to switch to (and
%    from) the Turkish language.
%
% \begin{macro}{\captionsturkish}
%    The macro |\captionsturkish| defines all strings used in the four
%    standard documentclasses provided with \LaTeX.
% \changes{turkish-1.1}{1993/07/15}{\cs{headpagename} should be
%    \cs{pagename}}
% \changes{turkish-1.2b}{1994/06/04}{Added braces behind \cs{i} in
%    strings}
% \changes{v1.2f}{1995/07/04}{Added \cs{proofname} for AMS-\LaTeX}
%    \begin{macrocode}
\addto\captionsturkish{%
  \def\prefacename{Preface}% <-- This needs translation!!
  \def\refname{Ba\c svurulan Kitaplar}%
  \def\abstractname{Konu}%
  \def\bibname{Bibliografi}%
  \def\chaptername{Anab\"ol\"um}%
  \def\appendixname{Appendix}%
  \def\contentsname{\.I\c cindekiler}%
  \def\listfigurename{\c Sekiller Listesi}%
  \def\listtablename{Tablolar\i{}n Listesi}%
  \def\indexname{\.Index}%
  \def\figurename{\c Sekiller}%
  \def\tablename{Tablo}%
  \def\partname{B\"ol\"um}%
  \def\enclname{Ekler}%
  \def\ccname{G\"onderen}%
  \def\headtoname{Al\i{}c\i}%
  \def\pagename{Sayfa}%
  \def\subjectname{To}% <-- This needs translation!!
  \def\seename{see}% <-- This needs translation!!
  \def\alsoname{see also}% <-- This needs translation!!
  \def\proofname{Proof}% <-- This needs translation!!
}%
%    \end{macrocode}
% \end{macro}
%
% \begin{macro}{\dateturkish}
%    The macro |\dateturkish| redefines the command |\today| to
%    produce Turkish dates.
% \changes{turkish-1.2b}{1994/06/04}{Added braces behind \cs{i} in
%    strings}
% \changes{turkish-1.2d}{1995/01/31}{removed extra closing brace,
%    \cs{mont} should be \cs{month}}
%    \begin{macrocode}
\def\dateturkish{%
  \def\today{\number\day.~\ifcase\month\or
    Ocak\or \c Subat\or Mart\or Nisan\or May\i{}s\or Haziran\or
    Temmuz\or A\u gustos\or Eyl\"ul\or Ekim\or Kas\i{}m\or
    Aral\i{}k\fi
    \space\number\year}}
%    \end{macrocode}
% \end{macro}
%
% \begin{macro}{\extrasturkish}
% \changes{turkish-1.2e}{1995/05/15}{Completely rewrote macro}
% \begin{macro}{\noextrasturkish}
%    The macro |\extrasturkish| will perform all the extra definitions
%    needed for the Turkish language. The macro |\noextrasturkish| is
%    used to cancel the actions of |\extrasturkish|.
%
%    Turkish typographic rules specify that a little `white space'
%    should be added before the characters `\texttt{:}', `\texttt{!}'
%    and `\texttt{=}'. In order to insert this white space
%    automatically these characters are made |\active|, so they have
%    to be treated in a special way.
%    \begin{macrocode}
\initiate@active@char{:}
\initiate@active@char{!}
\initiate@active@char{=}
%    \end{macrocode}
%    We specify that the turkish group of shorthands should be used.
%    \begin{macrocode}
\addto\extrasturkish{\languageshorthands{turkish}}
%    \end{macrocode}
%    These characters are `turned on' once, later their definition may
%    vary. 
%    \begin{macrocode}
\addto\extrasturkish{%
  \bbl@activate{:}\bbl@activate{!}\bbl@activate{=}}
%    \end{macrocode}
%
%    For Turkish texts |\frenchspacing| should be in effect. We
%    make sure this is the case and reset it if necessary.
% \changes{turkish-1.2e}{1995/05/15}{now use \cs{bbl@frenchspacing}
%    and \cs{bbl@nonfrenchspacing}}
%    \begin{macrocode}
\addto\extrasturkish{\bbl@frenchspacing}
\addto\noextrasturkish{\bbl@nonfrenchspacing}
%    \end{macrocode}
% \end{macro}
% \end{macro}
%
% \begin{macro}{\turkish@sh@:@}
% \begin{macro}{\turksih@sh@!@}
% \begin{macro}{\turkish@sh@=@}
%    The definitions for the three active characters were made using
%    intermediate macros. These are defined now. The insertion of
%    extra `white space' should only happen outside math mode, hence
%    the check |\ifmmode| in the macros.
% \changes{turkish-1.2d}{1995/01/31}{Added missing \cs{def}}
% \changes{turkish-1.2e}{1995/05/15}{Use the new mechanism of
%    \cs{declare@shorthand}}
%    \begin{macrocode}
\declare@shorthand{turkish}{:}{%
  \ifmmode
    \string:%
  \else\relax
    \ifhmode
      \ifdim\lastskip>\z@
        \unskip\penalty\@M\thinspace
      \fi
    \fi
    \string:%
  \fi}
\declare@shorthand{turkish}{!}{%
  \ifmmode
    \string!%
  \else\relax
    \ifhmode
      \ifdim\lastskip>\z@
        \unskip\penalty\@M\thinspace
      \fi
    \fi
    \string!%
  \fi}
\declare@shorthand{turkish}{=}{%
  \ifmmode
    \string=%
  \else\relax
    \ifhmode
      \ifdim\lastskip>\z@
        \unskip\kern\fontdimen2\font
        \kern-1.4\fontdimen3\font
      \fi
    \fi
    \string=%
  \fi}
%    \end{macrocode}
% \end{macro}
% \end{macro}
% \end{macro}
%
%    It is possible that a site might need to add some extra code to
%    the babel macros. To enable this we load a local configuration
%    file, \file{turkish.cfg} if it is found on \TeX' search path.
% \changes{turkish-1.2f}{1995/07/02}{Added loading of configuration
%    file}
%    \begin{macrocode}
\loadlocalcfg{turkish}
%    \end{macrocode}
%
%    Our last action is to make a note that the commands we have just
%    defined, will be executed by calling the macro |\selectlanguage|
%    at the beginning of the document.
%    \begin{macrocode}
\main@language{turkish}
%    \end{macrocode}
%    Finally, the category code of \texttt{@} is reset to its original
%    value. The macrospace used by |\atcatcode| is freed.
%    \begin{macrocode}
\catcode`\@\atcatcode \let\atcatcode\relax
%</code>
%    \end{macrocode}
%
% \Finale
%%
%% \CharacterTable
%%  {Upper-case    \A\B\C\D\E\F\G\H\I\J\K\L\M\N\O\P\Q\R\S\T\U\V\W\X\Y\Z
%%   Lower-case    \a\b\c\d\e\f\g\h\i\j\k\l\m\n\o\p\q\r\s\t\u\v\w\x\y\z
%%   Digits        \0\1\2\3\4\5\6\7\8\9
%%   Exclamation   \!     Double quote  \"     Hash (number) \#
%%   Dollar        \$     Percent       \%     Ampersand     \&
%%   Acute accent  \'     Left paren    \(     Right paren   \)
%%   Asterisk      \*     Plus          \+     Comma         \,
%%   Minus         \-     Point         \.     Solidus       \/
%%   Colon         \:     Semicolon     \;     Less than     \<
%%   Equals        \=     Greater than  \>     Question mark \?
%%   Commercial at \@     Left bracket  \[     Backslash     \\
%%   Right bracket \]     Circumflex    \^     Underscore    \_
%%   Grave accent  \`     Left brace    \{     Vertical bar  \|
%%   Right brace   \}     Tilde         \~}
%%
\endinput
}
\DeclareOption{uppersorbian}{% \iffalse meta-comment
%
% Copyright 1989-1995 Johannes L. Braams and any individual authors
% listed elsewhere in this file.  All rights reserved.
% 
% For further copyright information any other copyright notices in this
% file.
% 
% This file is part of the Babel system release 3.5.
% --------------------------------------------------
%   This system is distributed in the hope that it will be useful,
%   but WITHOUT ANY WARRANTY; without even the implied warranty of
%   MERCHANTABILITY or FITNESS FOR A PARTICULAR PURPOSE.
% 
%   For error reports concerning UNCHANGED versions of this file no more
%   than one year old, see bugs.txt.
% 
%   Please do not request updates from me directly.  Primary
%   distribution is through the CTAN archives.
% 
% 
% IMPORTANT COPYRIGHT NOTICE:
% 
% You are NOT ALLOWED to distribute this file alone.
% 
% You are allowed to distribute this file under the condition that it is
% distributed together with all the files listed in manifest.txt.
% 
% If you receive only some of these files from someone, complain!
% 
% Permission is granted to copy this file to another file with a clearly
% different name and to customize the declarations in that copy to serve
% the needs of your installation, provided that you comply with
% the conditions in the file legal.txt from the LaTeX2e distribution.
% 
% However, NO PERMISSION is granted to produce or to distribute a
% modified version of this file under its original name.
%  
% You are NOT ALLOWED to change this file.
% 
% 
% \fi
% \CheckSum{330}
% \iffalse
%    Tell the \LaTeX\ system who we are and write an entry on the
%    transcript.
%<*dtx>
\ProvidesFile{usorbian.dtx}
%</dtx>
%<code>\ProvidesFile{usorbian.ldf}
        [1995/07/04 v1.0b Upper Sorbian support from the babel system]
%
% Babel package for LaTeX version 2e
% Copyright (C) 1989 - 1995
%           by Johannes Braams, TeXniek
%
% Upper Sorbian Language Definition File
% Copyright (C) 1994 - 1995
%           by Eduard Werner
%           Werner, Eduard",
%           Serbski institut z. t.,
%           Dw\'orni\v{s}\'cowa 6
%           02625 Budy\v{s}in/Bautzen
%           Germany",
%           (??)3591 497223",
%           edi@kaihh.hanse.de",
%
% Please report errors to: Eduard Werner <edi@kaihh.hanse.de>
%
%    This file is part of the babel system, it provides the source
%    code for the Upper Sorbian definition file.
%<*filedriver>
\documentclass{ltxdoc}
\newcommand*\TeXhax{\TeX hax}
\newcommand*\babel{\textsf{babel}}
\newcommand*\langvar{$\langle \it lang \rangle$}
\newcommand*\note[1]{}
\newcommand*\Lopt[1]{\textsf{#1}}
\newcommand*\file[1]{\texttt{#1}}
\newfont{\logo}{logo10}
\newcommand*\MF{{\logo METAFONT}}
\begin{document}
 \DocInput{usorbian.dtx}
\end{document}
%</filedriver>
%\fi
% \GetFileInfo{usorbian.dtx}
%
% \changes{usorbian-0.1}{1994/10/10}{First version}
% \changes{usorbian-0.1b}{1994/10/18}{Made it possible to run through
%    \LaTeX; added \cs{MF} and removed extra \cs{end{macro}}}
%
%  \section{The Upper Sorbian language}
%
%    The file \file{\filename}\footnote{The file described in this
%    section has version number \fileversion\ and was last revised on
%    \filedate.  It was written by Eduard Werner
%    (\texttt{edi@kaihh.hanse.de}).}  It defines all the
%    language-specific macros for Upper Sorbian.
%
% \StopEventually{}
%
%    As this file needs to be read only once, we check whether it was
%    read before. If it was, the command |\captionusorbian| is already
%    defined, so we can stop processing. If this command is undefined
%    we proceed with the various definitions and first show the
%    current version of this file.
%
%    \begin{macrocode}
\ifx\undefined\captionsusorbian
\else
  \selectlanguage{usorbian}
  \expandafter\endinput
\fi
%    \end{macrocode}
%
%  \begin{macro}{\atcatcode}
%    This file, \file{usorbian.sty}, may have been read while \TeX\ is
%    in the middle of processing a document, so we have to make sure
%    the category code of \texttt{@} is `letter' while this file is
%    being read.  We save the category code of the @-sign in
%    |\atcatcode| and make it `letter'. Later the category code can be
%    restored to whatever it was before.
%
%    \begin{macrocode}
\chardef\atcatcode=\catcode`\@
\catcode`\@=11\relax
%    \end{macrocode}
% \end{macro}
%
%    Now we determine whether the the common macros from the file
%    \file{babel.def} need to be read. We can be in one of two
%    situations: either another language option has been read earlier
%    on, in which case that other option has already read
%    \file{babel.def}, or \texttt{usorbian} is the first language
%    option to be processed. In that case we need to read
%    \file{babel.def} right here before we continue.
%
%    \begin{macrocode}
\ifx\undefined\babel@core@loaded\input babel.def\relax\fi
%    \end{macrocode}
%
%    Another check that has to be made, is if another language
%    definition file has been read already. In that case its definitions
%    have been activated. This might interfere with definitions this
%    file tries to make. Therefore we make sure that we cancel any
%    special definitions. This can be done by checking the existence
%    of the macro |\originalTeX|. If it exists we simply execute it,
%    otherwise it is |\let| to |\empty|.
%
%    \begin{macrocode}
\ifx\undefined\originalTeX \let\originalTeX\empty \else\originalTeX\fi
%    \end{macrocode}
%
%    When this file is read as an option, i.e. by the |\usepackage|
%    command, \texttt{usorbian} will be an `unknown' languagein which
%    case we have to make it known. So we check for the existence of
%    |\l@usorbian| to see whether we have to do something here.
%
%    \begin{macrocode}
\ifx\undefined\l@usorbian
    \@nopatterns{Usorbian}
    \adddialect\l@usorbian0\fi
%    \end{macrocode}
%
%    The next step consists of defining commands to switch to (and
%    from) the Upper Sorbian language.
%
% \begin{macro}{\captionsusorbian}
%    The macro |\captionsusorbian| defines all strings used in the four
%    standard documentclasses provided with \LaTeX.
% \changes{usorbian-0.1c}{1994/11/27}{Removed two typos (Kapitel and
%    Dodatki)}
% \changes{usorbian-1.0b}{1995/07/04}{Added \cs{proofname} for
%    AMS-\LaTeX}
%    \begin{macrocode}
\addto\captionsusorbian{%
  \def\prefacename{Zawod}%
  \def\refname{Referency}%
  \def\abstractname{Abstrakt}%
  \def\bibname{Literatura}%
  \def\chaptername{Kapitl}%
  \def\appendixname{Dodawki}%
  \def\contentsname{Wobsah}%
  \def\listfigurename{Zapis wobrazow}%
  \def\listtablename{Zapis tabulkow}%
  \def\indexname{Indeks}%
  \def\figurename{Wobraz}%
  \def\tablename{Tabulka}%
  \def\partname{D\'z\v el}%
  \def\enclname{P\v r\l oha}%
  \def\ccname{CC}%
  \def\headtoname{Komu}%
  \def\pagename{Strona}%
  \def\seename{hl.}%
  \def\alsoname{hl.~te\v z}
  \def\proofname{Proof}%  <-- needs translation
  }%
%    \end{macrocode}
% \end{macro}
%
% \begin{macro}{\newdateusorbian}
%    The macro |\newdateusorbian| redefines the command |\today| to
%    produce Upper Sorbian dates.
%    \begin{macrocode}
\def\newdateusorbian{%
\def\today{\number\day.~\ifcase\month\or
januara\or februara\or m\v erca\or apryla\or meje\or junija\or
  julija\or awgusta\or septembra\or oktobra\or
  nowembra\or decembra\fi
    \space \number\year}}
%    \end{macrocode}
% \end{macro}
%
% \begin{macro}{\olddateusorbian}
%    The macro |\olddateusorbian| redefines the command |\today| to
%    produce old-style Upper Sorbian dates.
%    \begin{macrocode}
\def\olddateusorbian{%
\def\today{\number\day.~\ifcase\month\or
  wulkeho r\'o\v zka\or ma\l eho r\'o\v zka\or nal\v etnika\or
  jutrownika\or r\'o\v zownika\or  sma\v znika\or pra\v znika\or
  \v znjenca\or po\v znjenca\or winowca\or nazymnika\or
  hodownika\fi \space \number\year}}
%    \end{macrocode}
% \end{macro}
%
%    The default will be the new-style dates.
%    \begin{macrocode}
\let\dateusorbian\newdateusorbian
%    \end{macrocode}
%
% \begin{macro}{\extrasusorbian}
%    The macro |\extrasusorbian| will perform all the extra
%    definitions needed for the Upper Sorbian language. It's pirated
%    from |germanb.sty|.  The macro |\noextrasusorbian| is used to
%    cancel the actions of |\extrasusorbian|.
%
%    Because for Upper Sorbian (as well as for Dutch) the \texttt{"}
%    character is made active. This is done once, later on its
%    definition may vary.
%    \begin{macrocode}
\initiate@active@char{"}
\addto\extrasusorbian{\languageshorthands{usorbian}}
\addto\extrasusorbian{\bbl@activate{"}}
%\addto\noextrasusorbian{\bbl@deactivate{"}}
%    \end{macrocode}
%
%    In order for \TeX\ to be able to hyphenate German Upper Sorbian
%    words which contain `\ss' we have to give the character a nonzero
%    |\lccode| (see Appendix H, the \TeX book).
%    \begin{macrocode}
\addto\extrasusorbian{\babel@savevariable{\lccode`\^^Y}%
  \lccode`\^^Y`\^^Y}
%    \end{macrocode}
%    The umlaut accent macro |\"| is changed to lower the umlaut dots.
%    The redefinition is done with the help of |\umlautlow|.
%    \begin{macrocode}
\addto\extrasusorbian{\babel@save\"\umlautlow}
\addto\noextrasusorbian{\umlauthigh}
%    \end{macrocode}
%    The Upper Sorbian hyphenation patterns can be used with
%    |\lefthyphenmin| and |\righthyphenmin| set to~2.
%    \begin{macrocode}
\def\usorbianhyphenmins{\tw@\tw@}
%    \end{macrocode}
% \end{macro}
%
% \changes{usorbian-1.0a}{1995/05/27}{Removed stuff that has been
%    moved to \file{babel.def}}
%
%  \begin{macro}{\dq}
%    We save the original double quote character in |\dq| to keep it
%    available, the math accent |\"| can now be typed as |"|.  Also we
%    store the original meaning of the command |\"| for future use.
%    \begin{macrocode}
\begingroup \catcode`\"12
\def\x{\endgroup
  \def\@SS{\mathchar"7019 }
  \def\dq{"}}
\x
%    \end{macrocode}
% \end{macro}
%
%    Now we can define the doublequote macros: the umlauts,
%    \begin{macrocode}
\declare@shorthand{usorbian}{"a}{\textormath{\"{a}}{\ddot a}}
\declare@shorthand{usorbian}{"o}{\textormath{\"{o}}{\ddot o}}
\declare@shorthand{usorbian}{"u}{\textormath{\"{u}}{\ddot u}}
\declare@shorthand{usorbian}{"A}{\textormath{\"{A}}{\ddot A}}
\declare@shorthand{usorbian}{"O}{\textormath{\"{O}}{\ddot O}}
\declare@shorthand{usorbian}{"U}{\textormath{\"{U}}{\ddot U}}
%    \end{macrocode}
%    tremas,
%    \begin{macrocode}
\declare@shorthand{usorbian}{"e}{\textormath{\"{e}}{\ddot e}}
\declare@shorthand{usorbian}{"E}{\textormath{\"{E}}{\ddot E}}
\declare@shorthand{usorbian}{"i}{\textormath{\"{\i}}{\ddot\imath}}
\declare@shorthand{usorbian}{"I}{\textormath{\"{I}}{\ddot I}}
%    \end{macrocode}
%    usorbian es-zet (sharp s),
%    \begin{macrocode}
\declare@shorthand{usorbian}{"s}{\textormath{\ss{}}{\@SS{}}}
\declare@shorthand{usorbian}{"S}{SS}
%    \end{macrocode}
%    german and french quotes,
%    \begin{macrocode}
\declare@shorthandusorbian{}{"`}{%
  \textormath{\quotedblbase{}}{\mbox{\quotedblbase}}}
\declare@shorthand{usorbian}{"'}{%
  \textormath{\textquotedblleft{}}{\mbox{\textquotedblleft}}}
\declare@shorthand{usorbian}{"<}{%
  \textormath{\guillemotleft{}}{\mbox{\guillemotleft}}}
\declare@shorthand{usorbian}{">}{%
  \textormath{\guillemotright{}}{\mbox{\guillemotright}}}
%    \end{macrocode}
%    discretionary commands
%    \begin{macrocode}
\declare@shorthand{usorbian}{"c}{%
  \textormath{\usorbian@dq@disc ck}{c}}
\declare@shorthand{usorbian}{"C}{%
  \textormath{\usorbian@dq@disc CK}{C}}
\declare@shorthand{usorbian}{"f}{%
  \textormath{\usorbian@dq@disc f{ff}}{f}}
\declare@shorthand{usorbian}{"F}{%
  \textormath{\usorbian@dq@disc F{FF}}{F}}
\declare@shorthand{usorbian}{"l}{%
  \textormath{\usorbian@dq@disc l{ll}}{l}}
\declare@shorthand{usorbian}{"L}{%
  \textormath{\usorbian@dq@disc L{LL}}{L}}
\declare@shorthand{usorbian}{"m}{%
  \textormath{\usorbian@dq@disc m{mm}}{m}}
\declare@shorthand{usorbian}{"M}{%
  \textormath{\usorbian@dq@disc M{MM}}{M}}
\declare@shorthand{usorbian}{"n}{%
  \textormath{\usorbian@dq@disc n{nn}}{n}}
\declare@shorthand{usorbian}{"N}{%
  \textormath{\usorbian@dq@disc N{NN}}{N}}
\declare@shorthand{usorbian}{"p}{%
  \textormath{\usorbian@dq@disc p{pp}}{p}}
\declare@shorthand{usorbian}{"P}{%
  \textormath{\usorbian@dq@disc P{PP}}{P}}
\declare@shorthand{usorbian}{"t}{%
  \textormath{\usorbian@dq@disc t{tt}}{t}}
\declare@shorthand{usorbian}{"T}{%
  \textormath{\usorbian@dq@disc T{TT}}{T}}
%    \end{macrocode}
%    and some additional commands:
%    \begin{macrocode}
\declare@shorthand{usorbian}{"-}{\penalty\@M\-\allowhyphens}
\declare@shorthand{usorbian}{"|}{%
  \textormath{\penalty\@M\discretionary{-}{}{\kern.03em}%
              \allowhyphens}{}}
\declare@shorthand{usorbian}{""}{\hskip\z@skip}
%    \end{macrocode}
%
%  \begin{macro}{\mdqon}
%  \begin{macro}{\mdqoff}
%  \begin{macro}{\ck}
%    All that's left to do now is to  define a couple of commands
%    for reasons of compatibility with \file{german.sty}.
%    \begin{macrocode}
\def\mdqon{\bbl@activate{"}}
\def\mdqoff{\bbl@deactivate{"}}
\def\ck{\allowhyphens\discretionary{k-}{k}{ck}\allowhyphens}
%    \end{macrocode}
%  \end{macro}
%  \end{macro}
%  \end{macro}
%
%    It is possible that a site might need to add some extra code to
%    the babel macros. To enable this we load a local configuration
%    file, \file{usorbian.cfg} if it is found on \TeX' search path.
% \changes{usorbian-1.0b}{1995/07/02}{Added loading of configuration
%    file}
%    \begin{macrocode}
\loadlocalcfg{usorbian}
%    \end{macrocode}
%
%    Our last action is to make a note that the commands we have just
%    defined, will be executed by calling the macro |\selectlanguage|
%    at the beginning of the document.
%    \begin{macrocode}
\main@language{usorbian}
%    \end{macrocode}
%    Finally, the category code of \texttt{@} is reset to its original
%    value.
%
%    \begin{macrocode}
\catcode`\@=\atcatcode
%    \end{macrocode}
%
% \Finale
%%
%% \CharacterTable
%%  {Upper-case    \A\B\C\D\E\F\G\H\I\J\K\L\M\N\O\P\Q\R\S\T\U\V\W\X\Y\Z
%%   Lower-case    \a\b\c\d\e\f\g\h\i\j\k\l\m\n\o\p\q\r\s\t\u\v\w\x\y\z
%%   Digits        \0\1\2\3\4\5\6\7\8\9
%%   Exclamation   \!     Double quote  \"     Hash (number) \#
%%   Dollar        \$     Percent       \%     Ampersand     \&
%%   Acute accent  \'     Left paren    \(     Right paren   \)
%%   Asterisk      \*     Plus          \+     Comma         \,
%%   Minus         \-     Point         \.     Solidus       \/
%%   Colon         \:     Semicolon     \;     Less than     \<
%%   Equals        \=     Greater than  \>     Question mark \?
%%   Commercial at \@     Left bracket  \[     Backslash     \\
%%   Right bracket \]     Circumflex    \^     Underscore    \_
%%   Grave accent  \`     Left brace    \{     Vertical bar  \|
%%   Right brace   \}     Tilde         \~}
%%
\endinput
}
%    \end{macrocode}
%
%    Apart from all the language options we also have a few options
%    that influence the behaviour of language definition files.
%
%    The following options don't do anything themselves, they are just
%    defined in order to make it possible for language definition
%    files to check if one of them was specified by the user.
% \changes{babel~3.5d}{1995/07/04}{Added options to influence
%    behaviour of active acute and grave accents}
%    \begin{macrocode}
\DeclareOption{activeacute}{}
\DeclareOption{activegrave}{}
%    \end{macrocode}
%
%    The options have to be processed in the order in which the user
%    specified them:
%    \begin{macrocode}
\ProcessOptions*
%</package>
%    \end{macrocode}
%
% \section{The Kernel of Babel}
%
%    The kernel of the \babel\ system is stored in either
%    \file{hyphen.cfg} or \file{switch.def} and \file{babel.def}. The
%    file \file{hyphen.cfg} is a file that can be loaded into the
%    format, which is necessary when you want to be able to switch
%    hyphenation patterns. The file \file{babel.def} contains some
%    \TeX\ code that can be read in at run time. When \file{babel.def}
%    is loaded it checks if \file{hyphen.cfg} is in the format; if
%    not the file \file{switch.def} is loaded.
%
%    Because plain \TeX\ users might want to use some of the features
%    of the \babel{} system too, care has to be taken that plain \TeX\
%    can process the files. For this reason the current format will
%    have to be checked in a number of places. Some of the code below
%    is common to plain \TeX\ and \LaTeX, some of it is for the
%    \LaTeX\ case only.
%
%    When the command |\AtBeginDocument| doesn't exist we assume that
%    we are dealing with a plain-based format. In that case the file
%    \file{plain.def} is needed.
%
%    \begin{macrocode}
\ifx\AtBeginDocument\undefined
  \input plain.def\relax
\fi
%    \end{macrocode}
%
%    Check the presence of the command |\iflanguage|, if it is
%    undefined read the file \file{switch.def}.
% \changes{babel~3.0d}{1991/10/29}{Removed use of \cs{@ifundefined}}
%    \begin{macrocode}
%<*core>
\ifx\undefined\iflanguage
  \input switch.def\relax
\fi
%    \end{macrocode}
%
%    To communicate to the language definition files that the core of
%    the \babel{} system has been loaded, the following control
%    sequence is just |\let| equal to |\relax|.
%    \begin{macrocode}
\let\babel@core@loaded\relax
%</core>
%    \end{macrocode}
%
% \subsection{Multiple languages}
%
%    With \TeX\ version~3.0 it has become possible to load hyphenation
%    patterns for more than one language. This means that some extra
%    administration has to be taken care of.  The user has to know for
%    which languages patterns have been loaded, and what values of
%    |\language| have been used.
%
%    Some discussion has been going on in the \TeX\ world about how to
%    use |\language|. Some have suggested to set a fixed standard,
%    i.\,e., patterns for each language should \emph{always} be loaded
%    in the same location. It has also been suggested to use the
%    \textsc{iso} list for this purpose. Others have pointed out that
%    the \textsc{iso} list contains more than 256~languages, which
%    have \emph{not} been numbered consecutively.
%
%    I think the best way to use |\language|, is to use it
%    dynamically.  This code implements an algorithm to do so. It uses
%    an external file in which the person who maintains a \TeX\
%    environment has to record for which languages he has hyphenation
%    patterns \emph{and} in which files these are stored\footnote{This
%    is because different operating systems sometimes use \emph{very}
%    different filenaming conventions.}. When hyphenation exceptions
%    are stored in a separate file this can be indicated by naming
%    that file \emph{after} the file with the hyphenation patterns.
%
%    This ``configuration file'' can contain empty lines and comments,
%    as well as lines which start with an equals (\texttt{=})
%    sign. Such a line will instruct \LaTeX\ that the hyphenation
%    patterns just processed have to be known under an alternative
%    name. Here is an example:
%  \begin{verbatim}
%    % File    : language.dat
%    % Purpose : tell iniTeX what files with patterns to load.
%    english    english.hyphenations
%    =british
%
%    dutch      hyphen.dutch exceptions.dutch % Nederlands
%    german hyphen.ger
%  \end{verbatim}
%
%    As the file \file{switch.def} needs to be read only once, we
%    check whether it was read before.  If it was, the command
%    |\iflanguage| is already defined, so we can stop processing.
%    \begin{macrocode}
%<*kernel>
%<*!patterns>
\expandafter\ifx\csname iflanguage\endcsname\relax \else
\expandafter\endinput
\fi
%</!patterns>
%    \end{macrocode}
%
%  \begin{macro}{\language}
%    Plain \TeX\ version~3.0 provides the primitive |\language| that
%    is used to store the current language. When used with a pre-3.0
%    version this function has to be implemented by allocating a
%    counter. 
%    \begin{macrocode}
\ifx\language\undefined
  \csname newcount\endcsname\language
\fi
%    \end{macrocode}
%  \end{macro}
%
%  \begin{macro}{\last@language}
%    Another counter is used to store the last language defined.  For
%    pre-3.0 formats an extra counter has to be allocated,
%    \begin{macrocode}
\ifx\newlanguage\undefined
  \csname newcount\endcsname\last@language
%    \end{macrocode}
%    plain \TeX\ version 3.0 uses |\count 19| for this purpose.
%    \begin{macrocode}
\else
  \countdef\last@language=19
\fi
%    \end{macrocode}
%  \end{macro}
%
%  \begin{macro}{\addlanguage}
%
%    To add languages to \TeX's memory plain \TeX\ version~3.0
%    supplies |\newlanguage|, in a pre-3.0 environment a similar macro
%    has to be provided. For both cases a new macro is defined here,
%    because the original |\newlanguage| was defined to be |\outer|.
%
%    For a format based on plain version~2.x, the definition of
%    |\newlanguage| can not be copied because |\count 19| is used for
%    other purposes in these formats. Therefore |\addlanguage| is
%    defined using a definition based on the macros used to define
%    |\newlanguage| in plain \TeX\ version~3.0.
% \changes{hyphen-1.1a}{1991/11/11}{Added a \texttt{\%}, removed
%    \texttt{by}}
%    \begin{macrocode}
\ifx\newlanguage\undefined
  \def\addlanguage#1{%
    \global\advance\last@language \@ne
    \ifnum\last@language<\@cclvi
    \else
        \errmessage{No room for a new \string\language!}%
    \fi
    \global\chardef#1\last@language
    \wlog{\string#1 = \string\language\the\last@language}}
%    \end{macrocode}
%
%    For formats based on plain version~3.0 the definition of
%    |\newlanguage| can be simply copied, removing |\outer|.
%
%    \begin{macrocode}
\else
  \def\addlanguage{\alloc@9\language\chardef\@cclvi}
\fi
%    \end{macrocode}
%  \end{macro}
%
%  \begin{macro}{\adddialect}
%    The macro |\adddialect| can be used to add the name of a dialect
%    or variant language, for which an already defined hyphenation
%    table can be used.
% \changes{hyphen-1.1a}{1991/11/11}{Added \cs{relax}}
%    \begin{macrocode}
\def\adddialect#1#2{%
    \global\chardef#1#2\relax
    \wlog{\string#1 = a dialect from \string\language#2}}
%    \end{macrocode}
%  \end{macro}
%
%  \begin{macro}{\iflanguage}
%    Users might want to test (in a private package for instance)
%    which language is currently active. For this we provide a test
%    macro, |\iflanguage|, that has three arguments.  It checks
%    whether the first argument is a known language. If so, it
%    compares the first argument with the value of |\language|. Then,
%    depending on the result of the comparison, it executes either the
%    second or the third argument.
% \changes{hyphen-1.0b}{1991/05/29}{Added \cs{@bsphack} and
%    \cs{@esphack}}
% \changes{hyphen-1.0d}{1991/07/21}{Added comment character after
%    \texttt{\#2}}
% \changes{hyphen-1.0e}{1991/08/08}{Removed superfluous
%    \cs{expandafter}}
% \changes{hyphen-1.0h}{1991/10/07}{Removed space hacks and use of
%    \cs{@ifundefined}}
% \changes{hyphen-1.1a}{1991/11/11}{Refrased \cs{ifnum} test}
%    \begin{macrocode}
\def\iflanguage#1#2#3{%
  \expandafter\ifx\csname l@#1\endcsname\relax
    \@nolanerr{#1}%
  \else
    \ifnum\csname l@#1\endcsname=\language #2%
    \else#3\fi
  \fi}
%    \end{macrocode}
%  \end{macro}
%
%  \begin{macro}{\selectlanguage}
%    The macro |\selectlanguage| checks whether the language is
%    already defined before it performs its actual task, which is to
%    update |\language| and activate language-specific definitions.
%
%    To allow the call of |\selectlanguage| either with a control
%    sequence name or with a simple string as argument, we have to use
%    a trick to delete the optional escape character.
%
%    To convert a control sequence to a string, we use the |\string|
%    primitive.  Next we have to look at the first character of this
%    string and compare it with the escape character.  Because this
%    escape character can be changed by setting the internal integer
%    |\escapechar| to a character number, we have to compare this
%    number with the character of the string.  To do this we have to
%    use \TeX's backquote notation to specify the character as a
%    number.
%
%    If the first character of the |\string|'ed argument is the
%    current escape character, the comparison has stripped this
%    character and the rest in the `then' part consists of the rest of
%    the control sequence name.  Otherwise we know that either the
%    argument is not a control sequence or |\escapechar| is set to a
%    value outside of the character range~$0$--$255$.
%
%    If the user gives an empty argument, we provide a default
%    argument for |\string|.  This argument should expand to nothing.

% \changes{hyphen-1.0c}{1991/06/06}{Made \cs{selectlanguage}
%    robust}
% \changes{hyphen-1.1a}{1991/11/11}{Modified to allow arguments that
%    start with an escape character}
% \changes{hyphen-1.1b}{1991/11/17}{Simplified the modification to
%    allow the use in a \cs{write} command}
% \changes{babel~3.5b}{1995/05/13}{Store the name of the current
%    language in a control sequence instead of passing the whole macro
%    construct to strip the escape character in the argument of
%    \cs{selectlanguage }.}
%    \begin{macrocode}
\def\selectlanguage#1{%
  \edef\languagename{%
    \ifnum\escapechar=\expandafter`\string#1\empty
     \else \string#1\empty\fi}
  \expandafter\protect\csname selectlanguage \expandafter\endcsname
  \expandafter{\languagename}}
%    \end{macrocode}
%    Because the command |\selectlanguage| could be used in a moving
%    argument it expands to \verb*=\protect\selectlanguage =.
%    Therefore, we have to make sure that a macro |\protect| exists.
%    If it doesn't it is |\let| to |\relax|.
%    \begin{macrocode}
\ifx\undefined\protect\let\protect\relax\fi
%    \end{macrocode}
%
%    {\small Remark: If the |\selectlanguage| command is written to a
%    file, a possible control sequence argument gets totally expanded
%    to the string without the leading escape character.  In the
%    normal case we have to deal with the fact that the argument of
%    |\selectlanguage | is totally unexpanded at the moment.  There
%    is only one disadvantage in the current implementation:
%    |\originalTeX| contains this unexpanded argument and therefore
%    needs more memory for its macro definition.  \par}
%
% \changes{babel~3.5b}{1995/03/04}{Changed the name of the internal
%    macro to \cs{selectlanguage }.}
% \changes{babel~3.5b}{1995/03/05}{Added an extra level of expansion to
%    separate the switching mechanism from writing to aux files}
%    \begin{macrocode}
\expandafter\def\csname selectlanguage \endcsname#1{%
  \select@language{#1}%
%    \end{macrocode}
%    We also write a command to change the current language the
%    auxiliary files.
% \changes{babel~3.5a}{1995/02/17}{write the language change to the
%    auxiliary files}
%    \begin{macrocode}
  \if@filesw
    \protected@write\@auxout{}{\string\select@language{#1}}%
    \addtocontents{toc}{\string\select@language{#1}}%
    \addtocontents{lof}{\string\select@language{#1}}%
    \addtocontents{lot}{\string\select@language{#1}}%
  \fi}
%    \end{macrocode}
%    
%    First, check if the user asks for a known language. If so,
%    update the value of |\language| and call |\originalTeX|
%    to bring \TeX\ in a certain pre-defined state.
% \changes{hyphen-1.0b}{1991/05/29}{Added \cs{@bsphack} and
%    \cs{@esphack}}
% \changes{hyphen-1.0e}{1991/08/08}{Removed superfluous
%    \cs{expandafter}}
% \changes{hyphen-1.0h}{1991/10/07}{Removed space hacks and use of
%    \cs{@ifundefined}}
% \changes{hyphen-1.1b}{1991/11/17}{Added \cs{relax} as first command
%    to stop an expansion if \cs{protect} is empty}
%    \begin{macrocode}
\def\select@language#1{%
  \expandafter\ifx\csname l@#1\endcsname\relax
    \@nolanerr{#1}%
  \else
    \language=\csname l@#1\endcsname\relax
    \originalTeX
%    \end{macrocode}
%    The name of the language is stored in the control sequence
%    |\languagename|. The contents of this control sequence could be
%    tested in the following way:
%  \begin{verbatim}
%    \edef\tmp{\string english}
%    \ifx\languagename\tmp
%        ...
%    \else
%        ...
%    \fi
%  \end{verbatim}
%    The construction with |\string| is necessary because
%    |\languagename| returns the name with characters of category code
%    \texttt{12} (other).  Then we have to \emph{re}define
%    |\originalTeX| to compensate for the things that have been
%    activated.  To save memory space for the macro definition of
%    |\originalTeX|, we construct the control sequence name for the
%    |\noextras|\langvar command at definition time by expanding the
%    |\csname| primitive.
% \changes{hyphen-1.0c}{1991/06/06}{Replaced \cs{gdef} with \cs{def}}
% \changes{hyphen-1.1}{1991/10/31}{\cs{originalTeX} should only be
%    executed once}
% \changes{hyphen-1.1b}{1991/11/17}{Added three \cs{expandafter}s
%    to save macro space for \cs{originalTeX}}
% \changes{hyphen-1.1c}{1991/11/20}{Moved definition of
%    \cs{originalTeX} before \cs{extras\langvar}}
% \changes{hyphen-1.1d}{1991/11/24}{Set \cs{originalTeX} to
%    \cs{empty}, because it should be expandable.}
%    \begin{macrocode}
    \expandafter\def\expandafter\originalTeX
        \expandafter{\csname noextras#1\endcsname
                     \let\originalTeX\empty}%
    \babel@beginsave
%    \end{macrocode}
%    Now activate the language-specific definitions. This is done by
%    constructing the names of three macros by concatenating three
%    words with the argument of |\selectlanguage|, and calling these
%    macros.
% \changes{babel~3.5b}{1995/05/13}{Seperated the setting of the
% hyphenmin values}
%    \begin{macrocode}
    \csname captions#1\endcsname
    \csname date#1\endcsname
    \csname extras#1\endcsname\relax
%    \end{macrocode}
%    The switching of the values of |\lefthyphenmin| and
%    |\righthyphenmin| is somewhart different. First we save their
%    current values, then we check if |\|\langvar|hyphenmins| is
%    defined. If it is not we set default values (2 and 3), otherwise
%    the values in |\|\langvar|hyphenmins| will be used.
% \changes{babel~3.5b}{1995/06/05}{Addedd default setting of hyphenmin
%    parameters}
%    \begin{macrocode}
    \babel@savevariable\lefthyphenmin
    \babel@savevariable\righthyphenmin
    \expandafter\ifx\csname #1hyphenmins\endcsname\relax
      \lefthyphenmin\tw@\righthyphenmin\thr@@
    \else
      \expandafter\expandafter\expandafter\set@hyphenmins
        \csname #1hyphenmins\endcsname
    \fi
  \fi}
%    \end{macrocode}
%  \end{macro}
%
%  \begin{environment}{otherlanguage}
%    The \textsf{otherlanguage} environment can be used as an
%    alternative to using the |\selectlanguage| declarative
%    command. When you are typesetting a document with mixes
%    left-to-right and right-to-left typesetting you have to use this
%    environment in order to let things work as you expect them to.
%
%    The first thing this environment does is store the name of the
%    language in |\languagename|; it then calls
%    \verb*=\selectlanguage = to switch on everything that is needed for
%    this language The |\ignorespaces| command is necessary to hide
%    the environment when it is entered in horizontal mode.
% \changes{babel~3.5d}{1995/06/22}{environment added}
% \changes{babel~3.5e}{1995/07/07}{changed name}
%    \begin{macrocode}
\long\def\otherlanguage#1{%
  \def\languagename{#1}%
  \csname selectlanguage \endcsname{#1}%
  \ignorespaces
  }
%    \end{macrocode}
%    The |\endotherlanguage| part of the environment calls
%    |\originalTeX| to restore (most of) the settings and tries to
%    hide itself when it is called in horizontal mode.
%    \begin{macrocode}
\long\def\endotherlanguage{%
  \originalTeX
  \global\@ignoretrue\ignorespaces
  }
%    \end{macrocode}
%  \end{environment}
%
%  \begin{macro}{\foreignlanguage}
%    The |\foreignlanguage| command is another substitute for the
%    |\selectlanguage| command. This command takes two arguments, the
%    first argument is the name of the language to use for typesetting
%    the text specified in the second argument. 
%
%    Unlike |\selectlanguage| this command doesn't switch
%    \emph{everything}, it only switches the hyphenation rules and the
%    extra definitions for the language specified. It does this within
%    a group and assumes the |\extras|\langvar\ command doesn't make
%    any |\global| changes. The coding is very similar to part of
%    |\selectlanguage|.
% \changes{babel~3.5d}{1995/06/22}{Macro added}
%    \begin{macrocode}
\def\foreignlanguage#1#2{%
  \begingroup
    \def\languagename{#1}%
    \expandafter\ifx\csname l@#1\endcsname\relax
      \@nolanerr{#1}%
    \else
      \language=\csname l@#1\endcsname\relax
      \csname extras#1\endcsname
%    \end{macrocode}
% \changes{babel~3.5e}{1995/07/06}{Use \cs{relax} instead of
%    \cs{undefined}}
%    \begin{macrocode}
      \expandafter\ifx\csname #1hyphenmins\endcsname\relax
        \lefthyphenmin\tw@\righthyphenmin\thr@@
      \else
        \expandafter\expandafter\expandafter\set@hyphenmins
          \csname #1hyphenmins\endcsname
      \fi
    \fi
    #2
    \csname noextras#1\endcsname
    \endgroup
  }
%    \end{macrocode}
%  \end{macro}
%
%  \begin{macro}{\set@hyphenmins}
%    This macro sets the values of |\lefthyphenmin| and
%    |\righthyphenmin|. It expects two values as its argument.
%    \begin{macrocode}
\def\set@hyphenmins#1#2{\lefthyphenmin#1\righthyphenmin#2}
%</kernel>
%    \end{macrocode}
%  \end{macro}
%
%
%  \begin{macro}{\main@language}
% \changes{babel~3.5a}{1995/02/17}{Macro added}
%  \begin{macro}{\bbl@main@language}
% \changes{babel~3.5a}{1995/02/17}{Macro added}
%    This command should be used in the various language definition
%    files. It stores its argument in |\bbl@main@language|; to be used
%    to switch to the correct language ath the beginning of the
%    document. 
%    \begin{macrocode}
%<*core>
\def\main@language#1{\def\bbl@main@language{#1}}
%    \end{macrocode}
%    The default is to use English as the main language.
%    \begin{macrocode}
\main@language{english}
%    \end{macrocode}
%    We also have to make sure that some code gets executed at the
%    beginning of the document.
%    \begin{macrocode}
\AtBeginDocument{%
  \expandafter\selectlanguage\expandafter{\bbl@main@language}}
%</core>
%    \end{macrocode}
%  \end{macro}
%  \end{macro}
%
%    The macro|\originalTeX| should be known to \TeX\ at this moment.
%    As it has to be expandable we |\let| it to |\empty| instead of
%    |\relax|.
% \changes{hyphen-1.1d}{1991/11/24}{Set \cs{originalTeX} to
%    \cs{empty}, because it should be expandable.}
%    \begin{macrocode}
%<*kernel>
\ifx\undefined\originalTeX\let\originalTeX\empty\fi
%    \end{macrocode}
%    Because this part of the code can be included in a format, we
%    make sure that the macro which initialises the save mechanism,
%    |\babel@beginsave|, is not considered to be undefined.
%    \begin{macrocode}
\ifx\undefined\babel@beginsave\let\babel@beginsave\relax\fi
%    \end{macrocode}
%
%  \begin{macro}{\@nolanerr}
% \changes{babel~3.4e}{1994/06/25}{Use \cs{PackageError} in \LaTeXe\
%    mode}
%  \begin{macro}{\@nopatterns}
% \changes{babel~3.4e}{1994/06/25}{Macro added}
%    The \babel\ package will signal an error when a documents tries
%    to select a language that hasn't been defined earlier. When a
%    user selects a language for which no hyphenation patterns were
%    loaded into the format he will be given a warning about that
%    fact. We revert to the patterns for |\language|=0 in that case.
%    In most formats that will be (US)english, but it might also be
%    empty.
%
%    When the format knows about |\PackageError| it must be \LaTeXe,
%    so we can safely use its error handling interface. Otherwise
%    we'll have to `keep it simple'.
% \changes{hyphen-1.0h}{1991/10/07}{Added a percent sign to remove
%    unwanted white space}
% \changes{babel~3.5a}{1995/02/15}{Added \cs{@activated} to log active
%    characters}
% \changes{babel~3.5c}{1995/06/19}{Added missing closing brace}
%    \begin{macrocode}
\ifx\PackageError\undefined
  \def\@nolanerr#1{%
    \errhelp{Your command will be ignored, type <return> to proceed}%
    \errmessage{You haven't defined the language #1\space yet}}
  \def\@nopatterns#1{%
    \message{No hyphenation patterns were loaded for}
    \message{the language `#1'}
    \message{I will use the patterns loaded for \string\language=0
          instead}}
  \def\@activated#1{%
    \wlog{Package babel Info: Making #1 an active character}}
\else
  \newcommand*{\@nolanerr}[1]{%
    \PackageError{babel}%
                 {You haven't defined the language #1\space yet}%
        {Your command will be ignored, type <return> to proceed}}
  \newcommand*{\@nopatterns}[1]{%
    \PackageWarningNoLine{babel}%
        {No hyphenation patterns were loaded for\MessageBreak
          the language `#1'\MessageBreak
          I will use the patterns loaded for \string\language=0
          instead}}
  \newcommand*{\@activated}[1]{%
    \PackageInfo{babel}{%
      Making #1 an active character}}
\fi
%    \end{macrocode}
%  \end{macro}
%  \end{macro}
%
%    The following code is meant to be read by ini\TeX\ because it
%    should instruct \TeX\ to read hyphenation patterns. To this end
%    the \texttt{docstrip} option \texttt{patterns} can be used to
%    include this code in the file \file{hyphen.cfg}.
%    \begin{macrocode}
%<*patterns>
%    \end{macrocode}
%
%  \begin{macro}{\patterns@loaded}
% \changes{hyphen-1.0i}{1991/10/27}{Added a token register for
%    collecting the names of patterns that are loaded by ini\TeX.}
%    It has been suggested to add a remark to \LaTeX's |\everyjob|
%    message, stating which hyphenation patterns have been loaded.
%    This can be done by first collecting (in a token register) the
%    names when processing the file \file{language.dat} and
%    afterwards adding a string to the |\everyjob| message. The token
%    register is initially empty.
%
%    \begin{macrocode}
\newtoks\patterns@loaded \global\patterns@loaded={}
%    \end{macrocode}
%  \end{macro}
%
%  \begin{macro}{\process@line}
% \changes{babel~3.5b}{1995/04/28}{added macro}
%    Each line in the file \file{language.dat} is processed by
%    |\process@line| after it is read. The first thing this macro does
%    is to check wether the line starts with \texttt{=}.
%  \begin{macro}{\bbl@eq@}
% \changes{babel~3.5b}{1995/04/28}{added macro}
%    To be able to do that we need an \texttt{=}, stored in a macro.
%    \begin{macrocode}
\def\bbl@eq@{=}
%    \end{macrocode}
%  \end{macro}
%    When the first token of a line is an \texttt{=}, the macro
%    |\process@synonym| is called; otherwise the macro
%    |\process@language| will continue.
%    \begin{macrocode}
\def\process@line#1#2/{%
  \def\bbl@tmp{#1}
  \ifx\bbl@tmp\bbl@eq@
    \process@synonym#2/
  \else
    \process@language#1#2/%
  \fi
  }
%    \end{macrocode}
%  \end{macro}
%
%  \begin{macro}{\process@synonym}
% \changes{babel~3.5b}{1995/04/28}{added macro}
%    This macro takes care of the lines which start with an
%    \texttt{=}.
%    \begin{macrocode}
\def\process@synonym#1 /{%
  \ifnum\last@language=\m@ne
%    \end{macrocode}
%    When no languages have been loaded yet the name following the
%    \texttt{=} will be a synonym for hyphenation register 0.
%    \begin{macrocode}
    \expandafter\global
    \expandafter\chardef\csname l@#1\endcsname0\relax
    \wlog{\string\l@#1=\string\language0}
  \else
%    \end{macrocode}
%    Otherwise the name will be a synonym for the language loaded last.
%    \begin{macrocode}
    \expandafter\global
    \expandafter\chardef\csname l@#1\endcsname\last@language
    \wlog{\string\l@#1=\string\language\the\last@language}
  \fi
  }
%    \end{macrocode}
%  \end{macro}
%
%
%  \begin{macro}{\process@language}
%    The macro |\process@language| is used to process a non-empty line
%    from the `configuration file'. It has three arguments, each
%    delimited by white space. The third argument is optional,
%    therfore a |/| character is expected to delimit the last
%    argument.  The first argument is the `name' of a language, the
%    second is the name of the file that contains the patterns. The
%    optional third argument is the name of a file containing
%    hyphenation exceptions.
%
%    The first thing to do is call |\addlanguage| to allocate a
%    pattern register and to make that register `active'.
% \changes{hyphen-1.0e}{1991/08/08}{Removed superfluous
%    \cs{expandafter}}
% \changes{hyphen-1.0f}{1991/08/21}{Reinserted \cs{expandafter}}
% \changes{hyphen-1.0i}{1991/10/27}{Added the collection of pattern
%    names.}
%    \begin{macrocode}
\def\process@language#1 #2 #3/{%
    \expandafter\addlanguage\csname l@#1\endcsname
    \expandafter\language\csname l@#1\endcsname
%    \end{macrocode}
%    Then the `name' of the language that will be loaded now is
%    added to the token register |\patterns@loaded|. and finally
%    the pattern file is read.
%    \begin{macrocode}
    \global\patterns@loaded\expandafter{\the\patterns@loaded#1, }%
%    \end{macrocode}
%
% \changes{babel~3.4e}{1994/06/24}{Added code to detect assignments to
%    left- and righthyphenmin in the patternfile.}
%
%    Some pattern files contain assignments to |\lefthyphenmin| and
%    |\righthyphenmin|. \TeX\ does not keep track of these
%    assignments. Therefore we try to detect such assignments and
%    store them in the |\|\langvar|hyphenmins| macro. When no
%    assignments were made we provide a default setting.
%    \begin{macrocode}
    \lefthyphenmin\m@ne
    \input #2\relax
    \ifnum\lefthyphenmin=\m@ne
      \lefthyphenmin\tw@
      \righthyphenmin\thr@@
    \fi
%    \end{macrocode}
%    When the hyphenation patterns have been processed we need to see
%    if a file with hyphenation exceptions needs to be read. This is
%    the case when the third argument is not empty and when it does
%    not contain a space token.
% \changes{babel~3.5b}{1995/04/28}{Added optional reading of file with
%    hyphenation exceptions}
%    \begin{macrocode}
    \def\bbl@tmp{#3}
    \ifx\bbl@tmp\@empty
    \else
      \ifx\bbl@tmp\space
      \else
        \input #3\relax
      \fi
    \fi
%    \end{macrocode}
%    Finally we store the settings of |\lefthyphenmin| and
%    |\righthyphenmin|.
%    \begin{macrocode}
    \expandafter\edef\csname #1hyphenmins\endcsname{%
      \the\lefthyphenmin\the\righthyphenmin}}
%    \end{macrocode}
%  \end{macro}
%
%    The configuration file can now be opened for reading.
%    \begin{macrocode}
\openin1 = language.dat
%    \end{macrocode}
%
%    See if the file exists, if not, use the default hyphenation file
%    \file{hyphen.tex}. The user will be informed about this.
%
%    \begin{macrocode}
\ifeof1
  \message{I couldn't find the file language.dat,\space
           I will try the file hyphen.tex}
  \input hyphen.tex\relax
\else
%    \end{macrocode}
%
%    Pattern registers are allocated using count register
%    |\last@language|. Its initial value is~0. The definition of the
%    macro |\newlanguage| is such that it first increments the count
%    register and then defines the language. In order to have the
%    first patterns loaded in pattern register number~0 we initialize
%    |\last@language| with the value~$-1$.
%
% \changes{hyphen-1.1}{1991/05/21}{Removed use of \cs{toks0}}
%    \begin{macrocode}
  \last@language\m@ne
%    \end{macrocode}
%
%    We now read lines from the file until the end is found
%
%    \begin{macrocode}
  \loop
%    \end{macrocode}
%
%    While reading from the input it is useful to switch off
%    recognition of the end-of-line character. This saves us stripping
%    off spaces from the contents of the controlsequence.
%
%    \begin{macrocode}
    \endlinechar\m@ne
    \read1 to \bbl@line
    \endlinechar`\^^M
%    \end{macrocode}
%
%    Empty lines are skipped.
%    \begin{macrocode}
    \ifx\bbl@line\empty
    \else
%    \end{macrocode}
%
%    Now we add a space and a |/| character to the end of
%    |\bbl@line|. This is needed to be able to recognize the third,
%    optional, argument of |\process@language| later on.
% \changes{babel~3.5b}{1995/04/28}{Now add a \cs{space} and a /
%    character}
%    \begin{macrocode}
      \edef\bbl@line{\bbl@line\space/}
      \expandafter\process@line\bbl@line
    \fi
%    \end{macrocode}
%
%    Check for the end of the file.  To avoid a new \texttt{if}
%    control sequence we create the necessary |\iftrue| or |\iffalse|
%    with the help of |\csname|.  But there is one complication with
%    this approach: when skipping the \texttt{loop...repeat} \TeX\ has
%    to read |\if|/|\fi| pairs.  So we have to insert a `dummy'
%    |\iftrue|.
% \changes{hyphen-1.1}{1991/10/31}{Removed the extra \texttt{if}
%    control sequence}
%    \begin{macrocode}
    \iftrue \csname fi\endcsname
    \csname if\ifeof1 false\else true\fi\endcsname
  \repeat
%    \end{macrocode}
%
%    Reactivate the default patterns,
%    \begin{macrocode}
  \language=0
\fi
%    \end{macrocode}
%    and close the configuration file.
% \changes{hyphen-1.1c}{1991/11/20}{Free macro space for
%    \cs{process@language}}
%    \begin{macrocode}
\closein1
%    \end{macrocode}
%    Also remove some macros from memory
%    \begin{macrocode}
\let\process@language\undefined
\let\process@synonym\undefined
\let\process@line\undefined
\let\bbl@tmp\undefined
\let\bbl@eq@\undefined
\let\bbl@line\undefined
%    \end{macrocode}
%
% \changes{hyphen-1.1}{1991/10/31}{Added redefinition of \cs{dump} to
%    add a message to \cs{everyjob}}
%    We want to add a message to the message \LaTeX\ puts in the
%    |\everyjob| register. This could be done by the following code:
%    \begin{verbatim}
%    \let\orgeveryjob\everyjob
%    \def\everyjob#1{%
%      \orgeveryjob{#1}%
%      \orgeveryjob\expandafter{\the\orgeveryjob\immediate\write16{%
%          hyphenation patterns for \the\loaded@patterns loaded.}}%
%      \let\everyjob\orgeveryjob\let\orgeveryjob\undefined}
%    \end{verbatim}
%    The code above redefines the control sequence \cs{everyjob}
%    in order to be able to add something to the current contents of
%    the register. This is necessary because the processing of
%    hyphenation patterns happens long before \LaTeX\ fills the
%    register.\\
%    There are some problems with this approach though.
%  \begin{itemize}
%    \item When someone wants to use several hyphenation patterns with
%    \SliTeX\ the above scheme won't work. The reason is that \SliTeX\
%    overwrites the contents of the |\everyjob| register with its own
%    message.
%    \item Plain \TeX\ does not use the |\everyjob| register so the
%    message would not be displayed.
%  \end{itemize}
%    To circumvent this a `dirty trick' can be used. As this code is
%    only processed when creating a new format file there is one
%    command that is sure to be used, |\dump|. Therefore the orginal
%    |\dump| is saved in |\org@dump| and a new definition is supplied.
%    \begin{macrocode}
\let\orig@dump=\dump
\def\dump{%
%    \end{macrocode}
%    This new definition starts by adding an instruction to write a
%    message on the terminal and in the transcript file to inform the
%    user of the preloaded hyphenation patterns.
%    \begin{macrocode}
  \everyjob\expandafter{\the\everyjob%
    \immediate\write16{Hyphenation patterns for \the\patterns@loaded
      loaded.}}%
%    \end{macrocode}
%    Then everything is restored to the old situation and the format
%    is dumped.
%    \begin{macrocode}
  \let\dump\orig@dump\let\orig@dump\undefined\dump}
%    \end{macrocode}
%
%    Here the code for ini\TeX\ ends.
%    \begin{macrocode}
%</patterns>
%</kernel>
%    \end{macrocode}
%
% \subsection{Support for active characters}
%
%  \begin{macro}{\bbl@add@special}
% \changes{babel~3.2}{1991/11/10}{Added macro}
%    The macro |\bbl@add@special| is used to add a new character (or
%    single character control sequence) to the macro |\dospecials|
%    (and |\@sanitize| if \LaTeX\ is used).
%
%    To keep all changes local, we begin a new group.  Then we
%    redefine the macros |\do| and |\@makeother| to add themselves and
%    the given character without expansion.
%    \begin{macrocode}
%<*core|shorthands>
\def\bbl@add@special#1{\begingroup
    \def\do{\noexpand\do\noexpand}%
    \def\@makeother{\noexpand\@makeother\noexpand}%
%    \end{macrocode}
%    To add the character to the macros, we expand the original macros
%    with the additional character inside the redefinition of the
%    macros.  Because |\@sanitize| can be undefined, we put the
%    definition inside a conditional.
%    \begin{macrocode}
    \edef\x{\endgroup
      \def\noexpand\dospecials{\dospecials\do#1}%
      \expandafter\ifx\csname @sanitize\endcsname\relax \else
        \def\noexpand\@sanitize{\@sanitize\@makeother#1}%
      \fi}%
%    \end{macrocode}
%    The macro |\x| contains at this moment the following:\\
%    |\endgroup\def\dospecials{|\textit{old contents}%
%    |\do|\meta{char}|}|.\\
%    If |\@sanitize| is defined, it contains an additional definition
%    of this macro.  The last thing we have to do, is the expansion of
%    |\x|.  Then |\endgroup| is executed, which restores the old
%    meaning of |\x|, |\do| and |\@makeother|.  After the group is
%    closed, the new definition of |\dospecials| (and |\@sanitize|) is
%    assigned.
%    \begin{macrocode}
  \x}
%    \end{macrocode}
%  \end{macro}
%
%  \begin{macro}{\bbl@remove@special}
% \changes{babel~3.2}{1991/11/10}{Added macro}
%    The companion of the former macro is |\bbl@remove@special|.  It
%    is used to remove a character from the set macros |\dospecials|
%    and |\@sanitize|.
%
%    To keep all changes local, we begin a new group.  Then we define
%    a help macro |\x|, which expands to empty if the characters
%    match, otherwise it expands to its nonexpandable input.  Because
%    \TeX\ inserts a |\relax|, if the corresponding |\else| or |\fi|
%    is scanned before the comparison is evaluated, we provide a `stop
%    sign' which should expand to nothing.
%    \begin{macrocode}
\def\bbl@remove@special#1{\begingroup
    \def\x##1##2{\ifnum`#1=`##2\noexpand\empty
                 \else\noexpand##1\noexpand##2\fi}%
%    \end{macrocode}
%    With the help of this macro we define |\do| and |\make@other|.
%    \begin{macrocode}
    \def\do{\x\do}%
    \def\@makeother{\x\@makeother}%
%    \end{macrocode}
%    The rest of the work is similar to |\bbl@add@special|.
%    \begin{macrocode}
    \edef\x{\endgroup
      \def\noexpand\dospecials{\dospecials}%
      \expandafter\ifx\csname @sanitize\endcsname\relax \else
        \def\noexpand\@sanitize{\@sanitize}%
      \fi}%
  \x}
%    \end{macrocode}
%  \end{macro}
%
%  \subsection{Support for active characters}
%
%  \begin{macro}{\initiate@active@char}
% \changes{babel~3.5a}{1995/02/11}{Added macro}
% \changes{babel~3.5b}{1995/03/03}{Renamed macro}
%    A language definition file can call this macro to make a
%    character active. This macro takes one argument, the character
%    that is to be made active. When the character was already active
%    this macro does nothing. Otherwise, this macro defines the
%    control sequence |\normal@char|\m{char} to expand to the
%    character in its `normal state' and it defines the active
%    character to expand to |\normal@char|\m{char} by default
%    (\m{char} being the character to be made active). Later its
%    definition can be be changed to expand to |\active@char|\m{char}
%    by calling |\bbl@activate{|\m{char}|}|.
%
%    For example, to make the double quote character active one could
%    have the following line in a language definition file:
%  \begin{verbatim}
%    \initiate@active@char{"}
%  \end{verbatim}
%
%  \begin{macro}{\bbl@afterelse}
%  \begin{macro}{\bbl@afterfi}
%    Because the code that is used in the handling of active
%    characters may need to look ahead, we take extra care to `throw'
%    it over the |\else| and |\fi| parts of an
%    |\if|-statement\footnote{This code is based on code presented in
%    TUGboat vol. 12, no2, June 1991 in ``An expansion Power Lemma''
%    by Sonja Maus.}.
%    \begin{macrocode}
\def\bbl@afterelse#1\else#2\fi{\fi#1}
\def\bbl@afterfi#1\fi{\fi#1}
%    \end{macrocode}
%  \end{macro}
%  \end{macro}
%
%    Note that the definition of |\initiate@active@char| needs an
%    active character, for this the |~| is used. Some of the changes
%    we need do not have to become available later on, so we do it
%    inside a group. 
%    \begin{macrocode}
\begingroup
  \catcode`\~\active
  \def\x{\endgroup
    \def\initiate@active@char##1{%
%    \end{macrocode}
%    If the character is already active we don't do anything.
%    \begin{macrocode}
      \ifcat\noexpand##1\noexpand~\relax
      \else
%    \end{macrocode}
%    Otherwise we write a message in the transcript file,
%    \begin{macrocode}
        \@activated{##1}%
%    \end{macrocode}
%    and define |\normal@char|\m{char} to expand to the character in
%    its default state.
%    \begin{macrocode}
        \@namedef{normal@char\string##1}{##1}%
%    \end{macrocode}
%    If we are making the right quote active we need to change
%    |\pr@m@s| as well.
% \changes{babel-3.5a}{1995/03/10}{Added a check for right quote and
%    adapt \cs{pr@m@s} if necessary}
%    \begin{macrocode}
        \ifx##1'%
          \let\pr@m@s\bbl@pr@m@s
        \fi
%    \end{macrocode}
%    Now we set the lowercase code of the |~| equal to that of the
%    character to be made active and execute the rest of the code
%    inside a |\lowercase| `environment'.
%    \begin{macrocode}
        \lccode`~=`##1%
        \lowercase{%
%    \end{macrocode}
%    Make the character active and add it to |\dospecials| and
%    |\@sanitize|. 
%    \begin{macrocode}
          \catcode`~\active
          \expandafter\bbl@add@special
            \csname \string##1\endcsname
%    \end{macrocode}
%    Define the character to expand to 
%    \begin{center}
%    |\active@prefix| \m{char} |\active@char|\m{char}
%    \end{center}
%    (where |\active@char|\m{char} is \emph{one} control sequence!).
%    \begin{macrocode}
          \expandafter\gdef
            \expandafter~%
            \expandafter{%
            \expandafter\active@prefix\expandafter##1%
            \csname normal@char\string##1\endcsname}}%
%    \end{macrocode}
%    We define the first level expansion of |\active@char|\m{char} to
%    check the status of the |@safe@actives| flag. If it is set to
%    true we expand to the `normal' version of this character,
%    otherwise we call |\@active@char|\m{char}. 
%    \begin{macrocode}
        \@namedef{active@char\string##1}{%
          \if@safe@actives
            \bbl@afterelse\csname normal@char\string##1\endcsname
          \else
            \bbl@afterfi\csname user@active\string##1\endcsname
          \fi}%
%    \end{macrocode}
%    The next level of the code checks whether a user has defined a
%    shorthand for himself with this character. First we check for a
%    single character shorthand. If that doesn't exist we check for a
%    shorthand with an argument.
% \changes{babel~3.5d}{1995/07/02}{Skip the user-level active char
%    with argument if no shorthands with arguments were defined}
%    \begin{macrocode}
        \@namedef{user@active\string##1}{%
          \expandafter\ifx
          \csname \user@group @sh@\string##1@\endcsname
          \relax
            \bbl@afterelse\csname @sh@\string##1@sel\endcsname
              {user@active@arg\string##1}{language@active\string##1}%
          \else
            \bbl@afterfi\csname \user@group @sh@\string##1@\endcsname
          \fi}%
%    \end{macrocode}
%    When there is also no user-level shorthand with an argument we
%    will check whether there is a language defined shorthand for
%    this active character.
%    \begin{macrocode}
        \long\@namedef{user@active@arg\string##1}####1{%
          \expandafter\ifx
          \csname \user@group @sh@\string##1\string####1@\endcsname
          \relax
            \bbl@afterelse
            \csname language@active\string##1\endcsname####1%
          \else
            \bbl@afterfi
            \csname \user@group @sh@\string##1\string####1@%
            \endcsname
          \fi}%
%    \end{macrocode}
%    Like the shorthands that can be defined by the user, a language
%    definition file can also define shorthands with and without an
%    argument, so we need two more macros to check if they exist.
% \changes{babel~3.5d}{1995/07/02}{Skip the language-level active char
%    with argument if no shorthands with arguments were defined}
%    \begin{macrocode}
        \@namedef{language@active\string##1}{%
          \expandafter\ifx
          \csname \language@group @sh@\string##1@\endcsname
          \relax
            \bbl@afterelse\csname @sh@\string##1@sel\endcsname
              {language@active@arg\string##1}{system@active\string##1}%
          \else
            \bbl@afterfi
            \csname \language@group @sh@\string##1@\endcsname
          \fi}%
        \long\@namedef{language@active@arg\string##1}####1{%
          \expandafter\ifx
          \csname \language@group @sh@\string##1\string####1@\endcsname
          \relax
            \bbl@afterelse
            \csname system@active\string##1\endcsname####1%
          \else
            \bbl@afterfi
            \csname \language@group @sh@\string##1\string####1@%
            \endcsname
          \fi}%
%    \end{macrocode}
%    And the same goes for the system level.
%    \begin{macrocode}
        \@namedef{system@active\string##1}{%
          \expandafter\ifx
          \csname \system@group @sh@\string##1@\endcsname
          \relax
            \bbl@afterelse\csname @sh@\string##1@sel\endcsname
              {system@active@arg\string##1}{normal@char\string##1}%
          \else
            \bbl@afterfi\csname \system@group @sh@\string##1@\endcsname
          \fi}%
%    \end{macrocode}
%    When no shorthands were found the `normal' version of the active
%    character is inserted.
%    \begin{macrocode}
        \long\@namedef{system@active@arg\string##1}####1{%
          \expandafter\ifx
          \csname \system@group @sh@\string##1\string####1@\endcsname
          \relax
            \bbl@afterelse\csname normal@char\string##1\endcsname####1%
          \else
            \bbl@afterfi
            \csname \system@group @sh@\string##1\string####1@\endcsname
          \fi}%
        \fi}%
    }\x
%    \end{macrocode}
%  \end{macro}
%
%  \begin{macro}{\active@prefix}
%    The comand |\active@prefix| which is used in the expansion of
%    active characters has a function similar to |\OT1-cmd| in that it
%    |\protect|s the active character whenever |\protect| is
%    \emph{not} |\@typeset@protect|.
% \changes{babel~3.5d}{1995/07/02}{\cs{@protected@cmd} has vanished
%    from \file{ltoutenc.dtx}}
%    \begin{macrocode}
\def\active@prefix#1{%
  \ifx\protect\@typeset@protect
  \else
    \bbl@afterfi\protect#1\@gobble
  \fi}
%    \end{macrocode}
%  \end{macro}
%
%  \begin{macro}{\if@safe@actives}
%    In some circumstances it is necessary to be able to change the
%    expansion of an active character on the fly. For this purpose the
%    switch |@safe@actives| is available. This setting of this switch
%    should be checked in the first level expansion of
%    |\active@char|\m{char}. 
%    \begin{macrocode}
\newif\if@safe@actives
\@safe@activesfalse
%    \end{macrocode}
%  \end{macro}
%
%  \begin{macro}{\bbl@activate}
% \changes{babel~3.5a}{1995/02/11}{Added macro}
%
%    This macro takes one argument, like |\initiate@active@char|. The
%    macro is used to change the definition of an active character to
%    expand to |\active@char|\m{char} instead of
%    |\normal@char|\m{char}.
%    \begin{macrocode}
\def\bbl@activate#1{%
  \expandafter\def
  \expandafter#1\expandafter{%
    \expandafter\active@prefix
    \expandafter#1\csname active@char\string#1\endcsname}%
}
%    \end{macrocode}
%  \end{macro}
%
%  \begin{macro}{\bbl@deactivate}
% \changes{babel~3.5a}{1995/02/11}{Added macro}
%    This macro takes one argument, like |\Activate|. The macro
%    doesn't really make a character non-active; it changes its
%    definition to expand to |\normal@char|\m{char}.
%    \begin{macrocode}
\def\bbl@deactivate#1{%
  \expandafter\def
  \expandafter#1\expandafter{%
    \expandafter\active@prefix
    \expandafter#1\csname normal@char\string#1\endcsname}%
}
%    \end{macrocode}
%  \end{macro}
%
%  \begin{macro}{\bbl@firstcs}
%  \begin{macro}{\bbl@scndcs}
%    These macros have two arguments. They use one of their arguments
%    to build a control sequence from.
%    \begin{macrocode}
\def\bbl@firstcs#1#2{\csname#1\endcsname}
\def\bbl@scndcs#1#2{\csname#2\endcsname}
%    \end{macrocode}
%  \end{macro}
%  \end{macro}
%
%  \begin{macro}{\declare@shorthand}
%    The command |\declare@shorthand| is used to declare a shorthand
%    on a certain level. It takes three arguments:
%    \begin{enumerate}
%    \item a name for the collection of shorthands, i.e. `system', or
%    `dutch';
%    \item the character (sequence) that makes up the shorthand,
%    i.e. |~| or |"a|;
%    \item the code to be executed when the shorthand is encountered.
%    \end{enumerate}
% \changes{babel~3.5d}{1995/07/02}{Make a `note' when a shorthand with
%    an argument is defined.}
%    \begin{macrocode}
\def\declare@shorthand#1#2{\@decl@short{#1}#2\@nil}
\def\@decl@short#1#2#3\@nil#4{%
  \def\tmp{#3}%
  \ifx\tmp\empty
    \expandafter\let\csname @sh@\string#2@sel\endcsname\bbl@scndcs
  \else
    \expandafter\let\csname @sh@\string#2@sel\endcsname\bbl@firstcs
  \fi
  \@namedef{#1@sh@\string#2\string#3@}{#4}}
%    \end{macrocode}
%  \end{macro}
%
%  \begin{macro}{\textormath}
%    Some of the shorthands that will be declared by the language
%    definition files have to be useable in both text and mathmode. To
%    achieve this the helper macro |\textormath| is provided.
%    \begin{macrocode}
\def\textormath#1#2{%
  \ifmmode
    \bbl@afterelse#2%
  \else
    \bbl@afterfi#1%
  \fi}
%    \end{macrocode}
%  \end{macro}
%
%  \begin{macro}{\user@group}
%  \begin{macro}{\language@group}
%  \begin{macro}{\system@group}
%    The current concept of `shorthands' supports three levels or
%    groups of shorthands. For each level the name of the level or
%    group is stored in a macro. The default is to have no user group;
%    use language group `english' and have a system group called
%    `system'.
%    \begin{macrocode}
\def\user@group{}
\def\language@group{english}
\def\system@group{system}
%    \end{macrocode}
%  \end{macro}
%  \end{macro}
%  \end{macro}
%
%  \begin{macro}{\useshorthands}
%    This is the user level command to tell \LaTeX\ that user level
%    shorthands will be used in the document. It takes one argument,
%    the character that starts a shorthand.
%    \begin{macrocode}
\def\useshorthands#1{%
  \def\user@group{user}%
  \initiate@active@char{#1}%
  \bbl@activate{#1}}
%    \end{macrocode}
%  \end{macro}
%
%  \begin{macro}{\defineshorthand}
%    Currently we only support one group of user level shorthands,
%    called `user'.
%    \begin{macrocode}
\def\defineshorthand{\declare@shorthand{user}}
%    \end{macrocode}
%  \end{macro}
%
%  \begin{macro}{\languageshorthands}
%    A user level command to change the language from which shorthands
%    are used.
%    \begin{macrocode}
\def\languageshorthands#1{\def\language@group{#1}}
%    \end{macrocode}
%  \end{macro}
%
%    To prevent problems with constructs such as |\char"01A| when the
%    double quote is made active, we define a shorthand on
%    system level. 
% \changes{babel-3.5a}{1995/03/10}{Replaced 16 system shorthands to
%    deal with hex numbers by one}
%    \begin{macrocode}
\declare@shorthand{system}{"}{\csname normal@char\string"\endcsname}
%    \end{macrocode}
%
%    When the right quote is made active we need to take care of
%    handling it correctly in mathmode. Therefore we define a
%    shorthand at system level to make it expand to a non-active right
%    quote in textmode, but expand to its original definition in
%    mathmode. (Note that the right quote is `active' in mathmode
%    because of its mathcode.)
% \changes{babel-3.5a}{1995/03/10}{Added a system shorthand for the
%    right quote}
%    \begin{macrocode}
\declare@shorthand{system}{'}{%
  \textormath{\csname normal@char\string'\endcsname}%
             {\sp\bgroup\prim@s}}
%    \end{macrocode}
%
%  \begin{macro}{\bbl@pr@m@s}
% \changes{babel-3.5a}{1995/03/10}{Added macro}
%    One of the internal macros that are involved in substituting
%    |\prime| for each right quote in mathmode is |\pr@m@s|. This
%    checks if the next character is a right quote. When the right
%    quote is active, the definition of this macro needs to be adapted
%    to look for an active right quote.
%    \begin{macrocode}
\begingroup
  \catcode`\'\active\let'\relax
  \def\x{\endgroup
    \def\bbl@pr@m@s{%
      \ifx'\@let@token
        \expandafter\pr@@@s
      \else
        \ifx^\@let@token
          \expandafter\expandafter\expandafter\pr@@@t
        \else
          \egroup
        \fi
      \fi}%
    }
\x
%</core|shorthands>
%    \end{macrocode}
%  \end{macro}
%
%    Normally the |~| is active and expands to \verb*=\penalty\@M\ =.
%    When it is written to the \file{.aux} file is written
%    expanded. To prevent that and to be able to use the character |~|
%    as a start character for a shorthand, it is redefined here as a
%    one character shorthand on system level. To acheive that the
%    character |~| first needs to have character code 12 (other).
%    \begin{macrocode}
%<*core>
\catcode`~12\relax
\initiate@active@char{~}
\declare@shorthand{system}{~}{\penalty\@M\ }
\bbl@activate{~}
%    \end{macrocode}
%
%  \begin{macro}{\OT1dqpos}
%  \begin{macro}{\T1dqpos}
%    The position of the double quote character is different for the
%    OT1 and T1 encodings. It will later be selected using the
%    |\f@encoding| macro. Therefore we define two macros here to store
%    the position of the character in these encodings.
%    \begin{macrocode}
\expandafter\def\csname OT1dqpos\endcsname{127}
\expandafter\def\csname T1dqpos\endcsname{4}
%    \end{macrocode}
%    When the macor |\f@encoding| is undefined (as it is in plain
%    \TeX) we define it here to expand to \texttt{OT1}
%    \begin{macrocode}
\ifx\f@encoding\undefined
  \def\f@encoding{OT1}
\fi
%    \end{macrocode}
%  \end{macro}
%  \end{macro}
%
%  \subsection{Support for saving macro definitions}
%
%    To save the meaning of control sequences using |\babel@save|, we
%    use temporary control sequences.  To save hash table entries for
%    these control sequences, we don't use the name of the control
%    sequence to be saved to construct the temporary name.  Instead we
%    simply use the value of a counter, which is reset to zero each
%    time we begin to save new values.  This works well because we
%    release the saved meanings before we begin to save a new set of
%    control sequence meanings (see |\selectlanguage| and
%    |\originalTeX|).
%
%  \begin{macro}{\babel@savecnt}
% \changes{babel~3.2}{1991/11/10}{Added macro}
%  \begin{macro}{\babel@beginsave}
% \changes{babel~3.2}{1991/11/10}{Added macro}
%    The initialization of a new save cycle: reset the counter to
%    zero.
%    \begin{macrocode}
\def\babel@beginsave{\babel@savecnt\z@}
%    \end{macrocode}
%    Before it's forgotten, allocate the counter and initialize all.
%    \begin{macrocode}
\newcount\babel@savecnt
\babel@beginsave
%    \end{macrocode}
%  \end{macro}
%  \end{macro}
%
%  \begin{macro}{\babel@save}
% \changes{babel~3.2}{1991/11/10}{Added macro}
%    The macro |\babel@save|\meta{csname} saves the current meaning of
%    the control sequence \meta{csname} to
%    |\originalTeX|\footnote{\cs{originalTeX} has to be
%    expandable, i.\,e.\ you shouldn't let it to \cs{relax}.}.
%    To do this, we let the current meaning to a temporary control
%    sequence, the restore commands are appended to |\originalTeX| and
%    the counter is incremented.
% \changes{babel~3.2c}{1992/03/17}{missing backslash led to errors
%    when executing \cs{originalTeX}}
% \changes{babel~3.2d}{1992/07/02}{saving in \cs{babel@i} and
%    restoring from \cs{@babel@i} doesn't work very well...}
%    \begin{macrocode}
\def\babel@save#1{%
  \expandafter\let\csname babel@\number\babel@savecnt\endcsname #1\relax
  \begingroup
    \toks@\expandafter{\originalTeX \let#1=}%
    \edef\x{\endgroup
      \def\noexpand\originalTeX{\the\toks@ \expandafter\noexpand
         \csname babel@\number\babel@savecnt\endcsname\relax}}%
  \x
  \advance\babel@savecnt\@ne}
%    \end{macrocode}
%  \end{macro}
%
%  \begin{macro}{\babel@savevariable}
% \changes{babel~3.2}{1991/11/10}{Added macro}
%    The macro |\babel@savevariable|\meta{variable} saves the value of
%    the variable.  \meta{variable} can be anything allowed after the
%    |\the| primitive.
%    \begin{macrocode}
\def\babel@savevariable#1{\begingroup
    \toks@\expandafter{\originalTeX #1=}%
    \edef\x{\endgroup
      \def\noexpand\originalTeX{\the\toks@ \the#1\relax}}%
  \x}
%    \end{macrocode}
%  \end{macro}
%
%  \begin{macro}{\bbl@frenchspacing}
%  \begin{macro}{\bbl@nonfrenchspacing}
%    Some languages need to have |\frenchspacing| in effect. Others
%    don't want that. The command |\bbl@frenchspacing| switches it on
%    when it isn't already in effect and |\bbl@nonfrenchspacing|
%    switches it off if necessary.
%    \begin{macrocode}
\def\bbl@frenchspacing{%
  \ifnum\the\sfcode`\.=\@m
    \let\bbl@nonfrenchspacing\relax
  \else
    \frenchspacing
    \let\bbl@nonfrenchspacing\nonfrenchspacing
  \fi}
\let\bbl@nonfrenchspacing\nonfrenchspacing
%    \end{macrocode}
%  \end{macro}
%  \end{macro}
%
% \subsection{Support for extending macros}
%
%  \begin{macro}{\addto}
%    For each language four control sequences have to be defined that
%    control the language-specific definitions. To be able to add
%    something to these macro once they have been defined the macro
%    |\addto| is introduced. It takes two arguments, a \meta{control
%    sequence} and \TeX-code to be added to the \meta{control
%    sequence}.
%
%    If the \meta{control sequence} has not been defined before it is
%    defined now.
% \changes{babel~3.1}{1991/11/05}{Added macro}
% \changes{babel~3.4}{1994/02/04}{Changed to use toks register}
%    \begin{macrocode}
\def\addto#1#2{%
  \ifx#1\undefined
    \def#1{#2}
  \else
%    \end{macrocode}
%    Otherwise the replacement text for the \meta{control sequence} is
%    expanded and stored in a token register, together with the
%    \TeX-code to be added.  Finally the \meta{control sequence} is
%    \emph{re}defined, using the contents of the token register.
%    \begin{macrocode}
  {\toks@\expandafter{#1#2}%
   \xdef#1{\the\toks@}}%
  \fi
}
%    \end{macrocode}
%  \end{macro}
%
% \subsection{Macros common to a number of languages}
%
%  \begin{macro}{\allowhyphens}
% \changes{babel~3.2b}{1992/02/16}{Moved macro from language
%    definition files}
%    This macro makes hyphenation possible. Basically its definition
%    is nothing more than |\nobreak| |\hskip| \texttt{0pt plus
%    0pt}\footnote{\TeX\ begins and ends a word for hyphenation at a
%    glue node. The penalty prevents a linebreak at this glue node.}.
%    \begin{macrocode}
\def\allowhyphens{\penalty\@M \hskip\z@skip}
%    \end{macrocode}
%  \end{macro}
%
%  \begin{macro}{\set@low@box}
% \changes{babel~3.2b}{1992/02/16}{Moved macro from language
%    definition files}
%    The following macro is used to lower quotes to the same level as
%    the comma.  It prepares its argument in box register~0.
%    \begin{macrocode}
\def\set@low@box#1{\setbox\tw@\hbox{,}\setbox\z@\hbox{#1}%
    \dimen\z@\ht\z@ \advance\dimen\z@ -\ht\tw@%
    \setbox\z@\hbox{\lower\dimen\z@ \box\z@}\ht\z@\ht\tw@ \dp\z@\dp\tw@}
%    \end{macrocode}
%  \end{macro}
%
%  \begin{macro}{\save@sf@q}
% \changes{babel~3.2b}{1992/02/16}{Moved macro from language
%    definition files}
%    The macro |\save@sf@q| is used to save and reset the current
%    space factor.
%    \begin{macrocode}
\def\save@sf@q#1{{\ifhmode
    \edef\@SF{\spacefactor\the\spacefactor}\else
    \let\@SF\empty \fi \leavevmode #1\@SF}}
%    \end{macrocode}
%  \end{macro}
%
% \changes{babel~3.5c}{1995/06/14}{Repaired a typo (itlaic, PR1652)}
%
%  \subsection{Making glyphs available}
%
%    The file \file{\filename}\footnote{The file described in this
%    section has version number \fileversion, and was last revised on
%    \filedate.} makes a number of glyphs available that either do not
%    exist in the \texttt{OT1} encoding and have to be `faked', or
%    that are not accessible through \file{T1enc.def}.
%
%  \subsection{Quotation marks}
%
%  \begin{macro}{\quotedblbase}
%    In the \texttt{T1} encoding the opening double quote at the
%    baseline is available as a separate character, accessible via
%    |\quotedblbase|. In the \texttt{OT1} encoding it is not
%    available, therefore we make it available by lowering the normal
%    open quote character to the baseline.
%    \begin{macrocode}
\ProvideTextCommand{\quotedblbase}{OT1}{%
  \save@sf@q{\set@low@box{\textquotedblright\/}%
    \box\z@\kern-.04em\allowhyphens}}
%    \end{macrocode}
%    Make sure that when an encoding other then \texttt{OT1} ot
%    \texttt{T1} is used this glyph can still be typeset.
%    \begin{macrocode}
\ProvideTextCommandDefault{\quotedblbase}{%
  \UseTextSymbol{OT1}{\quotedblbase}}
%    \end{macrocode}
%  \end{macro}
%
%  \begin{macro}{\quotesinglbase}
%    We also need the single quote character at the baseline.
%    \begin{macrocode}
\ProvideTextCommand{\quotesinglbase}{OT1}{%
  \save@sf@q{\set@low@box{\textquoteright\/}%
    \box\z@\kern-.04em\allowhyphens}}
%    \end{macrocode}
%    Make sure that when an encoding other then \texttt{OT1} ot
%    \texttt{T1} is used this glyph can still be typeset.
%    \begin{macrocode}
\ProvideTextCommandDefault{\quotesinglbase}{%
  \UseTextSymbol{OT1}{\quotesinglbase}}
%    \end{macrocode}
%  \end{macro}
%
%  \begin{macro}{\guillemotleft}
%  \begin{macro}{\guillemotright}
%    The guillemot characters are not available in \texttt{OT1}
%    encoding. They are faked.
%    \begin{macrocode}
\ProvideTextCommand{\guillemotleft}{OT1}{%
  \ifmmode
    \ll
  \else
    \save@sf@q{\penalty\@M
      \raise.2ex\hbox{$\scriptscriptstyle\ll$}\allowhyphens}%
  \fi}
\ProvideTextCommand{\guillemotright}{OT1}{%
  \ifmmode
    \gg
  \else
    \save@sf@q{\penalty\@M
      \raise.2ex\hbox{$\scriptscriptstyle\gg$}\allowhyphens}%
  \fi}
%    \end{macrocode}
%    Make sure that when an encoding other then \texttt{OT1} ot
%    \texttt{T1} is used these glyphs can still be typeset.
%    \begin{macrocode}
\ProvideTextCommandDefault{\guillemotleft}{%
  \UseTextSymbol{OT1}{\guillemotleft}}
\ProvideTextCommandDefault{\guillemotright}{%
  \UseTextSymbol{OT1}{\guillemotright}}
%    \end{macrocode}
%  \end{macro}
%  \end{macro}
%
%  \begin{macro}{\guilsinglleft}
%  \begin{macro}{\guilsinglright}
%    The single guillemots are not available in \texttt{OT1}
%    encoding. They are faked.
%    \begin{macrocode}
\ProvideTextCommand{\guilsinglleft}{OT1}{%
  \ifmmode
    <%
  \else
    \save@sf@q{\penalty\@M
      \raise.2ex\hbox{$\scriptscriptstyle<$}\allowhyphens}%
  \fi}
\ProvideTextCommand{\guilsinglright}{OT1}{%
  \ifmmode
    >%
  \else
    \save@sf@q{\penalty\@M
      \raise.2ex\hbox{$\scriptscriptstyle>$}\allowhyphens}%
  \fi}
%    \end{macrocode}
%    Make sure that when an encoding other then \texttt{OT1} ot
%    \texttt{T1} is used these glyphs can still be typeset.
%    \begin{macrocode}
\ProvideTextCommandDefault{\guilsinglleft}{%
  \UseTextSymbol{OT1}{\guilsinglleft}}
\ProvideTextCommandDefault{\guilsinglright}{%
  \UseTextSymbol{OT1}{\guilsinglright}}
%    \end{macrocode}
%  \end{macro}
%  \end{macro}
%
%
%  \subsection{Letters}
%
%  \begin{macro}{\ij}
%  \begin{macro}{\IJ}
%    The dutch language uses the letter `ij'. It is available in
%    \texttt{T1} encoded fonts, but not in the \texttt{OT1} encoded
%    fonts. Therefore we fake it for the \texttt{OT1} encoding.
% \changes{dutch-3.7a}{1995/02/04}{Changed the kerning in the faked ij
%    to match the dc-version of it}
%    \begin{macrocode}
\DeclareTextCommand{\ij}{OT1}{%
  \allowhyphens i\kern-0.02em j\allowhyphens}
\DeclareTextCommand{\IJ}{OT1}{%
  \allowhyphens I\kern-0.02em J\allowhyphens}
\DeclareTextCommand{\ij}{T1}{\char188}
\DeclareTextCommand{\IJ}{T1}{\char156}
%    \end{macrocode}
%    Make sure that when an encoding other then \texttt{OT1} or
%    \texttt{T1} is used these glyphs can still be typeset.
%    \begin{macrocode}
\ProvideTextCommandDefault{\ij}{%
  \UseTextSymbol{OT1}{\ij}}
\ProvideTextCommandDefault{\IJ}{%
  \UseTextSymbol{OT1}{\IJ}}
%    \end{macrocode}
%  \end{macro}
%  \end{macro}
%
%  \begin{macro}{\dj}
%  \begin{macro}{\DJ}
%    The croatian language needs the letters |\dj| and |\DJ|; they are
%    availabel in the \texttt{T1} encoding, but not in the
%    \texttt{OT1} encoding by default.
%
%    Some code to construct these glyphs for the \texttt{OT1} encoding
%    was made available to my by Stipcevic Mario,
%    (\texttt{stipcevic@olimp.irb.hr}).
%    \begin{macrocode}
\def\crrtic@{\hrule height0.1ex width0.3em}
\def\crttic@{\hrule height0.1ex width0.33em}
%
\def\dj@@#1#2#3{%
  \leavevmode\kern#1em\raise#2ex\vbox{\crrtic@}\kern#3em}
\def\dj@{%
  \ifcase\the\fam
  \dj@@{0.25}{1.28}{-0.50}\or %roman
  \dj@@{0.25}{1.28}{-0.50}\or %roman
  \dj@@{0.25}{1.28}{-0.50}\or %roman
  \dj@@{0.25}{1.28}{-0.50}\or %roman
  \dj@@{0.33}{1.29}{-0.63}\or %italic
  \dj@@{0.35}{1.30}{-0.65}\or %slanted
  \dj@@{0.25}{1.25}{-0.55}\or %bold
  \dj@@{0.25}{1.15}{-0.55}\or %tt
  \dj@@{-0.04}{0.72}{-0.26}\or%csc
  \dj@@{0.35}{1.30}{-0.65} %slanted
  \else\dj@@{0.25}{1.25}{-0.55}\fi}%bold
%
\def\DJ@@#1#2#3{%
  \leavevmode\kern#1em\raise#2ex\vbox{\crttic@}\kern#3em}
\def\DJ@{%
  \ifcase\the\fam
  \DJ@@{0.11}{0.92}{-0.36}\or
  \DJ@@{0.11}{0.92}{-0.36}\or
  \DJ@@{0.11}{0.92}{-0.36}\or
  \DJ@@{0.11}{0.92}{-0.36}\or
  \DJ@@{0.25}{0.92}{-0.50}\or
  \DJ@@{0.22}{0.92}{-0.47}\or
  \DJ@@{0.09}{0.90}{-0.34}\or
  \DJ@@{0.02}{0.85}{-0.27}\or
  \DJ@@{0.1}{0.86}{-0.35}\or
  \DJ@@{0.22}{0.92}{-0.47}\else
  \DJ@@{0.09}{0.90}{-0.34}\fi}
\DeclareTextCommand{\dj}{OT1}{\dj@ d}
\DeclareTextCommand{\DJ}{OT1}{\DJ@ D}
%    \end{macrocode}
%    Make sure that when an encoding other then \texttt{OT1} or
%    \texttt{T1} is used these glyphs can still be typeset.
%    \begin{macrocode}
\ProvideTextCommandDefault{\dj}{%
  \UseTextSymbol{OT1}{\dj}}
\ProvideTextCommandDefault{\DJ}{%
  \UseTextSymbol{OT1}{\DJ}}
%    \end{macrocode}
%  \end{macro}
%  \end{macro}
%
% \subsection{Shorthands for quotation marks}
%
%    Shorthands are provided for a number of different quotation
%    marks, which make them useable both outside and inside mathmode.
%
%  \begin{macro}{\glq}
%  \begin{macro}{\grq}
%    The `german' single quotes.
%    \begin{macrocode}
\DeclareRobustCommand{\glq}{%
  \ifmmode\mbox{\quotesinglbase}\else\quotesinglbase\fi}
\DeclareRobustCommand{\grq}{%
  \ifmmode\mbox{\textquoteleft}\else\textquoteleft\fi}
%    \end{macrocode}
%  \end{macro}
%  \end{macro}
%
%  \begin{macro}{\glqq}
%  \begin{macro}{\grqq}
%    The `german' double quotes.
%    \begin{macrocode}
\DeclareRobustCommand{\glqq}{%
  \ifmmode\mbox{\quotedblbase}\else\quotedblbase\fi}
\DeclareRobustCommand{\grqq}{%
  \ifmmode\mbox{\textquotedblleft}\else\textquotedblleft\fi}
%    \end{macrocode}
%  \end{macro}
%  \end{macro}
%
%  \begin{macro}{\flq}
%  \begin{macro}{\frq}
%    The `french' single quillemets.
%    \begin{macrocode}
\DeclareRobustCommand{\flq}{%
  \ifmmode\mbox{\quilsinglleft}\else\quilsinglleft\fi}
\DeclareRobustCommand{\frq}{%
  \ifmmode\mbox{\quilsinglright}\else\quilsinglright\fi}
%    \end{macrocode}
%  \end{macro}
%  \end{macro}
%
%  \begin{macro}{\flqq}
%  \begin{macro}{\frqq}
%    The `french' double quillemets.
%    \begin{macrocode}
\DeclareRobustCommand{\flqq}{%
  \ifmmode\mbox{\quillemotleft}\else\quillemotleft\fi}
\DeclareRobustCommand{\frqq}{%
  \ifmmode\mbox{\quillemotright}\else\quillemotright\fi}
%    \end{macrocode}
%  \end{macro}
%  \end{macro}
%
%  \subsection{Umlauts and trema's}
%
%    The command |\"| needs to have a different effect for different
%    languages. For German for instance, the `umlaut' should be
%    positioned lower than the default position for placing it over
%    the letters a, o, u, A, O and U. When placed over an e, i, E or I
%    it can retain its normal position. For Dutch the same glyph is
%    always placed in the lower position.
%
%  \begin{macro}{\umlauthigh}
%  \begin{macro}{\umlautlow}
%    To be able to provide both positions of |\"| we provide two
%    commands to switch the positioning, the default will be
%    |\umlauthigh| (the normal positioning).
%    \begin{macrocode}
\def\umlauthigh{%
  \def\bbl@umlauta##1{{%
      \expandafter\accent\csname\f@encoding dqpos\endcsname
      ##1\allowhyphens}}%
  \let\bbl@umlaute\bbl@umlauta}
\def\umlautlow{%
  \def\bbl@umlauta{\protect\lower@umlaut}}
\def\umlautelow{%
  \def\bbl@umlaute{\protect\lower@umlaut}}
\umlauthigh
%    \end{macrocode}
%  \end{macro}
%  \end{macro}
%
%  \begin{macro}{\lower@umlaut}
%    The command |\lower@umlaut| is used to position the |\"| closer
%    the the letter.
%
%    We want the umlaut character lowered, nearer to the letter. To do
%    this we need an extra \meta{dimen} register.
%    \begin{macrocode}
\expandafter\ifx\csname U@D\endcsname\relax
  \csname newdimen\endcsname\U@D
\fi
%    \end{macrocode}
%    The following code fools \TeX's \texttt{make\_accent} procedure
%    about the current x-height of the font to force another placement
%    of the umlaut character.
%    \begin{macrocode}
\def\lower@umlaut#1{%
%    \end{macrocode}
%    First we have to save the current x-height of the font, because
%    we'll change this font dimension and this is always done
%    globally.
%    \begin{macrocode}
  {\U@D 1ex%
%    \end{macrocode}
%    Then we compute the new x-height in such a way that the umlaut
%    character is lowered to the base character.  The value of
%    \texttt{.45ex} depends on the \MF\ parameters with which the
%    fonts were built.  (Just try out, which value will look best.)
%    \begin{macrocode}
  {\setbox\z@\hbox{%
      \expandafter\char\csname\f@encoding dqpos\endcsname}%
    \dimen@ -.45ex\advance\dimen@\ht\z@
%    \end{macrocode}
%    If the new x-height is too low, it is not changed.
%    \begin{macrocode}
  \ifdim 1ex<\dimen@ \fontdimen5\font\dimen@ \fi}%
%    \end{macrocode}
%    Finally we call the |\accent| primitive, reset the old x-height
%    and insert the base character in the argument.
%    \begin{macrocode}
  \expandafter\accent\csname\f@encoding dqpos\endcsname
  \fontdimen5\font\U@D #1}}
%    \end{macrocode}
%  \end{macro}
%
%    For all vowels we declare |\"| to be a composite command which
%    uses |\bbl@umlauta| or|\bbl@umlaute| to position the umlaut
%    character. We need to be sure that these definitions override the
%    ones that are provided when the package \textsf{fontenc} with
%    option \textsf{OT1} is used. Therefore these declarations are
%    postponed until the beginning of the document.
%    \begin{macrocode}
\AtBeginDocument{%
  \DeclareTextCompositeCommand{\"}{OT1}{a}{\bbl@umlauta{a}}%
  \DeclareTextCompositeCommand{\"}{OT1}{e}{\bbl@umlaute{e}}%
  \DeclareTextCompositeCommand{\"}{OT1}{i}{\bbl@umlaute{\i}}%
  \DeclareTextCompositeCommand{\"}{OT1}{\i}{\bbl@umlaute{\i}}%
  \DeclareTextCompositeCommand{\"}{OT1}{o}{\bbl@umlauta{o}}%
  \DeclareTextCompositeCommand{\"}{OT1}{u}{\bbl@umlauta{u}}%
  \DeclareTextCompositeCommand{\"}{OT1}{A}{\bbl@umlauta{A}}%
  \DeclareTextCompositeCommand{\"}{OT1}{E}{\bbl@umlaute{E}}%
  \DeclareTextCompositeCommand{\"}{OT1}{I}{\bbl@umlaute{I}}%
  \DeclareTextCompositeCommand{\"}{OT1}{O}{\bbl@umlauta{O}}%
  \DeclareTextCompositeCommand{\"}{OT1}{U}{\bbl@umlauta{U}}%
}
%    \end{macrocode}
%
% \subsection{The redefinition of the style commands}
%
%    The rest of the code in this file can only be processed by
%    \LaTeX, so we check the current format. If it is plain \TeX,
%    processing should stop here. But, because of the need to limit
%    the scope of the definition of |\format|, a macro that is used
%    locally in the following |\if|~statement, this comparison is done
%    inside a group. To prevent \TeX\ from complaining about an
%    unclosed group, the processing of the command |\endinput| is
%    deferred until after the group is closed. This is accomplished by
%    the command |\aftergroup|.
%    \begin{macrocode}
{\def\format{lplain}
\ifx\fmtname\format
\else
  \def\format{LaTeX2e}
  \ifx\fmtname\format
  \else
    \aftergroup\endinput
  \fi
\fi}
%    \end{macrocode}
%
%    Now that we're sure that the code is seen by \LaTeX\ only, we
%    have to find out what the main (primary) document style is
%    because we want to redefine some macros.  This is only necessary
%    for releases of \LaTeX\ dated before december~1991. Therefore
%    this part of the code can optionally be included in
%    \file{babel.def} by specifying the \texttt{docstrip} option
%    \texttt{names}.
%    \begin{macrocode}
%<*names>
%    \end{macrocode}
%
%    The standard styles can be distinguished by checking whether some
%    macros are defined. In table~\ref{styles} an overview is given of
%    the macros that can be used for this purpose.
%  \begin{table}[htb]
%  \begin{center}
% \DeleteShortVerb{\|}
%  \begin{tabular}{|lcp{8cm}|}
%   \hline
%   article         & : & both the \verb+\chapter+ and \verb+\opening+
%                         macros are undefined\\
%   report and book & : & the \verb+\chapter+ macro is defined and
%                         the \verb+\opening+ is undefined\\
%   letter          & : & the \verb+\chapter+ macro is undefined and
%                         the \verb+\opening+ is defined\\
%   \hline
%  \end{tabular}
% \caption{How to determine the main document style}\label{styles}
% \MakeShortVerb{\|}
%  \end{center}
%  \end{table}
%
%    \noindent The macros that have to be redefined for the
%    \texttt{report} and \texttt{book} document styles happen to be
%    the same, so there is no need to distinguish between those two
%    styles.
%
%  \begin{macro}{\doc@style}
%    First a parameter |\doc@style| is defined to identify the current
%    document style. This parameter might have been defined by a
%    document style that already uses macros instead of hard-wired
%    texts, such as \file{artikel1.sty}~\cite{BEP}, so the existence of
%    |\doc@style| is checked. If this macro is undefined, i.\,e., if
%    the document style is unknown and could therefore contain
%    hard-wired texts, |\doc@style| is defined to the default
%    value~`0'.
% \changes{babel~3.0d}{1991/10/29}{Removed use of \cs{@ifundefined}}
%    \begin{macrocode}
\ifx\undefined\doc@style
  \def\doc@style{0}%
%    \end{macrocode}
%    This parameter is defined in the following \texttt{if}
%    construction (see table~\ref{styles}):
%
%    \begin{macrocode}
  \ifx\undefined\opening
    \ifx\undefined\chapter
      \def\doc@style{1}%
    \else
      \def\doc@style{2}%
    \fi
  \else
    \def\doc@style{3}%
  \fi%
\fi%
%    \end{macrocode}
%  \end{macro}
%
% \changes{babel~3.1}{1991/11/05}{Removed definition of
%    \cs{if@restonecol}}
%
%    \subsubsection{Redefinition of macros}
%
%    Now here comes the real work: we start to redefine things and
%    replace hard-wired texts by macros. These redefinitions should be
%    carried out conditionally, in case it has already been done.
%
%    For the \texttt{figure} and \texttt{table} environments we have
%    in all styles:
%    \begin{macrocode}
\@ifundefined{figurename}{\def\fnum@figure{\figurename{} \thefigure}}{}
\@ifundefined{tablename}{\def\fnum@table{\tablename{} \thetable}}{}
%    \end{macrocode}
%
%    The rest of the macros have to be treated differently for each
%    style.  When |\doc@style| still has its default value nothing
%    needs to be done.
%    \begin{macrocode}
\ifcase \doc@style\relax
\or
%    \end{macrocode}
%
%    This means that \file{babel.def} is read after the
%    \texttt{article} style, where no |\chapter| and |\opening|
%    commands are defined\footnote{A fact that was pointed out to me
%    by Nico Poppelier and was already used in Piet van Oostrum's
%    document style option~\texttt{nl}.}.
%
%    First we have the |\tableofcontents|,
%    |\listoffigures| and |\listoftables|:
%    \begin{macrocode}
\@ifundefined{contentsname}%
    {\def\tableofcontents{\section*{\contentsname\@mkboth
          {\uppercase{\contentsname}}{\uppercase{\contentsname}}}%
      \@starttoc{toc}}}{}

\@ifundefined{listfigurename}%
    {\def\listoffigures{\section*{\listfigurename\@mkboth
          {\uppercase{\listfigurename}}{\uppercase{\listfigurename}}}
     \@starttoc{lof}}}{}

\@ifundefined{listtablename}%
    {\def\listoftables{\section*{\listtablename\@mkboth
          {\uppercase{\listtablename}}{\uppercase{\listtablename}}}
      \@starttoc{lot}}}{}
%    \end{macrocode}
%
% Then the |\thebibliography| and |\theindex| environments.
%
%    \begin{macrocode}
\@ifundefined{refname}%
    {\def\thebibliography#1{\section*{\refname
      \@mkboth{\uppercase{\refname}}{\uppercase{\refname}}}%
      \list{[\arabic{enumi}]}{\settowidth\labelwidth{[#1]}%
        \leftmargin\labelwidth
        \advance\leftmargin\labelsep
        \usecounter{enumi}}%
        \def\newblock{\hskip.11em plus.33em minus.07em}%
        \sloppy\clubpenalty4000\widowpenalty\clubpenalty
        \sfcode`\.=1000\relax}}{}

\@ifundefined{indexname}%
    {\def\theindex{\@restonecoltrue\if@twocolumn\@restonecolfalse\fi
     \columnseprule \z@
     \columnsep 35pt\twocolumn[\section*{\indexname}]%
       \@mkboth{\uppercase{\indexname}}{\uppercase{\indexname}}%
       \thispagestyle{plain}%
       \parskip\z@ plus.3pt\parindent\z@\let\item\@idxitem}}{}
%    \end{macrocode}
%
% The |abstract| environment:
%
%    \begin{macrocode}
\@ifundefined{abstractname}%
    {\def\abstract{\if@twocolumn
    \section*{\abstractname}%
    \else \small
    \begin{center}%
    {\bf \abstractname\vspace{-.5em}\vspace{\z@}}%
    \end{center}%
    \quotation
    \fi}}{}
%    \end{macrocode}
%
% And last but not least, the macro |\part|:
%
%    \begin{macrocode}
\@ifundefined{partname}%
{\def\@part[#1]#2{\ifnum \c@secnumdepth >\m@ne
        \refstepcounter{part}%
        \addcontentsline{toc}{part}{\thepart
        \hspace{1em}#1}\else
      \addcontentsline{toc}{part}{#1}\fi
   {\parindent\z@ \raggedright
    \ifnum \c@secnumdepth >\m@ne
      \Large \bf \partname{} \thepart
      \par \nobreak
    \fi
    \huge \bf
    #2\markboth{}{}\par}%
    \nobreak
    \vskip 3ex\@afterheading}%
}{}
%    \end{macrocode}
%
%    This is all that needs to be done for the \texttt{article} style.
%
%    \begin{macrocode}
\or
%    \end{macrocode}
%
%    The next case is formed by the two styles \texttt{book} and
%    \texttt{report}.  Basically we have to do the same as for the
%    \texttt{article} style, except now we must also change the
%    |\chapter| command.
%
%    The tables of contents, figures and tables:
%    \begin{macrocode}
\@ifundefined{contentsname}%
    {\def\tableofcontents{\@restonecolfalse
      \if@twocolumn\@restonecoltrue\onecolumn
      \fi\chapter*{\contentsname\@mkboth
          {\uppercase{\contentsname}}{\uppercase{\contentsname}}}%
      \@starttoc{toc}%
      \csname if@restonecol\endcsname\twocolumn
      \csname fi\endcsname}}{}

\@ifundefined{listfigurename}
    {\def\listoffigures{\@restonecolfalse
      \if@twocolumn\@restonecoltrue\onecolumn
      \fi\chapter*{\listfigurename\@mkboth
          {\uppercase{\listfigurename}}{\uppercase{\listfigurename}}}%
      \@starttoc{lof}%
      \csname if@restonecol\endcsname\twocolumn
      \csname fi\endcsname}}{}

\@ifundefined{listtablename}
    {\def\listoftables{\@restonecolfalse
      \if@twocolumn\@restonecoltrue\onecolumn
      \fi\chapter*{\listtablename\@mkboth
          {\uppercase{\listtablename}}{\uppercase{\listtablename}}}%
      \@starttoc{lot}%
      \csname if@restonecol\endcsname\twocolumn
      \csname fi\endcsname}}{}
%    \end{macrocode}
%
%    Again, the |bibliography| and |index| environments; notice that
%    in this case we use |\bibname| instead of |\refname| as in the
%    definitions for the \texttt{article} style.  The reason for this
%    is that in the \texttt{article} document style the term
%    `References' is used in the definition of |\thebibliography|. In
%    the \texttt{report} and \texttt{book} document styles the term
%    `Bibliography' is used.
%    \begin{macrocode}
\@ifundefined{bibname}
    {\def\thebibliography#1{\chapter*{\bibname
     \@mkboth{\uppercase{\bibname}}{\uppercase{\bibname}}}%
     \list{[\arabic{enumi}]}{\settowidth\labelwidth{[#1]}%
     \leftmargin\labelwidth \advance\leftmargin\labelsep
     \usecounter{enumi}}%
     \def\newblock{\hskip.11em plus.33em minus.07em}%
     \sloppy\clubpenalty4000\widowpenalty\clubpenalty
     \sfcode`\.=1000\relax}}{}

\@ifundefined{indexname}
    {\def\theindex{\@restonecoltrue\if@twocolumn\@restonecolfalse\fi
    \columnseprule \z@
    \columnsep 35pt\twocolumn[\@makeschapterhead{\indexname}]%
      \@mkboth{\uppercase{\indexname}}{\uppercase{\indexname}}%
    \thispagestyle{plain}%
    \parskip\z@ plus.3pt\parindent\z@ \let\item\@idxitem}}{}
%    \end{macrocode}
%
% Here is the |abstract| environment:
%    \begin{macrocode}
\@ifundefined{abstractname}
    {\def\abstract{\titlepage
    \null\vfil
    \begin{center}%
    {\bf \abstractname}%
    \end{center}}}{}
%    \end{macrocode}
%
%     And last but not least the |\chapter|, |\appendix| and
%    |\part| macros.
%    \begin{macrocode}
\@ifundefined{chaptername}{\def\@chapapp{\chaptername}}{}
%
\@ifundefined{appendixname}
    {\def\appendix{\par
      \setcounter{chapter}{0}%
      \setcounter{section}{0}%
      \def\@chapapp{\appendixname}%
      \def\thechapter{\Alph{chapter}}}}{}
%
\@ifundefined{partname}
    {\def\@part[#1]#2{\ifnum \c@secnumdepth >-2\relax
            \refstepcounter{part}%
            \addcontentsline{toc}{part}{\thepart
            \hspace{1em}#1}\else
            \addcontentsline{toc}{part}{#1}\fi
       \markboth{}{}%
       {\centering
        \ifnum \c@secnumdepth >-2\relax
          \huge\bf \partname{} \thepart
        \par
        \vskip 20pt \fi
        \Huge \bf
        #1\par}\@endpart}}{}%
%    \end{macrocode}
%
%    \begin{macrocode}
\or
%    \end{macrocode}
%
%    Now we address the case where \file{babel.def} is read after the
%    \texttt{letter} style. The \texttt{letter} document style
%    defines the macro |\opening| and some other macros that are
%    specific to \texttt{letter}. This means that we have to redefine
%    other macros, compared to the previous two cases.
%
%    First two macros for the material at the end of a letter, the
%    |\cc| and |\encl| macros.
%    \begin{macrocode}
\@ifundefined{ccname}%
    {\def\cc#1{\par\noindent
     \parbox[t]{\textwidth}%
     {\@hangfrom{\rm \ccname : }\ignorespaces #1\strut}\par}}{}

\@ifundefined{enclname}%
    {\def\encl#1{\par\noindent
     \parbox[t]{\textwidth}%
     {\@hangfrom{\rm \enclname : }\ignorespaces #1\strut}\par}}{}
%    \end{macrocode}
%
%    The last thing we have to do here is to redefine the
%    \texttt{headings} pagestyle:
% \changes{babel~3.3}{1993/07/11}{\cs{headpagename} should be
%    \cs{pagename}}
%    \begin{macrocode}
\@ifundefined{headtoname}
    {\def\ps@headings{%
        \def\@oddhead{\sl \headtoname{} \ignorespaces\toname \hfil
                      \@date \hfil \pagename{} \thepage}%
        \def\@oddfoot{}}}{}
%    \end{macrocode}
%
%    This was the last of the four standard document styles, so if
%    |\doc@style| has another value we do nothing and just close the
%    \texttt{if} construction.
%    \begin{macrocode}
\fi
%    \end{macrocode}
%    Here ends the code that can be optionally included when a version
%    of \LaTeX\ is in use that is dated \emph{before} december~1991.
%    \begin{macrocode}
%</names>
%</core>
%    \end{macrocode}
%
% \subsection{Cross referencing macros}
%
%    The \LaTeX\ book states:
%  \begin{quote}
%    The \emph{key} argument is any sequence of letters, digits, and
%    punctuation symbols; upper- and lowercase letters are regarded as
%    different.
%  \end{quote}
%    When the above quote should still be true when a document is
%    typeset in a language that has active characters, special care
%    has to be taken of the category codes of these characters when
%    they appear in an argument of the cross referencing macros.
%
%    When a cross referencing command processes its argument, all
%    tokens in this argument should be character tokens with category
%    `letter' or `other'.
%
%    The only way to accomplish this in most cases is to use the trick
%    described in the \TeX book~\cite{DEK} (Appendix~D, page~382).
%    The primitive |\meaning| applied to a token expands to the
%    current meaning of this token.  For example, `|\meaning\A|' with
%    |\A| defined as `|\def\A#1{\B}|' expands to the characters
%    `|macro:#1->\B|' with all category codes set to `other' or
%    `space'.
%
%  \begin{macro}{\babel@sanitize@arg}
%    To call a macro with a `sanitized' argument, instead of |\A{\B}|
%    one would write |\babel@sanitize@arg{\A}{\B}|.  (But be careful,
%    this macro is not fully expandable!)
%    \begin{macrocode}
%tmp%\long\def\babel@sanitize@arg#1#2{\bgroup\def\@tempa{#2}%
%tmp%  \expandafter\babel@strip@meaning\meaning\@tempa\relax{#1}}
%tmp%\def\babel@strip@meaning#1->#2\relax#3{\egroup #3{#2}}
%    \end{macrocode}
%  \end{macro}
%
%    To redefine a command, we save the old meaning of the macro.
%    Then we redefine it to call the original macro with the
%    `sanitized' argument.  The reason why we do it this way is that
%    we don't want to redefine the \LaTeX\ macros completely incase
%    their definitions change (they have changed in the past).
%
%  \begin{macro}{\label}
%    The |\label| macro is one of the cross referencing macros
%    affected. First we save its original definition.
%    \begin{macrocode}
%tmp%\let\LTX@label=\label
%    \end{macrocode}
%    Then the macro |label| is redefined.
%    \begin{macrocode}
%tmp%\def\label#1{\babel@sanitize@arg\LTX@label{#1}}
%    \end{macrocode}
%  \end{macro}
%
%  \begin{macro}{\newlabel}
%    The macro |\label| writes a line with a |\newlabel| command
%    into the |.aux| file to define labels.
%    \begin{macrocode}
%<*core|shorthands>
\let\bbl@newlabel\newlabel
\def\newlabel#1#2{%
  \@safe@activestrue\bbl@newlabel{#1}{#2}\@safe@activesfalse}
%    \end{macrocode}
%  \end{macro}
%
%  \begin{macro}{\@testdef}
%    An internal \LaTeX\ macro used to test if the labels that have
%    been written on the |.aux| file have changed.  It is called by
%    the |\enddocument| macro.
% \changes{babel~3.4g}{1994/08/30}{Moved the \cs{def} inside the
%    macrocode environment}
%    \begin{macrocode}
\let\bbl@@testdef\@testdef
\def\@testdef#1#2#3{%
  \@safe@activestrue\bbl@@testdef{#1}{#2}{#3}\@safe@activesfalse}
%    \end{macrocode}
%  \end{macro}
%
%  \begin{macro}{\ref}
%  \begin{macro}{\pageref}
%    The same holds for the macro |\ref| that references a label
%    and |\pageref| to reference a page. While we change these macros,
%    we make them robust as well to prevent problems when they should
%    become expanded at the wrong moment.
% \changes{babel~3.5b}{1995/03/07}{Made \cs{ref} and \cs{pageref}
%    robust (PR1353)}
% \changes{babel~3.5d}{1995/07/04}{use a different control sequence
%    while making \cs{ref} and \cs{pageref} robust}
%    \begin{macrocode}
\let\bbl@ref\ref
\edef\ref{\noexpand\protect
  \expandafter\noexpand\csname bblref \endcsname}
\expandafter\def\csname bblref \endcsname#1{%
  \@safe@activestrue\bbl@ref{#1}\@safe@activesfalse}
%    \end{macrocode}
%    \begin{macrocode}
\let\bbl@pageref\pageref
\edef\pageref{\noexpand\protect
  \expandafter\noexpand\csname bblpageref \endcsname}
\expandafter\def\csname bblpageref \endcsname#1{%
  \@safe@activestrue\bbl@pageref{#1}\@safe@activesfalse}
%    \end{macrocode}
%  \end{macro}
%  \end{macro}
%
%  \begin{macro}{\@citex}
%    The macro used to cite from a bibliography, |\cite| uses an
%    internal macro, |\@citex|.
%    It is this internal macro that picks up the argument,
%    so we redefine this internal macro and leave |\cite| alone.
%    \begin{macrocode}
%tmp%\let\bbl@@citex\@citex
%tmp%\def\@citex[#1]#2{\babel@sanitize@arg{\bbl@@citex[#1]}{#2}}
%    \end{macrocode}
%  \end{macro}
%
%  \begin{macro}{\nocite}
%    The macro |\nocite| which is used to instruct BiB\TeX\ to
%    extract uncited references from the database.
%    \begin{macrocode}
%tmp%\let\bbl@nocite\nocite
%tmp%\def\nocite#1{\babel@sanitize@arg\bbl@nocite{#1}}
%    \end{macrocode}
%  \end{macro}
%
%  \begin{macro}{\bibcite}
%    The macro that is used in the |.aux| file to define citation
%    labels.
%    \begin{macrocode}
%tmp%\let\bbl@bibcite\bibcite
%tmp%\def\bibcite#1#2{\babel@sanitize@arg\bbl@bibcite{#1}{#2}}
%    \end{macrocode}
%  \end{macro}
%
%  \begin{macro}{\@bibitem}
%    One of the two internal \LaTeX\ macros called by |\bibitem|
%    that write the citation label on the |.aux| file.
%    \begin{macrocode}
%tmp%\let\bbl@@bibitem\@bibitem
%tmp%\def\@bibitem#1{\babel@sanitize@arg\bbl@@bibitem{#1}}
%    \end{macrocode}
%  \end{macro}
%
%  \begin{macro}{\@lbibitem}
%    The other of the two internal \LaTeX\ macros called by |\bibitem|
%    that write the citation label on the |.aux| file.
%    \begin{macrocode}
%tmp%\let\bbl@@lbibitem\@lbibitem
%tmp%\def\@lbibitem[#1]#2{\babel@sanitize@arg{\bbl@@lbibitem[#1]}{#2}}
%    \end{macrocode}
%  \end{macro}
%
%    \begin{macrocode}
%</core|shorthands>
%    \end{macrocode}
%
% \section{Local Language Configuration}
%
%  \begin{macro}{\loadlocalcfg}
%    At some sites it may be necessary to add sitespecific actions to
%    a language definition file. This can be done by creating a file
%    with the same name as the language defintion file, but with the
%    extension \file{.cfg}. For instance the file \file{norsk.cfg}
%    will be loaded when the langauge definition file \file{norsk.ldf}
%    is loaded.
%
% \changes{babel~3.5d}{1995/06/22}{Added macro}
%    \begin{macrocode}
%<*core>
\def\loadlocalcfg#1{%
  \InputIfFileExists{#1.cfg}
           {\typeout{*************************************^^J%
                     * Local config file #1.cfg used^^J%
                     *}%
            }
           {}}
%</core>
%    \end{macrocode}
%  \end{macro}
%
%
% \clearpage
% \section{Driver files for the documented source code}
%
%    Since \babel\ version 3.4 all source files that are part of the
%    \babel\ system can be typeset separately. But in order to typeset
%    them all in one document the file \file{babel.drv} can be used.
%    If you only want the information on how to use the \babel\ system
%    and what goodies are provided by the language spcific files you
%    can run the file \file{user.drv} through \LaTeX\ to get a user
%    guide.
%
% \changes{babel~3.4b}{1994/05/18}{Use the ltxdoc class instead of
%    article}
%    \begin{macrocode}
%<*driver>
\documentclass{ltxdoc}
\DoNotIndex{\!,\',\,,\.,\-,\:,\;,\?,\/,\^,\`,\@M}
\DoNotIndex{\@,\@ne,\@m,\@afterheading,\@date,\@endpart}
\DoNotIndex{\@hangfrom,\@idxitem,\@makeschapterhead,\@mkboth}
\DoNotIndex{\@oddfoot,\@oddhead,\@restonecolfalse,\@restonecoltrue}
\DoNotIndex{\@starttoc,\@unused}
\DoNotIndex{\accent,\active}
\DoNotIndex{\addcontentsline,\advance,\Alph,\arabic}
\DoNotIndex{\baselineskip,\begin,\begingroup,\bf,\box,\c@secnumdepth}
\DoNotIndex{\catcode,\centering,\char,\chardef,\clubpenalty}
\DoNotIndex{\columnsep,\columnseprule,\crcr,\csname}
\DoNotIndex{\day,\def,\dimen,\discretionary,\divide,\dp,\do}
\DoNotIndex{\edef,\else,\empty,\end,\endgroup,\endcsname,\endinput}
\DoNotIndex{\errhelp,\errmessage,\expandafter,\fi,\filedate}
\DoNotIndex{\fileversion,\fmtname,\fnum@figure,\fnum@table,\fontdimen}
\DoNotIndex{\gdef,\global}
\DoNotIndex{\hbox,\hidewidth,\hfil,\hskip,\hspace,\ht,\Huge,\huge}
\DoNotIndex{\ialign,\if@twocolumn,\ifcase,\ifcat,\ifhmode,\ifmmode}
\DoNotIndex{\ifnum,\ifx,\immediate,\ignorespaces,\input,\item}
\DoNotIndex{\kern}
\DoNotIndex{\labelsep,\Large,\large,\labelwidth,\lccode,\leftmargin}
\DoNotIndex{\lineskip,\leavevmode,\let,\list,\ll,\long,\lower}
\DoNotIndex{\m@ne,\mathchar,\mathaccent,\markboth,\month,\multiply}
\DoNotIndex{\newblock,\newbox,\newcount,\newdimen,\newif,\newwrite}
\DoNotIndex{\nobreak,\noexpand,\noindent,\null,\number}
\DoNotIndex{\onecolumn,\or}
\DoNotIndex{\p@,par, \parbox,\parindent,\parskip,\penalty}
\DoNotIndex{\protect,\ps@headings}
\DoNotIndex{\quotation}
\DoNotIndex{\raggedright,\raise,\refstepcounter,\relax,\rm,\setbox}
\DoNotIndex{\section,\setcounter,\settowidth,\scriptscriptstyle}
\DoNotIndex{\sfcode,\sl,\sloppy,\small,\space,\spacefactor,\strut}
\DoNotIndex{\string}
\DoNotIndex{\textwidth,\the,\thechapter,\thefigure,\thepage,\thepart}
\DoNotIndex{\thetable,\thispagestyle,\titlepage,\tracingmacros}
\DoNotIndex{\tw@,\twocolumn,\typeout,\uppercase,\usecounter}
\DoNotIndex{\vbox,\vfil,\vskip,\vspace,\vss}
\DoNotIndex{\widowpenalty,\write,\xdef,\year,\z@,\z@skip}
%    \end{macrocode}
%
%     Here |\dlqq| is defined so that  an example of |"'| can be
%     given.
%    \begin{macrocode}
\makeatletter
\gdef\dlqq{{\setbox\tw@=\hbox{,}\setbox\z@=\hbox{''}%
  \dimen\z@=\ht\z@ \advance\dimen\z@-\ht\tw@
  \setbox\z@=\hbox{\lower\dimen\z@\box\z@}\ht\z@=\ht\tw@
  \dp\z@=\dp\tw@ \box\z@\kern-.04em}}
%    \end{macrocode}
%
%    The code lines are numbered within sections,
%    \begin{macrocode}
%<*!user>
\@addtoreset{CodelineNo}{section}
\renewcommand\theCodelineNo{%
  \reset@font\scriptsize\thesection.\arabic{CodelineNo}}
%    \end{macrocode}
%    which should also be visible in the index; hence this
%    redefinition of a macro from \file{doc.sty}.
%    \begin{macrocode}
\renewcommand\codeline@wrindex[1]{\if@filesw
        \immediate\write\@indexfile
            {\string\indexentry{#1}%
            {\number\c@section.\number\c@CodelineNo}}\fi}
%    \end{macrocode}
%
%    The glossary environment is used or the change log, but its
%    definition needs changing for this document.
%    \begin{macrocode}
\renewenvironment{theglossary}{%
    \glossary@prologue%
    \GlossaryParms \let\item\@idxitem \ignorespaces}%
   {}
%</!user>
\makeatother
%    \end{macrocode}
%
%    A few shorthands used in the documentation
%    \begin{macrocode}
\font\manual=logo10 % font used for the METAFONT logo, etc.
\newcommand*\MF{{\manual META}\-{\manual FONT}}
\newcommand*\TeXhax{\TeX hax}
\newcommand*\babel{\textsf{babel}}
\newcommand*\m[1]{\mbox{$\langle$\it#1\/$\rangle$}}
\newcommand*\langvar{\m{lang}}
%    \end{macrocode}
%
%     Some more definitions needed in the documentation.
%    \begin{macrocode}
\newcommand*\note[1]{\textbf{#1}}
%\newcommand*\note[1]{}
\newcommand*\bsl{\protect\bslash}
\newcommand*\Lopt[1]{\textsf{#1}}
\newcommand*\file[1]{\texttt{#1}}
\newcommand*\pkg[1]{\texttt{#1}}
\newcommand*\langdeffile[1]{%
%<-user>  \clearpage
  \DocInput{#1}}
%    \end{macrocode}
%
%    When a full index should be generated unomment the line with
%    |\EnableCrossres|. Beware, processing may take some time.
%    Use |\DisableCrossrefs| when the index is ready.
%    \begin{macrocode}
%  \EnableCrossrefs
\DisableCrossrefs
%    \end{macrocode}
%
%    Inlude the change log.
%    \begin{macrocode}
%<-user>\RecordChanges
%    \end{macrocode}
%    The index should use the linenumbers of the code.
%    \begin{macrocode}
%<-user>\CodelineIndex
%    \end{macrocode}
%
% Set everything in |\MacroFont| instead of |\AltMacroFont|
%    \begin{macrocode}
\setcounter{StandardModuleDepth}{1}
%    \end{macrocode}
%
%    For the user guide we only want the description parts of all the
%    files.
%    \begin{macrocode}
%<+user>\OnlyDescription
%    \end{macrocode}
%    Here starts the document
%    \begin{macrocode}
\begin{document}
\DocInput{babel.dtx}
%    \end{macrocode}
%
%    All the language definition files.
% \changes{babel~3.2e}{1992/07/07}{Added slovak}
% \changes{babel~3.3}{1993/07/11}{Added catalan and galician}
% \changes{babel~3.3}{1993/07/11}{Added turkish}
% \changes{babel~3.4}{1994/02/28}{Added bahasa}
% \changes{babel~3.5a}{1995/02/16}{Added breton, irish, scottish}
% \changes{babel~3.5b}{1995/05/19}{Added lsorbian, usorbian}
% \changes{babel~3.5c}{1995/06/14}{Changed the order of including the
%    language files somwhat (PR1652)}
%    \begin{macrocode}
%<+user>\clearpage
\langdeffile{esperant.dtx}
\langdeffile{dutch.dtx}
\langdeffile{english.dtx}
\langdeffile{germanb.dtx}
%
\langdeffile{breton.dtx}
\langdeffile{irish.dtx}
\langdeffile{scottish.dtx}
%
\langdeffile{francais.dtx}
\langdeffile{italian.dtx}
\langdeffile{portuges.dtx}
\langdeffile{spanish.dtx}
\langdeffile{catalan.dtx}
\langdeffile{galician.dtx}
\langdeffile{romanian.dtx}
%
\langdeffile{danish.dtx}
\langdeffile{norsk.dtx}
\langdeffile{swedish.dtx}
%
\langdeffile{finnish.dtx}
\langdeffile{magyar.dtx}
\langdeffile{estonian.dtx}
%
\langdeffile{croatian.dtx}
\langdeffile{czech.dtx}
\langdeffile{polish.dtx}
\langdeffile{slovak.dtx}
\langdeffile{slovene.dtx}
%\langdeffile{russian.dtx}
%
\langdeffile{lsorbian.dtx}
\langdeffile{usorbian.dtx}
\langdeffile{turkish.dtx}
%
\langdeffile{bahasa.dtx}
%    \end{macrocode}
%    Finally print the index and change log (not for the user guide).
%    \begin{macrocode}
%<*!user>
\clearpage
\def\filename{index}
\PrintIndex
\clearpage
\def\filename{changes}
\PrintChanges
%</!user>
\end{document}
%</driver>
%    \end{macrocode}
%
% \Finale
%
%%
%% \CharacterTable
%%  {Upper-case    \A\B\C\D\E\F\G\H\I\J\K\L\M\N\O\P\Q\R\S\T\U\V\W\X\Y\Z
%%   Lower-case    \a\b\c\d\e\f\g\h\i\j\k\l\m\n\o\p\q\r\s\t\u\v\w\x\y\z
%%   Digits        \0\1\2\3\4\5\6\7\8\9
%%   Exclamation   \!     Double quote  \"     Hash (number) \#
%%   Dollar        \$     Percent       \%     Ampersand     \&
%%   Acute accent  \'     Left paren    \(     Right paren   \)
%%   Asterisk      \*     Plus          \+     Comma         \,
%%   Minus         \-     Point         \.     Solidus       \/
%%   Colon         \:     Semicolon     \;     Less than     \<
%%   Equals        \=     Greater than  \>     Question mark \?
%%   Commercial at \@     Left bracket  \[     Backslash     \\
%%   Right bracket \]     Circumflex    \^     Underscore    \_
%%   Grave accent  \`     Left brace    \{     Vertical bar  \|
%%   Right brace   \}     Tilde         \~}
\endinput
