%%% ====================================================================
%%% @LaTeX-file{
%%%   filename  = "amsopn.dtx",
%%%   version   = "1.2a",
%%%   date      = "1995/02/20",
%%%   time      = "12:11:47 EST",
%%%   author    = "American Mathematical Society",
%%%   copyright = "Copyright (C) 1995 American Mathematical Society,
%%%                all rights reserved.  Copying of this file is
%%%                authorized only if either:
%%%                (1) you make absolutely no changes to your copy,
%%%                including name; OR
%%%                (2) if you do make changes, you first rename it
%%%                to some other name.",
%%%   address   = "American Mathematical Society,
%%%                Technical Support,
%%%                Electronic Products and Services,
%%%                P. O. Box 6248,
%%%                Providence, RI 02940,
%%%                USA",
%%%   telephone = "401-455-4080 or (in the USA and Canada)
%%%                800-321-4AMS (321-4267)",
%%%   FAX       = "401-331-3842",
%%%   checksum  = "48778 244 1038 9330",
%%%   email     = "tech-support@math.ams.org (Internet)",
%%%   codetable = "ISO/ASCII",
%%%   keywords  = "amslatex, ams-latex, operator name, amsopn",
%%%   supported = "yes",
%%%   abstract  = "This is part of the AMS-\LaTeX{} distribution. It
%%%                provides a variety of extra mathematical features,
%%%                largely derived from AMS-\TeX{}.",
%%%   docstring = "The checksum field above contains a CRC-16 checksum
%%%                as the first value, followed by the equivalent of
%%%                the standard UNIX wc (word count) utility output of
%%%                lines, words, and characters.  This is produced by
%%%                Robert Solovay's checksum utility.",
%%% }
%%% ====================================================================
%
%\iffalse
%<*driver>
\documentclass{amsdtx}
\begin{document}
\title{The \pkg{amsopn} package}
\author{Michael Downes}
\date{Version \fileversion, \filedate}
\hDocInput{amsopn.dtx}
\end{document}
%</driver>
%\fi
%
% \maketitle
%
% \MakeShortVerb\|
%
% \section{Introduction}
%
%    The \pkg{amsopn} package provides a command
%    \cn{DeclareMathOperator} for defining new `math operator names'
%    similar to the standard function names \cn{sin}, \cn{lim},
%    \cn{max}, etc.
%
% \StopEventually{}
%
%    Standard file identification.
%    \begin{macrocode}
\ProvidesPackage{amsopn}[1995/02/20 v1.2a operator names]
%    \end{macrocode}
%
%    The definition of \cs{nolimits@} shows a standard \amstex/
%    construction.  We want to find out what the next token is going
%    to be, so we use
% \begin{verbatim}
%    \futurelet\@let@token\next@
% \end{verbatim}
%    The \cs{@let@token} is assigned the value of the next token and
%    then we must define \cs{next@} to look at \cs{@let@token} and
%    decide what to do. We could name \cs{next@} to be \cs{nolimits@@}
%    or something, but by always using \cs{next@} in similar
%    constructions we cut down on the number of control sequence names
%    that we use. But this means that \cs{nolimits@} must incorporate
%    the appropriate definition of \cs{next@}. Notice that \cs{next@}
%    turns around and redefines itself again; once more this is to save
%    a control sequence name.
%
%    What \cs{nolimits@} does is keep a \cn{limits} typed by the
%    user from having any effect. This is used for operatornames whose
%    standard usage is never to have limits.
%    \begin{macrocode}
\def\nolimits@{%
  \DN@{\nolimits\ifx\@let@token\limits\expandafter\@gobble\fi}%
  \FN@\next@}
%    \end{macrocode}
%
%    In operator names, it is sometimes desired to have text-mode
%    punctuation characters such as |*-/:'|.
%    Because the body of an
%    operator name is set in math mode, these few punctuation
%    characters will not come out right (wrong symbol/and or wrong
%    spacing). The purpose of \cs{newmcodes@} is to make them act like
%    their normal text versions.
%
%    Where practical, we use decimal numbers to cut down main mem
%    usage (\qc{\"} not needed).
%    \begin{macrocode}
\begingroup \catcode`\"=12
\gdef\newmcodes@{\mathcode`\'39\mathcode`\*42\mathcode`\."613A%
 \mathcode`\-45\mathcode`\/47\mathcode`\:"603A\relax}
\endgroup
%    \end{macrocode}
%
%    The command \cs{operatorname} prints its argument as a `math
%    operator' like \cs{sin} or \cs{det}, with proper font and spacing.
%    \begin{macrocode}
\DeclareRobustCommand{\operatorname}{%
  \@ifstar{\qopname\newmcodes@ m}%
          {\qopname\newmcodes@ o}}%
%    \end{macrocode}
%    In the interior of the \cs{mathop} we need a null object (we choose
%    a zero kern for minimum waste of main mem) in order to guard
%    against the case where \arg{3} is a single letter; \tex/ will seize
%    it and center it on the math axis if there is nothing else inside
%    the \cs{mathop} atom.
%    \begin{macrocode}
\DeclareRobustCommand{\qopname}[3]{%
  \mathop{#1\kern\z@\operator@font#3}%
  \csname n#2limits@\endcsname}
%    \end{macrocode}
%
%  \begin{macro}{\DeclareMathOperator}
%    The command \cn{DeclareMathOperator} defines the first argument to
%    be an operator name whose text is the second argument. The star
%    form means that the operator name should take limits (like \cn{max}
%    or \cn{lim}).
%    \begin{macrocode}
\newcommand{\DeclareMathOperator}{%
  \@ifstar{\@declmathop\@empty}{\@declmathop o}}
%    \end{macrocode}
%  \end{macro}
%
%    In the basic set of operator names (below) we did not use
%    \cn{DeclareRobustCommand} because of the hash table cost. But we
%    use it here to minimize the chances of trouble, since we are
%    producing a user-defined command.
%    \begin{macrocode}
\long\def\@declmathop#1#2#3{%
  \@ifdefinable{#2}{%
    \DeclareRobustCommand{#2}{\qopname\newmcodes@#1{#3}}}}
%
\@onlypreamble\DeclareMathOperator
\@onlypreamble\@declmathop
%    \end{macrocode}
%
%    \begin{macrocode}
\def\arccos{\qopname\relax o{arccos}}
\def\arcsin{\qopname\relax o{arcsin}}
\def\arctan{\qopname\relax o{arctan}}
\def\arg{\qopname\relax o{arg}}
\def\cos{\qopname\relax o{cos}}
\def\cosh{\qopname\relax o{cosh}}
\def\cot{\qopname\relax o{cot}}
\def\coth{\qopname\relax o{coth}}
\def\csc{\qopname\relax o{csc}}
\def\deg{\qopname\relax o{deg}}
\def\det{\qopname\relax\@empty{det}}
\def\dim{\qopname\relax o{dim}}
\def\exp{\qopname\relax o{exp}}
\def\gcd{\qopname\relax\@empty{gcd}}
\def\hom{\qopname\relax o{hom}}
\def\inf{\qopname\relax\@empty{inf}}
\def\injlim{\qopname\relax\@empty{inj\,lim}}
\def\ker{\qopname\relax o{ker}}
\def\lg{\qopname\relax o{lg}}
\def\lim{\qopname\relax\@empty{lim}}
\def\liminf{\qopname\relax\@empty{lim\,inf}}
\def\limsup{\qopname\relax\@empty{lim\,sup}}
\def\ln{\qopname\relax o{ln}}
\def\log{\qopname\relax o{log}}
\def\max{\qopname\relax\@empty{max}}
\def\min{\qopname\relax\@empty{min}}
\def\Pr{\qopname\relax\@empty{Pr}}
\def\projlim{\qopname\relax\@empty{proj\,lim}}
\def\sec{\qopname\relax o{sec}}
\def\sin{\qopname\relax o{sin}}
\def\sinh{\qopname\relax o{sinh}}
\def\sup{\qopname\relax\@empty{sup}}
\def\tan{\qopname\relax o{tan}}
\def\tanh{\qopname\relax o{tanh}}
%    \end{macrocode}
%
% \begin{macro}{\operator@font}
%    This command is provided to allow the document styles to decide in
%    which way math operators like `max' or `log' are typeset. The
%    default is to set them in \meta{math group} zero of the current
%    math version.
%    \begin{macrocode}
\def\operator@font{\mathgroup\symoperators}
%    \end{macrocode}
% \end{macro}
%
%    For backwards compatibility we keep this old command name for the
%    time being:
%    \begin{macrocode}
\def\operatornamewithlimits{\operatorname*}
%    \end{macrocode}
%
%    These macros use \cs{mathpalette}s so that they will change size
%    in script and scriptscript styles, though it's hard to imagine they
%    will ever be used there (the arrows, particularly, look bad in
%    subscript sizes). Notice that the use of \cs{ex@} means that the
%    vertical spacing may not be optimal in script and scriptscript
%    sizes. Unfortunately \tex/ provides no easy way to do math mode
%    vertical spacing that varies with current math style like mu units.
%    \begin{macrocode}
\def\varlim@#1#2{\mathop{\vtop{\ialign{##\crcr
 \hfil$#1\m@th\operator@font lim$\hfil\crcr
 \noalign{\nointerlineskip\kern\ex@}#2#1\crcr
 \noalign{\nointerlineskip\kern-\ex@}\crcr}}}}
\def\varinjlim{\mathpalette\varlim@\rightarrowfill@}
\def\varprojlim{\mathpalette\varlim@\leftarrowfill@}
\def\varliminf{\mathpalette\varliminf@{}}
\def\varliminf@#1{\mathop{\@@underline{\vrule\@depth.2\ex@\@width\z@
   \hbox{$#1\m@th\operator@font lim$}}}}
\def\varlimsup{\mathpalette\varlimsup@{}}
\def\varlimsup@#1{\mathop{\@@overline
  {\hbox{$#1\m@th\operator@font lim$}}}}
%    \end{macrocode}
%
%    \begin{macrocode}
\DeclareOption{namelimits}{\let\nmlimits@\displaylimits}
\DeclareOption{nonamelimits}{\let\nmlimits@\nolimits}
\ProcessOptions\relax
%    \end{macrocode}
%
%    The usual \cs{endinput} to ensure that random garbage at the end of
%    the file doesn't get copied by \fn{docstrip}.
%    \begin{macrocode}
\endinput
%    \end{macrocode}
%
% \changes{v1.2a}{1995/02/20}{Fixed operatorname
%   (shouldn't have same limits behavior as operatorname*)}
% \CheckSum{292}
% \Finale
